% $Id: CGroups.tex,v 1.1 2004/02/11 09:39:57 lcf Exp $

\section{Constructive Commutative Algebra}
We define the notions of commutative monoid, group, ring, integral
domain and field in a constructive way. In doing so, we follow
\cite{Ruit82} and \cite{TvD882}, by requiring the basic operations to
be strongly extensional. In the end this choice does not effect our
work, because in a real number structure, it can be proved from the
axioms that all basic operations and relations are strongly
extensional.

\begin{convention}\label{convstrext}
Without stating it explicitly, we require all operations on setoids to respect
the equality.
%, that is, for $f$ an operation in the structure we require 
%$\forall x, y [x = y \implies f(x) = f(y)].$ 
We also require all basic operations and relations on setoids to be
strongly extensional (Definition \ref{defstrext}).
\end{convention}

\subsection{Groups: One associative operation}
\begin{definition}[Constructive Semi-Group]\label{defsemigrp}
A {\em constructive semi-group\/} is a tuple \struct{S,+,=,\noto} with
\struct{S,=,\noto} a constructive setoid, $+$ a binary operation on
$S$ such that
  \begin{enumerate}
  \item $+$ is {\em associative}: $\forall x, y,z[(x+y)+z = x+(y+z)]$.
   \end{enumerate}
\end{definition}

\begin{definition}[Constructive Monoid]\label{defmonoid}
A {\em constructive monoid\/} is a tuple \struct{S,0,+,=,\noto} with
\struct{S,+,=,\noto} a constructive semi-group and $0$ an element of
$S$ such that
  \begin{enumerate}
  \item $0$ is the {\em identity\/} w.r.t.\ $+$: $\forall x[x+0=x]$.
  \end{enumerate}
\end{definition}

\begin{definition}[Constructive Group]\label{defgroup}
  A {\em constructive group\/} is a tuple \struct{S,0,+,-,=,\noto}
  with \struct{S,0,+,=,\noto} a constructive monoid, $-$ a unary
  operation on $S$ such that 
\begin{enumerate} 
\item $-x$ is the inverse of $x$: $\forall x[x+(-x)=0]$.
\end{enumerate}
In practice we write ``$x-y$'' for ``$x+(-y)$''.
\end{definition}


\begin{lemma}[Inverses are unique]\label{lemuninv}
The inverse of $+$ is unique, i.e.\ for all $x,y$,
$$x + y = 0  \implies y = -x.$$
As a consequence we find immediately that for all $x,y$,
\begin{eqnarray*}
  -(-x) &=& x,\\
  -(x+y) &=& (-y)+(-x).
\end{eqnarray*}
\end{lemma}

\begin{lemma}[Cancellation]\label{lemcanc} For all $x, y, z$,
\[
x+y=x+z \implies x=y.
\]
\end{lemma}

\subsubsection{Apartness in Groups}

\begin{lemma}\label{lemGrpHeyt}
For all $x,y$,
\[ x+y\noto 0 \implies x\noto 0 \vee y\noto 0. \]
\end{lemma}
\begin{proof}
By strong extensionality of $+$,  
$x+y\noto 0+0$ implies $x\noto 0\vee y\noto 0$.
\qed
\end{proof}

\begin{lemma}\label{lemgrstrext}
The operations of a group respect $\noto$, i.e.\ for all $x,y,z$,
\begin{eqnarray*}
x\noto y & \eqqe & x+z \noto y+z,\\
x\noto y & \eqqe & x-y \noto 0,\\
x\noto 0 & \eqqe & -x \noto 0.
\end{eqnarray*}
\end{lemma}
\begin{proof}
For direction $\implies$, $(x+z)-z = x\noto y = (y+z)-z$
(using that $\noto$ respects $=$, \ref{lemaprespeq}), so by strong
extensionality of $+$, $x+z\noto y+z$. The converse uses the forward
direction with $-z$. The second part follows from the first part
noting that $0=y-y$. \qed
\end{proof}

\begin{remark}As has already been pointed out, we {\em always} require 
functions to respect the equality and to be strongly extensional.
In general, you'd want a function to respect the inequality or the
apartness only if it has an inverse. See Lemma \ref{lemstrextinv}.
\end{remark}
