% $Id: CPolynomials.tex,v 1.1 2004/02/11 09:39:57 lcf Exp $

\section{Polynomials}

\subsection{Definition and general properties}
\begin{definition}
For $R$ a ring, we define the set of {\em polynomials over $R$},
$R[X]$, as the finite lists of elements of $R$. We define the operations $+$,
$*$ and $-$  on $R[X]$. Let $f= \langle f_0, \ldots,
f_n\rangle $ and $g= \langle g_0, \ldots, g_m\rangle $ be two
polynomials 
%with $m\leq n$. 
Then $f+g$ and $f-g$ are polynomials of length $\maxx\{ m+1,n+1\}$ and
$f*g$ is a polynomial of length $m+n+1$ defined as follows.
\begin{eqnarray*}
  (f+g)_i &:=& f_i + g_i \mbox{ for } i \leq \maxx\{ m,n\},\\
  (f-g)_i &:=& f_i - g_i \mbox{ for } i \leq \maxx\{ m,n\},\\
  (f*g)_i &:=& \Sigma_{j=0}^i f_j * g(i-j) \mbox{ for } i \leq m+n,
\end{eqnarray*}
where it is understood that we take $f_j$ (resp.\ $g_j$) to be $0$ if
$j>n$ (resp.\ $j>m$). The zero and unit are defined by
\begin{eqnarray*}
  0 &:=& \langle\; \rangle, (\mbox{the empty sequence}),\\
  1&:=&\langle 1\rangle.
\end{eqnarray*}
The apartness relation on $R[X]$ is defined by
$$ f \noto g \; := \; \exists i(f_i \noto g_i).$$
\end{definition}

Note that we use the terminology {\em length of a polynomial\/} when
talking about the length of the list of coefficients. The length of a
polynomial may not be the same as its {\em degree} (defined
precisely in \ref{defdegpol})may be $0$.

It is easy to see that
$$ f =g \; \eqqe \; \forall i(f_i = g_i).$$

\begin{definition}\label{defdegpol}
Let $f(X)=f_n X^n+f_{n-1}X^{n-1}+\ldots + f_1 X+f_0$ be a polynomial.
\begin{enumerate}
\item $f(X)$ {\em has degree $k$}, notation $\deg(f) = k$, if
$k\leq n$, $f_k \noto 0$ and $i=0$ for all $i$ with $k< i \leq n$,\\
\item $f(X)$ {\em has degree at most $k$}, notation $\deg(f) \leq k$,
if $k\leq n $, and $i=0$ for all $i$ with $k< i \leq n$,\\
\item $f(X)$ {\em has degree at least $k$}, notation $\deg(f) \geq k$,
if $k\leq n $, $f_k \noto 0$.
\end{enumerate}
So, not all polynomials have a degree: `degree' is not a function on
polynomials but a relation between polynomials and natural
numbers. However, it is always the case that 
the degree of $f(X)=f_n X^n+f_{n-1}X^{n-1}+\ldots + f_1 X+f_0$ is {\em
at most $n$}, and if we know that $f_k \noto 0$, it is {\em at least
$k$}.
\end{definition}

\begin{definition}\label{defregpol}
A polynomial $f(X)=f_n X^n+f_{n-1}X^{n-1}+\ldots + f_1 X+f_0$ is called
{\em regular} if for its leading coefficient one has $f_n\noto
0$. (That is: the polynomial has a degree, which is the same as its
length, $n$.) 
\end{definition}

\begin{lemma}\label{lempolring} For $R$ a ring, $R[X]$ is a ring.
\end{lemma}

In the following, unless stated otherwise, $R$ is a ring.

\begin{notation}
A polynomial $f= \langle f_0, \ldots, f_n\rangle$ will often be denoted by
$ f(X) = f_n X^n + f_{n-1} X^{n-1} + \ldots + f_0$ or by $f(X)=
\Sigma_{j=0}^n f_j X^j$.\\
The multiplication operation $*$ will usually be omitted.
\end{notation}

\begin{definition}\label{defpolyfun} For every polynomial $f(X)= f_n X^n
  + f_{n-1} 
  X^{n-1} + \ldots + f_0$  over $R$ we define a
  function $\overline{f}:R\rightarrow R$ in the canonical way:
$$f(a):=f_n a^n + f_{n-1}   a^{n-1} + \ldots + f_0.$$
In the following, we will often just write $f$ for this function
$\overline{f}$. 
\end{definition}

The following two Lemmas already hold for integral domains (rings with the
additional property $x\noto 0 \wedge y\noto 0 \implies xy \noto 0$,
see Lemma \ref{lemidfield}), but we have not introduced that notion here.

\begin{lemma}\label{lempolid}
  Let $F$ be a field and let $f = f_n X^n + \ldots + f_0$
  and $g= g_m X^m  + \ldots + g_0$ be polynomials over $F$. Write
  $h_{m+n} X^{m+n}  + \ldots + h_0$ for $fg$. Then
$$f_i g_j \noto 0 \implies \exists k[ i+j \leq k \leq n+m \wedge h_k
\noto 0].$$
\end{lemma}

\begin{proof}
  See \cite{TvD882}, p.\ 417.
\end{proof}

\begin{theorem}\label{thmpolid}
  If $F$ is a field, then $F[X]$ satisfies the integral domain
  property, i.e.\ for all $f, g \in F[X]$, if $f\noto 0$ and $g\noto
  0$, then $fg\noto 0$.
\end{theorem}

\begin{proof}
  Suppose $f= f_n X^n + \ldots + f_0 \noto 0$ and $g= g_m X^m  +
  \ldots + g_0\noto 0$ and let $h_{m+n} X^{m+n}  + \ldots + h_0$ be
  $fg$. Then $f_i g_j \noto 0$ for some 
  $i,j$, but then $h_k\noto 0$ for some $k$. \qed
\end{proof}

\subsection{Factorization and zeros}
Let $R$ be a constructive ring.

\begin{lemma}\label{lemremainder}
  Let $f = f_n X^n + \ldots + f_0$
  and $g= g_m X^m  + \ldots + g_0$ be polynomials over $R$. Then there
  exist $k\in \NN, q,r \in R[X]$ such that
$$(g_m)^k f(X) = q(X) g(X) + r(X)$$
  and $r(X)$ has length less then $m$ or $0$.
\end{lemma}

\begin{proof}
  See \cite{TvD882}, p.\ 418.
\end{proof}

\begin{theorem}\label{thmremainder}
  Let $f(X)\in R[X]$ and $a\in R$. Then
$$\exists ! q(X)\in R[X] ( f(X) = (X-a) q(X) + f(a)).$$
\end{theorem}

\begin{proof}
By Lemma \ref{lemremainder}, $f(X) =q(X) (X-a) + c$, for some
polynomial $q(X)$ and $c\in R$. By taking the value of the function
$f$ in $a$, we find that $c= f(a)$. Furthermore, $q(X) = q_{n-1}
X^{n-1} + \ldots + q_0$ and we can determine the coefficients of
$q(X)$ uniquely from the equation $f(X) =q(X) (X-a) + f(a)$. \qed
\end{proof}

\begin{corollary}\label{corremainder}
  For $f(X)\in R[X]$ and $a\in R$, 
$$ (X-a) | f(X) \eqqe f(a) = 0.$$
Moreover, if $f(X)$ has length $n$ and $f$ has $n+1$ zeros, then $f= 0$.
\end{corollary}

We now prove that if the polynomial $f$ has degree at least $k$
($n\geq k>0$) and we are given
$n+1$ distinct elements $(a_i)_{1\leq i \leq n+1}$, then $f(a_i)\noto
0$ for one of the $i$. This will be used to prove the Intermediate
Value Theorem for polynomials.

\begin{lemma}\label{lempolagrnpts} Let $f(X), g(X)\in R[X]$, both of
length $n$. Let $(a_i)_{0\leq i \leq n-1}$ be distinct elements of $R$ (I.e.\
$a_i \noto a_j$ if $i\neq j$.). If $f(a_i)=g(a_i)$ for all $i$ ($0\leq
i \leq n-1$), then $f=g$.
\end{lemma}

\begin{proof}
The polynomial $h:= f-g$ has length $n$ and has $n$ zeros,
so $h=0$ by Corollary \ref{corremainder}. Hence, $f=g$.\qed
\end{proof}

Let $F$ be a constructive field.

\begin{lemma}\label{lempolnptsform}
Let $f(X) \in R[X]$ of length $n$ and let $(a_i)_{1\leq i \leq
n}$ be distinct elements of $R$. Then
\begin{eqnarray*}
f(X) &=& f(a_1) \frac{(X-a_2)(X-a_3) \cdots (X-a_{n})}{(a_1-a_2)(a_1
-a_3) \cdots (a_1-a_{n})} +\\
&& f(a_2) \frac{(X-a_1)(X-a_3) \cdots (X-a_{n})}{(a_2-a_1)(a_2
-a_3) \cdots (a_2-a_{n})} +\\
&&\cdots\\
&& f(a_{n}) \frac{(X-a_1)(X-a_2) \cdots (X-a_{n-1})}{(a_{n}-a_1)(a_{n}
-a_2) \cdots (a_{n}-a_{n-1})}.\\
\end{eqnarray*}
\end{lemma}

\begin{proof}
The right hand side of the equation is a polynomial $h(X)$ of length
$n$ (note that all the fractions are defined, because all $a_i$ are
distinct). Furthermore $f$ and $h$ agree on all $a_i$, hence $f=h$ by
Lemma \ref{lempolagrnpts}. \qed
\end{proof}

\begin{lemma}\label{lempolnpts}
Let $f(X) \in R[X]$ of degree at least $k$ ($n\geq k > 0$) and let
$(a_i)_{1\leq i \leq 
n+1}$ be distinct elements of $R$. Then 
$$f(a_i) \noto 0$$
for some $i$.
\end{lemma}

\begin{proof}
Write $f(X) = f_n X^n + \ldots + f_0$.  By Lemma \ref{lempolnptsform}
we find that for the coefficient $f_k$ we
have
%$$f_k = \Sigma_{1\leq i \leq n+1} f(a_i)\frac{1}{\Pi_{1\leq j \leq
%n+1, i\neq j} (a_i - a_j)}.$$
$$f_k = \Sigma_{1\leq i \leq n+1} f(a_i) h_k^i,$$
where $h_k^i$ is the $k$-th coefficient of the $i$-th polynomial as above:
$$\frac{(X-a_1) \cdots
  (X-a_{i-1})(X-a_{i+1})\cdots(X-a_{n})}{(a_{i}-a_1)\cdots (a_{i}
  -a_{i-1})(a_{i} -a_{i+1}) \cdots (a_{i}-a_{n+1})}$$
As $f_k \noto 0$, we find that $f(a_i) \noto 0$ for at least one
$i$. \qed
\end{proof}


\subsection{Operations on polynomials}

We need some formal operations on polynomials. Let $F$ be an ordered
field (to make sure that always $n!\noto 0$; as ordered
field are infinite this is the case).


\begin{definition}
  For $f(X) = a_n X^n+a_{n-1}X^{n-1}+\ldots +a_1 X +a_0$ a polynomial,
  we define the {\em derivative of $f$}, $f\ac$ as follows.
$$f\ac (X) := na_n X^{n-1}+(n-1)a_{n-1}X^{n-2}+\ldots +a_1.$$
Taking $k$-times the deriavtive of $f$ is denoted as $f^{(k)}$.
\end{definition}

\begin{definition}\label{defpolyop} Let
  $f(x)=a_nx^n+a_{n-1}x^{n-1}+\ldots +a_0$ be a polynomial over
  $R$ and $c\in R$. Define the polynomials $f{{\verb+~+}}$ and $f_c$ as
  follows.
  \begin{eqnarray*}
    f{{\verb+~+}}(X)&=&a_0 X^n+\ldots +a_{n-1}X+a_n,\\
    f_c(X)&=&\frac{f^{(n)}(c)}{n!}X^n+\frac{f^{(n-1)}(c)}{(n-1)!}X^{n-1}+ 
    \ldots +\frac{f\ac(c)}{1!}X + f(c)  
  \end{eqnarray*}
\end{definition}

\begin{lemma}\label{lempolyop} Let $f(X)=a_n X^n+a_{n-1}X^{n-1}+\ldots
    +a_0$ be a polynomial over $R$ and $c\in R$. For the function
    $\overline{f}$ associated to this polynomial (\ref{defpolyfun}) we have
\begin{eqnarray*}
\overline{f{{\verb+~+}}}(x)= x^n \overline{f}(x^{-1}),\mbox{ if }x\noto
0,\\
\overline{f{{\verb+~+}}}(0)= a_n, \\
\overline{f_c}(x)=\overline{f}(x+c).
\end{eqnarray*}
\end{lemma}

\begin{proof}
We informally write $f$ where we refer to the function $\overline{f}$ etcetera.
\beqn
x^nf(\frac{1}{x})&=&x^n(a_nx^{-n}+\ldots+a_0)\\
&=&a_n+a_{n-1}+\ldots+a_0x^n\\
&=&f{{\verb+~+}}(x).
\eeqn
Clearly $f_c(x)$ is a polynomial of maximal degree $n$. Hence,
$$f_c(x)=b_nx^n+\ldots+b_0.$$
It follows that
\beqn
f(c)&=& f_c(0)\;=\;b_0,\hfill \mbox{ hence }b_0\;=\;f(c),\\
f\ac(c)&=&f\ac_c(0)\;=\;b_1,\mbox{ hence }b_1\;=\;f\ac(c)\\
f^{(2)}(c)&=&f^{(2)}_c(0)\;=\;2!b_2,\mbox{ hence }
b_2\;=\;\frac{f^{(2)}(c)}{2!}\\
\ldots&&\\
f^{(n)}(c)&=&f^{(n)}_c(0)\;=\;n!b_n,\mbox{ hence }
b_n\;=\;\frac{f^{(n)}(c)}{n!}. \qed
\eeqn
\end{proof}


\weg{
\begin{lemma}\label{1.6} Let $f(X) = a_n X^n+a_{n-1}X^{n-1}+\ldots +a_1
  X +a_0,$ be  a polynomial over an infinite field $F$ in which $i\neq
  0, a\noto 0 \implies ia\noto 0$.
  Suppose that $a_k\unneq 0,$ for some $0<k\leq n.$ Then for some
  $c\in F$ one has 
$$f(c)\unneq 0,\;f'(c)\unneq 0,\ldots,f^{(k)}(c)\unneq 0.$$
\end{lemma}

? Is $i\neq 0, a\noto 0 \implies ia\noto 0$ needed?
\begin{proof}

We show, for $0\leq l\leq k,$ the existence of an infinite subset
$A_l\subset F$ such that
$$a\in A_l\Rightarrow f(a)\unneq 0,\;f'(a)\unneq
0,\ldots,f^{(l)}(a)\unneq 0,$$
by induction on $l.$\\[1ex]
{\em Case }$l=0.$ Choose $u_0,\ldots,u_n\in F,\;\; u_i\noto u_j$ for
$i\neq j.$ Consider the system of $n+1$ equations
$$ (a_0\; a_1\;\ldots a_n)\;U =(f(u_0)\; f(u_1)\;\ldots f(u_n))$$
where
$$ U =
\left(
\begin{array}{llll}
1 &1 &\ldots &1 \\
u_0 &u_1 &\ldots &u_n \\
&&& \\
&&& \\
u_0^n &u_1^n &\ldots &u_n^n
\end{array}
\right).$$
We have
$$det(U) = \prod\limits_{0\leq j < i\leq n}(u_i - u_j)\noto
0\quad\quad\mbox{(Vandermonde \ref{lemVand})}.$$
Hence (by \ref{lemLinInvDet}), $U$ has an inverse $C$ such that 
$$ (a_0\; a_1\;\ldots a_n) = (f(u_0)\; f(u_1)\;\ldots f(u_n))\;C.$$
Especially
$$a_k = f(u_0)c_{0k}+\ldots +f(u_n)c_{nk}.$$

From $a_k\unneq 0$ follows $f(u_i)\unneq 0$ for some $i.$ This shows
the existence of the set $A_0.$\\[1ex]
{\em Case }$l>0.$ By the induction hypothesis there exists an infinite
subset $A_{l-1}\subset F$ such that
$$a\in A_{l-1}\Rightarrow f(a)\unneq 0,\;f'(a)\unneq
0,\ldots,f^{(l-1)}(a)\unneq 0.$$
We only need to show the existence of an infinite subset $A_l\subset
A_{l-1}$ such that
$$a\in A_l\Rightarrow f^{(l)}(a)\unneq 0.$$
The coefficient of degree $k-l$ in $f^{(l)}$ is
$k(k-1)\ldots(k-l+1)a_k\unneq 0.$ Hence if we choose
$u_0,\ldots,u_{n-l}\in A_{l-1},\;\; u_i\noto u_j \mbox{ if } i\neq j,$
then it follows similarly to the {\em Case }$l=0$ that
$f^{(l)}(u_i)\unneq 0$ for some $i,\;\; 0\leq i \leq n-l.$ This shows
the existence of $A_l.\hfill\qed$
\end{proof}
}
