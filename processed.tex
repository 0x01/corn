\documentclass[a4paper,10pt, runningheads]{llncs}
\usepackage{url}
\usepackage{bbm}
\usepackage{ucs}
% \usepackage[utf8x,utf8]{inputenc}
\newcommand{\N}{\ensuremath{\mathbbm{N}}}
\newcommand{\Z}{\ensuremath{\mathbbm{Z}}}
\newcommand{\Q}{\ensuremath{\mathbbm{Q}}}
%% ODER: format ==         = "\mathrel{==}"
%% ODER: format /=         = "\neq "
%
%
\makeatletter
\@ifundefined{lhs2tex.lhs2tex.sty.read}%
  {\@namedef{lhs2tex.lhs2tex.sty.read}{}%
   \newcommand\SkipToFmtEnd{}%
   \newcommand\EndFmtInput{}%
   \long\def\SkipToFmtEnd#1\EndFmtInput{}%
  }\SkipToFmtEnd

\newcommand\ReadOnlyOnce[1]{\@ifundefined{#1}{\@namedef{#1}{}}\SkipToFmtEnd}
\usepackage{amstext}
\usepackage{amssymb}
\usepackage{stmaryrd}
\DeclareFontFamily{OT1}{cmtex}{}
\DeclareFontShape{OT1}{cmtex}{m}{n}
  {<5><6><7><8>cmtex8
   <9>cmtex9
   <10><10.95><12><14.4><17.28><20.74><24.88>cmtex10}{}
\DeclareFontShape{OT1}{cmtex}{m}{it}
  {<-> ssub * cmtt/m/it}{}
\newcommand{\texfamily}{\fontfamily{cmtex}\selectfont}
\DeclareFontShape{OT1}{cmtt}{bx}{n}
  {<5><6><7><8>cmtt8
   <9>cmbtt9
   <10><10.95><12><14.4><17.28><20.74><24.88>cmbtt10}{}
\DeclareFontShape{OT1}{cmtex}{bx}{n}
  {<-> ssub * cmtt/bx/n}{}
\newcommand{\tex}[1]{\text{\texfamily#1}}	% NEU

\newcommand{\Sp}{\hskip.33334em\relax}


\newcommand{\Conid}[1]{\mathit{#1}}
\newcommand{\Varid}[1]{\mathit{#1}}
\newcommand{\anonymous}{\kern0.06em \vbox{\hrule\@width.5em}}
\newcommand{\plus}{\mathbin{+\!\!\!+}}
\newcommand{\bind}{\mathbin{>\!\!\!>\mkern-6.7mu=}}
\newcommand{\rbind}{\mathbin{=\mkern-6.7mu<\!\!\!<}}% suggested by Neil Mitchell
\newcommand{\sequ}{\mathbin{>\!\!\!>}}
\renewcommand{\leq}{\leqslant}
\renewcommand{\geq}{\geqslant}
\usepackage{polytable}

%mathindent has to be defined
\@ifundefined{mathindent}%
  {\newdimen\mathindent\mathindent\leftmargini}%
  {}%

\def\resethooks{%
  \global\let\SaveRestoreHook\empty
  \global\let\ColumnHook\empty}
\newcommand*{\savecolumns}[1][default]%
  {\g@addto@macro\SaveRestoreHook{\savecolumns[#1]}}
\newcommand*{\restorecolumns}[1][default]%
  {\g@addto@macro\SaveRestoreHook{\restorecolumns[#1]}}
\newcommand*{\aligncolumn}[2]%
  {\g@addto@macro\ColumnHook{\column{#1}{#2}}}

\resethooks

\newcommand{\onelinecommentchars}{\quad-{}- }
\newcommand{\commentbeginchars}{\enskip\{-}
\newcommand{\commentendchars}{-\}\enskip}

\newcommand{\visiblecomments}{%
  \let\onelinecomment=\onelinecommentchars
  \let\commentbegin=\commentbeginchars
  \let\commentend=\commentendchars}

\newcommand{\invisiblecomments}{%
  \let\onelinecomment=\empty
  \let\commentbegin=\empty
  \let\commentend=\empty}

\visiblecomments

\newlength{\blanklineskip}
\setlength{\blanklineskip}{0.66084ex}

\newcommand{\hsindent}[1]{\quad}% default is fixed indentation
\let\hspre\empty
\let\hspost\empty
\newcommand{\NB}{\textbf{NB}}
\newcommand{\Todo}[1]{$\langle$\textbf{To do:}~#1$\rangle$}

\EndFmtInput
\makeatother
%
%
%
%
%
%
% This package provides two environments suitable to take the place
% of hscode, called "plainhscode" and "arrayhscode". 
%
% The plain environment surrounds each code block by vertical space,
% and it uses \abovedisplayskip and \belowdisplayskip to get spacing
% similar to formulas. Note that if these dimensions are changed,
% the spacing around displayed math formulas changes as well.
% All code is indented using \leftskip.
%
% Changed 19.08.2004 to reflect changes in colorcode. Should work with
% CodeGroup.sty.
%
\ReadOnlyOnce{polycode.fmt}%
\makeatletter

\newcommand{\hsnewpar}[1]%
  {{\parskip=0pt\parindent=0pt\par\vskip #1\noindent}}

% can be used, for instance, to redefine the code size, by setting the
% command to \small or something alike
\newcommand{\hscodestyle}{}

% The command \sethscode can be used to switch the code formatting
% behaviour by mapping the hscode environment in the subst directive
% to a new LaTeX environment.

\newcommand{\sethscode}[1]%
  {\expandafter\let\expandafter\hscode\csname #1\endcsname
   \expandafter\let\expandafter\endhscode\csname end#1\endcsname}

% "compatibility" mode restores the non-polycode.fmt layout.

\newenvironment{compathscode}%
  {\par\noindent
   \advance\leftskip\mathindent
   \hscodestyle
   \let\\=\@normalcr
   \(\pboxed}%
  {\endpboxed\)%
   \par\noindent
   \ignorespacesafterend}

\newcommand{\compaths}{\sethscode{compathscode}}

% "plain" mode is the proposed default.
% It should now work with \centering.
% This required some changes. The old version
% is still available for reference as oldplainhscode.

\newenvironment{plainhscode}%
  {\hsnewpar\abovedisplayskip
   \advance\leftskip\mathindent
   \hscodestyle
   \let\hspre\(\let\hspost\)%
   \pboxed}%
  {\endpboxed%
   \hsnewpar\belowdisplayskip
   \ignorespacesafterend}

\newenvironment{oldplainhscode}%
  {\hsnewpar\abovedisplayskip
   \advance\leftskip\mathindent
   \hscodestyle
   \let\\=\@normalcr
   \(\pboxed}%
  {\endpboxed\)%
   \hsnewpar\belowdisplayskip
   \ignorespacesafterend}

% Here, we make plainhscode the default environment.

\newcommand{\plainhs}{\sethscode{plainhscode}}
\newcommand{\oldplainhs}{\sethscode{oldplainhscode}}
\plainhs

% The arrayhscode is like plain, but makes use of polytable's
% parray environment which disallows page breaks in code blocks.

\newenvironment{arrayhscode}%
  {\hsnewpar\abovedisplayskip
   \advance\leftskip\mathindent
   \hscodestyle
   \let\\=\@normalcr
   \(\parray}%
  {\endparray\)%
   \hsnewpar\belowdisplayskip
   \ignorespacesafterend}

\newcommand{\arrayhs}{\sethscode{arrayhscode}}

% The mathhscode environment also makes use of polytable's parray 
% environment. It is supposed to be used only inside math mode 
% (I used it to typeset the type rules in my thesis).

\newenvironment{mathhscode}%
  {\parray}{\endparray}

\newcommand{\mathhs}{\sethscode{mathhscode}}

% texths is similar to mathhs, but works in text mode.

\newenvironment{texthscode}%
  {\(\parray}{\endparray\)}

\newcommand{\texths}{\sethscode{texthscode}}

% The framed environment places code in a framed box.

\def\codeframewidth{\arrayrulewidth}
\RequirePackage{calc}

\newenvironment{framedhscode}%
  {\parskip=\abovedisplayskip\par\noindent
   \hscodestyle
   \arrayrulewidth=\codeframewidth
   \tabular{@{}|p{\linewidth-2\arraycolsep-2\arrayrulewidth-2pt}|@{}}%
   \hline\framedhslinecorrect\\{-1.5ex}%
   \let\endoflinesave=\\
   \let\\=\@normalcr
   \(\pboxed}%
  {\endpboxed\)%
   \framedhslinecorrect\endoflinesave{.5ex}\hline
   \endtabular
   \parskip=\belowdisplayskip\par\noindent
   \ignorespacesafterend}

\newcommand{\framedhslinecorrect}[2]%
  {#1[#2]}

\newcommand{\framedhs}{\sethscode{framedhscode}}

% The inlinehscode environment is an experimental environment
% that can be used to typeset displayed code inline.

\newenvironment{inlinehscode}%
  {\(\def\column##1##2{}%
   \let\>\undefined\let\<\undefined\let\\\undefined
   \newcommand\>[1][]{}\newcommand\<[1][]{}\newcommand\\[1][]{}%
   \def\fromto##1##2##3{##3}%
   \def\nextline{}}{\) }%

\newcommand{\inlinehs}{\sethscode{inlinehscode}}

% The joincode environment is a separate environment that
% can be used to surround and thereby connect multiple code
% blocks.

\newenvironment{joincode}%
  {\let\orighscode=\hscode
   \let\origendhscode=\endhscode
   \def\endhscode{\def\hscode{\endgroup\def\@currenvir{hscode}\\}\begingroup}
   %\let\SaveRestoreHook=\empty
   %\let\ColumnHook=\empty
   %\let\resethooks=\empty
   \orighscode\def\hscode{\endgroup\def\@currenvir{hscode}}}%
  {\origendhscode
   \global\let\hscode=\orighscode
   \global\let\endhscode=\origendhscode}%

\makeatother
\EndFmtInput
%





  % Without these, lhs2tex uses three-line equal signs.



\arrayhs%i.e. no page breaks in code blocks
\divide\abovedisplayskip 4
\divide\belowdisplayskip 4
\begin{document}
\title{The algebraic hierarchy in type theory using type classes}
\author{Bas Spitters \and Eelis van der Weegen}
\institute{Radboud University Nijmegen}
\maketitle
\begin{abstract}
We propose generic design patterns to organize algebraic structures using type classes inside the
Coq system. We support multiple inheritance, sharing of notations and theories, and automated
structure inference. A library for many-sorted universal algebra abstracts from the specific
algebraic structures and aids the organization of the library. Category theory provides a further
useful abstraction. We provide abstract
interfaces for $\N$, $\Z$, $\Q$ as the initial semiring, the initial ring and a field of
fractions. Similarly, we provide an interface for polynomials over a ring $R$ using $R$-algebras.
We show that the standard implementations of these are in fact instances of these type classes.
We report on the efficiency penality of these abstractions.

Finally, we use type classes to conveniently implement a quote function using the prolog-like
abilities of type class unification. We integrate this with our development of universal algebra.
\end{abstract}

\section{Introduction}
Developing formalized libraries for mathematics requires a detailed analysis of the
architecture of mathematics~\cite{C-corn,DBLP:conf/types/HaftmannW08}, similar to that of a large
software project.
The development of an algebraic hierarchy is an important aspect. Challenges include: 1. multiple
inheritance: A ring is a type with two operations. How do we inherit from the two theories of
monoids? 2. Sharing of notations. 3. Combining syntactic and semantic methods, usually this is
addressed by reflection.

Several solutions have been proposed for the Coq theorem prover~\cite{Coq,BC04}: dependent
records~\cite{DBLP:journals/jsc/GeuversPWZ02} (a.k.a.\ telescopes), packed classes~\cite{Packed} and
occasionally modules. Although the latter option is quickly discarded%
\footnote{Modules are not first class, but currently they seem to be superior for extraction to
OCaml~\cite{letouzey02}. Modules may also be seen as parametrised sections. They allow management
of namespaces, notations, hints, etc. An \ensuremath{\Conid{Import}} command imports all the constants from a
module. Thus allowing the user to type \ensuremath{\Conid{B}} instead of \ensuremath{\Conid{A.B}}. Similar functionality would be useful
for type classes.}
algebraic theories as `interfaces', as opposed to `modules', has been an important
guiding principle~\cite{Packed}.

In this fashion we implement\footnote{The sources are available
at:~\url{http://www.eelis.net/research/math-classes/}}
 algebraic theories as type classes, not instances, thus addressing the
challenges above; see Section~\ref{classes}. Type classes have recently entered
Coq~\cite{DBLP:conf/tphol/SozeauO08}. In Coq,
type classes are \emph{first class} citizens based on dependent records. The presence of dependent types makes them even more
powerful than the already succesful second class implementations in Haskell and Isabelle. Moreover,
Coq's type classes are well integrated with setoid-rewriting~\cite{Setoid-rewrite}.

We abstract from concrete algebra and formalize many-sorted universal algebra, including a
construction of
free $\Sigma$-algebras; see Section~\ref{univ}. We are also concerned with efficiency.
In fact, our aim is to use the present development as a new basis for Bishop's program in type
theory~\cite{typesreal-article}.\marginpar{More}

So, we would like to replace data structures with more efficient ones. Concretely,
Coq provides unary, binary and machine numbers. We use type classes polymorphism to
develop a unified library for them; see Section~\ref{modul}.
The interfaces for $\N$, $\Z,\Q$ are the initial semiring, the initial ring and a field of
fractions; see Section~\ref{interfaces}. In future work we aim to base our development of the reals
on an abstract dense set, allowing us to use the efficient dyadic
rationals~\cite{boldo2009combining} as a base for exact real number computation in
Coq~\cite{Riemann,Oconnor:real}.

The paper is organised as follows: Section~\ref{classes} starts with a development of the concrete
algebraic hierarchy addressing the issues raised above. Section~\ref{univ} abstracts to universal
algebra, we develop the basic theory to the point of the first homomorphism theorem.
Section~\ref{interfaces} provides interfaces to usual datastructures by providing their universal
properties. To allow symbolic manipulation of semantic objects one needs a quote function. Usually,
this function is implemented in Coq's tactics language. A more convenient way is to use type class
unification; see section~\ref{quote}. In section~\ref{canonical}, we end with a short comparision
between type classes and canonical structures.

In this paper we focus on the Coq proof assistant. We conjecture that the methods can be transfered
to any type theory based proof assistant supporting type classes such as
Matita~\cite{asperti2007user}.

\section{Preliminaries}
\subsection{Coq type theory}
The Coq proof assistant is based on the calculus of inductive
constructions~\cite{CoquandHuet,CoquandPaulin}.\marginpar{More}
This type theory lacks quotient types, as it would make type checking undecidable. Instead it
supports setoids (aka Bishop sets), a pair of a type and an equivalence
relation~\cite{Bishop67,Hofmann,Capretta}. Recently, improved support has been added to the
system~\cite{Setoid-rewrite}. Functions respecting these equivalence relations are called \ensuremath{\Conid{Proper}}.
As we will see in Section~\ref{univ2} below this functionality can also be used to work with
quotient algebras.

\subsection{Type classes}
Implementation in Ltac, Notation etc
Coercions


\section{To bunde or not to bundle? (Or: What to bundle?)}

How should we define reflexive, transitive, and symmetric relations? In a first attempt, we
might define:
\begin{hscode}\SaveRestoreHook
\column{B}{@{}>{\hspre}l<{\hspost}@{}}%
\column{3}{@{}>{\hspre}l<{\hspost}@{}}%
\column{5}{@{}>{\hspre}l<{\hspost}@{}}%
\column{20}{@{}>{\hspre}l<{\hspost}@{}}%
\column{E}{@{}>{\hspre}l<{\hspost}@{}}%
\>[3]{}\mathbf{Record}\;\Conid{ReflexiveRelation:}\;\Conid{Type}\;\Conid{:=}\;{}\<[E]%
\\
\>[3]{}\hsindent{2}{}\<[5]%
\>[5]{}\hspace*{0mm}\{\Varid{rr\char95 carrier:}\;\Conid{Type}{}\<[E]%
\\
\>[3]{}\hsindent{2}{}\<[5]%
\>[5]{};\Varid{rr\char95 rel:}\;\Varid{rr\char95 carrier}\;\rightarrow\;\Varid{rr\char95 carrier}\;\rightarrow\;\Conid{Prop}{}\<[E]%
\\
\>[3]{}\hsindent{2}{}\<[5]%
\>[5]{};\Varid{rr\char95 proof:}\;\Pi\;{}\<[20]%
\>[20]{}\Varid{x},\Varid{rr\char95 rel}\;\Varid{x}\;\Varid{x}\mskip1.5mu\}\;.{}\<[E]%
\ColumnHook
\end{hscode}\resethooks
and similar for transitive and symmetric relations.

With this approach, we run into problems when we try to define \ensuremath{\Conid{EquivalenceRelation}} in terms of
\ensuremath{\Conid{ReflexiveRelation}} and its two siblings, because when posit three objects of type
\ensuremath{\Conid{ReflexiveRelation}}, \ensuremath{\Conid{TransitiveRelation}}, and \ensuremath{\Conid{SymmetricRelation}}, respectively, we have to
add equalities expressing that they talk about the same carrier and relation, and these equalities
are a nightmare to work with (because the three carriers and relations won't be convertible).
Technically, this leads to the research in manifest fields, the \emph{with} construction, to which
we return in Section~\ref{manifest}.

As a first step we would define \ensuremath{\Conid{ReflexiveRelation}} as:
\begin{hscode}\SaveRestoreHook
\column{B}{@{}>{\hspre}l<{\hspost}@{}}%
\column{3}{@{}>{\hspre}l<{\hspost}@{}}%
\column{5}{@{}>{\hspre}l<{\hspost}@{}}%
\column{20}{@{}>{\hspre}l<{\hspost}@{}}%
\column{E}{@{}>{\hspre}l<{\hspost}@{}}%
\>[3]{}\mathbf{Record}\;\Conid{ReflexiveRelation}\;(\Varid{carrier:}\;\Conid{Type})\;\mathbin{:}\;\Conid{Type}\;\Conid{:=}\;{}\<[E]%
\\
\>[3]{}\hsindent{2}{}\<[5]%
\>[5]{}\hspace*{0mm}\{\Varid{rr\char95 rel:}\;\Varid{rr\char95 carrier}\;\rightarrow\;\Varid{rr\char95 carrier}\;\rightarrow\;\Conid{Prop}{}\<[E]%
\\
\>[3]{}\hsindent{2}{}\<[5]%
\>[5]{};\Varid{rr\char95 proof:}\;\Pi\;{}\<[20]%
\>[20]{}\Varid{x},\Varid{rr\char95 rel}\;\Varid{x}\;\Varid{x}\mskip1.5mu\}\;.{}\<[E]%
\ColumnHook
\end{hscode}\resethooks

This would solve the problem for the carrier, but \ensuremath{\Conid{EquivalenceRelation}} would still need equalities
for the relation:
\begin{hscode}\SaveRestoreHook
\column{B}{@{}>{\hspre}l<{\hspost}@{}}%
\column{3}{@{}>{\hspre}l<{\hspost}@{}}%
\column{5}{@{}>{\hspre}l<{\hspost}@{}}%
\column{E}{@{}>{\hspre}l<{\hspost}@{}}%
\>[3]{}\mathbf{Record}\;\Conid{EquivalenceRelation:}\;\Conid{Type}\;\Conid{:=}\;{}\<[E]%
\\
\>[3]{}\hsindent{2}{}\<[5]%
\>[5]{}\hspace*{0mm}\{\Varid{er\char95 carrier:}\;\Conid{Type}{}\<[E]%
\\
\>[3]{}\hsindent{2}{}\<[5]%
\>[5]{};\Varid{er\char95 refl:}\;\Conid{ReflexiveRelation}\;\Varid{er\char95 carrier}{}\<[E]%
\\
\>[3]{}\hsindent{2}{}\<[5]%
\>[5]{};\Varid{er\char95 sym:}\;\Conid{SymmetricRelation}\;\Varid{er\char95 carrier}{}\<[E]%
\\
\>[3]{}\hsindent{2}{}\<[5]%
\>[5]{};\Varid{er\char95 trans:}\;\Conid{TransitiveRelation}\;\Varid{er\char95 trans}{}\<[E]%
\\
\>[3]{}\hsindent{2}{}\<[5]%
\>[5]{};\Varid{same\char95 rel:}\;\Varid{rr\char95 rel}\;\Varid{er\char95 refl}\;\mathrel{=}\;\Varid{sr\char95 rel}\;\Varid{er\char95 sym}{}\<[E]%
\\
\>[3]{}\hsindent{2}{}\<[5]%
\>[5]{};\Varid{same\char95 rel':}\;\Varid{rr\char95 rel}\;\Varid{er\char95 refl}\;\mathrel{=}\;\Varid{tr\char95 rel}\;\Varid{er\char95 trans}\mskip1.5mu\}\;.{}\<[E]%
\ColumnHook
\end{hscode}\resethooks

A natural solution is to pull the relation out of \ensuremath{\Conid{ReflexiveRelation}} (and its siblings) as well:
\begin{hscode}\SaveRestoreHook
\column{B}{@{}>{\hspre}l<{\hspost}@{}}%
\column{3}{@{}>{\hspre}l<{\hspost}@{}}%
\column{5}{@{}>{\hspre}l<{\hspost}@{}}%
\column{20}{@{}>{\hspre}l<{\hspost}@{}}%
\column{28}{@{}>{\hspre}l<{\hspost}@{}}%
\column{E}{@{}>{\hspre}l<{\hspost}@{}}%
\>[3]{}\mathbf{Record}\;\Conid{ReflexiveRelation}\;(\Varid{carrier:}\;\Conid{Type})\;{}\<[E]%
\\
\>[3]{}\hsindent{25}{}\<[28]%
\>[28]{}(\Varid{rel:}\;\Varid{carrier}\;\rightarrow\;\Varid{carrier}\;\rightarrow\;\Conid{Prop})\;\mathbin{:}\;\Conid{Prop}\;\Conid{:=}\;{}\<[E]%
\\
\>[3]{}\hsindent{2}{}\<[5]%
\>[5]{}\hspace*{0mm}\{\Varid{rr\char95 proof:}\;\Pi\;{}\<[20]%
\>[20]{}\Varid{x},\Varid{rr\char95 rel}\;\Varid{x}\;\Varid{x}\mskip1.5mu\}\;.{}\<[E]%
\ColumnHook
\end{hscode}\resethooks
or equivalently:
\begin{hscode}\SaveRestoreHook
\column{B}{@{}>{\hspre}l<{\hspost}@{}}%
\column{3}{@{}>{\hspre}l<{\hspost}@{}}%
\column{5}{@{}>{\hspre}l<{\hspost}@{}}%
\column{8}{@{}>{\hspre}l<{\hspost}@{}}%
\column{32}{@{}>{\hspre}l<{\hspost}@{}}%
\column{E}{@{}>{\hspre}l<{\hspost}@{}}%
\>[3]{}\mathbf{Definition}\;\Conid{ReflexiveRelation}\;(\Varid{carrier:}\;\Conid{Type})\;{}\<[E]%
\\
\>[3]{}\hsindent{29}{}\<[32]%
\>[32]{}(\Varid{rel:}\;\Varid{carrier}\;\rightarrow\;\Varid{carrier}\;\rightarrow\;\Conid{Prop})\;\mathbin{:}\;\Conid{Prop}\;\Conid{:=}\;{}\<[E]%
\\
\>[3]{}\hsindent{2}{}\<[5]%
\>[5]{}\Pi\;{}\<[8]%
\>[8]{}\Varid{x},\Varid{rr\char95 rel}\;\Varid{x}\;\Varid{x.}{}\<[E]%
\ColumnHook
\end{hscode}\resethooks
This is how reflexivity is defined in the standard library, and this is entirely sensible. The
change from \ensuremath{\Conid{Type}} to \ensuremath{\Conid{Prop}} reflects the fact that \ensuremath{\Conid{ReflexiveRelation}} is now strictly a property,
proofs of which can be nicely kept opaque;%
\footnote{Coq distinguishes between transparent and opague definitions. The former can be unfolded,
whereas this is impossible for the latter.}
we will return to this issue in Section~\ref{explicit}.
We can now define \ensuremath{\Conid{EquivalenceRelation}} without awkward equalities:
\begin{hscode}\SaveRestoreHook
\column{B}{@{}>{\hspre}l<{\hspost}@{}}%
\column{3}{@{}>{\hspre}l<{\hspost}@{}}%
\column{5}{@{}>{\hspre}l<{\hspost}@{}}%
\column{E}{@{}>{\hspre}l<{\hspost}@{}}%
\>[3]{}\mathbf{Record}\;\Conid{EquivalenceRelation:}\;\Conid{Type}\;\Conid{:=}\;{}\<[E]%
\\
\>[3]{}\hsindent{2}{}\<[5]%
\>[5]{}\hspace*{0mm}\{\Varid{er\char95 carrier:}\;\Conid{Type}{}\<[E]%
\\
\>[3]{}\hsindent{2}{}\<[5]%
\>[5]{};\Varid{er\char95 rel:}\;\Varid{er\char95 carrier}\;\rightarrow\;\Varid{er\char95 carrier}\;\rightarrow\;\Conid{Prop}{}\<[E]%
\\
\>[3]{}\hsindent{2}{}\<[5]%
\>[5]{};\Varid{er\char95 refl:}\;\Conid{ReflexiveRelation}\;\Varid{er\char95 carrier}\;\Varid{er\char95 rel}{}\<[E]%
\\
\>[3]{}\hsindent{2}{}\<[5]%
\>[5]{};\Varid{er\char95 sym:}\;\Conid{SymmetricRelation}\;\Varid{er\char95 carrier}\;\Varid{er\char95 rel}{}\<[E]%
\\
\>[3]{}\hsindent{2}{}\<[5]%
\>[5]{};\Varid{er\char95 trans:}\;\Conid{TransitiveRelation}\;\Varid{er\char95 trans}\;\Varid{er\char95 rel}\mskip1.5mu\}\;.{}\<[E]%
\ColumnHook
\end{hscode}\resethooks

But looking at \ensuremath{\Conid{EquivalenceRelation}}, we see that it now essentially has the same form as the
original naive \ensuremath{\Conid{ReflexiveRelation}} record we started with. Clearly, equivalence relations
and reflexive relations are analogous concepts, and by the same argument we
conclude that \ensuremath{\Conid{EquivalenceRelation}} should really be defined as:

\begin{hscode}\SaveRestoreHook
\column{B}{@{}>{\hspre}l<{\hspost}@{}}%
\column{3}{@{}>{\hspre}l<{\hspost}@{}}%
\column{30}{@{}>{\hspre}l<{\hspost}@{}}%
\column{E}{@{}>{\hspre}l<{\hspost}@{}}%
\>[B]{}\mathbf{Record}\;\Conid{EquivalenceRelation}\;(\Varid{er\char95 carrier:}\;\Conid{Type})\;{}\<[E]%
\\
\>[B]{}\hsindent{30}{}\<[30]%
\>[30]{}(\Varid{er\char95 rel:}\;\Varid{er\char95 carrier}\;\rightarrow\;\Varid{er\char95 carrier}\;\rightarrow\;\Conid{Prop})\;\mathbin{:}\;\Conid{Prop}\;\Conid{:=}\;{}\<[E]%
\\
\>[B]{}\hsindent{3}{}\<[3]%
\>[3]{}\hspace*{0mm}\{\Varid{er\char95 refl:}\;\Conid{ReflexiveRelation}\;\Varid{er\char95 carrier}\;\Varid{er\char95 rel}{}\<[E]%
\\
\>[B]{}\hsindent{3}{}\<[3]%
\>[3]{};\Varid{er\char95 sym:}\;\Conid{SymmetricRelation}\;\Varid{er\char95 carrier}\;\Varid{er\char95 rel}{}\<[E]%
\\
\>[B]{}\hsindent{3}{}\<[3]%
\>[3]{};\Varid{er\char95 trans:}\;\Conid{TransitiveRelation}\;\Varid{er\char95 trans}\;\Varid{er\char95 rel}\mskip1.5mu\}\;.{}\<[E]%
\ColumnHook
\end{hscode}\resethooks

After all, when we define a type with
two different equivalence relations on it,
such as the quotient of a setoid,
we have the same considerations as for \ensuremath{\Conid{ReflexiveRelation}} above.

At the risk of overstressing the point, let us look at how the same problem and solution keep
appearing in composite structures. Suppose we use a definition of \ensuremath{\Conid{Monoid}} that bundles the
relation. Then if we proceed to define \ensuremath{\Conid{SemiRing}} as a double \ensuremath{\Conid{Monoid}}, we face a dilemma:
\begin{hscode}\SaveRestoreHook
\column{B}{@{}>{\hspre}l<{\hspost}@{}}%
\column{3}{@{}>{\hspre}l<{\hspost}@{}}%
\column{5}{@{}>{\hspre}l<{\hspost}@{}}%
\column{20}{@{}>{\hspre}l<{\hspost}@{}}%
\column{E}{@{}>{\hspre}l<{\hspost}@{}}%
\>[3]{}\mathbf{Record}\;\Conid{SemiRing}\;\Conid{A:}\;\Conid{Type}\;\Conid{:=}\;{}\<[E]%
\\
\>[3]{}\hsindent{2}{}\<[5]%
\>[5]{}\hspace*{0mm}\{\Varid{mult\char95 monoid:}\;\Conid{Monoid}\;\Conid{A}{}\<[E]%
\\
\>[3]{}\hsindent{2}{}\<[5]%
\>[5]{};\Varid{plus\char95 monoid:}\;\Conid{Monoid}\;\Conid{A}{}\<[E]%
\\
\>[3]{}\hsindent{2}{}\<[5]%
\>[5]{};\Varid{plus\char95 comm:}\;\Conid{Commutative}\;(\Varid{mon\char95 op}\;\Varid{plus\char95 monoid}){}\<[E]%
\\
\>[3]{}\hsindent{2}{}\<[5]%
\>[5]{};\Varid{mult\char95 comm:}\;\Conid{Commutative}\;(\Varid{mon\char95 op}\;\Varid{mult\char95 monoid}){}\<[E]%
\\
\>[3]{}\hsindent{2}{}\<[5]%
\>[5]{};\Varid{mult\char95 0\char95 l:}\;\Pi\;{}\<[20]%
\>[20]{}\Varid{x:}\;\Conid{A},\Varid{?}{}\<[E]%
\ColumnHook
\end{hscode}\resethooks
Which equality do we use in the expression ?, \ensuremath{\Varid{mult\char95 monoid}}'s equality or \ensuremath{\Varid{plus\char95 monoid}}'s?
Again, for this definition to be usable at all, we would need to add an equality expressing that the
two monoids contain the same equivalence relation. And again, the proper solution is to change
Monoid and unbundle the relation. But here, we have a choice as to how much we unbundle exactly.
(For the moment we ignore the question of whether the unit and binary operator should be unbundled
as well, focusing only on the equivalence relation and its equivalence proof.)

We \emph{could} become over-eager and decide to unbundle the equivalence proof term as well:
\begin{hscode}\SaveRestoreHook
\column{B}{@{}>{\hspre}l<{\hspost}@{}}%
\column{3}{@{}>{\hspre}l<{\hspost}@{}}%
\column{E}{@{}>{\hspre}l<{\hspost}@{}}%
\>[B]{}\mathbf{Record}\;\Conid{Monoid}\;(\Conid{A:}\;\Conid{Type})\;(\Conid{R:}\;\Conid{A}\;\rightarrow\;\Conid{A}\;\rightarrow\;\Conid{Prop})\;{}\<[E]%
\\
\>[B]{}\hsindent{3}{}\<[3]%
\>[3]{}(\Varid{e:}\;\Conid{Equivalence}\;\mathbb{R})\;\mathbin{:}\;\Conid{Type}\;\Conid{:=}\;\Varid{...}{}\<[E]%
\ColumnHook
\end{hscode}\resethooks
However, this way if we happen to have two proofs \ensuremath{\Varid{p}} and \ensuremath{\Varid{q}} that a certain relation is an
equivalence, we end up with two non-convertible ways to express that the relation (together with
some unit and binary operation) form a Monoid: 'Monoid A R p ...' and 'Monoid A R q ...'. We must
either try to deal with this (by proving that Monoid proofs with different Equivalence proofs are
interchangable, and hope for automation, such as type class instance resolution, to use them
interchangably), or take great care to ensure that \ensuremath{\Conid{Equivalence}} proof terms are always shared,
essentially elevating them to `values', which goes against the principle of proof
irrelevance.\marginpar{What if we add proof irrelevance?}

Neither of these options are very appealing. We really should just make \ensuremath{\Conid{Monoid}} take as parameters
only the operations/relations. If we treat the monoid's unit and binary operation the same way, this
means that \ensuremath{\Conid{Monoid}} ends up in \ensuremath{\Conid{Prop}}, and that its instances can be left opaque, which is really
how things ought to be.

Another benefit of this approach is that it works very smoothly when these records are in fact type
classes. When relations and operations are bundled and do \emph{not} appear as parameters of the
structure record/class, expressing as a type class constraint that some collection of relations and
operations posess the structure seems to require manifest fields; see Section~\ref{manifest}. With
our unbundled approach, on the other hand, such type class constraints are easily expressed without
any extensions as simple type inhabitation queries
% (e.g. ``hey Coq, find me a proof of [Monoid some_rel some_op some_unit]'')
of types living in Prop, so that we don't even care what instances are found (thus embracing proof
irrelevance).

We remark that leaving the equivalence proofs bundled means that when one has a proof that something
is a semiring, one has two proofs (via the two monoids) that the equivalence relation is an
equivalence. Fortunately, having redundant proofs like this is entirely harmless when one's terms
never refer (directly or indirectly) to proofs, which is the case when the relations and operations
they are composed of are left unbundled. Here, too, we enjoy the warm embrace of proof irrelevance.

From the discussion so far, we derive some principles:
\begin{enumerate}
 \item structural concepts (like Reflexive, Setoid, Monoid) should have their structural components
(i.e. carriers, relations, and operations) as parameters, and nothing else;
 \item (as a consequence) these concepts should be defined as predicates (which includes records and
classes living in Prop), and their proofs may be kept opaque;
 \item (as a second consequence) algebraic expressions need/should never refer (directly or
indirectly) to proofs.
\end{enumerate}
% \setcounter{enumi}{3}

This list will grow in subsequent sections.

The principles listed so far are not particularly controversial, in fact this
approach ensures maximum flexibility~\cite{Hints}. However,
there is a common perception that unbundling carriers, relations, and operations this way invariably
leads either to unmanagably big algebraic expressions (involving endless projections and/or implicit
arguments), or to unmanageably long and tedious enumeration of carriers/relations/operations, or to
problems establishing canonical names for operations (since we no longer project them out of
structure instances all the time).

In the next sections, we show how systematic use of type classes (and their support infrastructure
in Coq), combined with the use of a special form of type class we call an `operational type class'
(as opposed to a `structural' type class we might use in place of the records shown thus far), lets
us avoid these problems.

\section{Managing unbundled structure with type classes}

So far, we have only looked at the definitions of the structures themselves, and have not yet
considered the perspective of someone trying to prove some theorems about these abstract structures.
We would like to approximate the common mathematical vernacular, where one
simply says:

for $x, y, z$ in a semigroup $G$, $x * y * z = z * y * x$, $\ldots$

Suppose we have defined \ensuremath{\Conid{SemiGroup}} in line with the aforementioned principles:
\begin{hscode}\SaveRestoreHook
\column{B}{@{}>{\hspre}l<{\hspost}@{}}%
\column{3}{@{}>{\hspre}l<{\hspost}@{}}%
\column{E}{@{}>{\hspre}l<{\hspost}@{}}%
\>[3]{}\mathbf{Record}\;\Conid{SemiGroup}\;(\Conid{A:}\;\Conid{Type})\;(\Varid{eq:}\;\Conid{A}\;\rightarrow\;\Conid{A}\;\rightarrow\;\Conid{Prop})\;(\Varid{op:}\;\Conid{A}\;\rightarrow\;\Conid{A}\;\rightarrow\;\Conid{A})\;\mathbin{:}\;\Conid{Prop}\;\Conid{:=}\;\Varid{...}{}\<[E]%
\ColumnHook
\end{hscode}\resethooks
Then our starting point is the obvious but awful:
\begin{hscode}\SaveRestoreHook
\column{B}{@{}>{\hspre}l<{\hspost}@{}}%
\column{3}{@{}>{\hspre}l<{\hspost}@{}}%
\column{20}{@{}>{\hspre}l<{\hspost}@{}}%
\column{E}{@{}>{\hspre}l<{\hspost}@{}}%
\>[3]{}\mathbf{Lemma}\;\Varid{assoc1}\;\Conid{A}\;\Varid{eq}\;\Varid{op}\;(\Conid{P:}\;\Conid{SemiGroup}\;\Conid{A}\;\Varid{eq}\;\Varid{op})\;\Varid{x}\;\Varid{y}\;\Varid{z:}\;{}\<[E]%
\\
\>[3]{}\hsindent{17}{}\<[20]%
\>[20]{}\Varid{eq}\;(\Varid{op}\;(\Varid{op}\;\Varid{x}\;\Varid{y})\;\Varid{z})\;(\Varid{op}\;(\Varid{op}\;\Varid{z}\;\Varid{y})\;\Varid{x})\;.{}\<[E]%
\ColumnHook
\end{hscode}\resethooks
As a first cleaning measure, we might hope to introduce some notations to let us use infix notation
for the semigroup operator (say, \ensuremath{\Varid{\&}}) and for equality. Unfortunately, at this point the names \ensuremath{\Varid{eq}}
and \ensuremath{\Varid{op}} are just local names and there is no way to predeclare that they are to be associated with
any notation. What we really need are canonical names for these notions, so that we may fashion
notations for them.

This is a typical job for type classes, and they are used in this way in the \ensuremath{\Conid{Clean}} standard
library~\cite{Clean}. Similarly, Haskell uses type classes to overload notations.
We simply follow suit:
\begin{hscode}\SaveRestoreHook
\column{B}{@{}>{\hspre}l<{\hspost}@{}}%
\column{3}{@{}>{\hspre}l<{\hspost}@{}}%
\column{E}{@{}>{\hspre}l<{\hspost}@{}}%
\>[3]{}\mathbf{Class}\;\Conid{Equiv}\;\Conid{A}\;\Conid{:=}\;\Varid{equiv:}\;\Varid{relation}\;\Conid{A.}{}\<[E]%
\\
\>[3]{}\mathbf{Class}\;\Conid{SemiGroupOp}\;\Conid{A}\;\Conid{:=}\;\Varid{sg\char95 op:}\;\Conid{A}\;\rightarrow\;\Conid{A}\;\rightarrow\;\Conid{A.}{}\<[E]%
\\[\blanklineskip]%
\>[3]{}\mathbf{Infix}\;\text{\tt \char34 =\char34}\;\Conid{:=}\;\Varid{equiv:}\;\Varid{type\char95 scope.}{}\<[E]%
\\
\>[3]{}\mathbf{Infix}\;\text{\tt \char34 \&\char34}\;\Conid{:=}\;\Varid{sg\char95 op}\;(\Varid{at}\;\Varid{level}\;\Varid{50},\Varid{left}\;\Varid{associativity})\;.{}\<[E]%
\ColumnHook
\end{hscode}\resethooks
We choose to use \ensuremath{\Varid{\&}} here and reserve the notation \ensuremath{\Varid{*}} for ring multiplication.

We use = for setoid equality, since Leibniz equality seems less important for the formalization of
mathematics. More precisely, setoid equality is the basic concept. In special case, one can prove
that setoid equality coincides with Leibniz equality.

We would have like to use a haskell style definition
\begin{hscode}\SaveRestoreHook
\column{B}{@{}>{\hspre}l<{\hspost}@{}}%
\column{E}{@{}>{\hspre}l<{\hspost}@{}}%
\>[B]{}\mathbf{Class}\;\Conid{Equiv}\;\Conid{A}\;\Conid{:=}\;(\mathrel{=})\;\mathbin{:}\;\Varid{relation}\;\Conid{A.}{}\<[E]%
\ColumnHook
\end{hscode}\resethooks
Unfortunately, currently this is not supported in Coq.

These single-field type classes are given special treatment in that they are not translated into
records the way classes with any other number of fields are, but are instead turned into ordinary
definitions. The effect of this is that \ensuremath{\Conid{Equiv}\;\Conid{A}} immediately reduces to \ensuremath{\Varid{relation}\;\Conid{A}}, and that the
equiv `projection' is the identity function whose argument is inferred using instance resolution.
Thus, for a very modest cost in indirection, we gain the ability to use canonical names, and with
it, notations:
\begin{hscode}\SaveRestoreHook
\column{B}{@{}>{\hspre}l<{\hspost}@{}}%
\column{3}{@{}>{\hspre}l<{\hspost}@{}}%
\column{21}{@{}>{\hspre}l<{\hspost}@{}}%
\column{30}{@{}>{\hspre}l<{\hspost}@{}}%
\column{E}{@{}>{\hspre}l<{\hspost}@{}}%
\>[3]{}\mathbf{Record}\;\Conid{SemiGroup}\;(\Conid{A:}\;\Conid{Type})\;(\Varid{eq:}\;\Conid{Equiv}\;\Conid{A})\;{}\<[E]%
\\
\>[3]{}\hsindent{27}{}\<[30]%
\>[30]{}(\Varid{op:}\;\Conid{SemiGroupOp}\;\Conid{A})\;\mathbin{:}\;\Conid{Prop}\;\Conid{:=}\;\Varid{...}{}\<[E]%
\\
\>[3]{}\mathbf{Lemma}\;\Varid{assoc1}\;\Conid{A}\;\Varid{eq}\;\Varid{op}\;(\Conid{P:}\;\Conid{SemiGroup}\;\Conid{A}\;\Varid{eq}\;\Varid{op})\;\Varid{x}\;\Varid{y}\;\Varid{z:}\;{}\<[E]%
\\
\>[3]{}\hsindent{18}{}\<[21]%
\>[21]{}\Varid{x}\;\Varid{\&}\;\Varid{y}\;\Varid{\&}\;\Varid{z}\;\mathrel{=}\;\Varid{z}\;\Varid{\&}\;\Varid{y}\;\Varid{\&}\;\Varid{x.}{}\<[E]%
\ColumnHook
\end{hscode}\resethooks
We call these type classes \emph{operational type classes}. We note that operational type classes
allow us to avoid Coq's notation \ensuremath{\mathbf{scope}} mechanism.

It is important to point out that although the expression now \emph{looks} like it is bound to some
specific semigroup structure (which with traditional approaches would imply projections from bundles
and/or reference to proofs), it is really only referring near-directly to the actual operations
involved, with the semigroup proof existing only as opaque `knowledge about the operations' which we
may use in the proof. This lack of projections keeps our terms small and independent, and keeps
rewriting simple and sane.

On the other hand, it \emph{does} mean that all the carriers/relations/operations end up in the
context. The context for \ensuremath{\Varid{assoc1}} looks like:
\begin{hscode}\SaveRestoreHook
\column{B}{@{}>{\hspre}l<{\hspost}@{}}%
\column{3}{@{}>{\hspre}l<{\hspost}@{}}%
\column{E}{@{}>{\hspre}l<{\hspost}@{}}%
\>[3]{}\Conid{A:}\;\Conid{Type}{}\<[E]%
\\
\>[3]{}\Varid{eq:}\;\Conid{Equiv}\;\Conid{A}{}\<[E]%
\\
\>[3]{}\Varid{op:}\;\Conid{SemiGroupOp}\;\Conid{A}{}\<[E]%
\\
\>[3]{}\Conid{P:}\;\Conid{SemiGroup}\;\Conid{A}\;\Varid{eq}\;\Varid{op}{}\<[E]%
\ColumnHook
\end{hscode}\resethooks

While we are not particularly worried about overly large contexts (we feel that this will most
likely not be problematic for any but the most complex formalizations), having to \emph{declare}
these entities (like \ensuremath{\Varid{eq}} and \ensuremath{\Varid{op}} for \ensuremath{\Varid{assoc1}}) \emph{is} a bit of a chore. Fortunately, Coq
provides a feature called implicit generalization, which is exactly what we need.
\marginpar{describe how it works} With it, we can write \ensuremath{\Varid{assoc1}} as follows:
\begin{hscode}\SaveRestoreHook
\column{B}{@{}>{\hspre}l<{\hspost}@{}}%
\column{3}{@{}>{\hspre}l<{\hspost}@{}}%
\column{E}{@{}>{\hspre}l<{\hspost}@{}}%
\>[3]{}\mathbf{Lemma}\;\Varid{assoc1}\hspace*{0mm}`(\Conid{SemiGroup}\;\Conid{A})\;\Varid{x}\;\Varid{y}\;\Varid{z:}\;\Varid{x}\;\Varid{\&}\;\Varid{y}\;\Varid{\&}\;\Varid{z}\;\mathrel{=}\;\Varid{z}\;\Varid{\&}\;\Varid{y}\;\Varid{\&}\;\Varid{x.}{}\<[E]%
\ColumnHook
\end{hscode}\resethooks
Thus, we have reached the mathematical ideal we aimed for.

Let us now stop and think to what extent this approach suffers from all the problems commonly
associated with it. In particular, let us imagine what happens to our terms and contexts when we
want to talk about nested structures such as polynomials over the ring of matrices over a
given ring; see Section~\ref{interfaces}.

\section{Subset types}
BEGIN SPECULATIVE

A record type is \marginpar{mostly?} syntactic sugar for a $\Sigma$-type. In the presence of a
squash operator, \ensuremath{[\mskip1.5mu \mskip1.5mu]\;\Conid{:Type->Prop}}, we may define a subset type as \ensuremath{\lambda \;\hspace*{0mm}\{\Conid{A\char92 mid}\;\Conid{B\char92 }\mskip1.5mu\}\;\Varid{=Sig}\;\Conid{A.}\;[\mskip1.5mu \Conid{B}\mskip1.5mu]}. This is
pursued by Barras~\cite{Barras:subset,Werner}. The program machinery fakes subset types in the
present Coq implementation. Proof irrelevanve in Coq would allow a faithful
implementation.

Squash is a monad and allows many constructions: subsets, unions, images, heterogeneous equality,
\ldots.

Question: Would the meta-theory simplify if we would just add a conversion rule $x\triangleright
\epsilon$ for $x:Prop$? We could add a term constructor $[a]$ which will have type $[A]$. There
seems to be no reason to connect this to \ensuremath{\Conid{Prop}} or impredicativity.

END SPECULATIVE

We add another principle:
\begin{itemize}\setcounter{enumi}{3}
 \item Working hypothesis: `every record can be reordered as a subset'. Consequently, we
rewrite this subset \ensuremath{\lambda \;\hspace*{0mm}\{\Conid{A\char92 mid}\;\Conid{B\char92 }\mskip1.5mu\}} as \ensuremath{\mathbf{Record}\;\Varid{r}\;(\Varid{a:A})\;\Conid{:=}\;\hspace*{0mm}\{\Varid{b:B}\mskip1.5mu\}}.
\end{itemize}

\section{Algebraic hierarchy using type classes}\label{classes}
\paragraph{Hierarchy.}\label{hierarchy}
In our formalization of algebraic structures we use the structure as (implicit) parameters, but
pack the proofs. This can be nicely organized by using canonical names. For instance, we define:
\begin{hscode}\SaveRestoreHook
\column{B}{@{}>{\hspre}l<{\hspost}@{}}%
\column{3}{@{}>{\hspre}l<{\hspost}@{}}%
\column{18}{@{}>{\hspre}l<{\hspost}@{}}%
\column{E}{@{}>{\hspre}l<{\hspost}@{}}%
\>[B]{}\mathbf{Class}\;\Conid{SemiRing}\;\Conid{A}\;\hspace*{0mm}\{\Varid{e:}\;\Conid{Equiv}\;\Conid{A}\mskip1.5mu\}\;\hspace*{0mm}\{\Varid{plus:}\;\Conid{RingPlus}\;\Conid{A}\mskip1.5mu\}\;\hspace*{0mm}\{\Varid{mult:}\;\Conid{RingMult}\;\Conid{A}\mskip1.5mu\}\;{}\<[E]%
\\
\>[B]{}\hsindent{18}{}\<[18]%
\>[18]{}\hspace*{0mm}\{\Varid{zero:}\;\Conid{RingZero}\;\Conid{A}\mskip1.5mu\}\;\hspace*{0mm}\{\Varid{one:}\;\Conid{RingOne}\;\Conid{A}\mskip1.5mu\}\;\Conid{:Prop}\;\Conid{:=}\;{}\<[E]%
\\
\>[B]{}\hsindent{3}{}\<[3]%
\>[3]{}\hspace*{0mm}\{\Varid{semiring\char95 mult\char95 monoid:>}\;\Conid{Monoid}\;\Conid{A}\;(\Varid{op:=mult})\;(\Varid{unit:=one}){}\<[E]%
\\
\>[B]{}\hsindent{3}{}\<[3]%
\>[3]{};\Varid{semiring\char95 plus\char95 monoid:>}\;\Conid{Monoid}\;\Conid{A}\;(\Varid{op:=plus})\;(\Varid{unit:=zero}){}\<[E]%
\\
\>[B]{}\hsindent{3}{}\<[3]%
\>[3]{};\Varid{semiring\char95 plus\char95 comm:>}\;\Conid{Commutative}\;\Varid{plus}{}\<[E]%
\\
\>[B]{}\hsindent{3}{}\<[3]%
\>[3]{};\Varid{semiring\char95 mult\char95 comm:>}\;\Conid{Commutative}\;\Varid{mult}{}\<[E]%
\\
\>[B]{}\hsindent{3}{}\<[3]%
\>[3]{};\Varid{semiring\char95 distr:>}\;\Conid{Distribute}\;\Varid{mult}\;\Varid{plus}{}\<[E]%
\\
\>[B]{}\hsindent{3}{}\<[3]%
\>[3]{};\Varid{mult\char95 0\char95 l:}\;\forall\;\Varid{x},\Varid{0}\;\Varid{*}\;\Varid{x}\;= \;\Varid{0}\mskip1.5mu\}\;.{}\<[E]%
\ColumnHook
\end{hscode}\resethooks
where e.g.\ \ensuremath{\Conid{RingPlus}} is the operation class for the semigroup operator.
The notation \ensuremath{\Conid{:>}} makes \ensuremath{\Varid{semiring\char95 mult\char95 monoid}} a coercion: when asked for a monoid it suffices to
provide a ring.

\emph{Structure-as-parameters} helps setoid-rewriting: type class resolution
can find the equivalence relation in the context.
% A similar style should be possible for, say, the |Ring| tactic, instead of
% declaring the ring structure by a separate command, we would rely on type class resolution to find
% it.
We note that \ensuremath{\Varid{op}} does not depend on the proof that \ensuremath{\Varid{e}} is an equivalence. We use Coq's implicit
quantification (\ensuremath{\hspace*{0mm}`\hspace*{0mm}\{\mskip1.5mu\}}) to avoid having to write all the parameters when \emph{stating} a theorem
and Coq's maximally inserted implicit arguments to find the parameters when \emph{applying} a
theorem. Both features are new in Coq and stem from the type class implementation.

We mention the trade-off between bigger contexts versus bigger terms. Our contexts are bigger than
those of telescopes or packed classes. In our experience, this has been relatively
harmless\footnote{Coq's data structure for contexts is not very efficient however, we believe this
may be a bottleneck for huge developments. It seems that the data structure chosen
in~\cite{asperti2009compact} will behave better.}%
: most
terms in the context are there to support canonical names. Bigger terms
\emph{do} cause problems: 1. when proofs are part of mathematical objects we need to share these
proofs to allow rewriting. Moreover, it prohibits Opaque proofs and `proof irrelevance'. 2. The
projection paths may not be canonical.

Coercion pullbacks~\cite{Hints} were introduced to address problems with multiple coercions paths,
as in the definition of a semiring: a type with two monoid structures on it. We avoid some
of these problems by explicitly specifying the fields. We emphasize that the semiring properties are automatically derived from the ring
properties, although the properties of a semiring are not structurally included in the ring
properties.

\subsection{Manifest fields}\label{manifest}

Manifest fields in records allow us to fix a field of a dependent record.
Luo~\cite{DBLP:conf/types/Luo08} has proposed coersive subtyping to implement manifest fields.
It was conjectured~\cite{Hints} that implementing the algebraic hierarchy with type classes would
be awkward due to problems with equality. We avoided these problems by shifting the fields
from the dependent record to its argument, thus mimicking manifest fields by using type classes.
Since \ensuremath{\Varid{op}} is \emph{instantiated} with \ensuremath{\Varid{mult}}, we need not worry about the
equality \ensuremath{\Varid{op=mult}}.

\section{Universal algebra}\label{univ}
After having implemented concrete algebraic structures, we generalized some of our constructions
by implementing many-sorted universal algebra. A formalization of universal algebra
already exists in Coq~\cite{DBLP:conf/tphol/Capretta99}. It has recently been extended with more
categorical results~\cite{dominguez2008formalizing}. We have conveniently redeveloped most of this
using type classes. We defined equational theories and their varieties:
the categories of their models. A quote function helps to translate a statement for an instance of
a type class to a statement in the equational logic.

\subsection{Category theory}\label{cats}
Using type classes, we provide a basic library for category theory: categories, functors,
adjunctions, monads,\ldots We follow the usual type theoretical definition of a\marginpar{Use
notation for comp} category~\cite{saibi1995constructive}
\begin{hscode}\SaveRestoreHook
\column{B}{@{}>{\hspre}l<{\hspost}@{}}%
\column{3}{@{}>{\hspre}l<{\hspost}@{}}%
\column{4}{@{}>{\hspre}l<{\hspost}@{}}%
\column{7}{@{}>{\hspre}l<{\hspost}@{}}%
\column{E}{@{}>{\hspre}l<{\hspost}@{}}%
\>[B]{}\mathbf{Class}\;\Conid{Category}\;\Conid{O}\hspace*{0mm}`\hspace*{0mm}\{\Varid{!Arrows}\;\Conid{O}\mskip1.5mu\}\hspace*{0mm}`\hspace*{0mm}\{\forall\;\Varid{x}\;\Varid{y:}\;\Conid{O},\Conid{Equiv}\;(\Varid{x}\;\longrightarrow\;\Varid{y})\mskip1.5mu\}{}\<[E]%
\\
\>[B]{}\hsindent{4}{}\<[4]%
\>[4]{}\hspace*{0mm}`\hspace*{0mm}\{\Varid{!CatId}\;\Conid{O}\mskip1.5mu\}\hspace*{0mm}`\hspace*{0mm}\{\Varid{!CatComp}\;\Conid{O}\mskip1.5mu\}\;\mathbin{:}\;\Conid{Prop}\;\Conid{:=}{}\<[E]%
\\
\>[B]{}\hsindent{3}{}\<[3]%
\>[3]{}\hspace*{0mm}\{\Varid{arrow\char95 equiv:>}\;\forall\;\Varid{x}\;\Varid{y:}\;\Conid{O},\Conid{Setoid}\;(\Varid{x}\;\longrightarrow\;\Varid{y}){}\<[E]%
\\
\>[B]{}\hsindent{3}{}\<[3]%
\>[3]{};\Varid{comp\char95 proper:>}\;\forall\;\Varid{x}\;\Varid{y}\;\Varid{z:}\;\Conid{O},{}\<[E]%
\\
\>[3]{}\hsindent{4}{}\<[7]%
\>[7]{}\Conid{Proper}\;(\Varid{equiv}\;\Rightarrow \;\Varid{equiv}\;\Rightarrow \;\Varid{equiv})\;\Varid{\%signature}\;(\Varid{@comp}\;\anonymous \;\anonymous \;\anonymous \;\Varid{x}\;\Varid{y}\;\Varid{z}){}\<[E]%
\\
\>[B]{}\hsindent{3}{}\<[3]%
\>[3]{};\Varid{comp\char95 assoc:}\;\forall\;(\Varid{w}\;\Varid{x}\;\Varid{y}\;\Varid{z:}\;\Conid{O})\;(\Varid{a:}\;\Varid{w}\;\longrightarrow\;\Varid{x})\;(\Varid{b:}\;\Varid{x}\;\longrightarrow\;\Varid{y})\;(\Varid{c:}\;\Varid{y}\;\longrightarrow\;\Varid{z}),{}\<[E]%
\\
\>[3]{}\hsindent{4}{}\<[7]%
\>[7]{}\Varid{comp}\;\Varid{c}\;(\Varid{comp}\;\Varid{b}\;\Varid{a})\;= \;\Varid{comp}\;(\Varid{comp}\;\Varid{c}\;\Varid{b})\;\Varid{a}{}\<[E]%
\\
\>[B]{}\hsindent{3}{}\<[3]%
\>[3]{};\Varid{id\char95 l:}\;\forall\;(\Varid{x}\;\Varid{y:}\;\Conid{O})\;(\Varid{a:}\;\Varid{y}\;\longrightarrow\;\Varid{x}),\Varid{comp}\;\Varid{cat\char95 id}\;\Varid{a}\;= \;\Varid{a}{}\<[E]%
\\
\>[B]{}\hsindent{3}{}\<[3]%
\>[3]{};\Varid{id\char95 r:}\;\forall\;(\Varid{x}\;\Varid{y:}\;\Conid{O})\;(\Varid{a:}\;\Varid{x}\;\longrightarrow\;\Varid{y}),\Varid{comp}\;\Varid{a}\;\Varid{cat\char95 id}\;= \;\Varid{a}\mskip1.5mu\}\;.{}\<[E]%
\ColumnHook
\end{hscode}\resethooks
This is based on the 2-categorical idea that we have equality on arrows, but not on objects.
Similarly, we have equality on natural transformations, but not on functors.

For instance natural transformations are formalized as: \ensuremath{\circ}
\begin{hscode}\SaveRestoreHook
\column{B}{@{}>{\hspre}l<{\hspost}@{}}%
\column{3}{@{}>{\hspre}l<{\hspost}@{}}%
\column{5}{@{}>{\hspre}l<{\hspost}@{}}%
\column{6}{@{}>{\hspre}l<{\hspost}@{}}%
\column{E}{@{}>{\hspre}l<{\hspost}@{}}%
\>[3]{}\Conid{Context}\hspace*{0mm}`\hspace*{0mm}\{\Conid{Category}\;\Conid{C}\mskip1.5mu\}\hspace*{0mm}`\hspace*{0mm}\{\Conid{Category}\;\Conid{D}\mskip1.5mu\}\;(\Conid{F}\;\Conid{G:}\;\Conid{C}\;\to \;\Conid{D}){}\<[E]%
\\
\>[3]{}\hsindent{3}{}\<[6]%
\>[6]{}\hspace*{0mm}`\hspace*{0mm}\{\Varid{!Functor}\;\Conid{F}\;\Conid{Fa}\mskip1.5mu\}\hspace*{0mm}`\hspace*{0mm}\{\Varid{!Functor}\;\Conid{G}\;\Conid{Ga}\mskip1.5mu\}\;.{}\<[E]%
\\[\blanklineskip]%
\>[3]{}\mathbf{Class}\;\Conid{NaturalTransformation}\;(\Conid{η:}\;\forall\;\Varid{c:}\;\Conid{C},\Conid{F}\;\Varid{c}\;\longrightarrow\;\Conid{G}\;\Varid{c})\;\mathbin{:}\;\Conid{Prop}\;\Conid{:=}\;{}\<[E]%
\\
\>[3]{}\hsindent{2}{}\<[5]%
\>[5]{}\Varid{natural:}\;\forall\;(\Varid{x}\;\Varid{y:}\;\Conid{C})\;(\Varid{f:}\;\Varid{x}\;\longrightarrow\;\Varid{y}),\Conid{η}\;\Varid{y}\;\circ\;\Varid{fmap}\;\Conid{F}\;\Varid{f}\;= \;\Varid{fmap}\;\Conid{G}\;\Varid{f}\;\circ\;\Conid{η}\;\Varid{x.}{}\<[E]%
\ColumnHook
\end{hscode}\resethooks

The notation \ensuremath{\circ} denotes function composition.
Here \ensuremath{\Varid{fmap}\;\Conid{F}} explicitly refers to the arrow part of the functor \ensuremath{\Conid{F}}.\\
In mathematics one would apply $F$ to both objects and arrows ($F x$, $F f$). However, this does
not seem to fit with type theory, hence we follow the customs in functional programming.

The proofs are short and follow the textbook easily.\marginpar{really?}

We provide the category of setoids, the category of categories, the dual of a category, products,
and hence co-products, the category of $\Sigma$-algebras.

We provide the construction of the term-algebra for a signature $\Sigma$ and show that it is
initial in the category of $\Sigma$-algebras. We proceed to prove that this provides a left-adjoint
to the forgetful functor. The resulting expression monad will be used below.

\subsection{Universal algebra continued}\label{univ2}

We have developed basic universal algebra~\cite{meinke1993universal} upto the first homomorphism
theorem.~\marginpar{Description of the work.}
%perhaps the soundness theorem for equational logic.

In order to prove the first homophism theorem, we define a congruence:
%theory/ua_congruence.v
\begin{hscode}\SaveRestoreHook
\column{B}{@{}>{\hspre}l<{\hspost}@{}}%
\column{5}{@{}>{\hspre}l<{\hspost}@{}}%
\column{30}{@{}>{\hspre}l<{\hspost}@{}}%
\column{E}{@{}>{\hspre}l<{\hspost}@{}}%
\>[B]{}\mathbf{Class}\;\Conid{Congruence:}\;\Conid{Prop}\;\Conid{:=}\;{}\<[E]%
\\
\>[B]{}\hsindent{5}{}\<[5]%
\>[5]{}\hspace*{0mm}\{\Varid{congruence\char95 proper:>}\;\Pi\;{}\<[30]%
\>[30]{}\Varid{s},\Conid{Proper}\;(\Varid{equiv}\;\Rightarrow \;\Varid{equiv}\;\Rightarrow \;\Varid{iff})\;(\Varid{e}\;\Varid{s}){}\<[E]%
\\
\>[B]{}\hsindent{5}{}\<[5]%
\>[5]{};\Varid{congruence\char95 quotient:>}\;\Varid{@Algebra}\;\Varid{et}\;\Varid{v}\;\Varid{e}\;\anonymous {}\<[E]%
\\
\>[B]{}\hsindent{5}{}\<[5]%
\>[5]{}\mskip1.5mu\}\;.{}\<[E]%
\ColumnHook
\end{hscode}\resethooks
We note the use of \ensuremath{\Conid{Proper}}.

We define the forgetful functor from the category of $\Sigma$-algebras to the category of sets.

We define the term algebra consisting of closed terms, terms with the empty type as set of
variables. % extensionally equal closed term???
We show that the term algebra is initial. One may expect that this could be useful to
\emph{define}, say, the natural numbers as the initial semiring. However, it seems difficult to
prove decidability of equality, as it seems to require a normalization procedure.

We define the product of two algebras and show that it is the categorical product.

Given a sub\emph{set} of an algebra, we define the sub\emph{algebra} generated by it.
An interesting fact to mention is the use of the following heterogeneous equality between
elements of the algebra and of the subalgebra, i.e.\ terms of different types may be equal.
\begin{hscode}\SaveRestoreHook
\column{B}{@{}>{\hspre}l<{\hspost}@{}}%
\column{3}{@{}>{\hspre}l<{\hspost}@{}}%
\column{5}{@{}>{\hspre}l<{\hspost}@{}}%
\column{E}{@{}>{\hspre}l<{\hspost}@{}}%
\>[3]{}\mathbf{Fixpoint}\;\Varid{heq}\;\hspace*{0mm}\{\Varid{o}\mskip1.5mu\}\;\mathbin{:}\;\Varid{op\char95 type}\;\Varid{carrier}\;\Varid{o}\;\to \;\Varid{op\char95 type}\;\Varid{v}\;\Varid{o}\;\to \;\Conid{Prop}\;\Conid{:=}\;{}\<[E]%
\\
\>[3]{}\hsindent{2}{}\<[5]%
\>[5]{}\mathbf{match}\;\Varid{o}\;\mathbf{with}{}\<[E]%
\\
\>[3]{}\hsindent{2}{}\<[5]%
\>[5]{}\mid \;\Varid{constant}\;\Varid{x}\;\Rightarrow \;\lambda\hspace{-1mm} \;\Varid{a}\;\Varid{b}\;\Rightarrow \hspace*{0mm}`\Varid{a}\;= \;\Varid{b}{}\<[E]%
\\
\>[3]{}\hsindent{2}{}\<[5]%
\>[5]{}\mid \;\Varid{function}\;\Varid{x}\;\Varid{y}\;\Rightarrow \;\lambda\hspace{-1mm} \;\Varid{a}\;\Varid{b}\;\Rightarrow \;\forall\;\Varid{u},\Varid{heq}\;(\Varid{a}\;\Varid{u})\;(\Varid{b}\;(\hspace*{0mm}`\Varid{u})){}\<[E]%
\\
\>[3]{}\hsindent{2}{}\<[5]%
\>[5]{}\Varid{end.}{}\<[E]%
\ColumnHook
\end{hscode}\resethooks


We connect the abstract theory to the concrete theory. Concretely, let $\Sigma$ be the theory
of monoids. Given a monoid, we can construct the corresponding $\Sigma$-algebra and an
object in the category of $\Sigma$-algebras. Conversely, given an object in the category, we
construct a monoid.
\begin{hscode}\SaveRestoreHook
\column{B}{@{}>{\hspre}l<{\hspost}@{}}%
\column{E}{@{}>{\hspre}l<{\hspost}@{}}%
\>[B]{}\mathbf{Variable}\;\Varid{o:}\;\Conid{Variety}\;\Varid{theory.}{}\<[E]%
\\
\>[B]{}\mathbf{Global}\;\mathbf{Instance}\;\Varid{from\char95 object:}\;\Conid{Monoid}\;(\Varid{o}\;\Varid{tt})\;.{}\<[E]%
\ColumnHook
\end{hscode}\resethooks
Similarly, for arrows.



%There is a slight problem when defining the forgetful functor from the category of
%$\Sigma$-algebras to Sets. This problem is caused by our use of the category of all
%categories\footnote{Perhaps we should have used higher categories for this.}.
%We conjecture that this problem disappears when Coq's universe polymorphism is extended to
%definitions.
\section{Interfaces using category theory}\label{interfaces}\label{modul}
The current work was stimulated by our desire~\cite{Riemann} to move between several
implementations in a flexible way. In haskell type class polymorphism is used for this purpose.
Hence, we do the same in type theory.

We have characterized the naturals as the initial object in the category of semirings and derived
many of their properties from this interface. Unary, binary and machine numbers are
instances of this interface, so we can directly apply these results to all these instances.
Similarly, for the integers are the initial ring. The rationals are the field of fractions of the
integers. 
Given a ring $R$, the $R$-algebra $R[X]$ of polynomials, is the free $R$-algebra of a set $X$.
We provide two implementations of polynomials: the
standard representation using lists of coefficients and the Bernstein representation. A Bernstein
basis polynomial is one of the form:
\[b_{\nu,n}(x) = {n \choose \nu} x^{\nu} \left( 1 - x \right)^{n - \nu}, \quad \nu = 0, \ldots, n.\]
A Bernstein polynomial is a linear combination of these basic polynomials. Bernstein polynomials
have been used for efficient computations inside the Coq system~\cite{ZumkellerPhD}.

We encourage more efficient implementations by assigning the default implementation a
low priority. For example, the distance function on the natural numbers, which is derived from its
semiring structure, is assigned priority 10.
\begin{hscode}\SaveRestoreHook
\column{B}{@{}>{\hspre}l<{\hspost}@{}}%
\column{3}{@{}>{\hspre}l<{\hspost}@{}}%
\column{E}{@{}>{\hspre}l<{\hspost}@{}}%
\>[3]{}\mathbf{Global}\;\mathbf{Program}\;\Conid{Instance:}\;\Conid{NatDistance}\;\Conid{N}\;\mid \;\Varid{10}\;\Conid{:=}\;\Varid{...}{}\<[E]%
\ColumnHook
\end{hscode}\resethooks

The context grows as follows:\marginpar{expand}

\section{Type class quoting}\label{quote}
Unification hints~\cite{Hints} allow one to semi-automatically construct a quote
function\footnote{Gonthier provides similar functionality using canonical structures}.
Unification hints are absent in Coq. Fortunately, type classes provide similar functionality as
we will now show.

We define a term language.
\begin{hscode}\SaveRestoreHook
\column{B}{@{}>{\hspre}l<{\hspost}@{}}%
\column{E}{@{}>{\hspre}l<{\hspost}@{}}%
\>[B]{}\mathbf{Inductive}\;\Conid{Expr}\;(\Conid{V:}\;\Conid{Type})\;\Conid{:=}\;\Conid{Mult}\;(\Varid{a}\;\Varid{b:}\;\Conid{Expr}\;\Conid{V})\;\mid \;\Conid{Zero}\;\mid \;\Conid{Var}\;(\Varid{v:}\;\Conid{V})\;.{}\<[E]%
\ColumnHook
\end{hscode}\resethooks
This defines a monad on Type\marginpar{Do this!}.
The expression type is parameterized over the set of variable indices. Hence, we diverge
 from~\cite{Hints}, which uses \ensuremath{\mathbb{N}} for variables thereby introducing bounds problems and
 dummy variables.

\noindent An expression is only meaningful in the context of a variable assignment:\footnote{nat is
an Expr-algebra. eval is map vars composed with the algebra map.}
\begin{hscode}\SaveRestoreHook
\column{B}{@{}>{\hspre}l<{\hspost}@{}}%
\column{3}{@{}>{\hspre}l<{\hspost}@{}}%
\column{E}{@{}>{\hspre}l<{\hspost}@{}}%
\>[B]{}\mathbf{Definition}\;\Conid{Value}\;\Conid{:=}\;\Varid{nat.}{}\<[E]%
\\
\>[B]{}\mathbf{Definition}\;\Conid{Vars}\;\Conid{V}\;\Conid{:=}\;\Conid{V}\;\to \;\Conid{Value.}{}\<[E]%
\\[\blanklineskip]%
\>[B]{}\mathbf{Fixpoint}\;\Varid{eval}\;\hspace*{0mm}\{\Conid{V}\mskip1.5mu\}\;(\Varid{vs:}\;\Conid{Vars}\;\Conid{V})\;(\Varid{e:}\;\Conid{Expr}\;\Conid{V})\;\mathbin{:}\;\Conid{Value}\;\Conid{:=}\;{}\<[E]%
\\
\>[B]{}\hsindent{3}{}\<[3]%
\>[3]{}\mathbf{match}\;\Varid{e}\;\mathbf{with}{}\<[E]%
\\
\>[B]{}\hsindent{3}{}\<[3]%
\>[3]{}\mid \;\Conid{Zero}\;\Rightarrow \;\Varid{0}{}\<[E]%
\\
\>[B]{}\hsindent{3}{}\<[3]%
\>[3]{}\mid \;\Conid{Mult}\;\Varid{a}\;\Varid{b}\;\Rightarrow \;\Varid{eval}\;\Varid{vs}\;\Varid{a}\;\Varid{*}\;\Varid{eval}\;\Varid{vs}\;\Varid{b}{}\<[E]%
\\
\>[B]{}\hsindent{3}{}\<[3]%
\>[3]{}\mid \;\Conid{Var}\;\Varid{v}\;\Rightarrow \;\Varid{vs}\;\Varid{v}{}\<[E]%
\\
\>[B]{}\hsindent{3}{}\<[3]%
\>[3]{}\Varid{end.}{}\<[E]%
\ColumnHook
\end{hscode}\resethooks
%
Monads are trees with grafting. As explained by sigfpe~\cite{MonadsGrafting}, the tree monad
$TX:=1+T^2X$, may be seen as the prototypical monad. Indeed, the \ensuremath{\Conid{Expr}} monad, i.e.\ the free
$\Sigma$-algebra monad, behaves much like the tree monad. We now provide a direct definition of the
corresponding bind operation.

\begin{hscode}\SaveRestoreHook
\column{B}{@{}>{\hspre}l<{\hspost}@{}}%
\column{3}{@{}>{\hspre}l<{\hspost}@{}}%
\column{E}{@{}>{\hspre}l<{\hspost}@{}}%
\>[B]{}\mathbf{Fixpoint}\;\Varid{bind}\;\hspace*{0mm}\{\Conid{V}\;\Conid{W}\mskip1.5mu\}\;(\Varid{f:}\;\Conid{V}\;\to \;\Conid{Expr}\;\Conid{W})\;(\Varid{e:}\;\Conid{Expr}\;\Conid{V})\;\mathbin{:}\;\Conid{Expr}\;\Conid{W}\;\Conid{:=}\;{}\<[E]%
\\
\>[B]{}\hsindent{3}{}\<[3]%
\>[3]{}\mathbf{match}\;\Varid{e}\;\mathbf{with}{}\<[E]%
\\
\>[B]{}\hsindent{3}{}\<[3]%
\>[3]{}\mid \;\Conid{Zero}\;\Rightarrow \;\Conid{Zero}{}\<[E]%
\\
\>[B]{}\hsindent{3}{}\<[3]%
\>[3]{}\mid \;\Conid{Mult}\;\Varid{a}\;\Varid{b}\;\Rightarrow \;\Conid{Mult}\;(\Varid{bind}\;\Varid{f}\;\Varid{a})\;(\Varid{bind}\;\Varid{f}\;\Varid{b}){}\<[E]%
\\
\>[B]{}\hsindent{3}{}\<[3]%
\>[3]{}\mid \;\Conid{Var}\;\Varid{v}\;\Rightarrow \;\Varid{vs}\;\Varid{v}{}\<[E]%
\\
\>[B]{}\hsindent{3}{}\<[3]%
\>[3]{}\Varid{end.}{}\<[E]%
\ColumnHook
\end{hscode}\resethooks

We have shown that \ensuremath{\Conid{Expr}}, i.e.\ \ensuremath{\Varid{bind}} together with \ensuremath{\Conid{Var}}, is a monad on the category of Types
with extensional functions between them~\marginpar{Really?}.

\noindent Some simple combinators for variable packs:
%
\begin{hscode}\SaveRestoreHook
\column{B}{@{}>{\hspre}l<{\hspost}@{}}%
\column{3}{@{}>{\hspre}l<{\hspost}@{}}%
\column{E}{@{}>{\hspre}l<{\hspost}@{}}%
\>[B]{}\mathbf{Definition}\;\Varid{novars:}\;\Conid{Vars}\;\Conid{False}\;\Conid{:=}\;\Conid{False\char95 rect}\;\Varid{\char95 .}{}\<[E]%
\\
\>[B]{}\mathbf{Definition}\;\Varid{singlevar}\;(\Varid{x:}\;\Conid{Value})\;\mathbin{:}\;\Conid{Vars}\;\Varid{unit}\;\Conid{:=}\;\lambda\hspace{-1mm} \;\anonymous \;\Rightarrow \;\Varid{x.}{}\<[E]%
\\
\>[B]{}\mathbf{Definition}\;\Varid{merge}\;\hspace*{0mm}\{\Conid{A}\;\Conid{B}\mskip1.5mu\}\;(\Varid{a:}\;\Conid{Vars}\;\Conid{A})\;(\Varid{b:}\;\Conid{Vars}\;\Conid{B})\;\mathbin{:}\;\Conid{Vars}\;(\Conid{A+B})\;\Conid{:=}\;{}\<[E]%
\\
\>[B]{}\hsindent{3}{}\<[3]%
\>[3]{}\lambda\hspace{-1mm} \;\Varid{i}\;\Rightarrow \;\Varid{if}\;\Varid{i}\;\Varid{then}\;\Varid{a}\;\Varid{j}\;\Varid{else}\;\Varid{b}\;\Varid{j.}{}\<[E]%
\ColumnHook
\end{hscode}\resethooks

\noindent These last two combinators are the `constructors' of an implicitly defined subset of
 Gallina terms, representing heaps, for which we implement syntactic lookup with type classes.
Given a heap and value, \ensuremath{\Conid{Lookup}} instances give the value's index in the heap:\marginpar{A
canonical structures approach would allow us to do this in Coq?}
\begin{hscode}\SaveRestoreHook
\column{B}{@{}>{\hspre}l<{\hspost}@{}}%
\column{3}{@{}>{\hspre}l<{\hspost}@{}}%
\column{5}{@{}>{\hspre}l<{\hspost}@{}}%
\column{E}{@{}>{\hspre}l<{\hspost}@{}}%
\>[3]{}\mathbf{Class}\;\Conid{Lookup}\;\hspace*{0mm}\{\Conid{A}\mskip1.5mu\}\;(\Varid{x:}\;\Conid{Value})\;(\Varid{f:}\;\Conid{Vars}\;\Conid{A})\;\Conid{:=}\;{}\<[E]%
\\
\>[3]{}\hsindent{2}{}\<[5]%
\>[5]{}\hspace*{0mm}\{\Varid{lookup:}\;\Conid{A};\Varid{lookup\char95 correct:}\;\Varid{f}\;\Varid{lookup}\;\mathrel{=}\;\Varid{x}\mskip1.5mu\}\;.{}\<[E]%
\ColumnHook
\end{hscode}\resethooks
% Context (x: Value) {A B} (va: Vars A) (vb: Vars B).

If the heap is a merge of two heaps and we can find the value's index in the left heap, we can
access it by indexing the merged heap (and vice versa)
\begin{hscode}\SaveRestoreHook
\column{B}{@{}>{\hspre}l<{\hspost}@{}}%
\column{3}{@{}>{\hspre}l<{\hspost}@{}}%
\column{5}{@{}>{\hspre}l<{\hspost}@{}}%
\column{E}{@{}>{\hspre}l<{\hspost}@{}}%
\>[3]{}\mathbf{Global}\;\mathbf{Instance}\;\Varid{lookup\char95 left}\hspace*{0mm}`\hspace*{0mm}\{\Varid{!Lookup}\;\Varid{x}\;\Varid{va}\mskip1.5mu\}\;\mathbin{:}\;\Conid{Lookup}\;\Varid{x}\;(\Varid{merge}\;\Varid{va}\;\Varid{vb}){}\<[E]%
\\
\>[3]{}\hsindent{2}{}\<[5]%
\>[5]{}\Conid{:=}\;\hspace*{0mm}\{\Varid{lookup}\;\Conid{:=}\;\Varid{inl}\;(\Varid{lookup}\;\Varid{x}\;\Varid{va})\mskip1.5mu\}\;.{}\<[E]%
\ColumnHook
\end{hscode}\resethooks
If the heap is just a singlevar, we can easily index it.
\begin{hscode}\SaveRestoreHook
\column{B}{@{}>{\hspre}l<{\hspost}@{}}%
\column{3}{@{}>{\hspre}l<{\hspost}@{}}%
\column{E}{@{}>{\hspre}l<{\hspost}@{}}%
\>[3]{}\mathbf{Global}\;\mathbf{Program}\;\Conid{Instance:}\;\Conid{Lookup}\;\Varid{x}\;(\Varid{singlevar}\;\Varid{x})\;\Conid{:=}\;\hspace*{0mm}\{\Varid{lookup}\;\Conid{:=}\;\Varid{tt}\mskip1.5mu\}\;.{}\<[E]%
\ColumnHook
\end{hscode}\resethooks

One useful operation we need before we get to Quote relates to variables and expression
 evaluation. As its name suggests, \ensuremath{\Varid{map\char95 var}} maps an expression's variable indices.\marginpar{monad}

\begin{hscode}\SaveRestoreHook
\column{B}{@{}>{\hspre}l<{\hspost}@{}}%
\column{3}{@{}>{\hspre}l<{\hspost}@{}}%
\column{5}{@{}>{\hspre}l<{\hspost}@{}}%
\column{E}{@{}>{\hspre}l<{\hspost}@{}}%
\>[B]{}\mathbf{Definition}\;\Varid{map\char95 var}\;\hspace*{0mm}\{\Conid{V}\;\Conid{W:}\;\Conid{Type}\mskip1.5mu\}\;(\Varid{f:}\;\Conid{V}\;\to \;\Conid{W})\;\mathbin{:}\;{}\<[E]%
\\
\>[B]{}\hsindent{5}{}\<[5]%
\>[5]{}\Conid{Expr}\;\Conid{V}\;\to \;\Conid{Expr}\;\Conid{W}\;\Conid{:=}\;{}\<[E]%
\\
\>[B]{}\hsindent{3}{}\<[3]%
\>[3]{}\Varid{fix}\;\Conid{F}\;(\Varid{e:}\;\Conid{Expr}\;\Conid{V})\;\mathbin{:}\;\Conid{Expr}\;\Conid{W}\;\Conid{:=}\;{}\<[E]%
\\
\>[3]{}\hsindent{2}{}\<[5]%
\>[5]{}\mathbf{match}\;\Varid{e}\;\mathbf{with}{}\<[E]%
\\
\>[3]{}\hsindent{2}{}\<[5]%
\>[5]{}\mid \;\Conid{Mult}\;\Varid{a}\;\Varid{b}\;\Rightarrow \;\Conid{Mult}\;(\Conid{F}\;\Varid{a})\;(\Conid{F}\;\Varid{b}){}\<[E]%
\\
\>[3]{}\hsindent{2}{}\<[5]%
\>[5]{}\mid \;\Conid{Zero}\;\Rightarrow \;\Conid{Zero}{}\<[E]%
\\
\>[3]{}\hsindent{2}{}\<[5]%
\>[5]{}\mid \;\Conid{Var}\;\Varid{v}\;\Rightarrow \;\Conid{Var}\;(\Varid{f}\;\Varid{v}){}\<[E]%
\\
\>[3]{}\hsindent{2}{}\<[5]%
\>[5]{}\Varid{end.}{}\<[E]%
\ColumnHook
\end{hscode}\resethooks
In \ensuremath{\Conid{Quote}} below, the idea is that \ensuremath{\Conid{V},\Varid{l}}, and \ensuremath{\Varid{n}} are all input variables, while \ensuremath{\Conid{V'}} and \ensuremath{\Varid{r}} are
output variables (in the sense that we will rely on unification to generate them). \ensuremath{\Conid{V}} and \ensuremath{\Varid{l}}
represent the current heap, \ensuremath{\Varid{n}} represents the value we want to quote, and \ensuremath{\Conid{V'}} and \ensuremath{\Varid{r'}} represent
the heap of newly encountered variables during the quotation.
  This explains the type of quote: it is an expression that refers either to variables from
the old heap, or to newly encountered variables. Finally, \ensuremath{\Varid{eval\char95 quote}} is the usual correctness
property, which now merges the two heaps.

\begin{hscode}\SaveRestoreHook
\column{B}{@{}>{\hspre}l<{\hspost}@{}}%
\column{3}{@{}>{\hspre}l<{\hspost}@{}}%
\column{5}{@{}>{\hspre}l<{\hspost}@{}}%
\column{E}{@{}>{\hspre}l<{\hspost}@{}}%
\>[3]{}\mathbf{Class}\;\Conid{Quote}\;\hspace*{0mm}\{\Conid{V}\mskip1.5mu\}\;(\Varid{l:}\;\Conid{Vars}\;\Conid{V})\;(\Varid{n:}\;\Conid{Value})\;\hspace*{0mm}\{\Conid{V'}\mskip1.5mu\}\;(\Varid{r:}\;\Conid{Vars}\;\Conid{V'})\;\Conid{:=}\;{}\<[E]%
\\
\>[3]{}\hsindent{2}{}\<[5]%
\>[5]{}\hspace*{0mm}\{\Varid{quote:}\;\Conid{Expr}\;(\Conid{V}\;\Varid{+}\;\Conid{V'}){}\<[E]%
\\
\>[3]{}\hsindent{2}{}\<[5]%
\>[5]{};\Varid{eval\char95 quote:}\;\Varid{@eval}\;(\Conid{V+V'})\;(\Varid{merge}\;\Varid{l}\;\Varid{r})\;\Varid{quote}\;\mathrel{=}\;\Varid{n}\mskip1.5mu\}\;.{}\<[E]%
\ColumnHook
\end{hscode}\resethooks

Our first instance for \ensuremath{\Conid{Zero}} is easy. The `novars' in the result type reflects the fact that no
new variables are encountered.
\begin{hscode}\SaveRestoreHook
\column{B}{@{}>{\hspre}l<{\hspost}@{}}%
\column{3}{@{}>{\hspre}l<{\hspost}@{}}%
\column{5}{@{}>{\hspre}l<{\hspost}@{}}%
\column{E}{@{}>{\hspre}l<{\hspost}@{}}%
\>[3]{}\mathbf{Global}\;\mathbf{Program}\;\mathbf{Instance}\;\Varid{quote\char95 zero}\;\Conid{V}\;(\Varid{v:}\;\Conid{Vars}\;\Conid{V})\;\mathbin{:}\;{}\<[E]%
\\
\>[3]{}\hsindent{2}{}\<[5]%
\>[5]{}\Conid{Quote}\;\Varid{v}\;\Varid{0}\;\Varid{novars}\;\Conid{:=}{}\<[E]%
\\
\>[3]{}\hspace*{0mm}\{\Varid{quote}\;\Conid{:=}\;\Conid{Zero}\mskip1.5mu\}\;.{}\<[E]%
\ColumnHook
\end{hscode}\resethooks

The instance for multiplication is a bit more complex. The first line consists of
 variable declarations. The second line is important. `Quote x y z' must be read as
 `quoting y with existing heap x generates new heap z', so the second line basically just
shuffles heaps around.
 The third line contains some \ensuremath{\Varid{map\char95 var}}'s because the heap shuffling must be
reflected in the variable indices, but apart from that it's just constructing a \ensuremath{\Conid{Mult}} term with
quoted subterms.

\begin{hscode}\SaveRestoreHook
\column{B}{@{}>{\hspre}l<{\hspost}@{}}%
\column{3}{@{}>{\hspre}l<{\hspost}@{}}%
\column{E}{@{}>{\hspre}l<{\hspost}@{}}%
\>[B]{}\mathbf{Global}\;\mathbf{Program}\;\mathbf{Instance}\;\Varid{quote\char95 mult}\;\Conid{V}\;(\Varid{v:}\;\Conid{Vars}\;\Conid{V})\;{}\<[E]%
\\
\>[B]{}\hsindent{3}{}\<[3]%
\>[3]{}\Varid{n}\;\Conid{V'}\;(\Varid{v':}\;\Conid{Vars}\;\Conid{V'})\;\Varid{m}\;\Conid{V''}\;(\Varid{v'':}\;\Conid{Vars}\;\Conid{V''}){}\<[E]%
\\
\>[B]{}\hsindent{3}{}\<[3]%
\>[3]{}\hspace*{0mm}`\hspace*{0mm}\{\Varid{!Quote}\;\Varid{v}\;\Varid{n}\;\Varid{v'}\mskip1.5mu\}\hspace*{0mm}`\hspace*{0mm}\{\Varid{!Quote}\;(\Varid{merge}\;\Varid{v}\;\Varid{v'})\;\Varid{m}\;\Varid{v''}\mskip1.5mu\}\;\mathbin{:}{}\<[E]%
\\
\>[B]{}\hsindent{3}{}\<[3]%
\>[3]{}\Conid{Quote}\;\Varid{v}\;(\Varid{n}\;\Varid{*}\;\Varid{m})\;(\Varid{merge}\;\Varid{v'}\;\Varid{v''})\;\Conid{:=}{}\<[E]%
\\
\>[B]{}\hsindent{3}{}\<[3]%
\>[3]{}\hspace*{0mm}\{\Varid{quote}\;\Conid{:=}\;\Conid{Mult}\;(\Varid{map\char95 var}\;\Varid{bla}\;(\Varid{quote}\;\Varid{n}))\;(\Varid{map\char95 var}\;\Varid{sum\char95 assoc}\;(\Varid{quote}\;\Varid{m}))\mskip1.5mu\}\;.{}\<[E]%
\ColumnHook
\end{hscode}\resethooks
  Now follows the instance where we recognize values that are already in the heap. This
   is expressed by the Lookup requirement, which will only be fulfilled if the Lookup instances
   defined above can find the value in the heap. The novars in the \ensuremath{\Conid{Quote}\;\Varid{v}\;\Varid{x}\;\Varid{novars}} result
   reflects that this quotation does not generate new variables.

\begin{hscode}\SaveRestoreHook
\column{B}{@{}>{\hspre}l<{\hspost}@{}}%
\column{3}{@{}>{\hspre}l<{\hspost}@{}}%
\column{E}{@{}>{\hspre}l<{\hspost}@{}}%
\>[B]{}\mathbf{Global}\;\mathbf{Program}\;\mathbf{Instance}\;\Varid{quote\char95 old\char95 var}\;{}\<[E]%
\\
\>[B]{}\hsindent{3}{}\<[3]%
\>[3]{}\Conid{V}\;(\Varid{v:}\;\Conid{Vars}\;\Conid{V})\;\Varid{x}\;\hspace*{0mm}\{\Varid{i:}\;\Conid{Lookup}\;\Varid{x}\;\Varid{v}\mskip1.5mu\}\;\mathbin{:}\;{}\<[E]%
\\
\>[B]{}\hsindent{3}{}\<[3]%
\>[3]{}\Conid{Quote}\;\Varid{v}\;\Varid{x}\;\Varid{novars}\;\mid \;\Varid{8}\;\Conid{:=}\;\hspace*{0mm}\{\Varid{quote}\;\Conid{:=}\;\Conid{Var}\;(\Varid{inl}\;(\Varid{lookup}\;\Varid{x}\;\Varid{v}))\mskip1.5mu\}\;.{}\<[E]%
\ColumnHook
\end{hscode}\resethooks

\noindent Finally, the instance for new variables. We give this a lower priority so that it is only
used if Lookup fails. The \ensuremath{\Varid{8}} in the previous example is a random number less than \ensuremath{\Varid{9}} below. A
similar method is used in Coq's notation mechanism. We strongly recommend a change in
implementation supporting an agda style precedence relation which is only restricted to be a
directed acyclic graph~\cite{danielsson2009parsing}.

\begin{hscode}\SaveRestoreHook
\column{B}{@{}>{\hspre}l<{\hspost}@{}}%
\column{3}{@{}>{\hspre}l<{\hspost}@{}}%
\column{E}{@{}>{\hspre}l<{\hspost}@{}}%
\>[B]{}\mathbf{Global}\;\mathbf{Program}\;\mathbf{Instance}\;\Varid{quote\char95 new\char95 var}\;\Conid{V}\;(\Varid{v:}\;\Conid{Vars}\;\Conid{V})\;\Varid{x:}\;{}\<[E]%
\\
\>[B]{}\hsindent{3}{}\<[3]%
\>[3]{}\Conid{Quote}\;\Varid{v}\;\Varid{x}\;(\Varid{singlevar}\;\Varid{x})\;\mid \;\Varid{9}\;\Conid{:=}\;\hspace*{0mm}\{\Varid{quote}\;\Conid{:=}\;\Conid{Var}\;(\Varid{inr}\;\Varid{tt})\mskip1.5mu\}\;.{}\<[E]%
\ColumnHook
\end{hscode}\resethooks

\noindent As announced this provides a light-weight quoting function\footnote{\ensuremath{\Varid{eval\char95 quote'}} is a
modification of \ensuremath{\Varid{eval\char95 quote}} which works in the empty context.}:
\begin{hscode}\SaveRestoreHook
\column{B}{@{}>{\hspre}l<{\hspost}@{}}%
\column{3}{@{}>{\hspre}l<{\hspost}@{}}%
\column{E}{@{}>{\hspre}l<{\hspost}@{}}%
\>[B]{}\Conid{Goal}\;\forall\;\Varid{x}\;\Varid{y}\;(\Conid{P:}\;\Conid{Value}\;\to \;\Conid{Prop}),\Conid{P}\;((\Varid{x}\;\Varid{*}\;\Varid{y})\;\Varid{*}\;(\Varid{x}\;\Varid{*}\;\Varid{0}))\;.{}\<[E]%
\\
\>[B]{}\hsindent{3}{}\<[3]%
\>[3]{}\Varid{intros.}\;\Varid{rewrite}\;\leftarrow \;(\Varid{eval\char95 quote'}\;\anonymous )\;.{}\<[E]%
\ColumnHook
\end{hscode}\resethooks
This turns the goal into \ensuremath{(\Varid{eval}\;\Varid{variable\char95 pack}\;\Varid{quote})}.

It would be interesting to use this technique to implement the \ensuremath{\Varid{congruence}} tactic, which implements
the congruence closure algorithm~\cite{corbineau2007deciding}. One would naturally obtain a tactic
which moreover works on setoid equalities.

\section{Canonical structures}\label{canonical}
Packed classes use canonical structures for the algebraic hierarchy. Both canonical structures and
type classes may be seen as instances of hints in unification~\cite{Hints}. Some uses of canonical
structures can be replaced by type class instances. The user manual (2.7.15) uses canonical
structures to derive the setoid equality on the natural
numbers in the following example \ensuremath{(\Conid{S}\;\Varid{x})\;= \;(\Conid{S}\;\Varid{y})}. We have conveniently reimplemented this with
type classes with superior proof terms. Canonical structures give
\begin{hscode}\SaveRestoreHook
\column{B}{@{}>{\hspre}l<{\hspost}@{}}%
\column{E}{@{}>{\hspre}l<{\hspost}@{}}%
\>[B]{}\Varid{@equiv}\;(\Conid{Build\char95 Setoid}\;\mathbb{N}\;(\Varid{@eq}\;\mathbb{N})\;(\Varid{@eq\char95 equivalence}\;\mathbb{N}))\;(\Conid{S}\;\Varid{x})\;(\Conid{S}\;\Varid{y}){}\<[E]%
\ColumnHook
\end{hscode}\resethooks

which includes an explicit proof that \ensuremath{(\Varid{@eq}\;\mathbb{N})} is an equivalence,
whereas we obtain \ensuremath{\Varid{@equiv}\;\mathbb{N}\;(\Varid{@eq}\;\mathbb{N})\;(\Conid{S}\;\Varid{x})\;(\Conid{S}\;\Varid{y})}.

\subsection{Big operators}
Canonical structures have been used to provide a uniform treatment~\cite{bertot2008canonical} of big
operators (like $\Pi,\sum, \max$). These operators extend a pair of a binary and a 0-ary operation
to an $n$-ary operation for any $n$. Categorically, one considers the algebra maps from non-empty
lists, lists, inhabited finite sets and finite sets to the carrier of a semigroup, monoid,
commutative semigroup, commutative monoid. Hence we reuse the libraries for lists etc.

We use type classes to deduce the relevant monoid operation and write:
\begin{hscode}\SaveRestoreHook
\column{B}{@{}>{\hspre}l<{\hspost}@{}}%
\column{3}{@{}>{\hspre}l<{\hspost}@{}}%
\column{E}{@{}>{\hspre}l<{\hspost}@{}}%
\>[B]{}\mathbf{Definition}\;\Varid{seq\char95 sum}{}\<[E]%
\\
\>[B]{}\hsindent{3}{}\<[3]%
\>[3]{}\hspace*{0mm}`\hspace*{0mm}\{\Conid{Sequence}\;\Conid{A}\;\Conid{T}\mskip1.5mu\}\hspace*{0mm}`\hspace*{0mm}\{\Conid{RingPlus}\;\Conid{A}\mskip1.5mu\}\hspace*{0mm}`\hspace*{0mm}\{\Varid{z:}\;\Conid{RingZero}\;\Conid{A}\mskip1.5mu\}\;\mathbin{:}\;\Conid{T}\;\to \;\Conid{A}{}\<[E]%
\\
\>[B]{}\hsindent{3}{}\<[3]%
\>[3]{}\Conid{:=}\;\Varid{@seq\char95 to\char95 monoid}\;\Conid{A}\;\Conid{T}\;\anonymous \;\Conid{A}\;\Varid{ring\char95 plus}\;\Varid{z}\;\Varid{id.}{}\<[E]%
\\
\>[B]{}\hsindent{3}{}\<[3]%
\>[3]{}\mathbf{Eval}\;\Varid{compute}\;\mathbf{in}\;\Varid{seq\char95 sum}\;[\mskip1.5mu \Varid{3};\Varid{2}\mskip1.5mu]\;.{}\<[E]%
\ColumnHook
\end{hscode}\resethooks

\section{Constructive mathematics and proof irrelevance}\label{explicit}
As originally concieved~\cite{ITT,CMCP}, Propositions are Types and we can extract information from
them. In practice, it is convenient to assume that all proofs of a proposition are equal, the
principle of proof irrelevance. These two views clearly contradict eachother. In Coq, a middle way
was choosen, we are forbidden to extract information from proofs, but proofs are not identified.
Here we need to pay, but do not seem to get anything in return.%
\footnote{To be precise, what we get in return is the possibility to extract precisely the program
we want.}
There are various proposals to add the principle of proof irrelevance to Coq, perhaps a
promissing one is the Calculus of implicit
constructions~\cite{miquel2001implicit,barras2008implicit}. Such a change would fit naturally with
the Hoare-style separation between programs and proofs which is used by the \ensuremath{\mathbf{Program}} machinery.

On the other hand, we found instances where we needed proof \emph{relevance}. In this case, it lead
to a duplication of code which may have been avoided by universe polymorphism for \ensuremath{\mathbf{Fixpoint}}s. In
practice, manual dead-code analysis seems useful. Making Harrop formulas proof irrelevant seems to
be a good first approximation~\cite{lcf:spi:03}.

\section{Conclusions}
Telescopes have been criticized~\cite{Packed} for the lack of multiple inheritance and
the efficiency penalty of a long chain of coercion projections. Packed classes~\cite{Packed} provide
a solution to these problems. We provide an alternative solution. We have already
discussed multiple inheritance in Section~\ref{hierarchy}. Moreover, by exposing the
carrier we avoid the chains of projections.

We conclude that unification hints, canonical structures and type classes all appear to have
similar expressive powers. Canonical structures, being tied to the unifier, are more robust, but
seems to require more inguinity. Type classes are easier to use. Unification hints, may be a
good generalization, but are absent from Coq. We encourage their inclusion to the
Coq system.

An obvious topic for future research is the extension from equational logic to partial Horn
logic~\cite{palmgren2007partial}. Another topic would be to fully, but practically, embrase the
categorical approach to universal algebra~\cite{pitts2001categorical}.

According to \ensuremath{\Varid{coqwc}}, our development consists of 3103 lines of specifications and 707 lines of
proofs.\marginpar{update,targz}

% Alt-Ergo congruence closure parametrized by an equational theory.
% As usual either implement in Coq (verify the algorithm), or Check traces, or Check Alt-Ergo (Work
% in progress).

% http://lara.epfl.ch/dokuwiki/sav09:congruence_closure_algorithm_and_correctness
% I understand that this is the algorithm underlying congruence and perhaps first-order.
% 
% Type classes vs soft typing (Mizar types).
% Type classes and types in homalg?

\paragraph{Acknowledgements.}
This research was stimulated by the new possibilities provided by the introduction of type classes
in Coq. In fact, this seems to be the first substantial development that tries to exploid all their
possibilities. As a consequence, we found many small bugs and unintended behavior in the type
class implementation. All these issues were quickly solved by Matthieu Sozeau. Discussions with
Georges Gonthier and Claudio Sacerdoti Coen have helped to
sharpen our understanding of the relation with Canonical Structures and with Unification Hints.
%Jeremy, Thierry, Claudio, Georges?, Wouter
%\enlargethispage{5mm}
\bibliographystyle{plain}
\bibliography{alg}
\end{document}
