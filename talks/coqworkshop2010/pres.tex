\documentclass{beamer}


\mode<presentation>
{
  %\usetheme{Darmstadt}
  %\usetheme{Berlin}
  %\usetheme{default}
  \usetheme{Dresden}
    % Not too large, doesn't continuously list all the sections.
  \setbeamercovered{transparent}
}

\setbeamertemplate{navigation symbols}{}

\usepackage[english]{babel}
\usepackage[latin1]{inputenc}
\usepackage{times}
\usepackage{verbatim}
\usepackage[T1]{fontenc}
% \usepackage[usenames]{color}
\usepackage{marvosym}
\usepackage{amsmath}
\usepackage{amssymb}
\usepackage{upgreek}
\usepackage[mathletters]{ucs}
\usepackage[utf8x]{inputenx}

\definecolor{dkblue}{rgb}{0,0.1,0.5}
\definecolor{lightblue}{rgb}{0,0.5,0.5}
\definecolor{dkgreen}{rgb}{0,0.4,0}
\definecolor{dk2green}{rgb}{0.4,0,0}
\definecolor{dkviolet}{rgb}{0.6,0,0.8}

\usepackage{listings}
  % The package listingsutf8 supports utf8 for \lstinputlisting
  % However, we do not want to put all our code snippets in separate files.

\def\lstlanguagefiles{defManSSR.tex}
\lstset{language=SSR}
\lstset{literate=
  % symbols that actually occur as unicode in our source:
  {λ}{{$\uplambda\ $}}1
  {σ}{{$\upsigma$}}1
  {η}{{$\upeta$}}1
  {φ}{{$\upphi$}}1
  {∃}{{$\exists$}}1
  {→}{{$\to\ $}}1
  {≠}{{$\ne\ $}}1
  {¬}{{$\neg\ $}}1
  {⟶}{{$\longrightarrow\ $}}1
  {⇛}{{$\Longrightarrow\ $}}1
  {∧}{{$\land$}}1
  {∀}{{$\forall$}}1
  {Π}{{$\Uppi$ }}1
  {η}{{$\upeta$}}1
  {⊓}{{$\sqcap$}}1
  {∘}{{$\circ\ $}}1
  {◎}{{$\odot$}}1
  { ≡ }{{$\equiv\ $}}1
  % things we can't make pretty in the actual source, but can make pretty here!:
  {=>}{{$\,\Rightarrow\ $}}1
  {==>}{{$\Rightarrow\ $}}1
  {<-}{{$\leftarrow\ $}}1
  {~}{{$\sim$}}1
  {->}{{$\to$ }}1
  %{bumpeq}{{$\Bumpeq$}}1
}



\title{Type Classes for Mathematics}

\author[Eelis van der Weegen et al.]{
Eelis van der Weegen \and \\
{\small J.w.w. Bas Spitters}}

\institute{Radboud University Nijmegen\\Formath EU project}
 
\date{Coq Workshop 2010}

\begin{document}

\begin{frame}\titlepage\end{frame}
% todo: customary to list the other authors here, too?

\begin{frame}
\frametitle{Interfaces for mathematical structures:}

% so, if you want to do anything interesting in Coq, at some point you will very likely 

\begin{itemize}
\item Algebraic hierarchy (groups, rings, fields, ...)
\item Relations, orders, ...
\item Categories, functors, ...
\item Algebras over equational theories
\item Numbers ($\mathbb{N}$, $\mathbb{Z}$, $\mathbb{Q}$, $\mathbb{R}$, ...)
  % for some reason often don't do is building programs and theory on abstract interfaces for numbers. we are however very much interested in this because corn blabla swapping implementations, efficiency.
\end{itemize}
  % so, it's very important to have solid formalizations of these interfaces.
Need solid representations of these.
  % and preferably one day we'll even agree on what these should look like in Coq, but let's not get ahead of ourselves. ;-)
\end{frame}

% now, if you want to formalize these in a practical system like Coq, you have figure out a nice way to represent them that is convenient to work and hopefully still mathematically sound. and so this sort of becomes a software engineering challenge, and we can point out some of the typical things that solutions need to address:
\begin{frame}
\frametitle{Representing interfaces in Coq}

Engineering challenges:
\begin{itemize}
\item Structure inference
\item Multiple inheritance/sharing
\item Convenient algebraic manipulation (e.g. rewriting)
\item Idiomatic use of notations
\end{itemize}
\end{frame}

\begin{frame}
\frametitle{Solutions in Coq}
Existing solutions:
\begin{itemize}
\item Dependent records
\item Packed classes (Ssreflect)
\item Modules
\end{itemize}
\ \\
\ \\
% all of these have their strengths and weaknesses and make different use of the features that Coq provides.
\pause
% now, a while ago Coq got a /new/ feature, namely type classes, and we figured:

All of these have problems.

\ \\
New solution: Use type classes!

%Our work is one such solution.
% and so we started developing one, and we think it's turning out rather well, so i'll show you what it looks like.
\end{frame}

\begin{frame}
% now, i don't know how many of you have actually already started using type classes, so i'll summarize the key properties of the Coq implementation of type classes in one slide
\frametitle{Type classes}

Implementation in Coq is \emph{first class}:
\begin{itemize}
\item \emph{Classes}: records (``dictionaries'')
\item Class \emph{instances}: terms (of these record types)
\item ... registered as hints for instance resolution.
\item Class \emph{constraints}: implicit parameters
\item ... resolved during unification using instance hints.
\end{itemize}
% so it's a relatively light-weight tightly integrated implementation (very much taking advantage of the expressive power of dependent types), and in this sense it's really unlike the implementation of type classes in say Haskell or Isabelle, where type classes are really a completely distinct feature.

% ok, so that's type classes in a nutshell. i'll now show you how we actually use them.
\end{frame}

% core principle in our approach is to keep operations, relations, carriers _unbundled_. and i'll first very briefly motivate this approach and then show you what it looks like in terms of type classes.

\begin{frame}
\frametitle{Bundling}

Core principle in our approach: \\
\begin{itemize}
\item[] Represent algebraic structures as predicates,
\item[] ... over fully \emph{unbundled} components.
\end{itemize}
  % and by components i mean all the non-propositional stuff (i.e. carriers, relations, operations)
  % now, to illustrate how and why we do this, let's talk a bit about bundling.
\end{frame}

\begin{frame}[fragile]
% \frametitle{Bundling: The trade-off}
% so here's what a unbundled predicate thing
% i'll use the nice and simplistic example of a reflexive relation.

Fully {\color{blue}unbundled}:
\begin{lstlisting}
  Definition reflexive {A: Type} (R: relation A): Prop
    := Π a, R a a.
\end{lstlisting}

\begin{itemize}
\item Very flexible \emph{in theory}
  % but multiple inheritance and other kinds of sharing is trivial!
\item Inconvenient \emph{in practice} (without type classes!):
  \begin{itemize}
    \item Nothing to bind notations to % because no canonical names
    \item Declaring/passing inconvenient % enumerate them all all the time
    \item No structure inference % there's nothing to hold on to
  \end{itemize}
\pause
\item Hence: existing solutions choose to bundle. % which is entirely understable
\end{itemize}
% now our claim is that these problems can be very effectively solved by using type classes, but first let's talk about why this bundling is actually bad.
\end{frame}

\begin{frame}[fragile]
\frametitle{Bundling is bad}

Fully {\color{blue}unbundled} (the other end of the spectrum):
\begin{lstlisting}
  Record ReflexiveRelation: Type :=
    { A: Type; $\hspace{3mm}$R: relation A; $\hspace{3mm}$proof: Π a, R a a }.
\end{lstlisting}
Addresses \emph{some} of the problems:
\begin{itemize}
\item Structure inference % because you can make these fields canonical
\item Notations % because we now have canonical names. theoretically you could now bind a notation to this reflexive relation field R
\item Declaring/passing
\end{itemize}
\pause
But also introduces new ones:
\begin{itemize}
\item Prevents sharing
\item Multiple inheritance (diamond problem)
\item Long projection paths
\end{itemize}
  % these are very real problems. nevertheless, some form of bundling has been a perfectly reasonable approach because the unbundled approach was basically unworkable. however, now we get to the part where we argue that with type classes, the unbundled predicate approach can actually overcome the problems that were traditionally associated with it.
\end{frame}

\begin{frame}[fragile]
\frametitle{Solving problems with type classes}

Slightly more interesting example:
\begin{lstlisting}
  Record SemiGroup
      (G: Type) (e: relation G) (op: G -> G -> G): Prop :=
    { sg_setoid: Equivalence e
    ; sg_ass: Associative op
    ; sg_proper: Proper (e ==> e ==> e) op }.
\end{lstlisting}

\pause

Modifications we make:
\begin{enumerate}
  % first, we make it a type class, and we'll refer to such type classes as "predicate classes".
\item Make it a type class (``predicate class'')
\item Use \emph{operational type classes} for \lstinline|e| and \lstinline|op|
\end{enumerate}

\end{frame}

\begin{frame}[fragile]
\frametitle{Solving problems with type classes (cont'd)}
Revised \lstinline|SemiGroup|:
\begin{lstlisting}
  Class Equiv (A: Type) := equiv: relation A.
  Class SemiGroupOp (A: Type) := sg_op: A $\to$ A $\to$ A.

  Infix "=" := equiv.
  Infix "&" := sg_op.

  Class SemiGroup
      (G: Type) {e: Equiv G} {op: SemiGroupOp G}: Prop :=
    { sg_setoid:> Equivalence e
    ; sg_ass:> Associative op
    ; sg_proper:> Proper (e ==> e ==> e) op }.
\end{lstlisting}
% note, might look worrying, but Equiv and SemiGroup op special treatment

\end{frame}

\begin{frame}[fragile]
\frametitle{More syntax}
Theorem syntax:

\begin{lstlisting}
  Lemma bla `{SemiGroup G}:
    Π x y z: G, x & (y & z) = (x & y) & z.
\end{lstlisting}

Usage syntax:
\begin{lstlisting}
  apply bla
  rewrite bla
\end{lstlisting}

Instance syntax:
\begin{lstlisting}
  Instance: Equiv nat := @eq nat.
  Instance: SemiGroupOp nat := plus.
  Instance: SemiGroup nat.
  Proof. ... Qed.
\end{lstlisting}

\end{frame}


% - what we have developed in this style:

\begin{frame}[fragile]
\frametitle{Algebraic hierarchy}

\begin{columns}
  \column{.35\textwidth}
    \includegraphics[scale=0.8]{hierarchy.pdf}
  \column{.65\textwidth}
Features:
\begin{itemize}
\item No distinction between axiomatic and derived inheritance.
  % point to Ring→SemiRing derivation, which is just a separate instance declaration:
  % Instance Instance ring_as_semiring `{Ring R}: SemiRing R.
\item No sharing/multiple inheritance problems.
\item No rebundling.
\item No projection paths (hence, no ambiguous projection paths).
\item Instances opaque. % because it's just proofs
\item Terms never refer to proofs. % structural information just floats around in the context without terms referring to them.
\item Overlapping instances harmless. % happens in SemiRing's multiple inheritance, for example. that will give two Setoid instances for the equivalence relation.
\item Seamless setoid/rewriting support.
\end{itemize}
% full support for setoids and setoid rewriting; seemless integration thanks to the fact that equivalence relations and morphisms can now be registered using type class instances. only worth mentioning to contrast with the ssreflect hierarchy which does /not/ have built-in setoid support. we also don't restrict ourselves to decidable anything, but you're free to posit decidability of things, and by the way doing that with an operational type class is very neat.
\end{columns}
\end{frame}

% hm, if time left (unlikely), could show how SemiRing does multiple inheritance with named arguments.

\begin{frame}
% having shown how we use type classes and what advantages they generally provide, i'll continue with a very short summary of additional interfaces we have.

% these were really primarily motivated because we wanted to make interfaces for numbers that were both mathematically sound

\frametitle{Toward numerical interfaces}
Goal: Build theory/programs on \emph{abstract} numerical interfaces instead of concrete implementations.
\begin{itemize}
\item Cleaner
\item Mathematically sound
\item Can swap implementations
\end{itemize}

\ \\
Example:
\begin{itemize}
  \item[]Characterize $\mathbb{N}$ as \emph{initial semiring}. % in cat. of semirings.
\end{itemize}

\ \\
Need a bit of category theory.
\end{frame}

\begin{frame}[fragile]
\frametitle{Category theory}

Again, begin with operational type classes:
\begin{lstlisting}
  Class Arrows (O: Type): Type :=
    Arrow: O $\to$ O $\to$ Type.
  Class CatId O `{Arrows O}: Type :=
    cat_id: `(x ⟶ x).
  Class CatComp O `{Arrows O}: Type :=
    comp: Π {x y z}, (y ⟶ z) $\to$ (x ⟶ y) $\to$ (x ⟶ z).
\end{lstlisting}
... with notations bound to them:
\begin{lstlisting}
  Infix "⟶$\hspace{-1mm}$" := Arrow.
  Infix "◎" := comp.
\end{lstlisting}
\end{frame}

\begin{frame}[fragile]
\frametitle{Category theory (cont'd)}
\begin{lstlisting}
  Class Category (O: Type)
      `{Arrows O} `{Π $\hspace{-1mm}$x y: O, Equiv (x ⟶ y)}
      `{CatId O} `{CatComp O}: Prop :=
    { arrow_equiv:> Π x y, Setoid (x ⟶ y)
    ; comp_proper:> Π x y z,
        Proper (equiv ==> equiv ==> equiv) comp
    ; comp_assoc w x y z (a: w ⟶ x) (b: x ⟶ y) (c: y ⟶ z):
        c ◎$$ (b ◎$$ a) = (c ◎$$ b) ◎$$ a
    ; id_l `(a: x ⟶ y): cat_id ◎$$ a = a
    ; id_r `(a: x ⟶ y): a ◎$$ cat_id = a }.
\end{lstlisting}
% again, the emphasis here is not on "look, we can prove super hard category theory theorems", but on "look, this is a neat interface for categories"

% now, we also have classes for all the other stuff like functors, natural transformations, adjunctions, but i'm not going to show those. the point is that this is enough to define initality (which i'm also not going to show, but again it's split up into an operational and a predicate class)

\end{frame}

\begin{frame}
\frametitle{Next up: Building categories.}
% now to talk about initial semirings, we obvious need to have the category of semirings.

Could define category of semirings (etc) manually...

\ \\
Nicer: \emph{generate} category of equational theory of semirings.

\ \\
Need a bit of universal algebra.
\end{frame}

\begin{frame}
\frametitle{Universal algebra}
We formalize:
\begin{itemize}
\item multisorted universal algebra
\item equational theories
\item categories of algebras, equational theories
\item forgetful functors
\item open/closed term algebras
\item generic construction of initial objects
\item subalgebras/varieties, quotients
\item theory transference between isomorphic models
\end{itemize}

\ \\
All of it using type classes for optimum effect.
\end{frame}

\begin{frame}[fragile]
\frametitle{Universal algebra (cont'd)}
% now, i'm not going to show you all the code, i'll just give one example of how the operational type class and predicate class thing is again quite natural:
Operational type class:
\begin{lstlisting}
  Variables (σ: Signature) (carriers: sorts σ $\to$ Type).
  Class AlgebraOps: Type :=
    algebra_op: Π o: operation σ, op_type carriers (σ o).
\end{lstlisting}
Predicate class:
\begin{lstlisting}
  Class Algebra
    `{Π a, Equiv (carriers a)} `{AlgebraOps σ$$ carriers}: Prop :=
    { algebra_setoids:> Π a, Setoid (carriers a)
    ; algebra_propers:> Π o: σ, Proper (=) (algebra_op o) }.
\end{lstlisting}
\end{frame}


\begin{frame}[fragile]
\frametitle{Numerical interfaces}
% so, if we construct our semiring category using the univ. alg tools, then we could

Minimalistic interface for $\mathbb{N}$:
\begin{lstlisting}
  Class Naturals (A: ObjectInVariety semiring_theory)
      `{InitialArrows A}: Prop :=
    { naturals_initial:> Initial A }.
\end{lstlisting}
  % where ObjectInVariety constructs a category for an equational theory by bundling carriers with their operations and proofs (here we really cannot avoid bundling because the category needs a self-contained type for objects), and where InitialArrows is an operational type class for the homomorphisms to other semirings that you get from initiality.
  % Rather "high in the sky"; not easy to work with.
 % no link to the SemiRing type class here. and these initial arrows aren't ordinary functions.

More convenient: % and this is the one we actually use.
\begin{lstlisting}
  Class Naturals
      `{SemiRing A} `{NaturalsToSemiRing A}: Prop :=
    { naturals_ring:> SemiRing A
    ; naturals_to_semiring_mor:> Π `{SemiRing B},
        SemiRing_Morphism (naturals_to_semiring A B)
    ; naturals_initial:> Initial (bundle_semiring A) }.
\end{lstlisting}
% now, there is more to say about the advantages that we get from using universal algebra and category theory. for instance, it's trivial to show that initial objects are isomorphic, so any two implementations of these naturals are isomorphic. and because we use a generic universal algebra constructed category we can also very easily transfer statements in the language of semirings between such models, or lift theorems to the abstract interface, so that we immediately get most of the theory already proven for nat. so we don't have to start from scratch.
\end{frame}

\begin{frame}[fragile]
\frametitle{Specialization} % should only take a minute

Suppose you want to calculate things:

\begin{lstlisting}
  Definition calc `{Naturals N} (n m: N) := ... decide (n = m) ...
\end{lstlisting}
Generic instance:
\begin{lstlisting}
  Instance: Π `{Naturals N} (n m: N): Decision (n = m) | 9 := ...
\end{lstlisting}
Works, but inefficient.

\ \\
\pause
Specialized instance for \lstinline|nat|:
\begin{lstlisting}
  Instance: Π n m: nat, Decision (n = m).
\end{lstlisting}
Extra parameterization:
\begin{lstlisting}
  Definition calc `{Naturals N} `{Π$$n m: N, Decision (n = m)}
    (a b: nat) := ... decide (a = b) ... .
\end{lstlisting}
  % also appropriate for distance, minus, division/multiplication by 2
\end{frame}

\begin{frame}
\frametitle{Quoting}

Quoting:
\begin{itemize}
\item Find syntactic representation of semantic expression
\item Required for proof by reflection (ring, omega)
\end{itemize}

\ \\
Usually implemented at meta-level (Ltac, ML).

\ \\
Alternative: object level quoting.
\begin{itemize}
\item Unification hints (Matita)
\item Canonical structures (Ssreflect)
\end{itemize}

\end{frame}
\begin{frame}
\frametitle{Quoting (cont'd)}
Our implementation: type classes!

\ \\
Instance resolution
\begin{itemize}
\item Syntax-directed
\item Prolog-style resolution
\item Unification-based programming language
\end{itemize}

\ \\
Implementation in terms of type classes:
\begin{itemize}
\item Straightforward % unlike Ssreflect
\item Plan: integrate with universal algebra term types
\end{itemize}
\end{frame}

\begin{frame}
\frametitle{Conclusions}

Predicate type classes for mathematics:
\begin{itemize}
\item Works well in practice
\item Match mathematical practice
\item Compatible with efficient computation
\item Plan: use as basis for computational analysis (Formath)
\end{itemize}

\ \\
Pending issues: % prevent full endorsement
\begin{itemize}
\item instance resolution efficiency
\item universe polymorphism
\item ``infer if possible, generalize otherwise''
\end{itemize}

% Very promising.
% 
% We feel that type classes match mathematical practice.
% 
% Especially predicate classes very natural and general database of known information. And because of blabla, overlap and wild instance search no problem.
% 

\end{frame}

\setbeamercolor{background canvas}{bg=black}
\begin{frame}
{
\color{white}
Sources/papers: \\
\hspace{1cm} Google keywords: \texttt{coq math classes}
}
\end{frame}

\end{document}

