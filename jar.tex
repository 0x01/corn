\documentclass[a4paper,10pt,runningheads]{llncs}
\usepackage{url}
\usepackage{bbm}
\usepackage{amsmath}
\usepackage{upgreek}
\usepackage[mathletters]{ucs}
\usepackage[utf8x]{inputenx}
\usepackage{comment}
\newcommand{\N}{\ensuremath{\mathbbm{N}}}
\newcommand{\Z}{\ensuremath{\mathbbm{Z}}}
\newcommand{\Q}{\ensuremath{\mathbbm{Q}}}
\usepackage{color}

\usepackage{graphicx}


\definecolor{dkblue}{rgb}{0,0.1,0.5}
\definecolor{lightblue}{rgb}{0,0.5,0.5}
\definecolor{dkgreen}{rgb}{0,0.4,0}
\definecolor{dk2green}{rgb}{0.4,0,0}
\definecolor{dkviolet}{rgb}{0.6,0,0.8}

% preliminary choices:
%   don't discuss difference between maximally-inserted and non-maximally-inserted implicit arguments.

% todo:
  % what to say about terms referring on proofs?
  % cite that other type class based development in the context of the alternative bundling schemes.
  % maybe have a chapter called "interfaces are primary, not implementations" (also involve polynomials)
  % say this somewhere: \footnote{For an interesting (if fairly restrictive) alternative design in which unicity of term representations \emph{is} required for all objects, see the algebraic hierarchy in the ssreflect\cite{ssreflect} standard library.}

% listings:

\usepackage{listings}
  % The package listingsutf8 supports utf8 for \lstinputlisting
  % However, we do not want to put all our code snippets in separate files.

\def\lstlanguagefiles{defManSSR.tex}
\lstset{language=SSR}
\lstset{literate=
  % symbols that actually occur as unicode in our source:
  {λ}{{$\uplambda\ $}}1
  {∃}{{$\exists$}}1
  {→}{{$\to\ $}}1
  {≠}{{$\ne\ $}}1
  {¬}{{$\neg\ $}}1
  {⟶}{{$\longrightarrow\ $}}1
  {∧}{{$\land$}}1
  {∀}{{$\forall$}}1
  {Π}{{$\Uppi\ $}}1
  {η}{{$\upeta$}}1
  {⊓}{{$\sqcap$}}1
  {∘}{{$\circ\ $}}1
  {◎}{{$\odot\ $}}1
  { ≡ }{{$\equiv\ $}}1
  % things we can't make pretty in the actual source, but can make pretty here!:
  {=>}{{$\,\Rightarrow\ $}}1
  {==>}{{$\Rightarrow\ $}}1
  {<-}{{$\leftarrow\ $}}1
}


\begin{document}
\title{The algebraic hierarchy in type theory using type classes}
\author{Bas Spitters \and Eelis van der Weegen}
\institute{Radboud University Nijmegen}
\date{today}
\maketitle
\begin{abstract}
We present a new formalization of the algebraic hierarchy in Coq, exploiting its new type class mechanism to make practical a solution formerly thought infeasible. Our approach addresses both traditional challenges as well as new ones resulting from our ambition to build upon this development a library of constructive analysis in which abstraction penalties inhibiting efficient computation are reduced to a minimum. To support mathematically sound abstract interfaces for $\N$, $\Z$, and $\Q$, our formalization includes portions of category theory and multisorted universal algebra.
Algebra flourishes by the interplay between syntax and semantics. The Prolog-like
abilities of type class unification allow us to conveniently define a quote function thus facilitating the use of reflective techniques.
  % todo: handle.
\end{abstract}

\section{Introduction}
The development of libraries for formalized mathematics presents many software engineering challenges~\cite{C-corn,DBLP:conf/types/HaftmannW08}, because it is far from obvious how the clean, idealized concepts from everyday mathematics should be represented using the facilities provided by concrete theorem provers and their formalisms, in a way that is both mathematically faithful and convenient to work with.

For the algebraic hierarchy---a critical component in any library of formalized mathematics---these challenges include structure inference, handling of multiple inheritance, idiomatic use of notations, and convenient algebraic manipulation (e.g. rewriting).

Several solutions have been proposed for the Coq theorem prover: dependent records~\cite{DBLP:journals/jsc/GeuversPWZ02} (a.k.a. telescopes), packed classes~\cite{Packed}, and occasionally modules. We present a new approach based entirely on Coq's new type class mechanism, and show how it can be used to make fully ``unbundled'' predicate-representations of algebraic structures practical to work with.

Since we intend to use this development as a basis for constructive analysis with practical certified exact real arithmetic, an additional objective and motivation in our design is to facilitate \emph{efficient} computation. In particular, we want to be able to effortlessly swap implementations of number representations. Doing this requires that we have clean abstract interfaces, and mathematics tells us what these should look like: we represent $\N$, $\Z$, and $\Q$ as \emph{interfaces} specifying an initial semiring, an initial ring, and a field of integral fractions, respectively.

To express these elegantly and without duplication, our development\footnote{The sources are available
at:~\url{http://www.eelis.net/research/math-classes/}} includes an integrated formalization of parts of category theory and multi-sorted universal algebra, all expressed using type classes for optimum effect.

% ideally should be more or less readable by Haskell person. that would be neat. so that's what the preliminaries should prepare for.

\paragraph{Outline}
In section \ref{preliminaries} we briefly describe the Coq system and some of its facilities that are of particular interest to us (most notably its implementation of type classes). Then, in section \ref{bundling}, we give a very concrete introduction to the issue of \emph{bundling}, arguably the biggest design dimension when building 
interfaces for abstract structures. In section \ref{predicateclasses} we show how type classes can make the use of ``unbundled'' purely predicate based interfaces for abstract structures practical. Next, in section \ref{hierarchy}, we discuss our algebraic hierarchy implemented using such predicate classes. 


 %Having derived some principles concerning bundling, we then proceed to show (in section \ref{managing}) how type classes may be employed to adopt these principles without compromising ease of use. Next, in section \ref{hierarchy}, we introduce a number of additional conventions that let us complete a consistent idiom for interfaces to algebraic structures. With this in place, we continue with a discussion of

%to the 



% outline:
% 
% - preliminaries
%     -> coq (couple of sentences about programming in a combined framework with logic)
%     -> type classes
% - on bundling
%     -> introduce problem
%     -> naive solution
%     -> etc
%     -> "we first begin by showing how unbundling is in some sense the "right" way to do things"
%     in doing this, however, we will find that doing things this way is somewhat painful
%     -> derived principles
% - Managing unbundled structure with type classes
%     -> painful if flexible, which is why not done before
%     -> with type classes, managable.
% 
% - toward the algebraic hierarchy
% 
%   leaving things unbundled is one thing, but we need more for the alg. hierarchy. suppose we define a class for semigroups based on yada.
%   -> operational type classes (parameterized over a single operation = constant), structural type classes (parameterized over all the components that make up the structure).
% 
% - advantages
% 
% - numeric classes
% 
% - cat.theory
% 
% - univ. alg.
% 
% - quoting

%todo: have a word somewhere about "super classes"

In this paper we focus on the Coq proof assistant. We conjecture that the methods can be transferred
to any type theory based proof assistant supporting type classes such as
Matita~\cite{asperti2007user}.



\section{Preliminaries}
\label{preliminaries}

% ideally, this should make the paper readable for Haskell people

The Coq proof assistant is based on the calculus of inductive
constructions~\cite{CoquandHuet,CoquandPaulin}, a dependent type theory with (co)inductive types; see~\cite{Coq,BC04}. In true Curry-Howard fashion, it can be thought of either as an excessively pure (if somewhat pedantic) functional programming language with an extremely expressive type system, or as a language for mathematical statements and proofs. We highlight some aspect of Coq that are of particular relevance to our development.

\paragraph{Types and propositions}

Propositions in Coq are types, which themselves have types called \emph{sorts}. Coq features a distinguished sort called \lstinline|Prop| that one may voluntarily choose to use as the sort of types representing propositions. The distinguishing feature of the \lstinline|Prop| sort is that Coq will not permit terms of non-\lstinline|Prop| types to depend on the values of inhabitants of \lstinline|Prop| types (i.e. proof terms).

Adopting this regime of discrimination establishes a weak form of proof irrelevance, in that changing a proof can never affect the result of value computations. On a very practical level, this lets Coq safely erase all \lstinline|Prop| components when extracting certified programs to OCaml or Haskell.

A stronger notion of proof irrelevance in which any two inhabitants of a \lstinline|Prop| type are made \emph{convertible} is planned for a future Coq version.


\paragraph{Equality, setoids, and rewriting}

The ``native'' notion of equality in Coq is that of terms being convertible, which in this context constitutes Leibniz equality, naturally propositionalized by an inductive type family \lstinline|eq| with a single constructor \lstinline|eq_refl| of type \lstinline|Π (T: Type) (x: T), x ≡ x|, where ``\lstinline|a ≡ b|'' is notation for \lstinline|eq T a b|~\footnote{We diverge from Coq traditional and reserve ``\lstinline|=|'' for setoid equality, as this is the equality we will be working with most of the time.}. Importantly, since convertibility is a congruence, a proof of \lstinline|a ≡ b| lets us substitute \lstinline|b| for \lstinline|a| anywhere inside a term without further conditions.

We mention this explicitly only because rewriting \emph{does} give rise to conditions when we depart from Leibniz equality and introduce equivalence relations representing abstractions over common representations of the same conceptual object (for instance when identifying formal fractions of integers representing the same ratio). Rewriting a subterm using such an equality is permitted only if the subterm is an argument of a function that has been proven to \emph{respect} the equality. Such a function is called \emph{proper}, and propriety must be proved for each function in whose arguments we wish to enable rewriting.

Effectively keeping track of, resolving, and combining proofs of equivalence-ness and propriety when the user attempts to substitute a given (sub)term using a given equality, requires nontrivial infrastructure and support from the system. The Coq implementation of these mechanisms was largely rewritten in 200? by Matthieu Sozeau \ref{rewriting}. (todo: say that it's now much better than before)

\paragraph{Type classes}

Type classes have been a great success story in the Haskell functional programming language, as a means for organizing interfaces of abstract structures. The type class implementation that is now part of Coq provides a superset of their functionality, but implemented in a different way.

In Haskell and Isabelle, type classes and their instances are second class, in that they are handled as specialized syntactic constructs whose semantics are given specifically by the type class apparatus. By contrast, the expressivity of dependent types and inductive families as supported in Coq, combined with the use of pre-existing technology in the system (namely proof search and implicit arguments) enable a \emph{first class} type class implementation~\cite{DBLP:conf/tphol/SozeauO08}: classes are ordinary record types (``dictionaries''), instances are ordinary constants of these record types (registered as ``hints'' with the proof search machinery), class constraints are ordinary implicit parameters, and instance resolution is achieved by augmenting the unification algorithm to invoke ordinary proof search for implicit arguments of class type.

Thus, type classes in Coq are realized by relatively minor syntactic aids that bring together existing facilities of the theory and the system into a coherent idiom, rather than by introduction of a new category of qualitatively different definitions with their own dedicated semantics.


% The Coq type theory lacks quotient
% types, as it would make type checking undecidable. % todo: reference
% Instead it supports setoids (aka Bishop sets), a pair of a type and an equivalence relation~\cite{Bishop67,Hofmann,Capretta}. Recently, improved support for setoids has been added to the system~\cite{Setoid-rewrite}. Functions respecting these equivalence relations are called \lstinline|Proper|.
% As we will see in Section~\ref{univ} below this functionality can also be used to work with
% quotient algebras.


% \section{The Type-Classified Algebraic Hierarchy}\label{classes}
% We represent each structure in the algebraic hierarchy as a type class. This immediately leads to the familiar question of which components of the structure should become parameters of the class, and which should become fields. By far the most important design choice in our development is the decision to turn all \emph{structural} components (i.e. carriers, relations, and operations) into parameters, keeping only \emph{properties} as fields. Type classes defined in this way are essentially predicates with automatically resolved proofs.
% 
% Conventional wisdom warns that while this approach is theoretically very flexible, one risks extreme inconvenience both in having to declare and pass around all these structural components all the time, as well as in losing notations (because we no longer project named operations out of records). These legitimate concerns will be addressed in Sections~\ref{bundle}-\ref{classes}. Section~\ref{univ} abstracts to universal
% algebra, we develop the basic theory to the point of the first homomorphism theorem.
% Section~\ref{interfaces} provides interfaces to usual data structures by providing their universal
% properties. To allow symbolic manipulation of semantic objects one needs a quote function. Usually,
% this function is implemented in Coq's tactics language. A more convenient way is to use type class
% unification; see section~\ref{quote}. In section~\ref{canonical}, we end with a short comparison
% between type classes and canonical structures.



% \subsection{Type classes}
% Implementation in Ltac, Notation etc\\
% Coercions\\
% Unification hints

\section{Bundling is bad}\label{bundling}

Algebraic structures are expressed in terms of a number of carriers, a number of operations and relations on these carriers, and a number of laws that the operations and relations satisfy. In a system like Coq, we have different options when it comes to representing the grouping of these components. On one end of the spectrum, we can simply define the (conjunction of) laws as an $n$-ary predicate over $n$ components, forgoing explicit grouping altogether. For instance, for the mundane example of a reflexive relation, we could use:

\begin{lstlisting}
  Definition reflexive {A: Type} (R: relation A): Prop := Π a, R a a.
\end{lstlisting}
(The curly brackets used for \lstinline|A| mark it as an implicit argument.)

Higher structures, too, can all be expressed as predicates (expressing laws) over a number of carriers, relations, and operations. While optimally flexible in principle, in practice \emph{naive} adoption of this approach (i.e. without using type classes) leads to substantial inconveniences in actual use: when \emph{stating} theorems about abstract instances of such structures, one must enumerate all components along with the structure (predicate) of interest. And when \emph{applying} such theorems, one must either enumerate any non-inferrable components, or let the system spawn awkward metavariables to be resolved at a later time. Importantly, this also hinders proof search for proofs of the structure predicates, making any nontrivial use of theorems a laborious experience. Finally, the lack of \emph{canonical names} for particular components of abstract structures makes it impossible to associate idiomatic notations with them.

% hm, i used to have paragraphs here about how it would be unclear how this approach could support structure inference and structure inheritance, but those are actually pretty straightforward using ordinary proof search.

% hm, say something about opacity

In the absence of type classes, these are all very real problems, and for this reason the two biggest formalizations of abstract algebraic structures in Coq today (namely CoRN\ref{corn} and Ssreflect\ref{ssreflect}), both use \emph{bundled} representation schemes, using records with one or more of the components as fields instead of parameters. For reflexive relations, the following is a fully bundled representation---the other end of the spectrum:
\begin{lstlisting}
  Record ReflexiveRelation: Type :=
    { rr_carrier: Type
    ; rr_rel: relation rr_carrier
    ; rr_proof: Π x, rr_rel x x }.
\end{lstlisting}
Superficially, this instantly solves the problems described above: reflexive relations can now be declared and passed as self-contained packages, and the \lstinline|rr_rel| projection now constitutes a ``canonical name'' for relations that are known to be reflexive, and we could bind a notation to it\footnote{While there is no conventional notation for reflexive relations, the situation is the same in the context of, say, a semiring, where we would bind ``$+$'' and ``$*$'' notations to the record field projections for the addition and multiplication operations, respectively.}.

% While not observable in this example, when combined with other specialized facilities in Coq (namely coercions and/or canonical structures), this bundled representation also lets one address the problems of structure inference inheritance to some extent.

Unfortunately, despite its apparent virtues, the bundled representation introduces serious problems of its own, the most immediate and prominent one being a lack of support for \emph{sharing} components between structures, which is needed to cope with overlapping multiple inheritance.

In our example, lack of sharing support rears its head as soon as we try to define \lstinline|EquivalenceRelation| in terms of \lstinline|ReflexiveRelation| and its hypothetical siblings bundling symmetric and transitive relations. There, we would need some way to make sure that when we ``inherit'' \lstinline|ReflexiveRelation|, \lstinline|SymmetricRelation|, and \lstinline|TransitiveRelation| by adding them as fields in our bundled record, they all refer to the same carrier and relation. Adding additional fields stating equalities between the three bundled carriers and relations is neither easily accomplished (because one would need to work with heterogenous equality) nor would it permit natural use of the resulting structure (because one would constantly have to rewrite things back and forth).

Manifest fields~\ref{manifestrecords} have been proposed and implemented in the Matita system to address exactly this problem. We hope to convince the reader that type system extensions of this nature, designed to mitigate particular symptoms of the bundled approach, are less elegant than a solution that avoids the problem altogether by using predicate-like type classes in place of bundled records.

If we revert back to the predicate formulation of reflexive (et. al) relations, we \emph{could} still define \lstinline|EquivalenceRelation| in a bundled fashion without the need for equalities:
\begin{lstlisting}
  Record EquivalenceRelation: Type :=
    { er_carrier: Type
    ; er_rel: relation er_carrier
    ; er_refl: ReflexiveRelation er_carrier er_rel
    ; er_sym: SymmetricRelation er_carrier er_rel
    ; er_trans: TransitiveRelation er_trans er_rel }.
\end{lstlisting}
However, the same argument as before leads to the conclusion that \lstinline|EquivalenceRelation|, too, should be a predicate. Indeed, it would be rather strange for the interface of equivalence relations to differ qualitatively from the interface of reflexive relations.

Another attempt to recover some grouping might be to bundle the carrier with the relation into a (lawless) record, but this too hinders sharing: as soon as we try to define an algebraic structure with two reflexive relations on the same carrier, we need awkward hacks to establish equality between the carrier projections of two different (carrier, relation) bundles.

Even bundling just the operations of an algebraic structure together in a record (with the carrier as a parameter) does not work, as it again leads to the same problem when, for example, one attempts to define a hypothetical algebraic structure with two binary relations and a constant such that both binary relations form a monoid with the constant.

A second problem with bundling is that as the bundled records are stacked to represent higher and higher structures, the projection paths for their components grow longer and longer, resulting in ever more unwieldy terms (coercions and notations can make this less painful). Further, if one tries to implement some semblance of sharing in a bundled representation, these projection paths additionally become non-canonical, and still more extensions have been proposed to address this symptom (e.g. coercion pullbacks~\cite{Hints}).

Thus, bundled representations come at a substantial cost in flexibility. Historically, using bundled representations has nevertheless been an acceptable trade-off, because (1) the unbundled alternative was such a pain, and (2) the standard algebraic hierarchy (up to, say, fields and modules) is not all that wild.

In the next section, we show that type-classification of structure predicates and their component parameters has the potential to remedy the problems associated with the naive unbundled predicate approach, giving us the best of both worlds.

%For completeness, we mention an alternative record-based approach known as ``packed classes'', as used in the ssreflect library. In the packed classes idiom, each algebraic structure is represented by not one but a trinity of interdependent records along with packing/unpacking functions. These are then stacked in an intricate manner designed to reach a particular hybrid packing balance that, combined with the use of canonical structures and coercions, accomplishes some of the goals above.

%We have found it hard to assess the general merits of this approach, partly because of its complexity, and partly because the only actual usage example (the hierarchy in the ssreflect library) is not only wedded to a very specialized view on equality and decidability that does not support setoid relations from the outset, but also diverges significantly from standard mathematical dichotomy, making it hard to recognize familiar structures and presentation. The reasons for this (as we understand them) are pragmatic and twofold.

%First, the ssreflect library was originally developed strictly to support very practical work in a specialized domain of mathematics involving decidable and mostly finite structures. Second, at the time of ssreflect's initial development, support for setoid rewriting in Coq was not nearly as stable and refined as it is today, which made ``Leibniz equality at all costs'' a more reasonable strategy at the time than it might be considered today.

%Work to extend the ssreflect framework with techniques to allow representation of a less restricted class of mathematical structures has only recently begun and is still ongoing, though there are currently no plans to support undecidable structures.

%Because in our development we will be working with computable reals, equality on which can never be either Leibniz or decidable, we conclude that the ssreflect library is not usable for us.


% todo: mention proof irrelevance here somewhere, and terms that depend on proofs.

The observant reader may wonder whether it might be beneficial to go one step further and unbundle proof of laws and inherited substructures as well. This is not the case, because there is no point in sharing them. After all, by (weak) proof irrelevance, the ``value'' of such proofs can be of no consequence anyway. Indeed, parameterizing on proofs would be actively harmful because instantiations differing only in the proof argument would express the same thing yet be nonconvertible, requiring awkward conversions and hindering automation.

% Another benefit of our approach is that it works very smoothly when these records are in fact type
% classes. When relations and operations are bundled and do \emph{not} appear as parameters of the
% structure record/class, to express that some collection of relations and
% operations possess the structure as a type class constraint seems to require manifest fields; see
% Section~\ref{manifest}. With
% our unbundled approach, on the other hand, such type class constraints are easily expressed without
% any extensions as type inhabitation queries
% % (e.g. ``hey Coq, find me a proof of [Monoid some_rel some_op some_unit]'')
% of types living in Prop, so that we don't even care what instances are found (thus embracing proof
% irrelevance).

% Leaving the equivalence proofs bundled means that when one has a proof that a structure
% is a semiring, one has two proofs (via the two monoids) that the equivalence relation is an
% equivalence. Fortunately, this redundancy is entirely harmless since our terms
% never refer (directly or indirectly) to proofs, which is the case when the relations and operations
% they are composed from are left unbundled. Here, too, we enjoy the warm embrace of proof
% irrelevance.
% 
% From the discussion so far, we derive some principles:
% \begin{enumerate}
%  \item structural concepts (like Reflexive, Setoid, Monoid) should have their structural components
% (i.e. carriers, relations, and operations) as parameters, and nothing else;
%  \item (as a consequence) these concepts should be defined as predicates (which includes records and
% classes living in Prop), and their proofs may be kept opaque;
%  \item (as a second consequence) algebraic expressions need/should never refer (directly or
% indirectly) to proofs about the algebraic structure.\footnote{The reciprocal in a field
% famously does refer to a proof that the element is non-zero, but not to a proof that the operations
% indeed form a field.}
% \end{enumerate}
% % \setcounter{enumi}{3}
% 
% The principles listed so far are not particularly controversial, in fact this
% approach ensures maximum flexibility~\cite{Hints}. However,
% there is a common perception that unbundling carriers, relations, and operations this way invariably
% leads either to unmanageably big algebraic expressions (involving endless projections and/or implicit
% arguments), or to unmanageably long and tedious enumeration of carriers/relations/operations, or to
% problems establishing canonical names for operations (since we no longer project them out of
% structure instances all the time).
% 
% In the next sections, we show how systematic use of type classes (and their support infrastructure
% in Coq), combined with the use of a special form of type class we call an `operational type class'
% (as opposed to a `structural' type class we might use in place of the records shown thus far), lets
% us avoid these problems.

\section{Predicate classes}

We will tackle the problems associated with the structure predicates approach one by one, starting with those encountered during theorem \emph{application}. Suppose we have defined \lstinline|SemiGroup| as a structure predicate as follows \footnote{Note that defining \lstinline|SemiGroup| as a record instead of a straight conjunction does not make it any less of a predicate. The record form is simply more convenient in that it immediately gives us named projections for laws and substructures.}:
\begin{lstlisting}
  Record SemiGroup (G: Type) (e: relation G) (op: G → G → G): Prop :=
    { sg_setoid: Equivalence e
    ; sg_ass: Associative op
    ; sg_proper: Proper (e ==> e ==> e) op }.
\end{lstlisting}
Then by (1) making \lstinline|SemiGroup| a \emph{class} (by replacing the \lstinline|Record| keyword with the \lstinline|Class| keyword), (2) marking its proofs as \emph{instances} (again by replacing a single keyword), and (3) marking the \lstinline|SemiGroup| parameter of semigroup theorems as implicit (by using curly instead of round brackets), we no longer have to pass \lstinline|SemiGroup| proofs around manually ourselves, letting instance resolution do it for us instead. Because instance resolution is part of the unifier, this also works when the statement of the theorem we wish to apply only mentions some of the components (which admittedly doesn't make much sense for semigroups).

Next, we turn to problems concerning theorem \emph{declaration}. As a reference point, our ideal will be the common mathematical vernacular, where one simply says:
\begin{quote}
Theorem L: For $x, y, z$ in a semigroup $G$, $x * y * z = z * y * x$.
\end{quote}
Without further support from the system, this would have to be written as
\begin{lstlisting}
  Theorem L G e op {P: SemiGroup G e op}:
    Π x y z, e (op (op x y) z) (op (op z y) x).
\end{lstlisting}
(The curly brackets mark \lstinline|P| as an implicit argument.) Because \lstinline|e| and \lstinline{op} are freshly introduced local names, we cannot bind notations to them prior to this theorem. Hence, if we want notations, what we really need are canonical names for these components. This is easily accomplished with single-field type classes containing one component each, which we will call \emph{operational type classes}\footnote{These single-field type classes are used in the same way in the \lstinline|Clean| standard library \cite{something}.}:
\begin{lstlisting}
  Class Equiv A := equiv: relation A.
  Class SemiGroupOp A := sg_op: A $\to$ A $\to$ A.

  Infix "=" := equiv: type_scope.
  Infix "&" := sg_op (at level 50, left associativity).
\end{lstlisting}
We use \lstinline|&| here and reserve the notation \lstinline|*| for (semi)ring multiplication.

As an aside, we note that the distinction between the class field name and the infix operator notation bound to it is really just a mildly awkward Coq artifact. In Haskell, where operators can themselves be used as names, there would be no need to have the \lstinline|equiv| and \lstinline|sg_op| names in addition to the operator ``names''.

If we now retype \lstinline|SemiGroup| as:
\begin{lstlisting}
  Π (G: Type) (e: Equiv G) (op: SemiGroupOp G), Prop
\end{lstlisting}
then we can declare the theorem with:
\begin{lstlisting}
  Theorem L G e op {P: SemiGroup G e op}: Π x y z, x & y & z = z & y & x.
\end{lstlisting}
This works because instance resolution, invoked by the use of \lstinline|=| and \lstinline|&|, will find \lstinline|e| and \lstinline|op|, respectively. Hence, the above is really

\begin{lstlisting}
  Theorem L G e op {P: SemiGroup G e op}: Π x y z,
    equiv e (sg_op op (sg_op op x y) z) (sg_op op (sg_op op x y) z).
\end{lstlisting}
where e and the op's are filled in by instance resolution.

At this point, a legitimate worry might be that the \lstinline|Equiv|/\lstinline|SemiGroup| classes and their \lstinline|equiv|/\lstinline|sg_op| projections imply constant construction and deconstruction of records, harming the simplicity and flexibility of the predicate approach that we are trying so hard to preserve. No such construction and destruction takes place, however, because type classes with only a single field are not desugared into an actual record with field projections the way classes with any other number of fields are. Instead, both class itself and its field projection are defined as the identity function with a fancy type. Thus, the introduction of these canonical names is essentially free; the structure predicate's new type reduces straight back to what it was before.

A remaining eyesore in the theorem declaration is the enumeration of \lstinline|e| and \lstinline|op|. To remove these, we use a new parameter declaration feature called \emph{implicit generalization}, introduced in Coq specifically to support type classes. Using implicit generalization, we can write:

\begin{lstlisting}
  Lemma L `{SemiGroup G}: Π x y z: G, x & y & z = z & y & x.
\end{lstlisting}
% TODO: describe implicit generalization in two words
% TODO: at some point down the line in our coq-inadequacies section, describe the implicit generalization improvements we need.
Thus, we have reached the mathematical ideal we aimed for.




% We note that operational type classes allow us to avoid Coq's notation \lstinline|scope| mechanism.
% hm, i don't think this is true. we don't need scopes for different representations' use of operators because we properly recognize their commonality, but that can be done without operational type classes as well.

% TODO: get this in:
% Although the expression now \emph{looks} like it is bound to some
% specific semigroup structure (which with traditional approaches would imply projections from bundles
% and/or reference to proofs), it is really only referring near-directly to the actual operations
% involved, with the semigroup proof existing only as opaque `knowledge about the operations' which we
% may use in the proof. This lack of projections keeps our terms small and independent, and keeps
% rewriting simple and sane; see Section~\ref{canonical}.

One aspect of the predicate approach we have not mentioned thus far is that in proofs parameterized by abstract structures, all components become hypotheses in the context. For the example theorem above, the context looks like:
\begin{lstlisting}
  G: Type
  e: Equiv G
  op: SemiGroupOp G
  P: SemiGroup G e op
\end{lstlisting}
We are not particularly worried about overly large contexts, especially because most of the ``extra'' hypotheses we have compared to bundled approaches are declarations of relations, operators, and constants, which are all in some sense ``inert'' with respect to proof search. Hence, we do not foresee problems with large contexts for any but the most complex formalizations.

\section{The algebraic hierarchy}\label{classes}

We have developed an algebraic hierarchy composed entirely out of predicate classes and operational classes as described in the previous section. For instance, our semiring interface looks as follows:
\begin{lstlisting}
  Class SemiRing A {e: Equiv A}
      {plus: RingPlus A} {mult: RingMult A}
      {zero: RingZero A} {one: RingOne A}: Prop :=
    { semiring_mult_monoid:> CommutativeMonoid A (op:=mult)(unit:=one)
    ; semiring_plus_monoid:> CommutativeMonoid A (op:=plus)(unit:=zero)
    ; semiring_distr:> Distribute mult plus
    ; semiring_left_absorb:> LeftAbsorb mult zero }.
\end{lstlisting}
All of \lstinline|Equiv|, \lstinline|RingPlus|, \lstinline|RingMult|, \lstinline|RingZero|, and \lstinline|RingOne| are operational (single-field) classes, with bound notations \lstinline|=|, \lstinline|+|, \lstinline|*|, \lstinline|0|, and \lstinline|1|, respectively.

Let us briefly highlight some additional aspects of this style of structure definition in more detail.

Fields declared with \lstinline|:>| are registered as hints for instance resolution, so that in any context where \lstinline|(A, =, +, 0, *, 1)| is known to be a \lstinline|SemiRing|, \lstinline|(A, =, +, 0)| and \lstinline|(A, =, *, 1)| are automatically known to be \lstinline|CommutativeMonoid|s (and so on, transitively, because instance resolution is recursive). In our hierarchy, these fields by themselves establish the inheritance diagram in Figure \ref{inheritancediagram}. % todo: for some reason the figure number reference is wrong here..

\begin{figure}
\label{inheritancediagram}
\centering
\includegraphics{hierarchy.pdf}
\caption{Inheritance relations established by \lstinline|:>| fields.}
\end{figure}

However, we can easily add additional inheritance relations by declaring corresponding class instances. For instance, while our \lstinline|Ring| class does not have a \lstinline|SemiRing| field, the following instance declaration has the exact same effect for the purposes of instance resolution (at least once proved, which is trivial):
\begin{lstlisting}
  Instance ring_as_semiring `{Ring R}: SemiRing R.
\end{lstlisting}

Thus, axiomatic structural properties and inheritance have precisely the same status as separately proved structural properties and inheritance, reflecting natural mathematical ideology. Again, contrast this with bundled approaches, where axiomatic inheritance relations determine projection paths, and where additional inheritance relations lead to additional and ambiguous projection paths for the same operations.

% todo: problem with ambiguous projection paths: quoting will not notice convertability.

The declarations of the two inherited \lstinline|CommutativeMonoid| structures nicely illustrate how predicate classes naturally support not just multiple inheritance, but \emph{overlapping} multiple inheritance, where the inherited structures may share components (in this case carrier and equivalence relation). The carrier \lstinline|A|, being an explicit argument, is specified as normal. The equivalence relation, being an implicit argument of class type, is resolved automatically to \lstinline|e| and passed. The binary operation and constant would normally be automatically resolved and passed as well, but we override the inference mechanism locally using Coq's existing named argument facility (which is only syntactic sugar of the most superficial kind) in order to specify the correct pairings. 
Again, contrast this with type system extensions such as Matita's manifest records, which are required to make this work when the records bundle components such as \lstinline|op| and \lstinline{unit} as \emph{fields} instead of parameters.

Since \lstinline|CommutativeMonoid| indirectly includes a \lstinline|SemiGroup| field, which in turn includes a \lstinline|Equivalence| field, having a \lstinline|SemiRing| proof means having two distinct proofs that the equality relation is an equivalence. This kind of redundant knowledge (which arises naturally in mathematics) is never a problem in our setup, because the use of operational type classes ensures that terms composed of algebraic operations and relations never refer to structure proofs. We find that this is a tremendous relief compared to approaches that \emph{do} intermix the two and where one must be careful to ensure that such terms refer to the ``right'' proofs of properties. There, even \emph{strong} proof irrelevance (which would make terms convertible that differ only in what proofs they refer) would not make these difficulties go away entirely, because high-level tactics that rely on quotation of terms require syntactic identity (rather than ``mere'' convertibility) to recognize identical subterms.

% rather strict separation of operations and proofs of properties about them (resolved by instance resolution) provides a very pleasing environment to work in, without any need for hacks.
% terms composed of algebraic operations and relations 
% separation of , this kind of redundance is never a problem for us. 
% it is /never/ a problem to know a fact twice.
% multiple proofs of equality, and even of semigroup-ness (since both Monoid), but that's simply not a problem, since none of our terms ever refers to such proofs. We have found that this rather strict separation of operations and proofs of properties about them (resolved by instance resolution) provides a very pleasing environment to work in, without any need for hacks.

The reader will note that we use names for properties like distributivity and absorption, and in fact these are type classes as well (which is why we declare them with \lstinline|:>|). It has been our experience that almost any generic property worth naming is worth representing as a type class, so that its proofs are be resolved as instances behind the scenes whenever possible. Doing this consistently minimizes administrative noise in the code, bringing us closer to ordinary mathematical vernacular. Indeed, we believe that type classes are an elegant and apt formalization of the seemingly casual manner in which ordinary mathematical presentation assumes implicit administration and use of a ``database'' of structural properties previously proved, much more so than existing solutions using, for instance, canonical structures.


% Unfortunately, aggressive . More on taht in section bla

% Having argued that the \emph{all-structure-as-parameters} approach \emph{can} be made practical, we enumerate some of the benefits that make it worthwhile.
% 
% First, multiple inheritance becomes trivial: \lstinline|SemiRing| inherits two \lstinline|Monoid| structures on the same carrier and setoid relation, using ordinary named arguments to achieve ``manifest fields''; see section~\ref{manifest}.
% 
% Second, because our terms are small and independent and never refer to proofs, we are invulnerable to concerns about efficiency and ambiguity of projection paths that plague existing solutions, obviating the need for extensions like the proposed coercion pullbacks~\cite{Hints}.
% 
% Third, since our structural type classes are mere predicates, overlap between their instances is a non-issue. Together with the previous point, this gives us tremendous freedom to posit multiple structures on the same operations and relations, including ones derived implicitly via subclasses: by simply declaring a \lstinline|SemiRing| class instance showing that a ring is a semiring, results about semirings immediately apply implicitly to any known ring, without us having to explicitly encode this relation in the hierarchy definition itself, and without needing any projection or translation of carriers or operations.
% 
% Fourth, \emph{Structure-as-parameters} helps setoid-rewriting: type class resolution
% can find the equivalence relation in the context.\footnote{
% A similar style should be possible for, say, the \lstinline|Ring| tactic, instead of
% declaring the ring structure by a separate command, we would rely on type class resolution to find it.}
% We note that \lstinline|op| does not depend on the proof that \lstinline|e| is an equivalence. As explained above we use Coq's implicit quantification (\lstinline|`{}|) to avoid having to write all the parameters when \emph{stating} a theorem and Coq's maximally inserted implicit arguments to find the parameters when \emph{applying} a theorem. Both features are new in Coq and stem from the type class implementation.
% 
% We mention the trade-off between bigger contexts versus bigger terms. Our contexts are bigger than
% those of telescopes or packed classes. In our experience, this has been relatively
% harmless\footnote{However, Coq's data structure for contexts is not very efficient. Gonthier fears that this
% may be a bottleneck for huge developments. It seems that the data structure chosen
% in~\cite{asperti2009compact} will behave better.}%
% : most terms in the context are there to support canonical names. Bigger terms
% \emph{do} cause problems: 1. when proofs are part of mathematical objects we need to share these
% proofs to allow rewriting. Moreover, it prohibits Opaque proofs and `proof irrelevance'. 2. The
% projection paths may not be canonical.
% 
% Coercion pullbacks~\cite{Hints} were introduced to address problems with multiple coercions paths,
% as in the definition of a semiring: a type with two monoid structures on it. We avoid some
% of these problems by explicitly specifying the fields. We emphasize that the semiring properties are
% automatically derived from the ring properties, although the properties of a semiring are not
% structurally included in the ring properties.
% 
% Let us now stop and think to what extent this approach suffers from all the problems commonly
% associated with it. In particular, let us imagine what happens to our terms and contexts when we
% want to talk about nested structures such as polynomials over the ring. For a concrete
% representation of the polynomials the context will just contain the context for an abstract ring.
% For abstract reasoning about polynomials the context will grow with abstract operations on
% polynomials. Consequently, the context will grow linearly, as opposed to exponentially.

Having described the basic principles of our approach, in the remainder of this paper we tour other parts of our development, further illustrating what a state of the art formal development of foundational mathematical structures can look like with a modern proof assistant.

These parts were originally motivated by our desire to cleanly express interfaces for basic numeric data types such as $\N$ and $\Z$ in terms of their categorical characterization as initial objects in the categories of semirings and rings, respectively. Let us start, therefore, with basic category theory.

\section{Category theory}\label{cat}

Following our idiom, we introduce operational type classes for the \emph{components} of a category:

\begin{lstlisting}
Class Arrows (O: Type): Type := Arrow: O → O → Type.
Class CatId O `{Arrows O} := cat_id: `(x ⟶ x).
Class CatComp O `{Arrows O} :=
  comp: Π {x y z}, (y ⟶ z) → (x ⟶ y) → (x ⟶ z).

Infix "⟶" := Arrow (at level 90, right associativity).
Infix "◎" := comp (at level 40, left associativity).
\end{lstlisting}
(The categorical arrow is distinguished from the primitive function space arrow by its length.)

With these in place, our type class for categories follows the usual type-theoretical definition of a
category~\cite{saibi1995constructive}:

\begin{lstlisting}
Class Category (O: Type) `{!Arrows O} `{Π x y: O, Equiv (x ⟶ y)}
    `{!CatId O} `{!CatComp O}: Prop :=
  { arrow_equiv:> Π x y, Setoid (x ⟶ y)
  ; comp_proper:> Π x y z, Proper (equiv ==> equiv ==> equiv) comp
  ; comp_assoc w x y z (a: w ⟶ x) (b: x ⟶ y) (c: y ⟶ z):
      c ◎ (b ◎ a) = (c ◎ b) ◎ a
  ; id_l `(a: x ⟶ y): cat_id ◎ a = a
  ; id_r `(a: x ⟶ y): a ◎ cat_id = a }.
\end{lstlisting}
This definition is based on the 2-categorical idea of having equality only on arrows, not on objects.
Similarly, we will have equality on natural transformations, but not on functors. % todo: ORLY?

Initiality, too, is defined by a combination of an operational and a predicate class:

\begin{lstlisting}
  Context `{Category X}.
  Class InitialArrow (x: X): Type := initial_arrow: Π y, x ⟶ y.
  Class Initial (x: X) `{InitialArrow x}: Prop :=
    initial_arrow_unique: Π y (a: x ⟶ y), a = initial_arrow y.
\end{lstlisting}

Having \lstinline|InitialArrow| as a type class means that we can always simply say \lstinline|initial_arrow y| whenever \lstinline|y| is known to be an object in a category known to have an initial object (where ``known to'' refers to the instances available to instance resolution).

Strictly speaking the above is all we need in order to continue with the story line leading up to the numerical interfaces, but just to give a further taste of what category theory with this setup looks like in practice, we briefly mention a few more definitions and theorems.

\subsection{Functors}

In our definition of functors we see the by now familiar refrain once more:
\begin{lstlisting}
  Context `{Category C} `{Category D} (map_obj: C → D).

  Class Fmap: Type :=
    fmap: Π {v w: C}, (v ⟶ w) → (map_obj v ⟶ map_obj w).

  Class Functor `(Fmap): Prop :=
    { functor_from: Category C
    ; functor_to: Category D
    ; functor_morphism:> Π a b: C, Setoid_Morphism (@fmap _ a b)
    ; preserves_id: `(fmap (cat_id: a ⟶ a) = cat_id)
    ; preserves_comp `(f: y ⟶ z) `(g: x ⟶ y): fmap (f ◎ g) = fmap f ◎ fmap g }.
\end{lstlisting}
Some clarification is in order to explain the role of the \lstinline|Context| declaration of the two categories. While it may look we are parameterizing both \lstinline|Fmap| and \lstinline|Functor| on proofs (which would be a gross violation of our idiom), in fact they depend and are parameterized only on the \emph{components} declared through implicit generalization of the \lstinline|Category| declarations. Indeed, we use the latter strictly as a convenient way to declare the former.

Notice that \lstinline|map_obj| is \emph{not} made into an operational type class. The reason for this is that \lstinline|map_obj| is analogous to the carrier type in earlier structure predicate class definitions, in that it serves as the primary identification for the structure, and should therefore not be inferred.

The \lstinline|functor_to| and \lstinline|functor_from| fields in \lstinline|Functor| are not an absolute necessity, but eliminate the need for theorems to declare \lstinline|Category| parameters when they already declare \lstinline|Functor|s between them. This is an instance where we can freely posit structural properties without worrying about potential problems when such information turns out to be redundant in contexts where the source and target of the functor are already known to be \lstinline|Category|'s.

Unfortunately, there is actually one annoying wrinkle here, which will also explain why, as the observant reader will have noticed, we do not register them as instance resolution hints (by declaring them with \lstinline|:>|).

What we really want these fields to express is ``\emph{if} in a certain context we know something to be a functor, \emph{then} realize that the source and target are categories''. However, the current instance resolution implementation does not support this style of exclusively-forward implications, and only supports backward implications: had we registered \lstinline|functor_to| and \lstinline|functor_from| as instance resolution hints, we would in fact be saying ``\emph{if} trying to establish that something is a category, \emph{then} you might try finding a functor to or from it'', which quickly degenerates into a wild goose chase.

We will say more about this unfortunate limitation of the current type class implementation in section \ref{wishlist}.

Finally, we ought to say a few words about our use of \lstinline|fmap|. The usual mathematical notational convention for functor application is to use the name of the functor to refer to both its object map and its arrow map, relying on additional conventions regarding object/arrow names for disambiguation (i.e. ``\lstinline|F x|'' and ``\lstinline|F f|'' map an object and an arrow, respectively, because ``\lstinline|x|'' and ``\lstinline|f|'' are conventional names for objects and arrows, respectively.

In Coq, for a term \lstinline|F| to function as though it had two different types simultaneously (namely the object map and the arrow map), there must either (1) be coercions from the type of F to either function, or (2) F must be (coercible to) a single function that is able to consume both object and arrow arguments. In addition to not being supported by Coq, option (1) would violate our policy of leaving components unbundled. For (2), if it could be made to work at all (which is not clear at all), F would need a pretty egregious type considering that arrow types are indexed by objects, and that the type of the arrow map (namely ``\lstinline|Π x y, (x ⟶ y) → (F x ⟶ F y)|'') must refer to the object map.

We feel that these issues are not limitations of the Coq system, but merely reflect the fact that notationally identifying these two different and interdependent maps is a typical example of an ``abus de notation'' that by its very nature is ill-suited to a formal development where software engineering concerns apply. Hence, we do not adopt this practice, and use ``\lstinline|fmap F|'' (name taken from the Haskell standard library) to refer to the arrow map of a functor \lstinline|F|.


\subsection{Natural transformations}
blabla, routine.


%theory/categories.v
Natural transformations are formalized as:
\begin{lstlisting}
   Context `{Category C} `{Category D}
   `{!Functor (F: C ⟶ D) Fa} `{!Functor (G: C ⟶ D) Ga}.
   Class NaturalTransformation (η: Π c, F c ⟶ G c): Prop :=
     natural: Π (x y: C) (f: x ⟶ y), η$$ y ∘ fmap F f = fmap G f ∘ η$$ x.
\end{lstlisting}
  % Todo: Having to add "$$" to get whitespace is awful. we can't add "\ " in η's literate-replacement though, because then we also get space before the colon in "η: t".
 
Here \lstinline|fmap F| explicitly refers to the arrow part of the functor \lstinline|F|.
In mathematics one would apply $F$ to both objects and arrows ($F x$, $F f$). However, this does
not seem to fit with type theory as the type of \lstinline|fmap F| depends on the object part of
\lstinline|F|.

We have defined the category of setoids, the dual of a category, products, and hence co-products. 
We prove that two definitions of adjunction are equivalent. The proofs are concise and follow the textbooks.

The usual size problems in the definition of the category of categories can be avoided by using universe polymorphism. 
However, we need to avoid making \lstinline|Relation| a \lstinline|Definition|, since \lstinline|Definitions| are not (yet?) universe polymorphic.


\subsection{Adjunctions}
blabla, example of nontrivial theorem. nice stress-test, worked out mostly fine.



\section{Universal algebra}\label{univ}




Motivated originally by our desire to cleanly express interfaces for basic numeric data types such as $\N$ and $\Z$ in terms of their categorical characterization as initial objects in the categories of semirings and rings, respectively, we initially introduced only the very basics of category theory into our development, again using type classes where possible to achieve the same benefits mentioned before.

Realizing that much code duplication for the various algebraic structures in the hierarchy could be avoided by employing universal algebra constructions, we then proceeded to formalize some of the theory of multisorted universal algebra and equational theories, using it to automatically construct varieties of algebras. We avoided existing formalizations~\cite{DBLP:conf/tphol/Capretta99,dominguez2008formalizing} of universal algebra, because we aimed to find out what level of elegance, convenience, and integration can be achieved using the state of the art technology (of which type classes are the most important instance).

At the time of writing, our development includes a fully integrated formalization of a nontrivial portion of category theory and multisorted universal algebra, including various categories (e.g.\ the category \lstinline|Cat| of categories, and finitary algebraic categories  defined by a theory which we instantiate to obtain the categories of monoids, semirings, and rings), functors (including automatically generated forgetful functors), natural transformations, adjunctions, initial models of equational theories constructed from term algebras, transference of proofs between isomorphic models of equational theories, subalgebras, congruences, quotients, products, and the first homomorphism theorem~\cite{meinke1993universal}.

We consider the first homomorphism theorem in more detail:
\begin{theorem}[First homomorphism theorem]
If $A$ and $B$ are algebras, and $f$ is a homomorphism from $A$ to $B$, then the equivalence relation defined by $a\sim b$ if and only if $f(a)=f(b)$ is a congruence on $A$, and the algebra $A/\sim$ is isomorphic to the image of $f$, which is a subalgebra of B.
\end{theorem}

% theory/ua_subalgebraT.v
% find . -iname "*.v" |xargs grep "niverse"
We define congruence:
\begin{lstlisting}
Class Congruence: Prop :=
    { congruence_proper:> Π s, Proper (equiv ==> equiv ==> iff) (e s)
    ; congruence_quotient:> @Algebra et v e _
    }.
\end{lstlisting}

In set theory, a quotient is defined as a collection of equivalence classes. In type theory, a quotient is more naturally
defined by changing the equality, while leaving the carrier unchanged.%
\footnote{In Bishop's words~\cite[p.12]{Bishop/Bridges:1985}: The axiom of choice is used to extract
elements from equivalence classes where they should never have been put in the first place.}
 In this way we profit from the infrastructure for setoids: A congruence relation is coarser than the setoid relation, as indicated by the use of \lstinline|Proper| above.

Alternatively, we could have stated the compatibility of the relation \lstinline|e| by stating that the
relation as a set of pairs is a subalgebra in the product algebra. We have proved that both approaches
are equivalent. As said, the present definition seems natural in type theory. 

Coq provides different sorts. To simplify the discussion, we assume that there are only two sorts:
\lstinline|Prop|, for propositions, and \lstinline|Type|, for data types. A subalgebra is defined by a predicate on the algebra, that is a map \lstinline|A->Prop|. However, the image $Im f:=\{b:B \mid \exists a. b=(f a)\}$ is informative (in \lstinline|Type|) as it includes a witness for the existential. This witness is needed to map $b\in Im f$ to one of its pre-images in $A$. We thus need two flavors of subalgebras, one \lstinline|Prop|-valued an one \lstinline|Type|-valued; see also~\cite{coquand-towards}. Coq provides universe polymorphism for inductive types to avoid code duplication in such cases. Unfortunately, definitions and fixpoints are not (yet?) polymorphic. Hence we had to resort to changing some Fixpoints into Inductives.

Having discussed the first homomorphism theorem we now turn to category theory which helps to organize our development of universal algebra. 

\subsection{Categorical universal algebra}
Having discussed category theory, we now turn to its application to universal algebra.

%varieties/closed_terms.v
We define the forgetful functor $U$ from the category of τ-algebras to the category of sets and its left-adjoint $F$: the term algebra consisting of closed terms, i.e.\ terms with the empty type as set of variables. This term algebra is the initial τ-algebra. Every adjunction defines a monad by composition of the functors $U$ and $F$. We will use the resulting expression monad in Section~\ref{quote} below.


One may hope that this term algebra could be useful to \emph{define} inductive data types, say, to define the natural numbers 
as the term algebra for the theory of semirings. However, it seems difficult to prove decidability of equality, 
as this requires a normalization procedure.

%The product of two $\tau$-algebras is their categorical product.

\begin{comment}
Given a sub\emph{set} of an algebra, we define the sub\emph{algebra} generated by it.
An interesting fact to mention is the use of the following heterogeneous equality between
elements of the algebra and of the subalgebra, i.e.\ terms of different types may be equal.
\begin{lstlisting}
  Fixpoint heq {o}: op_type carrier o -> op_type v o -> Prop :=
    match o with
    | constant x => fun a b => `a == b
    | function x y => fun a b => forall u, heq (a u) (b (`u))
    end.
\end{lstlisting}
\end{comment}

We connect the abstract theory to the concrete theory. Concretely, let $\tau_m$ be the theory
of monoids. Given a (concrete) monoid, we can construct the corresponding $\tau_m$-algebra. 
Conversely, given a $\tau_m$-algebra, we construct an instance of the \lstinline|Monoid| type class.
%varieties/monoid.v
\begin{lstlisting}
Variable o: variety.Object theory.
Global Instance: SemiGroupOp (o tt) := algebra_op theory mult.
Global Instance: MonoidUnit (o tt) := algebra_op theory one.
Global Instance from_object: Monoid (o tt).
\end{lstlisting}

In the last line, instance resolution `automatically' finds the operation and the unit specified in
the lines before.

This interplay between concrete algebraic structure and their expressions on the one hand, and models of equational theories on the other is occasionally a source of tension (in that translation in either direction is not yet fully automatic). However, it opens the door to the possibility of fully internalized implementations of generic tactics for algebraic manipulation, no longer requiring plugins. We come back to this when we describe automatic quotation of concrete expressions into universal algebra expressions; see Section~\ref{quote}.


\section{Interfaces using category theory}\label{interfaces}\label{modul}
We have characterized the naturals as the initial object in the category of semirings and derived
many of their properties from this interface. Unary, binary and machine numbers are
instances of this interface, so we can directly apply these results to all these instances.
Similarly, the integers are the initial ring. The rationals are the field of fractions of the
integers. 
Given a ring $R$, the $R$-algebra $R[X]$ of polynomials, is the free $R$-algebra on a set $X$.
We provide two implementations of polynomials: the
standard representation using lists of coefficients and the Bernstein representation. A Bernstein
basis polynomial is one of the form:
\[b_{\nu,n}(x) = {n \choose \nu} x^{\nu} \left( 1 - x \right)^{n - \nu}, \quad \nu = 0, \ldots, n.\]
A Bernstein polynomial is a linear combination of these basic polynomials. Bernstein polynomials
have been used for efficient computations inside the Coq system~\cite{ZumkellerPhD}.

We encourage more efficient implementations by assigning the default implementation a
low priority. For example, the distance function on the natural numbers, which is derived from its
semiring structure, is assigned priority 10.
\begin{lstlisting}
  Global Program Instance: NatDistance N | 10 := ...
\end{lstlisting}

In future work we aim to base our development of the reals
on an abstract dense set, allowing us to use the efficient dyadic
rationals~\cite{boldo2009combining} as a base for exact real number computation in
Coq~\cite{Riemann,Oconnor:real}. The use of category theory has been important in these developments.

\section{Type class quoting}\label{quote}
Unification hints~\cite{Hints} allow one to semi-automatically construct a quote
function.\footnote{Gonthier provides similar functionality using canonical structures.} This feature is absent from Coq. Fortunately, type classes provide similar functionality as we will now show.

We define a term language for monoids
\begin{lstlisting}
Inductive Expr (V: Type) := Mult (a b: Expr V) | Zero | Var (v: V).
\end{lstlisting}

The expression type is parametrized over the set of variable indices. Hence, we diverge from~\cite{Hints}, 
which uses \lstinline|nat| for variables thereby introducing bound problems and dummy variables.

 An expression is only meaningful in the context of a variable
assignment:\footnote{Categorically: \lstinline|nat| is
an \lstinline|Expr|-algebra and \lstinline|eval| is \lstinline|map vars| composed with the algebra map.}
\begin{lstlisting}
Definition Value := nat.
Definition Vars V := V → Value.

Fixpoint eval {V} (vs: Vars V) (e: Expr V): Value :=
  match e with
  | Zero => 0
  | Mult a b => eval vs a * eval vs b
  | Var v => vs v
  end.
\end{lstlisting}

%Monads are trees with grafting~\cite{MonadsGrafting}: the tree monad
%$TX:=1+T^2X$, may be seen as the prototypical monad. 
The \lstinline|Expr| monad, i.e.\ the free \lstinline|Expr|-algebra monad on the category of \lstinline|Type|s with extensional functions%\marginpar{Proof in classquote}
, provides a bind operation which describes the arrows in the Kleisli category. Concretely,
\begin{lstlisting}
Fixpoint bind {V W} (f: V → Expr W) (e: Expr V): Expr W :=
  match e with
  | Zero => Zero
  | Mult a b => Mult (bind f a) (bind f b)
  | Var v => vs v
  end.
\end{lstlisting}


% We have shown that |Expr|, i.e.\ |bind| together with |Var|, is a monad on the category of Types
% with extensional functions between them~\marginpar{Really?}.

\noindent We define some simple combinators for variable packs:
%
\begin{lstlisting}
Definition novars: Vars False := False_rect _.
Definition singlevar (x: Value): Vars unit := λ _ => x.
Definition merge {A B} (a: Vars A) (b: Vars B): Vars (A+B) :=
  λ i => if i then a j else b j.
\end{lstlisting}


\noindent These last two combinators are the `constructors' of an implicitly defined subset of
 Galina terms, representing heaps, for which we implement syntactic lookup with type classes.
The heap can also be defined explicitly, with no essential change in the code.
Given a heap and value, \lstinline|Lookup| instances give the value's index in the heap:
% \marginpar{A
% canonical structures approach would allow us to do this in Coq?}
\begin{lstlisting}
  Class Lookup {A} (x: Value) (f: Vars A) :=
    { lookup: A; lookup_correct: f lookup = x }.
\end{lstlisting}

% Context (x: Value) {A B} (va: Vars A) (vb: Vars B).

If the heap is a merge of two heaps and we can find the value's index in the left heap, we can
access it by indexing the merged heap (and vice versa)
\begin{lstlisting}
  Global Instance lookup_left `{!Lookup x va}: Lookup x (merge va vb)
    := { lookup := inl (lookup x va) }.
\end{lstlisting}

If the heap is just a singlevar, we can easily index it.
\begin{lstlisting}
Global Program Instance: Lookup x (singlevar x) := { lookup := tt }.
\end{lstlisting}


%One useful operation we need before we get to Quote relates to variables and expression evaluation. 
 As its name suggests, \lstinline|map_var| maps an expression's variable indices.
This reindexing function is the the \lstinline|map| of the \lstinline|Expr|-monad.

\begin{lstlisting}
Definition map_var {V W: Type} (f: V → W):
    Expr V → Expr W :=
  fix F (e: Expr V): Expr W :=
    match e with
    | Mult a b => Mult (F a) (F b)
    | Zero => Zero
    | Var v => Var (f v)
    end.
\end{lstlisting}

In \lstinline|Quote| below, the idea is that \lstinline|V, l|, and \lstinline|n| are all input variables, while \lstinline|V'| and \lstinline|r| are
output variables (in the sense that we will rely on unification to generate them). \lstinline|V| and \lstinline|l|
represent the current heap, \lstinline|n| represents the value we want to quote, and \lstinline|V'| and \lstinline|r'| represent
the heap of newly encountered variables during the quotation.
  This explains the type of quote: it is an expression that refers either to variables from
the old heap, or to newly encountered variables. Finally, \lstinline|eval_quote| is the usual correctness property, which now merges the two heaps.

\begin{lstlisting}
  Class Quote {V} (l: Vars V) (n: Value) {V'} (r: Vars V') :=
    { quote: Expr (V + V')
    ; eval_quote: @eval (V+V') (merge l r) quote = n }.
\end{lstlisting}


Our first instance for \lstinline|Zero| is easy. The `novars' in the result type reflects the fact that no
new variables are encountered.
\begin{lstlisting}
  Global Program Instance quote_zero V (v: Vars V):
    Quote v 0 novars := { quote := Zero }.
\end{lstlisting}


The instance for multiplication is a bit more complex. The first line consists of
 variable declarations. The second line is important. `Quote x y z' must be read as
 `quoting y with existing heap x generates new heap z', so the second line basically just
shuffles heaps around.
 The third line contains some \lstinline|map_var|'s because the heap shuffling must be
reflected in the variable indices, but apart from that it is just constructing a \lstinline|Mult| term with
quoted subterms.

\begin{lstlisting}
Global Program Instance quote_mult V (v: Vars V)
  n V' (v': Vars V') m V'' (v'': Vars V'')
  `{!Quote v n v'} `{!Quote (merge v v') m v''}:
  Quote v (n * m) (merge v' v'') :=
  { quote := Mult (map_var bla (quote n)) (map_var sum_assoc (quote m)) }.
\end{lstlisting}

The instance where we recognize values that are already in the heap is expressed by the Lookup requirement, which will only be fulfilled if the Lookup instances defined above can find the value in the heap. The novars in the \lstinline|Quote v x novars| result
   reflects that this quotation does not generate new variables.

\begin{lstlisting}
Global Program Instance quote_old_var
  V (v: Vars V) x {i: Lookup x v}:
  Quote v x novars | 8 := { quote := Var (inl (lookup x v)) }.
\end{lstlisting}


\noindent Finally, the instance for new variables. We give this a lower priority so that it is only
used if Lookup fails. The \lstinline|8| in the previous example is a random number less than \lstinline|9| below. 
%A similar method is used in Coq's notation mechanism.
% We recommend a change in
% implementation supporting an agda style precedence relation which is only restricted to be a
% directed acyclic graph~\cite{danielsson2009parsing}.

\begin{lstlisting}
Global Program Instance quote_new_var V (v: Vars V) x:
  Quote v x (singlevar x) | 9 := { quote := Var (inr tt) }.
\end{lstlisting}


We have defined a light-weight quoting function. The following code turns the goal into \lstinline|(eval variable_pack quote)|.
Here \lstinline|eval_quote'| is a modification of \lstinline|eval_quote| which works in the empty context:
\begin{lstlisting}
Goal Π x y (P: Value $\to$ Prop), P ((x * y) * (x * 0)).
  intros. rewrite <- (eval_quote' _).
\end{lstlisting}

We have used this technique to implement a tactic to normalize monoid expressions.
%It would be interesting to use this technique to implement the \lstinline|congruence| tactic, which implements
%the congruence closure algorithm~\cite{corbineau2007deciding}. One would naturally obtain a tactic
%which moreover works on setoid equalities.

\section{Canonical structures}\label{canonical}
Packed classes use canonical structures for the algebraic hierarchy. Both canonical structures and
type classes may be seen as instances of hints in unification~\cite{Hints}. Some uses of canonical
structures can be replaced by type class instances. The user manual (2.7.15) uses canonical
structures to derive the setoid equality on the natural
numbers in the following example \lstinline|(S x)==(S y)|. In this case type classes provide similar proof
terms. Canonical structures give
\begin{lstlisting}
@equiv(Build_Setoid nat (@eq nat)(@eq_equivalence nat))(S x)(S y)
\end{lstlisting}

which includes an explicit proof that \lstinline|(@eq nat)| is an equivalence,
whereas we obtain \lstinline|@equiv nat (@eq nat) (S x) (S y)|.

\subsection{Big operators}
Canonical structures have been used to provide a uniform treatment~\cite{bertot2008canonical} of big
operators (like $\Pi,\sum, \max$). These operators extend a pair of a binary and a 0-ary operation
to an $n$-ary operation for any $n$. Categorically, one considers the algebra maps from non-empty
lists, lists, inhabited finite sets and finite sets to the carrier of a semigroup, monoid,
commutative semigroup, commutative monoid. Hence we want to reuse the libraries for lists etc.\ as much as possible.

Again, we can use type classes, instead of canonical structures, to deduce the relevant monoid operation:
\begin{lstlisting}
Definition seq_sum
  `{Sequence A T} `{RingPlus A} `{z: RingZero A}: T $\to$ A
  := @seq_to_monoid A T _ A ring_plus z id.
  Eval compute in seq_sum [3; 2].
\end{lstlisting}
\marginpar{Prove distributivity?}

\section{Subset types and proof irrelevance}\label{PI}
We have mentioned proof irrelevance a number of times before. It is an important theme in our
approach. The \lstinline|Program| machinery allows one to conveniently write programs with Hoare-style
pre- and post-conditions. I.e.\ functions $f: \{ A \mid P \} \to \{ B \mid Q \}$. Both side
conditions
are intended to be proof irrelevant, they are in \lstinline|Prop|. Presently, Coq does not support such subset
types, thus forcing the user to manually prove that \lstinline|f| does not depend on the proof of \lstinline|P|.
The addition of proof irrelevance and subset types to Coq is pursued by
Barras~\cite{Barras:subset,Werner08}. 
% Squash is a monad and allows many constructions: subsets, unions, images, heterogeneous equality,
% \ldots.

Our approach may be intuitively explained by `telescopic' subset types. We recall the use of telescopes in record types~\cite{pollack2000dependently}. Given a telescope\[
T=[x_1:A_1][x_2:A_2(x_1)]\cdots[x_k:A_k(x_1,\ldots,x_k)].
\]
There is an inductively defined type $\Sigma T$, with a single constructor, given by the following 
formation and introduction rule in pseudo-code for $k=2$:
\begin{align*}
 A_1 :& Type\\
 A_2 :& \Pi A_1,Type\\
\cline{1-2}
\Sigma T :& Type
\end{align*}

\begin{align*}
 \Sigma T:Type\quad a_1 &:A_1 \quad a_2 : A_2(a_1)\\
\cline{1-3}
(a_1,a_2) &: \Sigma T
\end{align*}

For our methodology to work we have to be able to transform the occurring record into a telescopic
subset $\{ A \mid  B\}$. We moreover conjecture that the terms in \lstinline|B| only depend on the
variables in \lstinline|A|, but not in \lstinline|B| itself. However, we do not need this.

We are naturally lead to the following methodology: Let $\phi$ be a statement in type
theory, i.e.\ an alternation of $\Pi$s and $\Sigma$s. By the type theoretic axiom of choice, we first Skolemize to $\Sigma \overrightarrow{h}. \Pi \overrightarrow{a}. P$ and then transfer the variable $\overrightarrow{h}$ to the parameters. Concretely, the statement of surjectivity is skolemized to a surjection with an explicit right inverse, i.e.\ a split epi. We move the right inverse to the parameters. As a side effect, the distinction between classical and constructive mathematics virtually disappears for such explicit statements.

The addition of implicit Σ-types to Coq would change many things. Perhaps it would even allow us
to pack the proofs together with the operations, but it is too early to tell.

\subsection{Propositions as Types}
As originally conceived~\cite{ITT,CMCP}, Propositions are Types and we can extract information from
them. Often we do \emph{not} want functions to depend on proofs. In Coq, this is the paradigmatic way of working. We are forbidden to extract information from proofs of \lstinline|Prop|s. Especially with the aid of the Program machinery, this allows us to flexibly work with simply typed program and their Hoare style correctness proofs. However, we would also like to use Coq as a dependently typed programming language. This requires a hybrid approach, were we put informative propositions in \lstinline|Type| and non-informative ones in \lstinline|Prop|. Making Harrop formulas proof irrelevant seems to be a good first approximation~\cite{lcf:spi:03}. An orthogonal and further reaching proposal is to replace the Coq type theory with the Calculus of implicit constructions with implicit
Σ-types~\cite{miquel2001implicit,barras2008implicit,Bernardo}. 

%One may be tempted to propose Propositions as [ ]-types~\cite{awodey2004propositions}. Here the
%construction $[P]:=\{\top\mid p:P\}$ squashes all the information about how we proved $P$. In Coq,
%the [ ]-types are isomorphic to the type of sort \lstinline|Prop|. However,
%there are places where we needed proof \emph{relevance}; e.g.\ in the proof of the first isomorphism
%theorem above.In practice, manual dead-code analysis seems useful. Making Harrop formulas proof
%irrelevant seems to be a good first approximation~\cite{lcf:spi:03}. However, a further refinement
%of Propositions as types seems necessary.

\section{Coq wishlist}
% todo: maybe merge with conclusions and add "and future work"

While we have already found the predicate-class-based style of formalization to work very well for our development thus far, and believe it has the potential to be \emph{the} modern solution to this problem, in our work we have encountered a number of limitations in the Coq implementation of key features of the tool, that keep us from making an unconditional recommendation for universal adoption of these techniques.

\paragraph{Universe-polymorphic definitions}
Coq has supported universe-polymorphism for parameterized inductive definitions (different instantiations of which are automatically of the lowest possible sort taking into account the universe level constraints on the arguments) for some time. However, the same functionality has not been available for \emph{definitions}, primarily because of efficiency concerns. In our development, bla bla misery and drama when doing this and that. In response to our inquiries about this issue, the Coq development team has agreed to implement ``opt-in'' universe polymorphism, where only expressly annotated definitions are made universe-polymorphic. This should solve our problems.

\paragraph{Implicit generalization}

The current syntax for implicit generalization 
%current syntax very subtle. various combinations of `, class-ness, {} vs (), ! or no !, make _ mean different things. Worse, one strategy of ``infer if possible, generalize otherwise'' is not supported while it seems like a very commonly needed thing. The Coq development team recognizes this as a legitimate problem, but fears the implementation might not be straightforward, and as of this writing we are not aware of plans to address this. That said, a new proof engine has recently, whcih might make this more attainable. Fortunately, not that big a problem. Just a remaining source of the occasional uglyness in declarations.

\paragraph{Instance resolution efficiency} ...

The current implementation is in some sense only a prototype.

Backward

Forward

It seems likely that known techniques from established resolution engines can be reused to significantly increase performance.


In reply to our inquiries about these issues the Coq development team has agreed that these are legitimate concerns and wishes, and indicated that they are committed to addressing them.





\section{Conclusions}
Telescopes have been criticized~\cite{Packed} for the lack of multiple inheritance and
the efficiency penalty of a long chain of coercion projections. Packed classes~\cite{Packed} provide
a solution to these problems. We have provided an alternative solution based on type classes. 

We conclude that unification hints, canonical structures and type classes all appear to have
similar expressive power. Canonical structures, being tied to the unifier, are more robust, but
seem to require more ingenuity. Type classes are easier to use. Unification hints, may be a
good generalization, but are absent from Coq. We encourage their inclusion to the
Coq system.

An obvious topic for future research is the extension from equational logic to partial Horn
logic~\cite{palmgren2007partial}. Another topic would be to fully, but practically, embrace the
categorical approach to universal algebra~\cite{pitts2001categorical}.

According to \lstinline|coqwc|, our development consists of 4895 lines of
specifications and 769 lines of proofs.\marginpar{update,targz}

% Alt-Ergo congruence closure parametrized by an equational theory.
% As usual either implement in Coq (verify the algorithm), or Check traces, or Check Alt-Ergo (Work
% in progress).

% http://lara.epfl.ch/dokuwiki/sav09:congruence_closure_algorithm_and_correctness
% I understand that this is the algorithm underlying congruence and perhaps first-order.
% 
% Type classes vs soft typing (Mizar types).
% Type classes and types in homalg?

\paragraph{Acknowledgements.}
This research was stimulated by the new possibilities provided by the introduction of type classes
in Coq. In fact, this seems to be the first substantial development that tries to exploit all their
possibilities. As a consequence, we found many small bugs and unintended behavior in the type
class implementation. All these issues were quickly solved by Matthieu Sozeau. Discussions with
Georges Gonthier and Claudio Sacerdoti Coen have helped to
sharpen our understanding of the relation with Canonical Structures and with Unification Hints.
%Jeremy, Thierry
\bibliographystyle{plain}
\bibliography{alg}
\end{document}