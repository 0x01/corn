\documentclass[a4paper,10pt, runningheads]{llncs}
\usepackage{url}
\usepackage{bbm}
\usepackage{ucs}
\usepackage{amsmath}
% \usepackage[utf8x,utf8]{inputenc}
\newcommand{\N}{\ensuremath{\mathbbm{N}}}
\newcommand{\Z}{\ensuremath{\mathbbm{Z}}}
\newcommand{\Q}{\ensuremath{\mathbbm{Q}}}
%include coq2tex.tex
\arrayhs%i.e. no page breaks in code blocks
\divide\abovedisplayskip 4
\divide\belowdisplayskip 4
\begin{document}
\title{DRAFT: The algebraic hierarchy in type theory using type classes}
\author{Bas Spitters \and Eelis van der Weegen}
\institute{Radboud University Nijmegen}
\maketitle
\begin{abstract}
We propose generic design patterns to organize algebraic structures using type classes inside the
Coq system. We support multiple inheritance, sharing of notations and theories, and automated
structure inference. A library for many-sorted universal algebra abstracts from the specific
algebraic structures and aids the organization of the library. Category theory provides a further
useful abstraction. We provide abstract
interfaces for $\N$, $\Z$, $\Q$ as the initial semiring, the initial ring and a field of
fractions. Similarly, we provide an interface for polynomials over a ring $R$ using $R$-algebras.
We show that the standard implementations of these are in fact instances of these type classes.
We report on the efficiency penality of these abstractions.

Finally, we use type classes to conveniently implement a quote function using the prolog-like
abilities of type class unification. We integrate this with our development of universal algebra.
\end{abstract}

\section{Introduction}
Developing formalized libraries for mathematics requires a detailed analysis of the
architecture of mathematics~\cite{C-corn,DBLP:conf/types/HaftmannW08}, similar to that of a large
software project.
The development of an algebraic hierarchy is an important aspect. Challenges include: 1. multiple
inheritance: A ring is a type with two operations. How do we inherit from the two theories of
monoids? 2. Sharing of notations. 3. Combining syntactic and semantic methods, usually this is
addressed by reflection.

Several solutions have been proposed for the Coq theorem prover~\cite{Coq,BC04}: dependent
records~\cite{DBLP:journals/jsc/GeuversPWZ02} (a.k.a.\ telescopes), packed classes~\cite{Packed} and
occasionally modules. Although the latter option is quickly discarded%
\footnote{Modules are not first class, but currently they seem to be superior for extraction to
OCaml~\cite{letouzey02}. Modules may also be seen as parametrised sections. They allow management
of namespaces, notations, hints, etc. An |Import| command imports all the constants from a
module. Thus allowing the user to type |B| instead of |A.B|. Similar functionality would be useful
for type classes.}
algebraic theories as `interfaces', as opposed to `modules', has been an important
guiding principle~\cite{Packed}.

In this fashion we implement\footnote{The sources are available
at:~\url{http://www.eelis.net/research/math-classes/}}
 algebraic theories as type classes, not instances, thus addressing the
challenges above; see Section~\ref{classes}. Type classes have recently entered
Coq~\cite{DBLP:conf/tphol/SozeauO08}. In Coq,
type classes are \emph{first class} citizens based on dependent records. The presence of dependent types makes them even more
powerful than the already succesful second class implementations in Haskell and Isabelle. Moreover,
Coq's type classes are well integrated with setoid-rewriting~\cite{Setoid-rewrite}.

We abstract from concrete algebra and formalize many-sorted universal algebra, including a
construction of free $\tau$-algebras; see Section~\ref{univ}.

% We are also concerned with efficiency.
% In fact, our aim is to use the present development as a new basis for Bishop's program in type
% theory~\cite{typesreal-article}.\marginpar{More}

So, we would like to replace data structures with more efficient ones. Concretely,
Coq provides unary, binary and machine numbers. We use type classes polymorphism to
develop a unified library for them; see Section~\ref{modul}.
The interfaces for $\N$, $\Z,\Q$ are the initial semiring, the initial ring and a field of
fractions; see Section~\ref{interfaces}. In future work we aim to base our development of the reals
on an abstract dense set, allowing us to use the efficient dyadic
rationals~\cite{boldo2009combining} as a base for exact real number computation in
Coq~\cite{Riemann,Oconnor:real}.

The paper is organised as follows: Section~\ref{classes} starts with a development of the concrete
algebraic hierarchy addressing the issues raised above. Section~\ref{univ} abstracts to universal
algebra, we develop the basic theory to the point of the first homomorphism theorem.
Section~\ref{interfaces} provides interfaces to usual datastructures by providing their universal
properties. To allow symbolic manipulation of semantic objects one needs a quote function. Usually,
this function is implemented in Coq's tactics language. A more convenient way is to use type class
unification; see section~\ref{quote}. In section~\ref{canonical}, we end with a short comparision
between type classes and canonical structures.

In this paper we focus on the Coq proof assistant. We conjecture that the methods can be transfered
to any type theory based proof assistant supporting type classes such as
Matita~\cite{asperti2007user}.

\section{Preliminaries}
\subsection{Coq type theory}
The Coq proof assistant is based on the calculus of inductive
constructions~\cite{CoquandHuet,CoquandPaulin}.\marginpar{More}
This type theory lacks quotient types, as it would make type checking undecidable. Instead it
supports setoids (aka Bishop sets), a pair of a type and an equivalence
relation~\cite{Bishop67,Hofmann,Capretta}. Recently, improved support has been added to the
system~\cite{Setoid-rewrite}. Functions respecting these equivalence relations are called |Proper|.
As we will see in Section~\ref{univ} below this functionality can also be used to work with
quotient algebras.

Coq has two kinds of sorts |Prop| for propostions and |Type| for data types. Coq
is scheduled to support proof irrelevance: every |p,q:P:Prop| are
convertible in the future; see section~\ref{PI}.

% \subsection{Type classes}
% Implementation in Ltac, Notation etc\\
% Coercions\\
% Unification hints

\section{To bunde or not to bundle? (Or: What to bundle?)}

How should we define reflexive, transitive, and symmetric relations? In a first attempt, we
might define:
%format → = "\rightarrow"
%format Π = "\Pi"
\begin{code}
  Record ReflexiveRelation: Type :=
    { rr_carrier: Type
    ; rr_rel: rr_carrier → rr_carrier → Prop
    ; rr_proof: Π x, rr_rel x x }.
\end{code}
and similar for transitive and symmetric relations.

With this approach, we run into problems when we try to define |EquivalenceRelation| in terms of
|ReflexiveRelation| and its two siblings, because when posit three objects of type
|ReflexiveRelation|, |TransitiveRelation|, and |SymmetricRelation|, respectively, we have to
add equalities expressing that they talk about the same carrier and relation, and these equalities
are a nightmare to work with (because the three carriers and relations won't be convertible).
Technically, this leads to the research in manifest fields, the \emph{with} construction, to which
we return in Section~\ref{manifest}.

As a first step we would define |ReflexiveRelation| as:
\begin{code}
  Record ReflexiveRelation (carrier: Type): Type :=
    { rr_rel: rr_carrier → rr_carrier → Prop
    ; rr_proof: Π x, rr_rel x x }.
\end{code}

This would solve the problem for the carrier, but |EquivalenceRelation| would still need equalities
for the relation:
\begin{code}
  Record EquivalenceRelation: Type :=
    { er_carrier: Type
    ; er_refl: ReflexiveRelation er_carrier
    ; er_sym: SymmetricRelation er_carrier
    ; er_trans: TransitiveRelation er_trans
    ; same_rel: rr_rel er_refl = sr_rel er_sym
    ; same_rel': rr_rel er_refl = tr_rel er_trans }.
\end{code}

A natural solution is to pull the relation out of |ReflexiveRelation| (and its siblings) as well:
\begin{code}
  Record ReflexiveRelation (carrier: Type)
                           (rel: carrier → carrier → Prop): Prop :=
    { rr_proof: Π x, rr_rel x x }.
\end{code}
or equivalently:
\begin{code}
  Definition ReflexiveRelation (carrier: Type)
                               (rel: carrier → carrier → Prop): Prop :=
    Π x, rr_rel x x.
\end{code}
This is how reflexivity is defined in the standard library, and this is entirely sensible. The
change from |Type| to |Prop| reflects the fact that |ReflexiveRelation| is now strictly a property,
proofs of which can be nicely kept opaque;%
\footnote{Coq distinguishes between transparent and opague definitions. The former can be unfolded,
whereas this is impossible for the latter.}
we will return to this issue in Section~\ref{PI}.
We can now define |EquivalenceRelation| without awkward equalities:
\begin{code}
  Record EquivalenceRelation: Type :=
    { er_carrier: Type
    ; er_rel: er_carrier → er_carrier → Prop
    ; er_refl: ReflexiveRelation er_carrier er_rel
    ; er_sym: SymmetricRelation er_carrier er_rel
    ; er_trans: TransitiveRelation er_trans er_rel }.
\end{code}

But looking at |EquivalenceRelation|, we see that it now essentially has the same form as the
original naive |ReflexiveRelation| record we started with. Clearly, equivalence relations
and reflexive relations are analogous concepts, and by the same argument we
conclude that |EquivalenceRelation| should really be defined as:

\begin{code}
Record EquivalenceRelation (er_carrier: Type)
                             (er_rel: er_carrier → er_carrier → Prop): Prop :=
  { er_refl: ReflexiveRelation er_carrier er_rel
  ; er_sym: SymmetricRelation er_carrier er_rel
  ; er_trans: TransitiveRelation er_trans er_rel }.
\end{code}

After all, when we define a type with
two different equivalence relations on it,
such as the quotient of a setoid,
we have the same considerations as for |ReflexiveRelation| above.

At the risk of overstressing the point, let us look at how the same problem and solution keep
appearing in composite structures. Suppose we use a definition of |Monoid| that bundles the
relation. Then if we proceed to define |SemiRing| as a double |Monoid|, we face a dilemma:
\begin{code}
  Record SemiRing A: Type :=
    { mult_monoid: Monoid A 
    ; plus_monoid: Monoid A 
    ; plus_comm: Commutative (mon_op plus_monoid)
    ; mult_comm: Commutative (mon_op mult_monoid)
    ; mult_0_l: Π x: A, ?
\end{code}
Which equality do we use in the expression ?, |mult_monoid|'s equality or |plus_monoid|'s?
Again, for this definition to be usable at all, we would need to add an equality expressing that the
two monoids contain the same equivalence relation. And again, the proper solution is to change
Monoid and unbundle the relation. But here, we have a choice as to how much we unbundle exactly.
(For the moment we ignore the question of whether the unit and binary operator should be unbundled
as well, focusing only on the equivalence relation and its equivalence proof.)

% We \emph{could} become over-eager and decide to unbundle the equivalence proof term as well:
% \begin{code}
% Record Monoid (A: Type) (R: A → A → Prop)
%   (e: Equivalence R) : Type := ...
% \end{code}
% However, this way if we happen to have two proofs |p| and |q| that a certain relation is an
% equivalence, we end up with two non-convertible ways to express that the relation (together with
% some unit and binary operation) form a Monoid: 'Monoid A R p ...' and 'Monoid A R q ...'. We must
% either try to deal with this (by proving that Monoid proofs with different Equivalence proofs are
% interchangable, and hope for automation, such as type class instance resolution, to use them
% interchangably), or take great care to ensure that |Equivalence| proof terms are always shared,
% essentially elevating them to `values', which goes against the principle of proof
% irrelevance.\marginpar{What if we add proof irrelevance?}
% BOTTOMLINE: Bundle as much as we can!
% 
% Neither of these options are very appealing. We really should just make |Monoid| take as
% parameters
% only the operations/relations. If we treat the monoid's unit and binary operation the same way,
% this
% means that |Monoid| ends up in |Prop|, and that its instances can be left opaque, which is really
% how things ought to be.

Another benefit of our approach is that it works very smoothly when these records are in fact type
classes. When relations and operations are bundled and do \emph{not} appear as parameters of the
structure record/class, to express that some collection of relations and
operations posess the structure as a type class constraint seems to require manifest fields; see
Section~\ref{manifest}. With
our unbundled approach, on the other hand, such type class constraints are easily expressed without
any extensions as type inhabitation queries
% (e.g. ``hey Coq, find me a proof of [Monoid some_rel some_op some_unit]'')
of types living in Prop, so that we don't even care what instances are found (thus embracing proof
irrelevance).

Leaving the equivalence proofs bundled means that when one has a proof that a structure
is a semiring, one has two proofs (via the two monoids) that the equivalence relation is an
equivalence. Fortunately, this redundancy is entirely harmless since our terms
never refer (directly or indirectly) to proofs, which is the case when the relations and operations
they are composed from are left unbundled. Here, too, we enjoy the warm embrace of proof
irrelevance.

From the discussion so far, we derive some principles:
\begin{enumerate}
 \item structural concepts (like Reflexive, Setoid, Monoid) should have their structural components
(i.e. carriers, relations, and operations) as parameters, and nothing else;
 \item (as a consequence) these concepts should be defined as predicates (which includes records and
classes living in Prop), and their proofs may be kept opaque;
 \item (as a second consequence) algebraic expressions need/should never refer (directly or
indirectly) to proofs about the algebraic structure.\footnote{The reciprocal in a field
famously does refer to a proof that the element is non-zero, but not to a proof that the operations
indeed form a field.}
\end{enumerate}
% \setcounter{enumi}{3}

The principles listed so far are not particularly controversial, in fact this
approach ensures maximum flexibility~\cite{Hints}. However,
there is a common perception that unbundling carriers, relations, and operations this way invariably
leads either to unmanagably big algebraic expressions (involving endless projections and/or implicit
arguments), or to unmanageably long and tedious enumeration of carriers/relations/operations, or to
problems establishing canonical names for operations (since we no longer project them out of
structure instances all the time).

In the next sections, we show how systematic use of type classes (and their support infrastructure
in Coq), combined with the use of a special form of type class we call an `operational type class'
(as opposed to a `structural' type class we might use in place of the records shown thus far), lets
us avoid these problems.

\section{Managing unbundled structure with type classes}

So far, we have only looked at the definitions of the structures themselves, and have not yet
considered the perspective of someone trying to prove some theorems about these abstract structures.
We would like to approximate the common mathematical vernacular, where one
simply says:

for $x, y, z$ in a semigroup $G$, $x * y * z = z * y * x$, $\ldots$

Suppose we have defined |SemiGroup| in line with the aforementioned principles:
\begin{code}
Record SemiGroup (A: Type) (eq: A → A → Prop)
 (op: A → A → A): Prop := ...
\end{code}
Then our starting point is the obvious but awful:
\begin{code}
Lemma perm A eq op (P: SemiGroup A eq op) x y z:
                   eq (op (op x y) z) (op (op z y) x).
\end{code}
As a first cleaning measure, we might hope to introduce some notations to let us use infix notation
for the semigroup operator (say, |&|) and for equality. Unfortunately, at this point the names |eq|
and |op| are just local names and there is no way to predeclare that they are to be associated with
any notation. What we really need are canonical names for these notions, so that we may fashion
notations for them.

This is a typical job for type classes, and they are used in this way in the |Clean| standard
library~\cite{Clean}. Similarly, Haskell uses type classes to overload notations.
We simply follow suit:
\begin{code}
  Class Equiv A := equiv: relation A.
  Class SemiGroupOp A := sg_op: A → A → A.

  Infix "=" := equiv: type_scope.
  Infix "&" := sg_op (at level 50, left associativity).
\end{code}
We choose to use |&| here and reserve the notation |*| for ring multiplication.

We use = for setoid equality, since Leibniz equality seems less important for the formalization of
mathematics. More precisely, setoid equality is the basic concept. In special case, one can prove
that setoid equality coincides with Leibniz equality.

We would have like to use a haskell style definition
\begin{code}
Class Equiv A := (=): relation A.
\end{code}
Unfortunately, currently this is not supported in Coq.

These single-field type classes are given special treatment in that they are not translated into
records the way classes with any other number of fields are, but are instead turned into ordinary
definitions. The effect of this is that |Equiv A| immediately reduces to |relation A|, and that the
equiv `projection' is the identity function whose argument is inferred using instance resolution.
Thus, for a very modest cost in indirection, we gain the ability to use canonical names, and with
it, notations:
\begin{code}
Record SemiGroup (A: Type) (eq: Equiv A)
                 (op: SemiGroupOp A): Prop := ...
Lemma perm A eq op (P: SemiGroup A eq op) x y z:
                 x & y & z = z & y & x.
\end{code}
We call these type classes \emph{operational type classes}. We note that operational type classes
allow us to avoid Coq's notation |scope| mechanism.

It is important to point out that although the expression now \emph{looks} like it is bound to some
specific semigroup structure (which with traditional approaches would imply projections from bundles
and/or reference to proofs), it is really only referring near-directly to the actual operations
involved, with the semigroup proof existing only as opaque `knowledge about the operations' which we
may use in the proof. This lack of projections keeps our terms small and independent, and keeps
rewriting simple and sane; see Section~\ref{canonical}.

On the other hand, it \emph{does} mean that all the carriers/relations/operations end up in the
context. The context for |perm| looks like:
\begin{code}
  A: Type
  eq: Equiv A
  op: SemiGroupOp A
  P: SemiGroup A eq op
\end{code}

While we are not particularly worried about overly large contexts (we feel that this will most
likely not be problematic for any but the most complex formalizations), having to \emph{declare}
these entities (like |eq| and |op| for |perm|) \emph{is} a bit of a chore. Fortunately, Coq
provides a feature called implicit generalization, which is exactly what we need.
With it, we can write |perm| as follows:
\begin{code}
  Lemma perm `(SemiGroup A) x y z: x & y & z = z & y & x.
\end{code}
Thus, we have reached the mathematical ideal we aimed for.

Let us now stop and think to what extent this approach suffers from all the problems commonly
associated with it. In particular, let us imagine what happens to our terms and contexts when we
want to talk about nested structures such as polynomials over the ring of matrices over a
given ring; see Section~\ref{interfaces}.

\section{Putting it all together}\label{classes}
Combining the ideas in the previous sections we end up with a formalization of algebraic
structures where we use the structure as (implicit) parameters, but
pack the proofs. This can be nicely organized by using canonical names. For instance, we define:
\begin{code}
Class SemiRing A {e: Equiv A}{plus: RingPlus A}{mult: RingMult A}
                 {zero: RingZero A}{one: RingOne A}:Prop :=
  { semiring_mult_monoid:> Monoid A (op:=mult)(unit:=one)
  ; semiring_plus_monoid:> Monoid A (op:=plus)(unit:=zero)
  ; semiring_plus_comm:> Commutative plus
  ; semiring_mult_comm:> Commutative mult
  ; semiring_distr:> Distribute mult plus
  ; mult_0_l: forall x, 0 * x == 0 }.
\end{code}
where e.g.\ |RingPlus| is the operation class for the semigroup operator.
The notation |:>| makes |semiring_mult_monoid| a coercion: when asked for a monoid it suffices to
provide a ring.

\emph{Structure-as-parameters} helps setoid-rewriting: type class resolution
can find the equivalence relation in the context.
% A similar style should be possible for, say, the |Ring| tactic, instead of
% declaring the ring structure by a separate command, we would rely on type class resolution to find
% it.
We note that |op| does not depend on the proof that |e| is an equivalence. As explained above we
use Coq's implicit quantification (|`{}|) to avoid having to write all the parameters when
\emph{stating} a theorem and Coq's maximally inserted implicit arguments to find the parameters when
\emph{applying} a theorem. Both features are new in Coq and stem from the type class implementation.

We mention the trade-off between bigger contexts versus bigger terms. Our contexts are bigger than
those of telescopes or packed classes. In our experience, this has been relatively
harmless\footnote{Coq's data structure for contexts is not very efficient. Gonthier fears that this
may be a bottleneck for huge developments. It seems that the data structure chosen
in~\cite{asperti2009compact} will behave better.}%
: most terms in the context are there to support canonical names. Bigger terms
\emph{do} cause problems: 1. when proofs are part of mathematical objects we need to share these
proofs to allow rewriting. Moreover, it prohibits Opaque proofs and `proof irrelevance'. 2. The
projection paths may not be canonical.

Coercion pullbacks~\cite{Hints} were introduced to address problems with multiple coercions paths,
as in the definition of a semiring: a type with two monoid structures on it. We avoid some
of these problems by explicitly specifying the fields. We emphasize that the semiring properties are automatically derived from the ring
properties, although the properties of a semiring are not structurally included in the ring
properties.

\subsection{Manifest fields}\label{manifest}

Manifest fields in records allow us to fix a field of a dependent record. For example, as in the
example above, |Monoid with op=mult|. 
Luo~\cite{DBLP:conf/types/Luo08} has proposed coersive subtyping to implement manifest fields.
It was conjectured~\cite{Hints} that implementing the algebraic hierarchy with type classes would
be awkward due to problems with equality. We avoided these problems by shifting the fields
from the dependent record to its argument, thus mimicking manifest fields by using type classes.
Since |op| is \emph{instantiated} with |mult|, we need not worry about the
equality |op=mult|. This seems to be a general phenomenon. For instance, instead of searching
for a |Monoid with op=mult| we would use type class resolution to find an instance of |Monoid mult|.
We have thus addressed the problem of multiple inheritance.

\section{Universal algebra}\label{univ}
After having implemented concrete algebraic structures, we generalize some of our constructions
by implementing many-sorted universal algebra. A formalization of universal algebra
already exists in Coq~\cite{DBLP:conf/tphol/Capretta99} and has recently been extended with
more categorical results~\cite{dominguez2008formalizing}. We have conveniently redeveloped most of
this using type classes. We define equational theories and their category of algebras. A quote
function helps to translate a statement for an instance of a type class to a statement in the
equational logic. We will come back to this in Section~\ref{quote}.

% \subsection{Universal algebra continued}\label{univ2}

We have developed basic universal algebra~\cite{meinke1993universal} upto the first homomorphism
theorem.~\marginpar{Description of the work.}
Discuss: the duplication of code which may have been avoided by universe
polymorphism for |Fixpoint|s.\marginpar{universe polymorphism}
%perhaps the soundness theorem for equational logic?

In order to prove the first homophism theorem, we define a congruence.
\begin{code}
Class Congruence: Prop :=
    { congruence_proper:> Π s, Proper (equiv ==> equiv ==> iff) (e s)
    ; congruence_quotient:> @Algebra et v e _
    }.
\end{code}
Alternatively, we could have state the compatibility of the relation |e| by stating that the
relation as a set of pairs is a subalgebra in the product algebra. We prove that both approaches
are equivalent.

A congruence relation is coarser than the setoid relation, as indicated by the use of |Proper|
above. Since our theory is founded on a congruence, the initial setoid equality, the development of 
congruences and quotients benefits from this.\marginpar{how}

We define the forgetful functor from the category of $\tau$-algebras to the category of sets.
We define the term algebra consisting of closed terms, terms with the empty type as set of
variables. The term algebra is the free $\tau$-algebra and hence\marginpar{?}
initial. Categorically: we have defined a left-adjoint to the forgetful functor. A formal proof
connects the categorical and the algebraic definition.
One may expect that this term algebra could be useful to
\emph{define}, say, the natural numbers as the term algebra for the theory of semiring. However, it
seems difficult to prove decidability of equality, which requires a normalization procedure.

The product of two $\tau$-algebras is their categorical product.

Given a sub\emph{set} of an algebra, we define the sub\emph{algebra} generated by it.
An interesting fact to mention is the use of the following heterogeneous equality between
elements of the algebra and of the subalgebra, i.e.\ terms of different types may be equal.
\begin{code}
  Fixpoint heq {o}: op_type carrier o -> op_type v o -> Prop :=
    match o with
    | constant x => fun a b => `a == b
    | function x y => fun a b => forall u, heq (a u) (b (`u))
    end.
\end{code}

We connect the abstract theory to the concrete theory. Concretely, let $\tau_m$ be the theory
of monoids. Given a monoid, we can construct the corresponding $\tau_m$-algebra and an
object in the category of $\tau$-algebras. Conversely, given an object in the category/variety, we
construct an instance of the |Monoid| type class.\marginpar{Update code}
%varieties/monoid.v
\begin{code}
Variable o: variety.Object theory.
Global Instance: SemiGroupOp (o tt) := algebra_op theory mult.
Global Instance: MonoidUnit (o tt) := algebra_op theory one.
Global Instance from_object: Monoid (o tt).
\end{code}
In the last line, instance resolution `automatically' finds the operation and the unit specified in
the lines before.\marginpar{Why this way?}

\subsection{Category theory}\label{cats}
Using type classes, we provide a basic library for category theory: categories, functors,
adjunctions, monads,\ldots We follow the usual type theoretical definition of a
category~\cite{saibi1995constructive}.\marginpar{Update code}

%format ◎ = "\circ"
%interfaces/abstract_algebra.v
\begin{code}
Class Category O `{!Arrows O} `{Π x y: O, Equiv (x ⟶ y)}
  `{!CatId O} `{!CatComp O}: Prop :=
  { arrow_equiv:> Π x y: O, Setoid (x ⟶ y)
  ; comp_proper:> Π x y z,
      Proper (equiv ==> equiv ==> equiv)%signature (@comp _ _ _ x y z)
  ; comp_assoc (w x y z: O) (a: w ⟶ x) (b: x ⟶ y) (c: y ⟶ z):
      c ◎ (b ◎ a) = (c ◎ b) ◎ a
  ; id_l `(a: x ⟶ y): cat_id ◎ a = a
  ; id_r `(a: x ⟶ y): a ◎ cat_id = a }.
\end{code}
This is based on the 2-categorical idea that we have equality on arrows, but not on objects.
Similarly, we have equality on natural transformations, but not on functors.

%theory/categories.v
Natural transformations are formalized as:
\begin{code}
  Context `{Category C} `{Category D}
  `{!Functor (F: C → D) Fa} `{!Functor (G: C → D) Ga}.

  Class NaturalTransformation (η: Π c, F c ⟶ G c): Prop :=
    natural: Π (x y: C) (f: x ⟶ y), η y ◎ fmap F f = fmap G f ◎ η x.
\end{code}

Here |fmap F| explicitly refers to the arrow part of the functor |F|.
In mathematics one would apply $F$ to both objects and arrows ($F x$, $F f$). However, this does
not seem to fit with type theory.\marginpar{Explain why.}

The proofs are short and follow the textbook easily.\marginpar{really?}

We have defined the category of setoids, the category of categories, the dual of a category,
products, and hence co-products and the category of $\tau$-algebras, for a given type $\tau$.

We provide the construction of the term-algebra for a type $\tau$ and show that it is
initial in the category of $\tau$-algebras. We proceed to prove that this provides a left-adjoint
to the forgetful functor. The resulting expression monad will be used in Section~\ref{quote} below.

\section{Interfaces using category theory}\label{interfaces}\label{modul}
The current work was stimulated by our desire~\cite{Riemann} to move between several
implementations in a flexible way. In haskell type class polymorphism is used for this purpose.
Hence, we do the same in type theory.

We have characterized the naturals as the initial object in the category of semirings and derived
many of their properties from this interface. Unary, binary and machine numbers are
instances of this interface, so we can directly apply these results to all these instances.
Similarly, the integers are the initial ring. The rationals are the field of fractions of the
integers. 
Given a ring $R$, the $R$-algebra $R[X]$ of polynomials, is the free $R$-algebra on a set $X$.
We provide two implementations of polynomials: the
standard representation using lists of coefficients and the Bernstein representation. A Bernstein
basis polynomial is one of the form:
\[b_{\nu,n}(x) = {n \choose \nu} x^{\nu} \left( 1 - x \right)^{n - \nu}, \quad \nu = 0, \ldots, n.\]
A Bernstein polynomial is a linear combination of these basic polynomials. Bernstein polynomials
have been used for efficient computations inside the Coq system~\cite{ZumkellerPhD}.

We encourage more efficient implementations by assigning the default implementation a
low priority. For example, the distance function on the natural numbers, which is derived from its
semiring structure, is assigned priority 10.
\begin{code}
  Global Program Instance: NatDistance N | 10 := ...
\end{code}

%There is a slight problem when defining the forgetful functor from the category of
%$\Sigma$-algebras to Sets. This problem is caused by our use of the category of all
%categories\footnote{Perhaps we should have used higher categories for this.}.
%We conjecture that this problem disappears when Coq's universe polymorphism is extended to
%definitions.

\section{Type class quoting}\label{quote}
Unification hints~\cite{Hints} allow one to semi-automatically construct a quote
function\footnote{Gonthier provides similar functionality using canonical structures}. However,
this feature is absent from Coq. Fortunately, type classes provide similar functionality as
we will now show.

We define a term language for monoids:
\begin{code}
Inductive Expr (V: Type) := Mult (a b: Expr V) | Zero | Var (v: V).
\end{code}
This defines a monad on the category of |Type|s with extensional functions.\marginpar{Proof this
formally?}
The expression type is parameterized over the set of variable indices. Hence, we diverge
 from~\cite{Hints}, which uses |nat| for variables thereby introducing bounds problems and
 dummy variables.

\noindent An expression is only meaningful in the context of a variable
assignment:\footnote{Categorically: |nat| is
an |Expr|-algebra and |eval| is |map vars| composed with the algebra map.}
\begin{code}
Definition Value := nat.
Definition Vars V := V -> Value.

Fixpoint eval {V} (vs: Vars V) (e: Expr V): Value :=
  match e with
  | Zero => 0
  | Mult a b => eval vs a * eval vs b
  | Var v => vs v
  end.
\end{code}
%
Monads are trees with grafting~\cite{MonadsGrafting}: the tree monad
$TX:=1+T^2X$, may be seen as the prototypical monad. Indeed, the |Expr| monad, i.e.\ the free
|Expr|-algebra monad, behaves much like the tree monad. We now provide a direct definition of the
corresponding bind operation, the arrows in the Kleisli category.

\begin{code}
Fixpoint bind {V W} (f: V -> Expr W) (e: Expr V): Expr W :=
  match e with
  | Zero => Zero
  | Mult a b => Mult (bind f a) (bind f b)
  | Var v => vs v
  end.
\end{code}

% We have shown that |Expr|, i.e.\ |bind| together with |Var|, is a monad on the category of Types
% with extensional functions between them~\marginpar{Really?}.

\noindent We define some simple combinators for variable packs:
%
\begin{code}
Definition novars: Vars False := False_rect _.
Definition singlevar (x: Value): Vars unit := fun _ => x.
Definition merge {A B} (a: Vars A) (b: Vars B): Vars (A+B) :=
  fun i => if i then a j else b j.
\end{code}

\noindent These last two combinators are the `constructors' of an implicitly defined subset of
 Gallina terms, representing heaps, for which we implement syntactic lookup with type classes.
The heap can also be defined explitly, with no essential change in the code.
Given a heap and value, |Lookup| instances give the value's index in the heap:
% \marginpar{A
% canonical structures approach would allow us to do this in Coq?}
\begin{code}
  Class Lookup {A} (x: Value) (f: Vars A) :=
    { lookup: A; lookup_correct: f lookup = x }.
\end{code}
% Context (x: Value) {A B} (va: Vars A) (vb: Vars B).

If the heap is a merge of two heaps and we can find the value's index in the left heap, we can
access it by indexing the merged heap (and vice versa)
\begin{code}
  Global Instance lookup_left `{!Lookup x va}: Lookup x (merge va vb)
    := { lookup := inl (lookup x va) }.
\end{code}
If the heap is just a singlevar, we can easily index it.
\begin{code}
  Global Program Instance: Lookup x (singlevar x) := { lookup := tt }.
\end{code}

One useful operation we need before we get to Quote relates to variables and expression
 evaluation. As its name suggests, |map_var| maps an expression's variable indices.
This reindexing function is the the |map| of the |Expr|-monad.

\begin{code}
Definition map_var {V W: Type} (f: V -> W):
    Expr V -> Expr W :=
  fix F (e: Expr V): Expr W :=
    match e with
    | Mult a b => Mult (F a) (F b)
    | Zero => Zero
    | Var v => Var (f v)
    end.
\end{code}
In |Quote| below, the idea is that |V, l|, and |n| are all input variables, while |V'| and |r| are
output variables (in the sense that we will rely on unification to generate them). |V| and |l|
represent the current heap, |n| represents the value we want to quote, and |V'| and |r'| represent
the heap of newly encountered variables during the quotation.
  This explains the type of quote: it is an expression that refers either to variables from
the old heap, or to newly encountered variables. Finally, |eval_quote| is the usual correctness
property, which now merges the two heaps.

\begin{code}
  Class Quote {V} (l: Vars V) (n: Value) {V'} (r: Vars V') :=
    { quote: Expr (V + V')
    ; eval_quote: @eval (V+V') (merge l r) quote = n }.
\end{code}

Our first instance for |Zero| is easy. The `novars' in the result type reflects the fact that no
new variables are encountered.
\begin{code}
  Global Program Instance quote_zero V (v: Vars V):
    Quote v 0 novars :=
  { quote := Zero }.
\end{code}

The instance for multiplication is a bit more complex. The first line consists of
 variable declarations. The second line is important. `Quote x y z' must be read as
 `quoting y with existing heap x generates new heap z', so the second line basically just
shuffles heaps around.
 The third line contains some |map_var|'s because the heap shuffling must be
reflected in the variable indices, but apart from that it's just constructing a |Mult| term with
quoted subterms.

\begin{code}
Global Program Instance quote_mult V (v: Vars V)
  n V' (v': Vars V') m V'' (v'': Vars V'')
  `{!Quote v n v'} `{!Quote (merge v v') m v''}:
  Quote v (n * m) (merge v' v'') :=
  { quote := Mult (map_var bla (quote n)) (map_var sum_assoc (quote m)) }.
\end{code}
  Now follows the instance where we recognize values that are already in the heap. This
   is expressed by the Lookup requirement, which will only be fulfilled if the Lookup instances
   defined above can find the value in the heap. The novars in the |Quote v x novars| result
   reflects that this quotation does not generate new variables.

\begin{code}
Global Program Instance quote_old_var
  V (v: Vars V) x {i: Lookup x v}:
  Quote v x novars | 8 := { quote := Var (inl (lookup x v)) }.
\end{code}

\noindent Finally, the instance for new variables. We give this a lower priority so that it is only
used if Lookup fails. The |8| in the previous example is a random number less than |9| below. A
similar method is used in Coq's notation mechanism.
% We recommend a change in
% implementation supporting an agda style precedence relation which is only restricted to be a
% directed acyclic graph~\cite{danielsson2009parsing}.

\begin{code}
Global Program Instance quote_new_var V (v: Vars V) x:
  Quote v x (singlevar x) | 9 := { quote := Var (inr tt) }.
\end{code}

\noindent As announced this provides a light-weight quoting function\footnote{|eval_quote'| is a
modification of |eval_quote| which works in the empty context.}:
\begin{code}
Goal forall x y (P: Value -> Prop), P ((x * y) * (x * 0)).
  intros. rewrite <- (eval_quote' _).
\end{code}
This turns the goal into | (eval variable_pack quote)|.

It would be interesting to use this technique to implement the |congruence| tactic, which implements
the congruence closure algorithm~\cite{corbineau2007deciding}. One would naturally obtain a tactic
which moreover works on setoid equalities.

\section{Canonical structures}\label{canonical}
Packed classes use canonical structures for the algebraic hierarchy. Both canonical structures and
type classes may be seen as instances of hints in unification~\cite{Hints}. Some uses of canonical
structures can be replaced by type class instances. The user manual (2.7.15) uses canonical
structures to derive the setoid equality on the natural
numbers in the following example |(S x)==(S y)|. In this case type classes provide superior proof
terms. Canonical structures give
\begin{code}
@equiv(Build_Setoid nat (@eq nat)(@eq_equivalence nat))(S x)(S y)
\end{code}
which includes an explicit proof that |(@eq nat)| is an equivalence,
whereas we obtain |@equiv nat (@eq nat) (S x) (S y)|.

\subsection{Big operators}
Canonical structures have been used to provide a uniform treatment~\cite{bertot2008canonical} of big
operators (like $\Pi,\sum, \max$). These operators extend a pair of a binary and a 0-ary operation
to an $n$-ary operation for any $n$. Categorically, one considers the algebra maps from non-empty
lists, lists, inhabited finite sets and finite sets to the carrier of a semigroup, monoid,
commutative semigroup, commutative monoid. Hence we want to reuse the libraries for lists etc.

We use type classes to deduce the relevant monoid operation and write:
\begin{code}
Definition seq_sum
  `{Sequence A T} `{RingPlus A} `{z: RingZero A}: T -> A
  := @seq_to_monoid A T _ A ring_plus z id.
  Eval compute in seq_sum [3; 2].
\end{code}
\marginpar{distributivity?}

\section{Subset types and proof irrelevance}\label{PI}
We have mentioned proof irrelevance a number of times before. It is an important theme in our
approach. The |Program| machinery allows one to conveniently write programs with Hoare-style
pre- and post-conditions. I.e.\ functions $f: \{ A \mid P \} \to \{ B \mid Q \}$. Both side
conditions
are intended to be proof irrelevant, they are in |Prop|. Presently, Coq does not support such subset
types, thus forcing the user to manually prove that |f| does not depend on the proof of |P|.
The addition of proof irrelevance and subset types to Coq is pursued by
Barras~\cite{Barras:subset,Werner}. 
% Squash is a monad and allows many constructions: subsets, unions, images, heterogeneous equality,
% \ldots.

Pollack~\cite{pollack2000dependently} introduces record types with telescopes. 
Given a telescope\[
T=[x_1:A_1][x_2:A_2(x_1)]\cdots[x_k:A_k(x_1,\ldots,x_k)].
\]
There is an inductively defined type $\Sigma T$, with a single constructor, given by the following 
formation and introduction rule in pseudo-code for $k=2$:
\begin{align*}
 A_1 :& Type\\
 A_2 :& \Pi A_1,Type\\
\cline{1-2}
\Sigma T :& Type
\end{align*}

\begin{align*}
 \Sigma T:Type\quad a_1 &:A_1 \quad a_2 : A_2(a_1)\\
\cline{1-3}
(a_1,a_2) &:\Sigma T
\end{align*}

For our methodology to work we have to be able to transform the occuring record into a telescopic
subset $\{ A \mid  B\}$. We moreover conjecture that the terms in |B| only depend on the
variables in |A|, but not in |B| itself. However, we do not need this.

\subsection{Propositions as [ ]-Types}
As originally concieved~\cite{ITT,CMCP}, Propositions are Types and we can extract information from
them. In practice, it is convenient to assume that all proofs of a proposition are equal: the
principle of proof irrelevance. These two views clearly contradict eachother. In Coq, a middle way
was choosen, we are forbidden to extract information from proofs, but proofs are not identified.
This allows us to extract the program we want, but we want more:
We would like to use Coq as a dependently typed programming language. To do so,
one would like to add the principle of proof irrelevance to Coq. A promissing proposal is the
Calculus of implicit constructions with implicit
$\Sigma$-types~\cite{miquel2001implicit,barras2008implicit,barrasSigma}. Such a change would fit
naturally with the Hoare-style separation between programs and proofs which is used by the |Program|
machinery. 

One may be tempted to propose Propositions as [ ]-types~\cite{awodey2004propositions}. Here the
construction $[P]:=\{\top\mid p:P\}$ squashes all the information about how we proved $P$. In Coq,
the [ ]-types are isomorphic to the type of sort |Prop|. However,
there are places where we needed proof \emph{relevance}; e.g.\ in the proof of the first isomorphism
theorem above.In practice, manual dead-code analysis seems useful. Making Harrop formulas proof
irrelevant seems to be a good first approximation~\cite{lcf:spi:03}. However, a further refinement
of Propositions as types seems necessary.

\section{Conclusions}
Telescopes have been criticized~\cite{Packed} for the lack of multiple inheritance and
the efficiency penalty of a long chain of coercion projections. Packed classes~\cite{Packed} provide
a solution to these problems. We provide an alternative solution. We have already
discussed multiple inheritance in Section~\ref{classes}. Moreover, by exposing the
carrier we avoid the chains of projections.

We conclude that unification hints, canonical structures and type classes all appear to have
similar expressive powers. Canonical structures, being tied to the unifier, are more robust, but
seem to require more inguinity. Type classes are easier to use. Unification hints, may be a
good generalization, but are absent from Coq. We encourage their inclusion to the
Coq system.

An obvious topic for future research is the extension from equational logic to partial Horn
logic~\cite{palmgren2007partial}. Another topic would be to fully, but practically, embrase the
categorical approach to universal algebra~\cite{pitts2001categorical}.

According to |coqwc|, our development consists of 3103 lines of specifications and 707 lines of
proofs.\marginpar{update,targz}

% Alt-Ergo congruence closure parametrized by an equational theory.
% As usual either implement in Coq (verify the algorithm), or Check traces, or Check Alt-Ergo (Work
% in progress).

% http://lara.epfl.ch/dokuwiki/sav09:congruence_closure_algorithm_and_correctness
% I understand that this is the algorithm underlying congruence and perhaps first-order.
% 
% Type classes vs soft typing (Mizar types).
% Type classes and types in homalg?

\paragraph{Acknowledgements.}
This research was stimulated by the new possibilities provided by the introduction of type classes
in Coq. In fact, this seems to be the first substantial development that tries to exploid all their
possibilities. As a consequence, we found many small bugs and unintended behavior in the type
class implementation. All these issues were quickly solved by Matthieu Sozeau. Discussions with
Georges Gonthier and Claudio Sacerdoti Coen have helped to
sharpen our understanding of the relation with Canonical Structures and with Unification Hints.
%Jeremy, Thierry, Claudio, Georges?, Wouter
\bibliographystyle{plain}
\bibliography{alg}
\end{document}
