\documentclass{article}
\usepackage{amssymb,url}

\usepackage{theorem}
\usepackage{enumerate}
\usepackage{ifthen}
\usepackage[all]{xy}

\theoremheaderfont{\scshape} \theorembodyfont{\upshape}
\newtheorem{definition}{Definition}[section]
\newtheorem{theorem}[definition]{Theorem}
\newtheorem{lemma}[definition]{Lemma}
\newtheorem{corollary}[definition]{Corollary}
\newtheorem{proposition}[definition]{Proposition}
\newtheorem{example}[definition]{Example}
\newtheorem{notation}[definition]{Notation}
\newtheorem{remark}[definition]{Remark}
\newenvironment{proof}{\smallskip\textsc{Proof.}}{\hspace*{\fill}$\Box$}

\newcommand{\D}{\textsf D}
\newcommand{\PP}{\textsf P}
\newcommand{\T}{\textsf T}
\newcommand{\X}{\textsf X}
\newcommand{\FOL}{\textsf{FOL}}

\newcommand{\NN}{\ensuremath{\mathbb N}}

\newcommand{\alt}{\mathrel{|}}
\newcommand{\ifte}{\textsf{if-then-else}}
\newcommand{\Ifte}{\textsf{If-then-else}}
\newcommand{\ifthelse}[3]{\ensuremath{\mathsf{if}\ {#1}\ \mathsf{then}\ {#2}\ \mathsf{else}\ {#3}}}
\newcommand{\starfun}[1]{\ensuremath{{#1}^\ast}}
\newcommand{\starmap}{\starfun\cdot}
\newcommand{\modelfunm}[1]{\ensuremath{\mathfrak{#1}_\ast}}
\newcommand{\modelfun}[1]{\ensuremath{{#1}_\ast}}
\newcommand{\modelmap}{\modelfun\cdot}
\newcommand{\restrfunm}[1]{\ensuremath{\mathfrak{#1}_|}}
\newcommand{\restrfun}[1]{\ensuremath{{#1}_|}}
\newcommand{\restrmap}{\restrfun\cdot}
\newcommand{\ofun}[1]{\ensuremath{{#1}^\circ}}
\newcommand{\omap}{\ofun\cdot}

\newcommand{\ok}{\mbox{\em wf}}
\newcommand{\wok}{\mbox{\em wwf}}
\newcommand{\wf}{\ \mbox{\em wf}}
\newcommand{\wwf}{\ \mbox{\em wwf}}

\newcommand{\pair}[2]{\ensuremath{\langle{#1},{#2}\rangle}}
\newcommand{\partto}{\ensuremath{\not\to}}
\newcommand{\defined}{\ensuremath{\!\downarrow}}
\newcommand{\undefined}{\ensuremath{\!\uparrow}}

\newcommand{\lang}[1]{\ensuremath{{\cal L}_{\mathsf{#1}}}}
\newcommand{\terms}[1]{\ensuremath{{\cal T}_{\mathsf{#1}}}}
\newcommand{\judg}[1]{\ensuremath{{\cal J}_{\mathsf{#1}}}}
\newcommand{\myvdash}[1]{\ensuremath{\vdash^{\mathsf{#1}}}}

\newcommand{\intm}[3]{\ensuremath{[\![{#3}]\!]^{\mathsf{#1}}_{\mathfrak{#2}}}}
\newcommand{\ints}[4]{\ensuremath{[\![{#4}]\!]^{\mathsf{#1}}_{\mathfrak{#2},{#3}}}}
\newcommand{\mymodels}[1]{\ensuremath{\models^{\mathsf{#1}}}}
\newcommand{\mymodelsm}[2]{\ensuremath{\models_{\mathfrak{#2}}^{\mathsf{#1}}}}
\newcommand{\mymodelss}[3]{\ensuremath{\models_{\mathfrak{#2},{#3}}^{\mathsf{#1}}}}
\newcommand{\yields}[1]{\ensuremath{\models^{\mathsf{#1}}}}
\newcommand{\synDC}[1][{}]{\ifthenelse{\equal{#1}{}}{%
{\ensuremath{{\cal{DC}}}}}{%
{\ensuremath{{\cal{DC}}_{#1}}}}}
\newcommand{\semDC}[2][{\mathfrak M},\rho]{\ensuremath{\overline{\cal{DC}}^{#1}_{#2}}}
\newcommand{\semDCa}[1]{\semDC[{\modelfunm M,\rho}]{#1}}

%% Freek's definitions for derivation rules (iaiks!)

\def\namerule#1{({\it #1\/})}
\def\rulex#1#2{\frac{\displaystyle\strut #1}{\displaystyle\strut #2}}
\def\rulexn#1#2#3{\makebox[0pt][r]{\hss\namerule{#1}$\;\;$}\frac{\displaystyle\strut #2}{\displaystyle\strut #3}}
\def\rulexnc#1#2#3#4{\makebox[0pt][r]{\hss\namerule{#1}$\;\;$}\frac{\displaystyle\strut #2}{\displaystyle\strut #3}\mrlap{#4}}
\def\mrlap#1{\makebox[0pt][l]{$\displaystyle\;\;#1$\hss}}
\def\rulexn#1#2#3{\mbox{\namerule{#1}}\;\frac{\displaystyle\strut #2}{\displaystyle\strut #3}}
\def\rulexnc#1#2#3#4{\mbox{\namerule{#1}}\;\frac{\displaystyle\strut #2}{\displaystyle\strut #3}\mrlap{#4}}
\def\mrlap#1{{\displaystyle\;#1}}
\def\sep{.\,}
\def\lmarg#1{\leqno{\hbox{\normalsize\hspace{-2em}$#1$:}}}

\begin{document}

\title{Partial First-Order Logic}
\author{Lu\'\i s Cruz-Filipe \& Herman Geuvers \& Freek Wiedijk}
\maketitle

\begin{abstract}
Still empty.
\end{abstract}

\section{Introduction}
In this paper we study the system for {\em partial\/} first order
logic {\D} \cite{wie:zwa:03} and relate it to (the well-known) first
order logic. The partiality of the system {\D} lies in the fact that
functions can be partial, as opposed to first order logic, where all
functions are total. There are various ways of making this partiality
formally precise, e.g.\ one can introduce a separate unary predicate
$E$, that holds for a term $t$ iff $t$ denotes something (the term $t$
``exists''). Then one can still write down, e.g.\ $\frac{1}{0}$, but
it doesn't denote anything, so $\neg E(\frac{1}{0})$. This is studied
in detail in so called $E$-logic, see \cite{sco:79,bee:85}. In our approach,
we avoid terms like $\frac{1}{0}$ alltogether by introducing for every
function symbol $f$ a domain predicate $D_f$ and putting a restriction
on the first order language: $f(t)$ is a well-formed expression only
if $D_f(t)$ holds. So, compared to $E$-logic, we have moved the issue
of ``well-definedness of expressions'' from the {\em logical\/} level
to the {\em linguistic\/} (pre-logical) level.

The reason for moving from a logical to a linguistic level is
motivated by mathematical practice. In mathematics, if a function is
partial, one just doesn't want to speak about the application of that
function to an element that is outside its domain, like
$\frac{1}{0}$. The assertion that $\frac{1}{0}$ is undefined is not
seen as a logical statement (like $0<1$), but as a meta-statement
about expressions. Type theory does credit to that point of view by
disallowing the term $\frac{1}{0}$. This is made formally precise by
letting the division function take {\em three\/} arguments: a
numerator $x$, a denominator $y$ and a {\em proof that the denominator
$y$ is not $0$}. So $x$ devided by $y$ then looks like
$\frac{x}{\langle y, p\rangle}$, where $p$ is a proof that $y\neq
0$. But now we are in the situation that a term depends on a proof,
which is also not fully convincing. Moreover, it is not completely
clear how such a system elates to well-known first order predicate
logic.

The system {\D} was introduced in \cite{wie:zwa:03} to bridge the gap
between first order predicate logic and type theory. It does not deal
with ``proof objects'', like the $p$ above that is needed in a type
theoetical system to make he division well-formed. But it does
disallow $\frac{x}{y}$ if we can't prove that $y\neq 0$. In \cite
{wie:zwa:03}, the connection between {\D} and its type-theoretical
counterpart (called {\PP} in that paper) was made precise. The
connection between {\D} and first-order logic was only partially
established, by rlating {\D} to the system {\T}, a first order logic
where all functions are total, but which has, e.g., an additional
\ifte\ construct. This construction introduces terms of the form 
$\ifthelse{\varphi}{t_1}{t_2}$, meaning to denote the term $t_1$ if
$\varphi$ holds ans $t_2$ if $\neg\varphi$ holds. But this introduces
formulas in the first order term language, a feature alien to ordinary
first order logic.  The connection betwen {\T} and first order logic
was only sketched in
\cite{wie:zwa:03}. In the present article we give a full description
of the connection between {\T} and first order logic. This connection
is made by defining a map $\omap$ from {\T} to first-order logic that
preserves derivability and is an equivalence on first order formulas.

We also compare {\D} and first order logic from a model-theoretic
perspective. We define a model notion for {\D} and relate {\D}-models to
first order models. This allows to prove a completeness result for {\D}
(w.r.t.\ the notion of {\D}-model). 

\subsection{Background and Approach}
In first order logic, there are no partial functions. As a matter of
fact, functions play a minor role in first order logic: to {\em
define\/} a (unary) function is understood as defining a formula
$\psi(x,y)$ such that $\forall x \exists ! y( \psi(x,y))$. So,
functions in first order logic are defined as relations: functions are
identified with their graphs. Similarly a partial funtion can be
defined as a relation $\psi(x,y)$ such that $\forall x \forall y, y'(
\psi(x,y)\wedge \psi(x,y') \rightarrow y = y')$. In mathematics
however, functions are (also) viewed as ``procedures'' for generating
(computing) the output from the input. To support that one usually
introduces a function symbol $f$ to write $f(t)$ for the unique $y$
such that $\psi(t,y)$ holds. That's fine in the case of total
funtions, but in the case of partal functions $f(t)$ may not denote
anything.

The issue of partiality becomes more apparent when one want to use a
theorem prover to do first order logic on a computer, to do formal
verification. In \cite{wie:zwa:03} the options for dealing wih
partiality in theorem provers are discussed, based on an analysis of
\cite{harrison}. We summarize the options here.
\begin{itemize}
\item Make all functions total, either by chooing a specific 
value for the function when it is applied outside its domain, or by
letting the function have an arbitrary (unknown) value outside its
domain.
\item Restricting the applicability of a function to its domain; 
applying the function to an element outside the domain is a type
error.
\item Having a logic with total and partial terms.
\end{itemize}
The difference between the third and the second alternative is that in
the second one cannot write down $\frac{1}{0}$, whereas in the third
one can.

When implementing a theorem prover, one has to make a choice of how to
deal with partiality. ACL2, HOL and Isabelle use the first approach,
while Coq, NuPRL and PVS use the second. The third approach is used by
IMPS. The system {\PP} of \cite{wie:zwa:03} is inspired by the type
theoretic way of dealing with partiality (used in systems like Coq and
NuPRL). The reciprocal function there has the following type
$$\frac{1}{-} : \Pi x{:}F (x\neq 0) \rightarrow F,$$ 

so the reciprocal takes an argument $x:F$ ($F$ is some field) and a
{\em proof\/} of $x\neq 0$ to produce a value in $F$. This means that
the reciprocal can only be applied to an element if we also give it a
proof that the element is different from $0$. In PVS, something
similar happens. The type of the reciprocal function in PVS is as
follows.  
$$\frac{1}{-} : \{x{:}F \mid x\neq 0\} \rightarrow F,$$

so the reciprocal function can only be applied to an element if we can
prove that the element is different from $0$. The difference is that
now this proof is not passed as an argument to the reciprocal
function. The system {\D} of \cite{wie:zwa:03} uses this approach: the
proof is not passed as an argument to the function, but applying the
reciprocal to $0$ does yield a type error (and is thus forbidden).

{\bf To Do} Say something about type theory, that all vars have to be declared and that that can be  a source of undefinedness. Hence we have four possible values and we analyze this in systems {\D} and {\T}

{\bf To Do} Update biblio


\section{The three systems}

In this section we briefly summarize the definitions of systems
{\FOL}, {\D} and {\T}.

Throughout this paper we assume a fixed signature $\Sigma$ containing:
\begin{itemize}
\item finitely many constant symbols $c_1,\ldots,c_k$;
\item finitely many function symbols $f_1,\ldots,f_n$ with arities
respectively $a_1,\ldots,a_n$;
\item finitely many predicate symbols $P_1,\ldots,P_m$ with arities
$r_1,\ldots,r_m$.
\end{itemize}

To each function symbol $f$ there is a predicate symbol $D_f$
associated with the same arity.  (This $D_f$ is one of the $P_i$s.)
Intuitively, $D_f$ is the predicate that represents the domain of $f$;
however this will only be explicit in system {\D}, as will be seen.
In {\FOL} and {\T} the $D_f$s play the role of `meta-information' which
will be essential when we want to state the equivalence between these
systems and {\D}.

Finally we assume a denumerable set of variables $x_0$, $x_1$, $x_2$,\ldots

\subsection{System {\FOL}}

System {\FOL} is simply first-order logic, so we will refrain from
commenting the definitions in this subsection.

\begin{definition}\label{defn:FOLterms}
The set of {\FOL}-terms is defined inductively as follows:
\begin{enumerate}[(i)]
\item $x_i$ is a term for any $i\in\NN$;
\item $c_i$ is a term for any $i=1,\ldots,k$ (with $k\geq1$);
\item if $f_i$ is a function symbol and $t_1,\ldots,t_{a_i}$ are terms
then $f_i(t_1,\ldots,t_{a_i})$ is a term.
\end{enumerate}
The set of {\FOL}-terms is denoted as {\terms{FOL}}; arbitrary terms are
denoted by $t$, $t'$, $t''$, $t_1$,\ldots
\end{definition}

\begin{definition}\label{defn:FOLformulas}
The set of {\FOL}-formulas is defined inductively as follows:
\begin{enumerate}[(i)]
\item $\bot$ is a formula;
\item if $P_i$ is a function symbol and $t_1,\ldots,t_{r_i}$ are terms
then $P_i(t_1,\ldots,t_{r_i})$ is a formula;
\item if $t_1$ and $t_2$ are terms then $t_1=t_2$ is a formula;
\item if $\varphi$ and $\psi$ are formulas then so are $\varphi\to\psi$
and $\forall x_i\sep\varphi$.
\end{enumerate}
The set of {\FOL}-formulas is denoted as {\lang{FOL}}; arbitrary formulas
are denoted by $\varphi$, $\psi$, $\varphi'$, $\psi''$, $\varphi_1$,\ldots
\end{definition}

\begin{definition}\label{defn:FOLcontext}
A {\FOL}-context is a list of {\FOL}-formulas.  The empty context is denoted
by $\epsilon$; arbitrary contexts will be denoted by $\Gamma$, $\Delta$,
$\Gamma'$, $\Delta_1$,\ldots
\end{definition}

\begin{definition}\label{defn:FOLderivation}\label{defn:FOLjudgement}
A {\FOL}-judgement is a statement of the form $\Gamma\myvdash{FOL}\varphi$.
A judgement is \emph{valid} iff it can be derived from the following rules.

$$\rulexnc{assum}{}{\Gamma\myvdash{FOL}\varphi}{\varphi\in\Gamma}
\quad
\rulexn{$\neg\neg$-E}{\Gamma\myvdash{FOL}\neg\neg\varphi}{\Gamma\myvdash{FOL}\varphi}$$
$$\rulexn{$\to$-I}{\Gamma,\varphi\myvdash{FOL}\psi}{\Gamma\myvdash{FOL}(\varphi\to\psi)}
\quad
\rulexn{$\to$-E}{\Gamma\myvdash{FOL}(\varphi\to\psi) \quad \Gamma\myvdash{FOL}\varphi}
                {\Gamma\myvdash{FOL}\psi}$$
$$\rulexnc{$\forall$-I}{\Gamma\myvdash{FOL}\varphi}
   {\Gamma\myvdash{FOL}(\forall x_i\sep\varphi)}{x_i\not\in FV(\Gamma)}
\quad
\rulexnc{$\forall$-E}{\Gamma\myvdash{FOL}(\forall x_i\sep\varphi)}
                     {\Gamma\myvdash{FOL}\varphi[x_i:=t]}{***}$$
$$\rulexn{refl}{}{\Gamma\myvdash{FOL} t=t}
\quad
\rulexn{sym}{\Gamma\myvdash{FOL} t=t'}{\Gamma\myvdash{FOL} t'=t}
\quad
\rulexn{trans}{\Gamma\myvdash{FOL} t_1=t_2 \quad \Gamma\myvdash{FOL} t_2=t_3}
              {\Gamma\myvdash{FOL} t_1=t_3}$$
$$\rulexn{$=$-fun}
   {\Gamma\myvdash{FOL} t_1=t'_1 \quad \cdots \quad \Gamma\myvdash{FOL} t_{a_i}=t'_{a_i}}
   {\Gamma\myvdash{FOL} f_i(t_1,\ldots, t_{a_i})=f_i(t'_1,\ldots, t'_{a_i})}$$
$$\rulexn{$=$-pred}
   {\Gamma\myvdash{FOL} t_1=t'_1 \quad \cdots \quad \Gamma\myvdash{FOL} t_{r_i}=t'_{r_i}}
   {\Gamma\myvdash{FOL} P_i(t_1,\ldots, t_{r_i}) \to P_i(t'_1,\ldots, t'_{r_i})}$$
\end{definition}

\begin{definition}\label{defn:FOLmodel} A {\FOL}-model $\mathfrak M$
is a tuple $\mathfrak M=\langle A,F,P,C\rangle$ where:
\begin{enumerate}[--]
\item $A$ is a non-empty set (the domain);
\item $F=\{\intm{FOL}M{f_1},\ldots,\intm{FOL}M{f_n}\}$ where
$\intm{FOL}M{f_i}:A^{a_i}\to A$ for each $i=1,\ldots,n$;
\item $P=\{\intm{FOL}M{P_1},\ldots,\intm{FOL}M{P_m}\}$ where
$\intm{FOL}M{P_i}\subseteq A^{r_i}$ for each $i=1,\ldots,m$;
\item $C=\{\intm{FOL}M{c_1},\ldots,\intm{FOL}M{c_k}\}\subseteq A$.
\end{enumerate}
A {\FOL}-substitution for $\mathfrak M$ is a function $\rho$ that assigns
a value in $A$ to each variable $x_i$.
\end{definition}

\begin{definition}\label{defn:FOLinterpretation}
Let $\mathfrak M$ be a {\FOL}-model and $\rho$ be a {\FOL}-substitution for
$\mathfrak M$.  The interpretation of a term $t\in\lang T$ is
recursively defined as follows.
\begin{eqnarray*}
\ints{FOL}M\rho{x_i} & := & \rho(x_i) \\
\ints{FOL}M\rho{c_i} & := & \intm{FOL}M{c_i} \\
\ints{FOL}M\rho{f_i(t_1,\ldots,t_{a_i})} & := & \intm{FOL}M{f_i}
  \left(\ints{FOL}M\rho{t_1},\ldots,\ints{FOL}M\rho{t_{a_i}}\right)
\end{eqnarray*}
\end{definition}

\begin{definition}\label{defn:FOLsatisfaction}
Let $\mathfrak M$ be a {\FOL}-model and $\rho$ be a {\FOL}-substitution
substitution for $\mathfrak M$.  Satisfaction of a formula by $\mathfrak M$
and $\rho$, $\mymodelss{FOL}M\rho\varphi$, is recursively defined as follows.
\begin{eqnarray*}
\not\!{\mymodelss{FOL}M\rho}\bot \\
\mymodelss{FOL}M\rho P_i(t_1,\ldots,t_{r_i}) & \mbox{iff} &
 \ints{FOL}M\rho{P_i}\left(
   \ints{FOL}M\rho{t_1},\ldots,\ints{FOL}M\rho{t_{r_i}}\right) \\
\mymodelss{FOL}M\rho t_1=t_2 & \mbox{iff} &
 \ints{FOL}M\rho{t_1}=\ints{FOL}M\rho{t_2} \\
\mymodelss{FOL}M\rho\varphi\to\psi & \mbox{iff} &
 \not\!{\mymodelss{FOL}M\rho\varphi} \mbox{ or } \mymodelss{FOL}M\rho\psi \\
\mymodelss{FOL}M\rho\forall x_i\sep\varphi & \mbox{iff} &
 \mymodelss{FOL}M{\rho[x_i:=a]}\varphi\mbox{ for all $a\in A$}
\end{eqnarray*}
\end{definition}

\begin{definition}\label{defn:FOLvalidity}\label{defn:FOLconsequence}
Let $\mathfrak M$ be a {\FOL}-model, $\Gamma$ a context and $\varphi$
a formula.
\begin{enumerate}[(i)]
\item $\varphi$ is valid in $\mathfrak M$, denoted
$\mymodelsm{FOL}M\varphi$, iff $\mymodelss{FOL}M\rho\varphi$ for all
{\FOL}-substitutions $\rho$ for $\mathfrak M$.
\item $\Gamma$ is valid in $\mathfrak M$, denoted
$\mymodelsm{FOL}M\Gamma$, iff $\mymodelsm{FOL}M\varphi$ for every
$\varphi\in\Gamma$.
\item $\varphi$ is a consequence of $\Gamma$, denoted
$\Gamma\yields{FOL}\varphi$, iff $\mymodelsm{FOL}M\varphi$ for all
{\FOL}-models $\mathfrak M$ such that $\mymodelsm{FOL}M\Gamma$.
\item In particular, $\varphi$ is valid (denoted $\mymodels{FOL}\varphi$)
iff $\epsilon\yields{FOL}\varphi$.
\end{enumerate}
\end{definition}

\subsection{System {\D}}

System {\D} is meant to represent a variant of first-order logic where
functions are allowed to be partial.  Because of the way it was defined
(from a specific type theory via the Curry--Howard isomorphism), it also
has other more subtle differences, namely that variables have to be
declared before they are used -- so {\D}-contexts contain formulas and
variables.  Also there is an {\ifte} constructor for terms, which
allows definitions of functions such as `if $x\neq 0$ then $1/x$ else $0$'
which abound in mathematics.  As a consequence, the inductive
definitions of terms and formulas have to be made simultaneously,
yielding an added degree of complexity because of the mutual
dependency between terms and formulas.

In system {\D} we will also have occasion to consider the
predicates $D_f$ associated to each function symbol $f$.

\begin{definition}%
\label{defn:Dterms}\label{defn:Dformulas}
The sets of {\D}-terms and {\D}-formulas are defined inductively as follows.
\begin{itemize}
\item Term-formation rules:
\begin{enumerate}[(i)]
\item $x_i$ is a term for any $i\in\NN$;
\item $c_i$ is a term for any $i=1,\ldots,k$;
\item if $f_i$ is a function symbol and $t_1,\ldots,t_{a_i}$ are terms
then $f_i(t_1,\ldots,t_{a_i})$ is a term;
\item if $\vartheta$ is a formula and $t_1$ and $t_2$ are terms, then
$\ifthelse\vartheta{t_1}{t_2}$ is a term.
\end{enumerate}
\item Formula-formation rules:
\begin{enumerate}[(i)]\setcounter{enumi}{4}
\item $\bot$ is a formula;
\item if $P_i$ is a function symbol and $t_1,\ldots,t_{r_i}$ are terms
then $P_i(t_1,\ldots,t_{r_i})$ is a formula;
\item if $t_1$ and $t_2$ are terms then $t_1=t_2$ is a formula;
\item if $\varphi$ and $\psi$ are formulas then so are $\varphi\to\psi$
and $\forall x_i\sep\varphi$.
\end{enumerate}
\end{itemize}
The set of {\D}-terms is denoted as {\terms{D}}; arbitrary terms are
denoted by $t$, $t'$, $t''$, $t_1$,\ldots

The set of {\D}-formulas is denoted as {\lang{D}}; arbitrary formulas
are denoted by $\varphi$, $\psi$, $\varphi'$, $\psi''$, $\varphi_1$,\ldots
\end{definition}

\begin{definition}\label{defn:Dcontext}
A {\D}-context is a list of {\D}-formulas and variables.  The empty
context is denoted by $\epsilon$; arbitrary contexts will be denoted
by $\Gamma$, $\Gamma'$, $\Delta$, $\Delta_1$,\ldots
\end{definition}

\begin{definition}\label{defn:Djudgement}
In system {\D} the following kinds of judgements exist.
\begin{enumerate}[(i)]
\item A context $\Gamma$ is well formed, $\Gamma\myvdash{D}\ok$.
\item A term $t$ is well formed in a context $\Gamma$, $\Gamma\myvdash{D}t\wf$.
\item A formula $\varphi$ is well formed in a context $\Gamma$,
$\Gamma\myvdash{D}\varphi\wf$.
\item A formula $\varphi$ is provable from a context $\Gamma$,
$\Gamma\myvdash{D}\varphi$.
\end{enumerate}
\end{definition}

\bigskip\noindent
The first three kinds of judgements are in fact auxiliary judgements, since
we are ultimately interested in derivability.  The idea is, in a
well-formed term (formula, context) all applications of partial functions
can be proved to be defined from the context.  Also, all variables that
are used in a well-formed term (formula, context) have been previously
declared.

\begin{definition}\label{defn:Dderivation}
A {\D}-judgement is \emph{valid} iff it can be derived using the following
rules.

$$\rulexn{$\epsilon$-wf}{}{\epsilon\myvdash{D}\ok}
\quad
\rulexnc{decl-wf}{\Gamma\myvdash{D}\ok}{\Gamma,x_i\myvdash{D}\ok}
{x_i\not\in\Gamma}
\quad
\rulexn{assum-wf}{\Gamma\myvdash{D}\varphi\wf}{\Gamma,\varphi\myvdash{D}\ok}
\lmarg{\mbox{Contexts}}$$
$$\rulexnc{var-wf}{\Gamma\myvdash{D}\ok}{\Gamma\myvdash{D}x_i\wf}{x_i\in\Gamma}
\quad
\rulexn{const-wf}{\Gamma\myvdash{D}\ok}{\Gamma\myvdash{D}c_i\wf}
\lmarg{\mbox{Terms}}$$
$$\rulexn{fun-wf}{\Gamma\myvdash{D}D_{f_i}(t_1,\ldots,t_{a_i})}
  {\Gamma\myvdash{D}f_i(t_1,\ldots,t_{a_i})\wf}
$$
$$\rulexn{if-wf}
   {\Gamma,\vartheta\myvdash{D}t_1\wf \quad \Gamma,\neg\vartheta\myvdash{D}t_2\wf}
   {\Gamma\myvdash{D}(\ifthelse\vartheta{t_1}{t_2})\wf}$$
$$\rulexn{$\bot$-wf}{\Gamma\myvdash{D}\ok}{\Gamma\myvdash{D}\bot\wf}
\quad
\rulexn{$\to$-wf}{\Gamma,\varphi\myvdash{D}\psi\wf}
                 {\Gamma\myvdash{D}(\varphi\to\psi)\wf}
\quad
\rulexn{$\forall$-wf}{\Gamma,x_i\myvdash{D}\varphi\wf}
                     {\Gamma\myvdash{D}(\forall x_i\sep\varphi)\wf}
\lmarg{\mbox{Formulas}}$$
$$
\rulexn{$=$-wf}{\Gamma\myvdash{D}t_1\wf \quad \Gamma\myvdash{D}t_2\wf}
               {\Gamma\myvdash{D}t_1=t_2\wf}
\quad
\rulexn{pred-wf}
   {\Gamma\myvdash{D}t_1\wf \quad\cdots\quad \Gamma\myvdash{D}t_{r_i}\wf
                            \quad \Gamma\myvdash{D}\ok}
   {\Gamma\myvdash{D}P_i(t_1,\ldots,t_{r_i})\wf}$$
$$\rulexnc{assum}{\Gamma\myvdash{D}\ok}{\Gamma\myvdash{D}\varphi}
                 {\varphi\in\Gamma}
\quad
\rulexn{$\neg\neg$-E}{\Gamma\myvdash{D}\neg\neg\varphi}{\Gamma\myvdash{D}\varphi}
\lmarg{\mbox{Proofs}}$$
$$\rulexn{$\to$-I}{\Gamma,\varphi\myvdash{D}\psi}
                  {\Gamma\myvdash{D}(\varphi\to\psi)}
\quad
\rulexn{$\to$-E}
   {\Gamma\myvdash{D}(\varphi\to\psi) \quad \Gamma\myvdash{D}\varphi}
   {\Gamma\myvdash{D}\psi}$$
$$\rulexn{$\forall$-I}{\Gamma,x_i\myvdash{D}\varphi}
                      {\Gamma\myvdash{D}(\forall x_i\sep\varphi)}
\quad
\rulexn{$\forall$-E}
   {\Gamma\myvdash{D}(\forall x_i\sep\varphi) \quad \Gamma\myvdash{D}t\wf}
   {\Gamma\myvdash{D}\varphi[x_i:=t]}$$
$$\rulexn{refl}{\Gamma\myvdash{D}t\wf}{\Gamma\myvdash{D}t=t}
\quad
\rulexn{sym}{\Gamma\myvdash{D}t_1=t_2}{\Gamma\myvdash{D}t_2=t_1}
\quad
\rulexn{trans}{\Gamma\myvdash{D}t_1=t_2 \quad \Gamma\myvdash{D}t_2=t_3}
              {\Gamma\myvdash{D}t_1=t_3}$$
$$\rulexn{$=$-fun}
   {\Gamma\myvdash{D}t_1=t'_1 \quad \cdots \quad
    \Gamma\myvdash{D}t_{a_i}=t'_{a_i} \quad
    \Gamma\myvdash{D}D_{f_i}(t_1,\ldots,t_{a_i}) \quad 
    \Gamma\myvdash{D}D_{f_i}(t'_1,\ldots,t'_{a_i})}
   {\Gamma\myvdash{D}f_i(t_1,\ldots,t_{a_i})=f_i(t'_1,\ldots,t'_{a_i})}$$
$$\rulexn{$=$-pred}
   {\Gamma\myvdash{D}t_1=t'_1\quad\cdots\quad\Gamma\myvdash{D}t_{r_i}=t'_{r_i}
                             \quad \Gamma\myvdash{D}\ok}
   {\Gamma\myvdash{D}P_i(t_1,\ldots, t_{r_i}) \to P_i(t'_1,\ldots, t'_{r_i})}$$
$$\rulexn{$=$-if-true}
   {\Gamma\myvdash{D}\vartheta \quad \Gamma,\vartheta\myvdash{D}t_1\wf
                          \quad \Gamma,\neg\vartheta\myvdash{D}t_2\wf}
   {\Gamma\myvdash{D}(\ifthelse\vartheta{t_1}{t_2})=t_1}$$
$$\rulexn{$=$-if-false}
   {\Gamma\myvdash{D}\neg\vartheta \quad \Gamma,\vartheta\myvdash{D}t_1\wf
                              \quad \Gamma,\neg\vartheta\myvdash{D}t_2\wf}
   {\Gamma\myvdash{D}(\ifthelse\vartheta{t_1}{t_2})=t_2}$$
\end{definition}

\bigskip\noindent
A few remarks about these definitions, most of which are already mentioned
in~\cite{wie:zwa:03}.
\begin{itemize}
\item The way the rules are presented, all declared variables have to
be distinct.  This allows us to remove the proviso on the rule
(\emph{$\forall$-I}): if $\Gamma,x_i\myvdash{T}\varphi$, then
$\Gamma,x_i\myvdash{T}\ok$ (Proposition~1 of~\cite{wie:zwa:03}),
hence $x_i$ cannot occur in $\Gamma$.
\item For a similar reason the proviso on the rule (\emph{$\forall$-E})
is also not needed: all variables bound in $(\forall x_i\sep\varphi)$
are distinct from the variables in $\Gamma$, and since
$\Gamma\myvdash{T}t\wf$ they cannot occur in $t$.
\item We identify expressions that are $\alpha$-equivalent and
assume the Barendregt convention.  This allows us to see e.g.\ 
$x_1,P_1(x_1),x_1,P_2(x_1)\myvdash{D}P_2(x_1)$ as a derivable judgement,
since it is $\alpha$-equivalent to
$x_1,P_1(x_1),x_2,P_2(x_2)\myvdash{D}P_2(x_2)$ and the latter can be
derivable from the rules above.  However, note that
$x_i,P(x_i),x_i\myvdash{D}P(x_i)$ is not derivable, since it is
$\alpha$-equivalent to
$x_i,P(x_i),x_j\myvdash{D}P(x_j)$, which is not derivable.

Notice that this does not conflict with the previous remarks, since we
are \emph{defining} a derivation of a judgement where the same variable
is declared more than once in terms of derivations from the rules.
\end{itemize}

The model theory for {\D} is similar to that of {\FOL}, but the partiality
of functions yields some striking differences.
The first difference -- quite expected -- is the definition of model itself.

\begin{definition}\label{defn:Dmodel} A {\D}-model $\mathfrak M$
is a tuple $\mathfrak M=\langle A,F,P,C\rangle$ where:
\begin{enumerate}[--]
\item $A$ is a non-empty set (the domain);
\item $F=\{\intm{D}M{f_1},\ldots,\intm{D}M{f_n}\}$ where
$\intm{D}M{f_i}:A^{a_i}\not\to A$ for each $i=1,\ldots,n$;
\item $P=\{\intm{D}M{P_1},\ldots,\intm{D}M{P_m}\}$ where
$\intm{D}M{P_i}\subseteq A^{r_i}$ for each $i=1,\ldots,m$;
\item $C=\{\intm{D}M{c_1},\ldots,\intm{D}M{c_k}\}\subseteq A$;
\item for each $i=1,\ldots,n$ and $e_1,\ldots,e_{a_i}\in A$,
$(e_1,\ldots,e_{a_i})\in\intm{D}M{D_{f_i}}$ iff
$f_i(e_1,\ldots,e_{a_i})$ is defined.
\end{enumerate}
A {\D}-substitution for $\mathfrak M$ is a partial function that assigns
values in $A$ to some variables $x_i$.
\end{definition}

The definition of satisfaction of a formula has some unexpected
complications.  If we want to get completeness of {\D} w.r.t.\ the
class of {\D}-models, the notion of consequence needs to capture the
effect of variable declarations.  We achieve this by allowing
\emph{partial} assignments, which will be defined (at least) for the
variables that have been declared.  We also need to have a semantical
judgement for a legal term $t$ (formula $\varphi$), in the sense that all
its variables can be interpreted.  We will denote this judgement by
$t\wwf$ ($\varphi\wwf$), where $\wwf$ should be read as
`weakly well-formed'.

Furthermore, since the interpretation of terms can be undefined (either due
to non-declaration of a variable or to an ill-applied function) we also
need a semantic judgement for well-formed formula.  This is relevant to
define the interpretation of $\varphi\to\psi$: the {\FOL}-definition is
not sufficient, as $\not\mymodelss{D}\rho\varphi$ can occur because
$\varphi$ contains undefined terms.  Therefore we will also have
judgements $t\wf$ and $\varphi\wf$, where $\wf$ is read `(strongly)
well-formed'.  (The judgement for terms is helpful, since terms and
formulas are mutually defined.)

This situation can be understood as follows.  A term can have three
kinds of interpretations: a value of the domain, undefined or illegal.
`Undefined' means that some function is applied outside its domain,
whereas `illegal' means that a variable has been used for which there
is no assigned value.  Likewise, formulas can have four truth-values: true,
false, undefined or illegal.  The first two cases are the common ones,
the third applies to a formula that refers to undefined terms, and the
last to a formula that uses illegal terms.  These truth-values form a
lattice as depicted in Figure~\ref{truth-values}, which helps clear up
the following definition.
\begin{figure}[ht]
\[
\xymatrix{\mathbf{t\!t} & & \mathbf{f\!f} \\
 & \mathbf{und}\ar[ul]+D-<0cm,1mm>\ar[ur]+D-<0cm,1mm> & \\
 & \bot\ar[u]+D-<0cm,1mm>
 \save"1,1"."1,3"*[F]\frm{}\restore
 \save{"1,1"+UL+<-1mm,1mm>}.{"2,3"+<3mm,-2mm>}*[F]\frm{}\restore
 \save"1,2"+<0mm,-5mm>*\txt{\wf}\restore
 \save"2,2"+<10mm,-5mm>*\txt{\wwf}\restore
}
\]
\caption{Truth-values and judgements in {\D}-models}\label{truth-values}
\end{figure}

%% The following only made sense when T came first. (lcf)

%The first problem that arises is with well-formedness of terms and
%formulas.  In {\T}, the only situation when a term (formula) could be
%undefined in a model with a substitution $\rho$ was if it contained an
%occurrence of a variable $x_i$ for which $\rho(x_i)$ was undefined.
%But in {\D} there is another possibility, namely that some function is
%being applied outside of its domain.

%However, if we want to get completeness of the logic w.r.t.\ the class
%of {\D}-models, we must have some way of distinguish these two cases.
%Consider any unary function symbol $f$; then $\bot,x\myvdash{D}P(f(x))$
%for any variable $x$, but $\bot\not\myvdash{D}P(f(x))$ because $x$ must
%be declared.  Looking at the definition of consequence for {\T}, we
%quickly see that there is no way we can distinguish these situations
%only with the judgements earlier introduced.

%The solution we propose is unfolding the {\T}-concept of `being
%well-formed' in two {\D}-concepts: being well-formed and being
%weakly well-formed.  The first has the same meaning as above (the
%interpretation of a term is defined, a formula only speaks about defined
%terms), the second simply checks that no variables whose interpretation
%is undefined occur in the term or formula.

%Bearing this in mind, the definition of {\D}-interpretation of terms and
%{\D}-satisfaction of formulas should now be clear.
\begin{definition}\label{defn:Dinterpretation}\label{defn:Dsatisfaction}
Let $\mathfrak M$ be a {\D}-model and $\rho$ be a {\D}-substitution for
$\mathfrak M$.
The interpretation of a term $t\in\terms D$,
{\ints{D}M\rho t}, is defined simultaneously with the judgements 
$\mymodelss{D}M\rho t\wwf$, $\mymodelss{D}M\rho\varphi\wwf$,
$\mymodelss{D}M\rho t\wf$, $\mymodelss{D}M\rho\varphi\wf$ and
$\mymodelss{D}M\rho\varphi$ (satisfaction) as follows.
\begin{enumerate}[(i)]
\item Rules for interpreting terms.  The interpretation of a term again is
undefined if any of the recursive calls returns undefined, or if none of the
cases hold.
\begin{eqnarray*}
\ints{D}M\rho{x_i} & := & \rho(x_i) \\
\ints{D}M\rho{c_i} & := & \intm{D}M{c_i} \\
\ints{D}M\rho{f_i(t_1,\ldots,t_{a_i})} & := &
 \intm{D}M{f_i}\left(\ints{D}M\rho{t_1},\ldots,\ints{D}M\rho{t_{a_i}}\right) \\
\ints{D}M\rho{\ifthelse\vartheta{t_1}{t_2}} & := & \left\{\begin{array}{ll}
   \ints{D}M\rho{t_1} & \mbox{ if $\mymodelss{D}M\rho\vartheta$ and $\mymodelss{D}M\rho t_2\wwf$} \\
   \ints{D}M\rho{t_2} & \mbox{ if $\mymodelss{D}M\rho\neg\vartheta$ and $\mymodelss{D}M\rho t_1\wwf$}
\end{array}\right. \\
\end{eqnarray*}
\item Rules for weak well-formation of terms and formulas.
\begin{eqnarray*}
\mymodelss{D}M\rho{x_i}\wwf & \mbox{iff} & \rho(x_i)\mbox{ is defined} \\
\mymodelss{D}M\rho{c_i}\wwf \\
\mymodelss{D}M\rho{f_i(t_1,\ldots,t_{a_i})}\wwf & \mbox{iff} &
  \mymodelss{D}M\rho{t_1}\wwf,\ldots,\mymodelss{D}M\rho{t_{a_i}}\wwf \\
\mymodelss{D}M\rho{\ifthelse\vartheta{t_1}{t_2}}\wwf & \mbox{iff} & 
  \mymodelss{D}M\rho\vartheta\wwf, \\
  & & \mymodelss{D}M\rho{t_1}\wwf\mbox{ and }\mymodelss{D}M\rho{t_2}\wwf\\
 \\
\mymodelss{D}M\rho\bot\wwf \\
\mymodelss{D}M\rho P_i(t_1,\ldots,t_{r_i})\wwf & \mbox{iff} &
 \mymodelss{D}M\rho{t_1}\wwf,\ldots,\mymodelss{D}M\rho{t_{r_i}}\wwf \\
\mymodelss{D}M\rho t_1=t_2\wwf & \mbox{iff} &
 \mymodelss{D}M\rho{t_1}\wwf\mbox{ and }\mymodelss{D}M\rho{t_2}\wwf \\
\mymodelss{D}M\rho(\varphi\to\psi)\wwf & \mbox{iff} &
 \mymodelss{D}M\rho\varphi\wwf\mbox{ and } \mymodelss{D}M\rho\psi\wwf \\
\mymodelss{D}M\rho(\forall x_i\sep\varphi)\wwf & \mbox{iff} &
 \mymodelss{D}M{\rho[x_i:=a]}\varphi\wwf \mbox{ for all $a\in A$}
\end{eqnarray*}
\item For any term $t$, $\mymodelss{D}M\rho t\wf$ iff $\ints{D}M\rho t$ is
defined.
\item Rules for well formation of formulas.
\begin{eqnarray*}
\mymodelss{D}M\rho\bot\wf \\
\mymodelss{D}M\rho P_i(t_1,\ldots,t_{r_i})\wf & \mbox{iff} &
 \mymodelss{D}M\rho{t_1}\wf,\ldots,\mymodelss{D}M\rho{t_{r_i}}\wf \\
\mymodelss{D}M\rho t_1=t_2\wf & \mbox{iff} &
 \mymodelss{D}M\rho{t_1}\wf\mbox{ and }\mymodelss{D}M\rho{t_2}\wf \\
\mymodelss{D}M\rho(\varphi\to\psi)\wf & \mbox{iff} &
 \mymodelss{D}M\rho\varphi\wf\mbox{ and }
 \left\{\begin{array}{l}
 \mymodelss{D}M\rho\varphi,\ \mymodelss{D}M\rho\psi\wf \\
 \not\mymodelss{D}M\rho\varphi,\ \mymodelss{D}M\rho\psi\wwf
 \end{array}\right. \\
\mymodelss{D}M\rho(\forall x_i\sep\varphi)\wf & \mbox{iff} &
 \mymodelss{D}M{\rho[x_i:=a]}\varphi\wf \mbox{ for all $a\in A$}
\end{eqnarray*}
\item Rules for satisfaction of formulas.  In the side conditions the
requirement that the interpretation of all terms mentioned be defined
is implicit.
\begin{eqnarray*}
\not\mymodelss{D}M\rho\bot \\
\mymodelss{D}M\rho P_i(t_1,\ldots,t_{r_i}) & \mbox{iff} &
 \left(\ints{D}M\rho{t_1},\ldots,\ints{D}M\rho{t_{r_i}}\right)
  \in\ints{D}M\rho{P_i} \\
\mymodelss{D}M\rho t_1=t_2 & \mbox{iff} &
 \ints{D}M\rho{t_1}=\ints{D}M\rho{t_2} \\
\mymodelss{D}M\rho\varphi\to\psi & \mbox{iff} &
 \mymodelss{D}M\rho(\varphi\to\psi)\wf \mbox{ and }
 \not\mymodelss{D}M\rho\varphi \mbox{ or } \mymodelss{D}M\rho\psi \\
\mymodelss{D}M\rho\forall x_i\sep\varphi & \mbox{iff} &
 \mymodelss{D}M{\rho[x_i:=a]}\varphi \mbox{ for all $a\in A$}
\end{eqnarray*}
\end{enumerate}
\end{definition}

The following are some trivial consequences of these definitions.
\begin{proposition}\label{modelDprops}
Let $\mathfrak M$ be a {\D}-model, $\rho$ be a {\D}-substitution for
$\mathfrak M$, $t$ be a {\D}-term and $\varphi$ be a formula.
\begin{enumerate}[(i)]
\item If $\mymodelss{D}M\rho t\wf$, then $\mymodelss{D}M\rho t\wwf$.
\item $\mymodelss{D}M\rho\varphi\wf$ iff $\mymodelss{D}M\rho\varphi$ or
$\mymodelss{D}M\rho\neg\varphi$.
\end{enumerate}
\end{proposition}

The other difference between the semantics for {\FOL} and for {\D} is
the definition of consequence.  This needs to be a local definition
(i.e.\ parameterized on $\rho$) and defined recursively on the length
of the context for two reasons: because $\rho$ needs to be changed every
time a variable is declared; and because formulas in the context can
influence well-formedness of terms that occur later, so order is
relevant, which was not the case in {\FOL}.  For example,
$x,x\neq 0,1/x=3$ is well-formed but $x,1/x=3,x\neq 0$ is not.
\begin{definition}\label{defn:Dconsequence}
Let $\mathfrak M$ be a {\D}-model and $\rho$ be a {\D}-substitution for
$\mathfrak M$.
\begin{enumerate}[(i)]
\item Well-formation of contexts is defined inductively as follows:
\begin{enumerate}
\item $\epsilon\mymodelss{D}M\rho\ok$;
\item $\varphi,\Gamma\mymodelss{D}M\rho\ok$ iff
$\mymodelss{D}M\rho\varphi\wf$ and $\Gamma\mymodelss{D}M\rho\ok$;
\item $x_i,\Gamma\mymodelss{D}M\rho\ok$ iff
$\Gamma\mymodelss{D}M{\rho[x_i:=a]}\ok$ for all $a\in A$.
\end{enumerate}
\item Let $\cal X$ stand for $t\wwf$ or $\psi\wwf$.  Then
$\Gamma\mymodelss{D}M\rho\cal X$ is defined inductively as follows:
\begin{enumerate}
\item $\epsilon\mymodelss{D}M\rho\cal X$ iff $\mymodelss{D}M\rho\cal X$;
\item $\varphi,\Gamma\mymodelss{D}M\rho\cal X$ iff
$\mymodelss{D}M\rho\varphi\wwf$ and $\Gamma\mymodelss{D}M\rho\cal X$;
\item $x_i,\Gamma\mymodelss{D}M\rho\cal X$ iff
$\Gamma\mymodelss{D}M{\rho[x_i:=a]}\cal X$ for all $a\in A$.
\end{enumerate}
\item Let $\cal X$ stand for $t\wf$ or $\psi\wf$.  Then
$\Gamma\mymodelss{T}M\rho\cal X$ is defined inductively as follows:
\begin{enumerate}
\item $\epsilon\mymodelss{D}M\rho\cal X$ iff $\mymodelss{D}M\rho\cal X$;
\item $\varphi,\Gamma\mymodelss{T}M\rho\cal X$ iff
(1) $\mymodelss{D}M\rho\varphi$ and $\Gamma\mymodelss{D}M\rho\cal X$ or
(2) $\mymodelss{D}M\rho\neg\varphi$ and $\Gamma\mymodelss{T}M\rho\cal X'$
(where $\cal X'$ stands for $t\wwf$ or $\psi\wwf$);
\item $x_i,\Gamma\mymodelss{D}M\rho\cal X$ iff
$\Gamma\mymodelss{D}M{\rho[x_i:=a]}\cal X$ for all $a\in A$.
\end{enumerate}
\item Consequence is defined inductively as follows:
\begin{enumerate}
\item $\epsilon\mymodelss{D}M\rho\psi$ iff $\mymodelss{D}M\rho\psi$;
\item $\varphi,\Gamma\mymodelss{D}M\rho\psi$ iff
(1) $\mymodelss{D}M\rho\varphi$ and $\Gamma\mymodelss{D}M\rho\psi$ or
(2) $\mymodelss{D}M\rho\neg\varphi$ and $\Gamma\mymodelss{D}M\rho\psi\wwf$;
\item $x_i,\Gamma\mymodelss{D}M\rho\psi$ iff
$\Gamma\mymodelss{D}M{\rho[x_i:=a]}\psi$ for all $a\in A$.
\end{enumerate}
\item Let $\cal X$ stand for $\ok$, $t\wwf$, $\psi\wwf$, $t\wf$,
$\psi\wf$ or $\psi$.  Then $\Gamma\mymodelsm{D}M\cal X$ iff
$\Gamma\mymodelss{D}M\emptyset\cal X$ and $\Gamma\yields{D}\cal X$ iff
$\Gamma\mymodelsm{D}M\cal X$ for all {\D}-models $\mathfrak M$.
\item In particular, a formula $\varphi$ is valid (denoted
$\mymodels{D}\varphi$) iff $\epsilon\yields{D}\varphi$.
\end{enumerate}
\end{definition}

Notice the effect of the declarations of the variables: intuitively,
the substitution `grows' as it goes through the context from left to
right, adding each new variable to its domain.  It is easy to see
that this captures the effect of declaring variables in the deductive
system.

Notice that only closed formulas can be valid.

\subsection{System {\T}}

System {\T} is meant as an in-between system between {\FOL} and {\D}, which
are the `interesting' systems.
%The former is a standard, well-known system
%which has been extensively studied, whereas the latter is the subject
%of~\cite{wie:zwa:03}.
System {\T} bridges the gap between the two others: morally
and intuitively it corresponds to {\FOL}, as it basically describes the
same logic; but formally it is closer to {\D}, sharing the same set of
terms and judgements.  System {\T} also makes it easier to study the
relationship between {\FOL} and {\D}.

System {\T} has the same language as {\D}.  Hence, it differs from
{\FOL} in two ways: the presence of the {\ifte} constructor (which
adds a degree of complexity to the system because of the mutual
dependency between terms and formulas) and the need to declare
variables in contexts.

\begin{definition}\label{defn:Tterms}\label{defn:Tformulas}\label{defn:Tcontexts}
The sets of {\T}-terms, {\T}-formulas and {\T}-contexts are the sets of
{\D}-terms, {\D}-formulas and {\D}-contexts.

The set of {\T}-terms is denoted as {\terms{T}}; arbitrary terms are
denoted by $t$, $t'$, $t''$, $t_1$,\ldots

The set of {\T}-formulas is denoted as {\lang{T}}; arbitrary formulas
are denoted by $\varphi$, $\psi$, $\varphi'$, $\psi''$, $\varphi_1$,\ldots

Arbitrary contexts will be denoted
by $\Gamma$, $\Gamma'$, $\Delta$, $\Delta_1$,\ldots
\end{definition}

\begin{definition}\label{defn:Tjudgement}
In system {\T} the same four kinds of judgements exist as in {\D}.
\begin{enumerate}[(i)]
\item A context $\Gamma$ is well formed, $\Gamma\myvdash{T}\ok$.
\item A term $t$ is well formed in a context $\Gamma$, $\Gamma\myvdash{T}t\wf$.
\item A formula $\varphi$ is well formed in a context $\Gamma$,
$\Gamma\myvdash{T}\varphi\wf$.
\item A formula $\varphi$ is provable from a context $\Gamma$,
$\Gamma\myvdash{T}\varphi$.
\end{enumerate}
\end{definition}

\bigskip\noindent
In system {\T} the first three kinds of judgements simply state that all
free variables are `declared' before they are used.

\begin{definition}\label{defn:Tderivation}
A {\T}-judgement is \emph{valid} iff it can be derived using the following
rules.

$$\rulexn{$\epsilon$-wf}{}{\epsilon\myvdash{T}\ok}
\quad
\rulexnc{decl-wf}{\Gamma\myvdash{T}\ok}{\Gamma,x_i\myvdash{T}\ok}
  {x_i\not\in\Gamma}
\quad
\rulexn{assum-wf}{\Gamma\myvdash{T}\varphi\wf}{\Gamma,\varphi\myvdash{T}\ok}
\lmarg{\mbox{Contexts}}$$
$$\rulexnc{var-wf}{\Gamma\myvdash{T}\ok}{\Gamma\myvdash{T}x_i\wf}{x_i\in\Gamma}
\quad
\rulexn{const-wf}{\Gamma\myvdash{T}\ok}{\Gamma\myvdash{T}c_i\wf}
\lmarg{\mbox{Terms}}$$
$$\rulexn{fun-wf}
   {\Gamma\myvdash{T}t_1\wf \quad \cdots \quad \Gamma\myvdash{T}t_{a_i}\wf
                            \quad \Gamma\myvdash{T}\ok}
   {\Gamma\myvdash{T}f_i(t_1,\ldots,t_{a_i})\wf}$$
$$\rulexn{if-wf}
   {\Gamma\myvdash{T}\vartheta\wf \quad \Gamma\myvdash{T}t_1\wf
                             \quad \Gamma\myvdash{T}t_2\wf}
   {\Gamma\myvdash{T}(\ifthelse\vartheta{t_1}{t_2})\wf}$$
$$\rulexn{$\bot$-wf}{\Gamma\myvdash{T}\ok}{\Gamma\myvdash{T}\bot\wf}
\quad
\rulexn{$\to$-wf}{\Gamma\myvdash{T}\varphi\wf \quad \Gamma\myvdash{T}\psi\wf}
                 {\Gamma\myvdash{T}(\varphi\to\psi)\wf}
\quad
\rulexn{$\forall$-wf}{\Gamma,x_i\myvdash{T}\varphi\wf}
                     {\Gamma\myvdash{T}(\forall x_i\sep\varphi)\wf}
\lmarg{\mbox{Formulas}}$$
$$
\rulexn{$=$-wf}{\Gamma\myvdash{T}t_1\wf \quad \Gamma\myvdash{T}t_2\wf}
               {\Gamma\myvdash{T}t_1=t_2\wf}
\quad
\rulexn{pred-wf}
   {\Gamma\myvdash{T}t_1\wf \quad\cdots\quad \Gamma\myvdash{T}t_{r_i}\wf
                            \quad \Gamma\myvdash{T}\ok}
   {\Gamma\myvdash{T}P_i(t_1,\ldots,t_{r_i})\wf}$$
$$\rulexnc{assum}{\Gamma\myvdash{T}\ok}{\Gamma\myvdash{T}\varphi}
                 {\varphi\in\Gamma}
\quad
\rulexn{$\neg\neg$-E}{\Gamma\myvdash{T}\neg\neg\varphi}{\Gamma\myvdash{T}\varphi}
\lmarg{\mbox{Proofs}}$$
$$\rulexn{$\to$-I}{\Gamma,\varphi\myvdash{T}\psi}
                  {\Gamma\myvdash{T}(\varphi\to\psi)}
\quad
\rulexn{$\to$-E}
   {\Gamma\myvdash{T}(\varphi\to\psi) \quad \Gamma\myvdash{T}\varphi}
   {\Gamma\myvdash{T}\psi}$$
$$\rulexn{$\forall$-I}{\Gamma,x_i\myvdash{T}\varphi}
                      {\Gamma\myvdash{T}(\forall x_i\sep\varphi)}
\quad
\rulexn{$\forall$-E}
   {\Gamma\myvdash{T}(\forall x_i\sep\varphi) \quad \Gamma\myvdash{T}t\wf}
   {\Gamma\myvdash{T}\varphi[x_i:=t]}$$
$$\rulexn{refl}{\Gamma\myvdash{T}t\wf}{\Gamma\myvdash{T}t=t}
\quad
\rulexn{sym}{\Gamma\myvdash{T}t_1=t_2}{\Gamma\myvdash{T}t_2=t_1}
\quad
\rulexn{trans}{\Gamma\myvdash{T}t_1=t_2 \quad \Gamma\myvdash{T}t_2=t_3}
              {\Gamma\myvdash{T}t_1=t_3}$$
$$\rulexn{$=$-fun}
   {\Gamma\myvdash{T}t_1=t'_1\quad\cdots\quad\Gamma\myvdash{T}t_{a_i}=t'_{a_i}
                             \quad \Gamma\myvdash{T}\ok}
   {\Gamma\myvdash{T}f_i(t_1,\ldots, t_{a_i})=f_i(t'_1,\ldots, t'_{a_i})}$$
$$\rulexn{$=$-pred}
   {\Gamma\myvdash{T}t_1=t'_1\quad\cdots\quad\Gamma\myvdash{T}t_{r_i}=t'_{r_i}
                             \quad \Gamma\myvdash{T}\ok}
   {\Gamma\myvdash{T}P_i(t_1,\ldots, t_{r_i}) \to P_i(t'_1,\ldots, t'_{r_i})}$$
$$\rulexn{$=$-if-true}
   {\Gamma\myvdash{T}\vartheta \quad \Gamma\myvdash{T}t_1\wf
                          \quad \Gamma\myvdash{T}t_2\wf}
   {\Gamma\myvdash{T}(\ifthelse\vartheta{t_1}{t_2})=t_1}$$
$$\rulexn{$=$-if-false}
   {\Gamma\myvdash{T}\neg\vartheta \quad \Gamma\myvdash{T}t_1\wf
                              \quad \Gamma\myvdash{T}t_2\wf}
   {\Gamma\myvdash{T}(\ifthelse\vartheta{t_1}{t_2})=t_2}$$
\end{definition}

\bigskip\noindent
A few remarks are in place about this definition.
\begin{itemize}
\item Most of the rules for {\T} are the same as for {\D}.  The rules
(\emph{if-wf}), (\emph{$\to$-wf}), (\emph{$=$-if-true}) and
(\emph{$=$-if-false}) are equivalent in {\T} (but not in {\D}) to their
{\D} analogues.  Only (\emph{fun-wf}) and (\emph{$=$-fun}) are essentially
different.
\item The definitions of $\Gamma\myvdash{T}\ok$, $\Gamma\myvdash{T}t\wf$
and $\Gamma\myvdash{T}\varphi\wf$ are mutually recursive, but no longer
depend on the validity of any judgement of the form
$\Gamma\myvdash{T}\varphi$.
\item As before, we identify $\alpha$-equivalent judgements and assume
the Barendregt convention.
\end{itemize}

The following are some simple results that will be needed later on.
\begin{proposition}\label{formwfimptermwfT}
Let $\Gamma$ be a {\T}-context,
$\varphi$ a {\T}-formula and $t$ a {\T}-term occurring in $\varphi$.
If $\Gamma\myvdash{T}\varphi\wf$, then $\Gamma\myvdash{T}t\wf$.
\end{proposition}
\begin{proof}
Straightforward induction on the derivation of 
$\Gamma\myvdash{T}\varphi\wf$.
\end{proof}

\begin{lemma}\label{Twf} Let $\Gamma$ be a {\T}-context,
$t\in\terms{T}$ and $\varphi\in\lang{T}$.  Suppose that $x_i\in\Gamma$
for all $x_i\in FV(t)\cup FV(\varphi)$.  Then $\Gamma\myvdash{T} t\wf$
and $\Gamma\myvdash{T}\varphi\wf$.
\end{lemma}
\begin{proof}
By induction on the derivations of
$\Gamma\myvdash{T}t \wf$ and $\Gamma\myvdash{T}\varphi\wf$.
\end{proof}

\begin{lemma}\label{Textravars} Let $\Gamma$ be a {\T}-contexts,
$\varphi\in\lang{T}$ and $x_i$ be a variable such that
$x_i\not\in FV(\varphi)$.  If $\Gamma,x_i\myvdash{T}\varphi$, then
$\Gamma\myvdash{T}\varphi$.
\end{lemma}
\begin{proof}
Suppose that $\Gamma,x_i\myvdash{T}\varphi$.  From this we can
infer $\Gamma\myvdash{T}\forall x_i\sep\varphi$ by (\emph{$\forall$-I}).
Let $c_j$ be a constant; then $\Gamma\myvdash{T}\ok$ by Proposition~3
of~\cite{wie:zwa:03}, hence $\Gamma\myvdash{T}c_j\wf$ by (\emph{const-wf}).
>From (\emph{$\forall$-E}) follows that $\Gamma\myvdash{T}\varphi[x_i:=c_j]$,
and since $x_i$ does not occur free in $\varphi$ the last judgement is simply
$\Gamma\myvdash{T}\varphi$.
%
%If $\Delta$ is $\Delta',\psi$, then from
%$\Gamma,x_i,\Delta',\psi\myvdash{T}\varphi$ we infer
%$\Gamma,x_i,\Delta'\myvdash{T}\psi\to\varphi$ by (\emph{$\to$-I}); from the
%induction hypothesis follows that 
%$\Gamma,\Delta'\myvdash{T}\psi\to\varphi$.  Also
%$\Gamma,\Delta',\psi\myvdash{T}\ok$, since
%$\Gamma,x_i,\Delta',\psi\myvdash{T}\ok$ (Proposition~3 of~\cite{wie:zwa:03}
%applied to the hypothesis) and $x_i$ does not occur in either $\Delta'$
%or $\psi$.  Hence we can apply (\emph{assum}) to conclude that
%$\Gamma,\Delta',\psi\myvdash{T}\psi$ and by (\emph{$\to$-E}) follows that
%$\Gamma,\Delta',\psi\myvdash{T}\varphi$.
%
%Finally if $\Delta$ is $\Delta',x_j$, then from
%$\Gamma,x_i,\Delta',x_j\myvdash{T}\varphi$ we infer
%$\Gamma,x_i,\Delta'\myvdash{T}\forall x_j\sep\varphi$
%by (\emph{$\forall$-I}); from the induction hypothesis follows that
%$\Gamma,\Delta'\myvdash{T}\forall x_j\sep\varphi$.  From this follows
%that $\Gamma,\Delta'\myvdash{T}\ok$ by Proposition~3
%of~\cite{wie:zwa:03}, hence $\Gamma,\Delta',x_j\myvdash{T}x_j\wf$
%by (\emph{decl-wf}) and (\emph{var-wf}).  We can then apply weakening
%and (\emph{$\forall$-E}) to conclude that
%$\Gamma,\Delta',x_j\myvdash{T}\varphi$, since $\varphi[x_j:=x_j]\equiv\varphi$.
\end{proof}

\bigskip\noindent
Notice that for this last lemma it is essential that there be at least one
constant in the signature.  Otherwise there is a counterexample which is
well known in minimal logic: the judgement
$\forall x_i\sep P_1(x_i)\to A,\forall x_j\sep P_1(x_j)\myvdash{FOL}A$
but $\forall x_i\sep P_1(x_i)\to A,\forall x_j\sep P_1(x_j)\not\myvdash{T}A$.
Incidentally, this would also be a counterexample to the Completeness
Theorem~\ref{Tcomplete}, since
$\forall x_i\sep P_1(x_i)\to A,\forall x_j\sep P_1(x_j)\yields{T}A$.

The model theory for {\T} is between that for {\FOL} and that for {\D}:
we use {\FOL}-models, but define consequence as in {\D}.
\begin{definition}\label{defn:Tmodel} A {\T}-model $\mathfrak M$
is a {\FOL}-model.
A {\T}-substitution for $\mathfrak M$ is a partial function that assigns
values in $A$ to some variables $x_i$.
\end{definition}

The interpretation of terms and satisfaction of formulas is simpler than
in {\D}.  Since all functions are now partial, the `undefined' and
`illegal' categories of terms (or formulas) collapse, as do the notions
of weakly and strongly well-formed.  This makes the definitions lighter.
\begin{definition}\label{defn:Tinterpretation}\label{defn:Tsatisfaction}
Let $\mathfrak M$ be a {\T}-model and $\rho$ be a {\T}-substitution for
$\mathfrak M$.
The interpretation of a term $t\in\terms T$,
{\ints{T}M\rho t}, is defined simultaneously with the judgements 
$\mymodelss{T}M\rho t\wf$, $\mymodelss{T}M\rho\varphi\wf$ and
$\mymodelss{T}M\rho\varphi$ (satisfaction) as follows.
\begin{enumerate}[(i)]
\item Rules for interpreting terms.  The interpretation of a term is
undefined if any of the recursive calls returns undefined, or if none of the
cases hold.
\begin{eqnarray*}
\ints{T}M\rho{x_i} & := & \rho(x_i) \\
\ints{T}M\rho{c_i} & := & \intm{T}M{c_i} \\
\ints{T}M\rho{f_i(t_1,\ldots,t_{a_i})} & := &
 \intm{T}M{f_i}\left(\ints{T}M\rho{t_1},\ldots,\ints{T}M\rho{t_{a_i}}\right) \\
\ints{T}M\rho{\ifthelse\vartheta{t_1}{t_2}} & := & \left\{\begin{array}{ll}
   \ints{T}M\rho{t_1} & \mbox{ if $\mymodelss{T}M\rho\vartheta$ and $\mymodelss{T}M\rho t_2\wf$} \\
   \ints{T}M\rho{t_2} & \mbox{ if $\mymodelss{T}M\rho\neg\vartheta$ and $\mymodelss{T}M\rho t_1\wf$}
\end{array}\right. \\
\end{eqnarray*}
\item For any term $t$, $\mymodelss{T}M\rho t\wf$ iff $\ints{T}M\rho t$ is
defined.
\item Rules for well formation of formulas.
\begin{eqnarray*}
\mymodelss{T}M\rho\bot\wf \\
\mymodelss{T}M\rho P_i(t_1,\ldots,t_{r_i})\wf & \mbox{iff} &
 \mymodelss{T}M\rho{t_1}\wf,\ldots,\mymodelss{T}M\rho{t_{r_i}}\wf \\
\mymodelss{T}M\rho t_1=t_2\wf & \mbox{iff} &
 \mymodelss{T}M\rho{t_1}\wf\mbox{ and }\mymodelss{T}M\rho{t_2}\wf \\
\mymodelss{T}M\rho(\varphi\to\psi)\wf & \mbox{iff} &
 \mymodelss{T}M\rho\varphi\wf\mbox{ and } \mymodelss{T}M\rho\psi\wf \\
\mymodelss{T}M\rho(\forall x_i\sep\varphi)\wf & \mbox{iff} &
 \mymodelss{T}M{\rho[x_i:=a]}\varphi\wf \mbox{ for all $a\in A$}
\end{eqnarray*}
\item Rules for satisfaction of formulas.  In the side conditions the
requirement that the interpretation of all terms mentioned be defined
is implicit.
\begin{eqnarray*}
\not\!{\mymodelss{T}M\rho}\bot \\
\mymodelss{T}M\rho P_i(t_1,\ldots,t_{r_i}) & \mbox{iff} &
 \left(\ints{T}M\rho{t_1},\ldots,\ints{T}M\rho{t_{r_i}}\right)
  \in\ints{T}M\rho{P_i} \\
\mymodelss{T}M\rho t_1=t_2 & \mbox{iff} &
 \ints{T}M\rho{t_1}=\ints{T}M\rho{t_2} \\
\mymodelss{T}M\rho\varphi\to\psi & \mbox{iff} &
 \mymodelss{T}M\rho(\varphi\to\psi)\wf \mbox{ and }
 \not\!{\mymodelss{T}M\rho\varphi} \mbox{ or } \mymodelss{T}M\rho\psi \\
\mymodelss{T}M\rho\forall x_i\sep\varphi & \mbox{iff} &
 \mymodelss{T}M{\rho[x_i:=a]}\varphi \mbox{ for all $a\in A$}
\end{eqnarray*}
\end{enumerate}
\end{definition}

The following is straightforward to prove.
\begin{proposition}\label{modelTwf}
Let $\mathfrak M$ be a {\T}-model, $\rho$ be a {\T}-substitution for
$\mathfrak M$ and $\varphi$ be a formula.
Then $\mymodelss{T}M\rho\varphi\wf$ iff $\mymodelss{T}M\rho\varphi$ or
$\mymodelss{T}M\rho\neg\varphi$.
\end{proposition}

As a consequence, the interpretation of $\ifthelse\vartheta{t_1}{t_2}$
is always defined if $\vartheta$ is a well-formed formula and the
interpretations of $t_1$ and $t_2$ are defined.

\begin{definition}\label{defn:Tconsequence}
Let $\mathfrak M$ be a {\T}-model and $\rho$ be a {\T}-substitution for
$\mathfrak M$.
\begin{enumerate}[(i)]
\item Well-formation of contexts is defined inductively as follows:
\begin{enumerate}
\item $\epsilon\mymodelss{T}M\rho\ok$;
\item $\varphi,\Gamma\mymodelss{T}M\rho\ok$ iff
$\mymodelss{T}M\rho\varphi\wf$ and $\Gamma\mymodelss{T}M\rho\ok$;
\item $x_i,\Gamma\mymodelss{T}M\rho\ok$ iff
$\Gamma\mymodelss{T}M{\rho[x_i:=a]}\ok$ for all $a\in A$.
\end{enumerate}
\item Let $\cal X$ stand for $t\wf$ or $\psi\wf$.  Then
$\Gamma\mymodelss{T}M\rho\cal X$ is defined inductively as follows:
\begin{enumerate}
\item $\epsilon\mymodelss{T}M\rho\cal X$ iff $\mymodelss{T}M\rho\cal X$;
\item $\varphi,\Gamma\mymodelss{T}M\rho\cal X$ iff
$\mymodelss{T}M\rho\varphi\wf$ and $\Gamma\mymodelss{T}M\rho\cal X$;
\item $x_i,\Gamma\mymodelss{T}M\rho\cal X$ iff
$\Gamma\mymodelss{T}M{\rho[x_i:=a]}\cal X$ for all $a\in A$.
\end{enumerate}
\item Consequence is defined inductively as follows:
\begin{enumerate}
\item $\epsilon\mymodelss{T}M\rho\psi$ iff $\mymodelss{T}M\rho\psi$;
\item $\varphi,\Gamma\mymodelss{T}M\rho\psi$ iff
(1) $\mymodelss{T}M\rho\varphi$ and $\Gamma\mymodelss{T}M\rho\psi$ or
(2) $\mymodelss{T}M\rho\neg\varphi$ and $\Gamma\mymodelss{T}M\rho\psi\wf$;
\item $x_i,\Gamma\mymodelss{T}M\rho\psi$ iff
$\Gamma\mymodelss{T}M{\rho[x_i:=a]}\psi$ for all $a\in A$.
\end{enumerate}
\item Let $\cal X$ stand for $\ok$, $t\wf$, $\psi\wf$ or $\psi$.  Then
$\Gamma\mymodelsm{T}M\cal X$ iff $\Gamma\mymodelss{T}M\emptyset\cal X$
and $\Gamma\yields{T}\cal X$ iff $\Gamma\mymodelsm{T}M\cal X$ for all
{\T}-models $\mathfrak M$.
\item In particular, a formula $\varphi$ is valid (denoted
$\mymodels{T}\varphi$) iff $\epsilon\yields{T}\varphi$.
\end{enumerate}
\end{definition}

Completeness of {\T} will be one of the results we will prove later on;
the following partial result, however, is quite simple.
\begin{lemma}\label{Twfsyntsem} Let $\Gamma$ be a {\T}-context and
$\varphi\in\lang{T}$.  Then $\Gamma\myvdash{T}\varphi\wf$ iff
$\Gamma\yields{T}\varphi\wf$.
\end{lemma}
\begin{proof}
We prove this for all formulas $\varphi$ simultaneously, by induction
on $\Gamma$.\marginpar{Hmmm.}

Let $\Gamma$ be empty.  On the one hand
$\myvdash{T}\varphi\wf$ iff $\varphi$ is a closed formula (the converse
implication follows from Lemma~\ref{Twf}, the direct from
Proposition~\ref{formwfimptermwfT} and the observation that no variable is
well formed in the empty context).
On the other hand $\yields{T}\varphi\wf$ holds iff
$\varphi$ is closed, as can be seen from
Definition~\ref{defn:Tsatisfaction}.
Hence $\myvdash{T}\varphi$ iff $\varphi$ is closed iff $\yields{T}\varphi$.

Let $\Gamma$ be $\Gamma',\psi$.  From the rules of {\T} it can be seen
that $\Gamma',\psi\myvdash{T}\varphi\wf$ iff
$\Gamma'\myvdash{T}\psi\to\varphi\wf$; by the induction hypothesis the
latter holds iff $\Gamma'\yields{T}\psi\to\varphi\wf$; and this is again
equivalent to $\Gamma',\psi\yields{T}\varphi\wf$, as can be seen from
Definition~\ref{defn:Tsatisfaction}.

Finally take $\Gamma$ to be $\Gamma',x_i$.  From the rules of {\T} it can
again be seen that $\Gamma',x_i\myvdash{T}\varphi\wf$ iff
$\Gamma'\myvdash{T}\forall x_i\sep\varphi\wf$; by the induction hypothesis
the latter holds iff $\Gamma'\yields{T}\forall x_i\sep\varphi\wf$, which is
equivalent to $\Gamma',x_i\yields{T}\varphi\wf$ as consequence of
Definition~\ref{defn:Tsatisfaction}.
\end{proof}

\section{Equivalence of {\FOL} and {\T}}

In this section we will show that system {\T} is equivalent to
{\FOL} in the following sense: for any context $\Gamma$ and
formula $\varphi$ without any occurrence of {\ifte},
$\Delta\myvdash{T}\varphi$ iff $\Gamma\myvdash{FOL}\varphi$
and $\Delta\yields{T}\varphi$ iff $\Gamma\yields{FOL}\varphi$,
where $\Delta$ is any context obtained from $\Gamma$ by adding
the necessary variable declarations so that $\Delta\myvdash{T}\varphi\wf$
or $\Delta\yields{T}\varphi\wf$.
As a corollary, we obtain completeness of {\T}.

We achieve this by defining a map $\omap:\lang{T}\to\lang{FOL}$
such that $\varphi$ is equivalent (semantically and sintactically) to
$\ofun\varphi$ and examining its behaviour on formulas of {\lang{FOL}}.

\subsection{Definition of {\omap}}

Since terms and formulas mutually depend on each other, we define {\omap}
by mutual recursion on terms and formulas.
For any {\T}-formula $\varphi$, $\ofun\varphi$ will be a {\FOL}-formula;
for any {\T}-term $t$, $\ofun t$ will be a set of pairs {\pair\psi{t'}}.
Intuitively, each pair corresponds to a possible value $t'$ of $t$ together
with a {\FOL}-formula $\psi$ stating when $t=t'$ actually holds.

\begin{definition}\label{defn:omap} The functions
$\omap : \terms{T} \to \wp\left(\lang{FOL}\times\terms{FOL}\right)$ and
$\omap : \lang{T} \to \lang{FOL}$ are simultaneously defined as follows.
\begin{eqnarray*}
\omap : \terms{T} & \to & \wp\left(\lang{FOL}\times\terms{FOL}\right) \\
x_i & \mapsto & \{\pair\top{x_i}\} \\
c_i & \mapsto & \{\pair\top{c_i}\} \\
f_i(t_1,\ldots,t_{a_i}) & \mapsto &
 \left\{\pair{\bigwedge_{k=1}^{a_i} \psi_k}{f_i(t'_1,\ldots,t'_{a_i})}\alt
   \forall k\sep\pair{\psi_k}{t'_k}\in\ofun{t_k}\right\}\\
(\ifthelse\varphi{t_1}{t_2}) & \mapsto &
 \left\{\pair{\ofun\varphi\wedge\psi}{t'_1}\alt
   \pair\psi{t'_1}\in\ofun{t_1}\right\}\cup \\
 & & \ \ 
 \left\{\pair{\ofun{\neg\varphi}\wedge\psi}{t'_2}\alt
   \pair\psi{t'_2}\in\ofun{t_2}\right\} \\
\\
\omap : \lang{T} & \to & \lang{FOL} \\
\bot & \mapsto & \bot \\
\varphi\to\psi & \mapsto & \ofun\varphi\to\ofun\psi \\
\forall x_i\sep\varphi & \mapsto & \forall x_i\sep\ofun\varphi \\
t_1=t_2 & \mapsto & 
 \bigwedge_{\pair{\varphi_k}{t'_k}\in\ofun{t_k}}
 \left(\varphi_1\wedge\varphi_2\to t'_1=t'_2\right) \\
P_i(t_1,\ldots,t_{r_i})& \mapsto & 
 \bigwedge_{\pair{\varphi_k}{t'_k}\in\ofun{t_k}}
 \left(\bigwedge_{k=1}^{r_i}\varphi_i\to P_i(t'_1,\ldots,t'_{r_i})\right)
\end{eqnarray*}
\end{definition}

This definition is well formed since it can be proved by induction that
{\ofun t} is a finite set for any $t$, whence all conjunctions and
disjunctions appearing in the definition of {\omap} are finite.

The extension of {\omap} to contexts is immediate: $\ofun\epsilon=\epsilon$,
$\ofun{(\Gamma,\varphi)}=\ofun\Gamma,\ofun\varphi$ and
$\ofun{(\Gamma,x_i)}=\ofun\Gamma$.

\begin{remark}\label{remark}
The {\omap} function is not stable under $\alpha$-conversion; in
particular it misbehaves if a variable is declared several times in a
context: $x_1,P_1(x_1),x_1,P_2(x_1)$ is {\T}-equivalent to
$x_1,P_1(x_1),x_2,P_2(x_2)$, but they are mapped respectively to
$P_1(x_1),P_2(x_1)$ and $P_1(x_1),P_2(x_2)$, which are \emph{not}
{\FOL}-equivalent.  So throughout this section we strongly appeal to
the Barendregt convention and assume no variable is declared twice in
a context, to avoid complicating the definition of {\omap}.
\end{remark}

\subsection{Syntactic equivalence of {\FOL} and {\T}}

We now state and prove the equivalence between these two systems.

The first proposition confirms the intuition that anything that can be
proved in {\FOL} can also be proved in {\T} modulo some bookkeeping.
\begin{proposition}\label{FOLtoT} Let $\Gamma$ be a {\FOL}-context
and $\varphi\in\lang{FOL}$, and assume $\Gamma\myvdash{FOL}\varphi$.
Let $\Delta$ be any {\T}-context containing exactly the same formulas
as $\Gamma$ such that $\Delta\myvdash{T}\varphi\wf$.  Then
$\Delta\myvdash{T}\varphi$.
\end{proposition}
\begin{proof}
By induction on the derivation of
$\Gamma\myvdash{FOL}\varphi$.  The cases (\emph{$\neg\neg$-E}),
(\emph{$\to$-I}), (\emph{$\forall$-I}) and (\emph{sym}) immediately follow
from the induction hypothesis; (\emph{assum}), (\emph{$=$-fun}) and
(\emph{$=$-pred}) are similar, since the extra condition becomes
$\Delta\myvdash{T}\ok$, which is a consequence of
$\Delta\myvdash{T}\varphi\wf$.

The cases (\emph{$\to$-E}) and (\emph{trans}) are very similar; we consider
the first of these.  If $\Gamma\myvdash{FOL}\varphi$ follows
from $\Gamma\myvdash{FOL}\psi\to\varphi$ and $\Gamma\myvdash{FOL}\psi$
by (\emph{$\to$-E}), let $\Delta$ be a {\T}-context such that $\Delta$
contains the same formulas as $\Gamma$ and $\Delta\myvdash{T}\varphi\wf$.
Notice that we cannot directly apply the induction hypothesis, since
there is no guarantee that $\Delta\myvdash{T}\psi\wf$; therefore, we need
to consider the context $\Delta'$ obtaining by appending to $\Delta$ the free
variables in $\psi$ which are not yet there.
Then $\Delta'\myvdash{T}\psi\wf$ and $\Delta'\myvdash{T}\psi\to\varphi\wf$;
by the induction hypothesis applied to $\Delta'$ follows that
$\Delta'\myvdash{T}\psi$ and $\Delta'\myvdash{T}\psi\to\varphi$.  Applying
(\emph{$\to$-E}) yields $\Delta'\myvdash{T}\varphi$, and by repeatedly
applying Lemma~\ref{Textravars} we prove the thesis.

If $\Gamma\myvdash{FOL}t=t$ is proved by (\emph{refl}), it suffices
observe that $\Delta\myvdash{T}t\wf$ follows from
$\Delta\myvdash{T}(t=t)\wf$ by Proposition~\ref{formwfimptermwfT}
and apply (\emph{refl}).

If $\Gamma\myvdash{FOL}\varphi[x_i:=t]$ is proved by (\emph{$\forall$-E}),
there are two cases.  If $x_i$ actually occurs (free) in $\varphi$,
we need to establish the extra hypothesis $\Delta\myvdash{T}t\wf$;
but $t$ occurs in $\varphi[x_i:=t]$ and
Proposition~\ref{formwfimptermwfT} is applicable.  Otherwise take some
constant symbol $c_j$; then $\Delta\myvdash{T}c_j\wf$ (since
$\Delta\myvdash{T}\ok$ as above argued) and we can prove
$\Delta\myvdash{T}\varphi[x_i:=c_j]$.  But since $x_i$ does not
occur in $\varphi$, $\varphi[x_i:=t]\equiv\varphi[x_i:=c_j]$, hence we
have established the thesis.
\end{proof}

\bigskip\noindent
Next, we look at the effect of {\omap} on {\FOL}-terms and formulas.
\begin{proposition}\label{ofunFOLterms} Let $t\in\terms{FOL}$.
Then $\ofun t$ is a singleton $\{\pair\varphi t\}$, with $\varphi$ a
propositional tautology.
\end{proposition}
\begin{proof}
By induction on $t$.  If $t$ is a variable or a constant, then the thesis
holds since $\top$ is a propositional tautology.  If $t$ is
$f_i(t_1,\ldots,t_{a_i})$, then by induction hypothesis each of
the $\ofun{t_k}$ is a singleton and
$\ofun t=\{\pair{\bigwedge_{k=1}^{a_i}\varphi_k}{f_i(t_1,\ldots,t_{a_i})}\}$
with each $\varphi_k$ a propositional tautology; since a finite conjunction
of propositional tautologies is a propositional tautology the thesis again
holds.
\end{proof}

\begin{proposition}\label{ofunFOLprops} Let $\varphi\in\lang{FOL}$.
Then $\myvdash{FOL}\varphi\leftrightarrow\ofun\varphi$.
\end{proposition}
\begin{proof}
By induction on $\varphi$.  If $\varphi$ is $\bot$ the result is
trivial; the cases $\vartheta\to\psi$ and $\forall x_i\sep\psi$
immediately follow from the induction hypothesis.  Finally, in the
cases $t_1=t_2$ and $P_i(t_1,\ldots,t_{r_i})$ we conclude from
Proposition~\ref{ofunFOLterms} that $\ofun\varphi$ is
$\psi\to\varphi$, with $\psi$ a propositional tautology, and the
result is again immediate.
\end{proof}

\bigskip\noindent
The formulas in the sets {\ofun t} are mutually exclusive and cover all
possibilities.
\begin{lemma}\label{ofunfalse}
Let $t$ be a term of {\terms{T}}, {\pair\varphi{t'}} and
{\pair\psi{t''}} be distinct elements of {\ofun{t}}.
Then $\myvdash{FOL}\neg(\varphi\wedge\psi)$.
\end{lemma}
\begin{proof}
By induction on the structure of $t$.  If $t$ is a variable or a constant
then {\ofun{t}} is a singleton, so the hypothesis cannot be satisfied.
Suppose that $t$ is $f_i(t_1,\ldots,t_{a_i})$; then
$\varphi\equiv\bigwedge_{k=1}^{a_i}\varphi_k$ and
$\psi\equiv\bigwedge_{k=1}^{a_i}\psi_k$.
Since the pairs {\pair\varphi{f(t'_1,\ldots,t'_{a_i})}} and
{\pair\psi{f(t''_1,\ldots,t''_{a_i})}} are different, there is an
index $k$ such that {\pair{\varphi_k}{t'_k}} and {\pair{\psi_k}{t''_k}}
differ; by induction hypothesis
$\myvdash{FOL}\neg(\varphi_k\wedge\psi_k)$, from which the thesis
follows.

Finally, if $t$ is {(\ifthelse\vartheta{t_1}{t_2})} there are three cases to
consider.  Suppose first that $\varphi\equiv\ofun\vartheta\wedge\varphi'$
and $\psi\equiv\ofun\vartheta\wedge\psi'$ with {\pair{\varphi'}{t'}} and
{\pair{\psi'}{t''}} both in {\ofun{t_1}}; then the induction
hypothesis can be applied as above to conclude that
$\myvdash{FOL}\neg(\varphi'\wedge\psi')$, from which follows
that $\myvdash{FOL}\neg(\varphi\wedge\psi)$.  The case when
$\varphi\equiv\neg\ofun\vartheta\wedge\varphi'$ and
$\psi\equiv\neg\ofun\vartheta\wedge\psi'$ with {\pair{\varphi'}{t'}} and
{\pair{\psi'}{t''}} both in {\ofun{t_2}} is analogous.  Finally, if
$\varphi\equiv\ofun\vartheta\wedge\varphi'$ and
$\psi\equiv\neg\ofun\vartheta\wedge\psi'$ with
${\pair{\varphi'}{t'}}\in{\ofun{t_1}}$ and
$\pair{\psi'}{t''}\in{\ofun{t_2}}$ (or vice-versa) then
$\myvdash{FOL}\neg(\ofun\vartheta\wedge\neg\ofun\vartheta)$, since this
formula is a propositional tautology, whence
$\myvdash{FOL}\neg(\varphi\wedge\psi)$.
\end{proof}

\begin{lemma}\label{ofuntrue}
Let $t$ be a term of {\terms{T}}.
Then $\myvdash{FOL}\bigvee_{\pair\varphi{t'}\in\ofun{t}}\varphi$.
\end{lemma}
\begin{proof}
Notice that $\bigvee_{\pair\varphi{t'}\in\ofun t}\varphi$ is a well-formed
formula because {\ofun t} is a finite set.
We again proceed by induction on $t$.  If $t$ is a variable or a
constant, then {\ofun t} is a singleton and
$\bigvee_{\pair\varphi{t'}\in\ofun t}\varphi$ is simply $\top$, which
is provable from any well-formed context.

In the case when $t$ is of the form $f_i(t_1,\ldots,t_{a_i})$, we need
to show that
\[\Gamma\myvdash{FOL}\bigvee_{\pair{\varphi_k}{t'_k}\in\ofun{t_k}}
\bigwedge_{k=1}^{a_i} \varphi_k;\]
we proceed by induction on $a_i$.  If $a_i=1$ then we get exactly the
induction hypothesis for $t$.  Else, we can use the distributive property
of disjunction over conjunction to show that
\begin{eqnarray*}
\bigvee_{\pair{\varphi_k}{t'_k}\in\ofun{t_k}}\bigwedge_{k=1}^{a_i} \varphi_k
 & \leftrightarrow & \bigvee_{\pair{\varphi_k}{t'_k}\in\ofun{t_k}}
       \left(\bigwedge_{k=1}^{a_i-1}\varphi_k\wedge\varphi_{a_i}\right) \\
 & \leftrightarrow & \bigvee_{\pair{\varphi_{a_i}}{t'_{a_i}}\in\ofun{t_{a_i}}}
       \left(\left(\bigvee_{\pair{\varphi_k}{t'_k}\in\ofun{t_k}}
         \bigwedge_{k=1}^{a_i-1}\varphi_k\right)\wedge\varphi_{a_i}\right) \\
 & \leftrightarrow & \bigvee_{\pair{\varphi_{a_i}}{t'_{a_i}}\in\ofun{t_{a_i}}}
       \varphi_{a_i} \\
 & \leftrightarrow & \top
\end{eqnarray*}
where we used the induction hypothesis for $a_i$ in the third equivalence
and the induction hypothesis for $t$ in the last.

In the case when $t$ is {(\ifthelse\vartheta{t_1}{t_2})}, we have to show that
\[\Gamma\myvdash{FOL}\bigvee_{\pair{\varphi}{t'_1}\in\ofun{t_1}}
(\vartheta\wedge\varphi)\vee\bigvee_{\pair{\psi}{t'_2}\in\ofun{t_2}}
(\neg\vartheta\wedge\psi).\]
As in the previous case, we use distributivity of disjunction over
conjunction.
\begin{eqnarray*}
\left(\bigvee_{\pair{\varphi}{t'_1}\in\ofun{t_1}}\vartheta\wedge\varphi\right)
\vee\left(\bigvee_{\pair{\psi}{t'_2}\in\ofun{t_2}}\neg\vartheta\wedge\psi\right)
 & \leftrightarrow &
\left(\vartheta\wedge\bigvee_{\pair{\varphi}{t'_1}\in\ofun{t_1}}\varphi\right)
\vee\left(\neg\vartheta\wedge\bigvee_{\pair{\psi}{t'_2}\in\ofun{t_2}}\psi\right) \\
 & \leftrightarrow & (\vartheta\wedge\top)\vee(\neg\vartheta\wedge\top) \\
 & \leftrightarrow & \vartheta\vee\neg\vartheta \\
 & \leftrightarrow & \top
\end{eqnarray*}
Again, the induction hypothesis is used in the third step.
\end{proof}

\bigskip\noindent
We now examine the effect of {\omap} on {\T} terms and
formulas.
\begin{proposition}\label{ofunTprops} Let $\Gamma$ be a {\T}-context.
\begin{enumerate}[(i)]
\item Let $t\in\terms T$ and $\pair\varphi{t'}\in\ofun t$.
If $\Gamma\myvdash{T}t\wf$ then $\Gamma,\varphi\myvdash{T} t=t'$.
\item Let $\varphi\in\lang T$.  If $\Gamma\myvdash{T}\varphi\wf$
then $\Gamma\myvdash{T}\varphi\leftrightarrow\ofun\varphi$.
\end{enumerate}
\end{proposition}
\begin{proof}
By simultaneous induction on the proof of $\Gamma\myvdash{T}t\wf$ and
$\Gamma\myvdash{T}\varphi\wf$.
If this proof ends with an application of (\emph{var-wf}) or (\emph{const-wf}),
then the result follows from $\myvdash{T}\top$ and (\emph{refl}).

Suppose the last step is an application of (\emph{fun-wf}) to
$\Gamma\myvdash{T} t_k\wf$, $k=1,\ldots,a_i$ and $\Gamma\myvdash{T}\ok$ to
conclude $\Gamma\myvdash{T} f_i(t_1,\ldots,t_{a_i})\wf$.
Then $\varphi$ and $t'$ must be of the forms, respectively,
$\bigwedge_{k=1}^{a_i}\psi_k$ and $t'_k$, with
$\pair{\psi_k}{t'_k}\in\ofun{t_k}$ for each $k$.  By induction hypothesis
$\Gamma,\psi_k\myvdash{T}t_k=t'_k$, from which follows that
$\Gamma,\bigwedge_{j=1}^{a_i}\psi_j\myvdash{T}t_k=t'_k$ for every $k$;
applying (\emph{$=$-fun}) yields the thesis.

If the last step concludes $\Gamma\myvdash{T}\ifthelse\vartheta{t_1}{t_2}\wf$
from assumptions $\Gamma\myvdash{T}\vartheta\wf$ and $\Gamma\myvdash{T} t_k\wf$
for $k=1,2$ using (\emph{if-wf}), then either $\varphi$ is
$\ofun\vartheta\wedge\psi$ with $\pair\psi{t'}\in\ofun{t_1}$ or $\varphi$
is $\neg\ofun\vartheta\wedge\psi$ with $\pair\psi{t'}\in\ofun{t_2}$.  In
the first case we get from induction hypothesis that
$\Gamma,\psi\myvdash{T} t_1=t'$, from which follows that
$\Gamma,\varphi\myvdash{T} t_1=t'$; also we get
$\Gamma,\varphi\myvdash{T}(\ifthelse\varphi{t_1}{t_2})=t_1$ from rule
(\emph{$=$-if-true}) and the induction hypothesis.  Applying (\emph{trans})
yields the result.  The second case is analogous.

As for formulas, the case (\emph{$\bot$-wf}) is trivial and the cases
(\emph{$\to$-wf}) and (\emph{$\forall$-wf}) follow directly from the
induction hypothesis.  We consider the case when
$\Gamma\myvdash{T} P_i(t_1,\ldots,t_{r_i})\wf$ follows from
$\Gamma\myvdash{T} t_k\wf$ for $k=1,\ldots,r_i$ and $\Gamma\myvdash{T}\ok$
by (\emph{pred-wf}).
Taking any pair $\pair{\psi_k}{t'_k}\in\ofun{t_k}$ and applying the
induction hypothesis we conclude that $\Gamma,\psi_k\myvdash{T} t_k=t'_k$;
choosing one such pair for each $k$ we conclude that
$\Gamma,P_i(t_1,\ldots,t_{r_i}),\bigwedge_{j=1}^{r_i}\psi_j\myvdash{T} t_k=t'_k$
for all $k$, hence 
$\Gamma,P_i(t_1,\ldots,t_{r_i}),\bigwedge_{j=1}^{r_i}\psi_j\myvdash{T}%
P(t'_1,\ldots,t'_{r_i})$ using (\emph{$=$-pred}) and propositional reasoning.
Since this holds for all choices of the $\psi_k$'s and $t'_k$'s, we conclude
that
$\Gamma\myvdash{T} P_i(t_1,\ldots,t_{r_i})\to\ofun{(P_i(t_1,\ldots,t_{r_i}))}$.
To prove the converse implication we prove by induction that
$\Gamma',\psi_1,\ldots,\psi_{k}\myvdash{T}%
P(t'_1,\ldots,t'_k,t_{k+1},\ldots,t_{r_i})$ for all choices of
$\pair{\psi_k}{t'_k}\in\ofun{t_k}$, where $\Gamma'$ is
$\Gamma,\ofun{(P_i(t_1,\ldots,t_{r_i}))}$.  If $k=r_i$ the result follows
by propositional reasoning, since the conclusion is one of the conjuncts
in {\ofun{(P_i(t_1,\ldots,t_{r_i}))}}.  Now fix $\psi_1,\ldots,\psi_{k-1}$
as well as $t'_1,\ldots,t'_{k-1}$; by induction hypothesis 
$\Gamma',\psi_1,\ldots,\psi_{k}\myvdash{T}%
P(t'_1,\ldots,t'_k,t_{k+1},\ldots,t_{r_i})$ is provable for all choices
of $\psi_k$; furthermore, from the induction hypothesis for the proposition
it also follows that $\Gamma',\psi_1,\ldots,\psi_k\myvdash{T} t'_k=t_k$,
whence by rule (\emph{$=$-pred}) we get
$\Gamma',\psi_1,\ldots,\psi_k\myvdash{T}%
P(t'_1,\ldots,t'_{k-1},t_k,t_{k+1},\ldots,t_{r_i})$.  By
Lemma~\ref{ofuntrue} and Proposition~\ref{FOLtoT},
$\Gamma',\psi_1,\ldots,\psi_{k-1}\myvdash{T}\bigvee \psi_k$, where
the disjunction ranges over all choices of $\psi_k$, hence by propositional
reasoning we conclude that $\Gamma',\psi_1,\ldots,\psi_{k-1}\myvdash{T}%
P(t'_1,\ldots,t'_{k-1},t_k,\ldots,t_{r_i})$.

The case of (\emph{$=$-wf}) is similar.
\end{proof}

\bigskip\noindent
At this stage we can prove that {\omap} actually embeds {\T} in {\FOL}.
\begin{proposition}\label{TtoFOL} Let $\Gamma$ be a {\T}-context and
$\varphi\in\lang{T}$.  If $\Gamma\myvdash{T}\varphi$, then
$\ofun\Gamma\myvdash{FOL}\ofun\varphi$.
\end{proposition}
\begin{proof}
By induction on the derivation of $\Gamma\myvdash{T}\varphi$.  Rules
(\emph{assum}), (\emph{$\neg\neg$-E}), (\emph{$\to$-I}) and
(\emph{$\to$-E}) are immediate.  For the rules (\emph{$\forall$-I})
and (\emph{$\forall$-E}) one only needs to check that the respective
provisos still holds; this was already mentioned in the remarks to
Definition~\ref{defn:Tderivation}.

For rule (\emph{refl}), assume that $\varphi$ is $t=t$.  Then,
according to Definition~\ref{defn:omap} we need to prove that
$\ofun\Gamma\myvdash{FOL}\bigwedge_{\pair{\varphi_k}{t'_k}\in\ofun{t}}%
\left(\varphi_1\wedge\varphi_2\to t'_1=t'_2\right)$.  Suppose
{\pair{\varphi_1}{t'_1}} and {\pair{\varphi_2}{t'_2}} coincide; then
$\myvdash{FOL}t'_1=t'_2$ by (\emph{refl}).  Using weakening we
conclude that $\myvdash{FOL}\varphi_1\wedge\varphi_2\to t'_1=t'_2$.
If {\pair{\varphi_1}{t'_1}} and {\pair{\varphi_2}{t'_2}} do not
coincide, then $\myvdash{FOL}\neg(\varphi_1\wedge\varphi_2)$ by
Lemma~\ref{ofunfalse}; by propositional reasoning follows that
$\myvdash{FOL}(\varphi_1\wedge\varphi_2)\to t'_1=t'_2$ also in this
case.  The thesis then follows by propositional reasoning and
weakening.

Rule (\emph{sym}) is simpler; if $\Gamma\myvdash T t_2=t_1$ is proved by
symmetry from $\Gamma\myvdash T t_1=t_2$, then by induction hypothesis
$\ofun\Gamma\myvdash{FOL} \bigwedge_{\pair{\varphi_k}{t'_k}\in\ofun{t_k}}
 \left(\varphi_1\wedge\varphi_2\to t'_1=t'_2\right)$; by propositional
reasoning follows that
$\ofun\Gamma\myvdash{FOL} \varphi_1\wedge\varphi_2\to t'_1=t'_2$ for each
$\pair{\varphi_k}{t'_k}\in\ofun{t_k}$.  By propositional reasoning and
(\emph{sym}) we conclude that
$\ofun\Gamma\myvdash{FOL} \varphi_2\wedge\varphi_1\to t'_2=t'_1$, whence
$\ofun\Gamma\myvdash{FOL} \bigwedge_{\pair{\varphi_k}{t'_k}\in\ofun{t_k}}
 \left(\varphi_2\wedge\varphi_1\to t'_2=t'_1\right)$.

Now suppose that $\Gamma\myvdash T t_1=t_3$ is proved from
$\Gamma\myvdash T t_1=t_2$ and $\Gamma\myvdash T t_2=t_3$ by
(\emph{trans}).  Reasoning as above, we get from the induction
hypothesis that
$\ofun\Gamma\myvdash{FOL} \varphi_1\wedge\varphi_2\to t'_1=t'_2$ and
$\ofun\Gamma\myvdash{FOL} \varphi_2\wedge\varphi_3\to t'_2=t'_3$ for
each $\pair{\varphi_k}{t'_k}\in\ofun{t_k}$.  From these follows by
propositional reasoning that
$\ofun\Gamma,\varphi_1,\varphi_2,\varphi_3\myvdash{FOL}t'_1=t'_3$ or,
equivalently,
$\ofun\Gamma,\varphi_2\myvdash{FOL}\varphi_1\wedge\varphi_3\to t'_1=t'_3$.
But by Proposition~\ref{ofuntrue} also $\ofun\Gamma\myvdash{FOL}%
\bigvee_{\pair{\varphi_2}{t'_2}\in\ofun{t_2}}\varphi_2$,
whence again by propositional reasoning we conclude that
$\ofun\Gamma\myvdash{FOL}\varphi_1\wedge\varphi_3\to t'_1=t'_3$.  This
allows us to establish
$\ofun\Gamma\myvdash{FOL}\bigwedge_{\pair{\varphi_k}{t'_k}\in\ofun{t_k}}%
\left(\varphi_1\wedge\varphi_3\to t'_1=t'_3\right)$.

The cases (\emph{$=$-fun}) and (\emph{$=$-pred}) are similar, so we only treat
the first one.  If
$\Gamma\myvdash T f_i(t_1,\ldots,t_{a_i})=f_i(u_1,\ldots,u_{a_i})$ is proved
from (\emph{$=$-fun}), then for all $k=1,\ldots,a_i$, all
$\pair{\varphi_k}{t'_k}\in\ofun{t_k}$ and all
$\pair{\psi_k}{u'_k}\in\ofun{u_k}$ the induction hypothesis ensures that
$\ofun\Gamma\myvdash{FOL}\varphi_k\wedge\psi_k\to t'_k=u'_k$.
Choosing one such pair {\pair{\varphi_k}{t'_k}} and
{\pair{\psi_k}{u'_k}} for each $k$ we can prove, using rule
(\emph{$=$-fun}) and propositional reasoning, that $\ofun\Gamma\myvdash{FOL}%
\bigwedge_{k=1}^{a_i}\varphi_k\wedge\bigwedge_{k=1}^{a_i}\psi_k\to
f_i(t'_1,\ldots,t'_{a_i})=f_i(u'_1,\ldots,u'_{a_i})$; taking the conjunction
over all possible choices of {\pair{\varphi_k}{t'_k}} and
{\pair{\psi_k}{u'_k}} yields the thesis.

Finally we consider the case of the rule (\emph{$=$-if-true}), the case
(\emph{$=$-if-false}) being analogous.  If we conclude
$\Gamma\myvdash T(\ifthelse\vartheta{t_1}{t_2})=t_1$ using this rule, then by
induction hypothesis we can conclude that
$\ofun\Gamma\myvdash{FOL}\ofun\vartheta$.
We want to show that
$\ofun\Gamma\myvdash{FOL}\bigwedge(\psi\wedge\psi')\to t'=t'_1$, where
{\pair\psi{t'}} ranges over {\ofun{(\ifthelse\vartheta{t_1}{t_2})}} and
{\pair{\psi'}{t'_1}} over {\ofun{t_1}}; there are three cases to
consider.
If $t'$ and $t'_1$ coincide, then
$\ofun\Gamma\myvdash{FOL}\psi\wedge\psi'\to t'=t'_1$ is provable using
(\emph{refl}) and weakening.
Otherwise suppose that $\psi$ is $\ofun\vartheta\wedge\varphi$; then $\varphi$ and
$\psi'$ must come from different elements of {\ofun{t_1}}, whence
$\myvdash{FOL}\neg(\varphi\wedge\psi')$ due to
Lemma~\ref{ofunfalse}.  By propositional reasoning again follows that
$\ofun\Gamma\myvdash{FOL}\psi\wedge\psi'\to t'=t'_1$.
Finally, if $\psi$ is $\neg\ofun\vartheta\wedge\varphi$, then
(since $\ofun\Gamma\myvdash{FOL}\ofun\vartheta$) by propositional reasoning
again follows that $\ofun\Gamma\myvdash{FOL}\psi\wedge\psi'\to t'=t'_1$,
from which we can establish the thesis.
\end{proof}

\bigskip\noindent
These results can all be condensed and strengthened in one theorem.
\begin{theorem}\label{TequivFOL}
\emph{(Equivalence)}
\begin{enumerate}[(i)]
\item Let $\Gamma$ be a {\T}-context and $\varphi\in\lang{T}$ such that
$\Gamma\myvdash{T}\varphi\wf$.  Then $\Gamma\myvdash{T}\varphi$ iff
$\ofun\Gamma\myvdash{FOL}\ofun\varphi$.
\item Let $\Gamma$ be a {\FOL}-context and $\varphi\in\lang{FOL}$.
Let $\Delta$ be any {\T}-context containing exactly the same formulas
as $\Gamma$ such that $\Delta\myvdash{T}\varphi\wf$.  Then
$\Gamma\myvdash{FOL}\varphi$ iff $\Delta\myvdash{T}\varphi$.
\end{enumerate}
\end{theorem}
\begin{proof}
\begin{enumerate}[(i)]
\item If $\Gamma\myvdash{T}\varphi$, then
$\ofun\Gamma\myvdash{FOL}\ofun\varphi$ by Proposition~\ref{TtoFOL}.
If $\ofun\Gamma\myvdash{FOL}\ofun\varphi$ then
$\Delta\myvdash{T}\ofun\varphi$ by Proposition~\ref{FOLtoT}, where $\Delta$
is the context obtained by replacing each formula $\psi$ in $\Gamma$ by
$\ofun\psi$ (to see that $\Delta\myvdash{T}\ofun\varphi\wf$ apply
Proposition~3 of~\cite{wie:zwa:03} to~(ii) of Proposition~\ref{ofunTprops}).

We now prove that, for all {\T}-formulas $\theta$, if
$\Delta\myvdash{T}\theta$ then $\Gamma\myvdash{T}\theta$.  The proof
is by induction on $\Gamma$.  If $\Gamma$ is $\epsilon$, then so is
$\Delta$ and there is nothing to prove.
Take $\Gamma\equiv\Gamma',\psi$; then
$\Delta\equiv\Delta',\ofun\psi$ with $\Delta'$ obtained from $\Gamma'$ as
above.  Assume $\Delta',\ofun\psi\myvdash{T}\theta$; then:
\begin{eqnarray*}
\Delta'\myvdash{T}\ofun\psi\to\theta & \mbox{by} &
  \mbox{(\emph{$\to$-I})} \\
\Gamma'\myvdash{T}\ofun\psi\to\theta & \mbox{by} &
  \mbox{induction hypothesis} \\
\Gamma'\myvdash{T}\psi\to\ofun\psi & \mbox{by} &
  \mbox{Proposition~\ref{ofunTprops}} \\
\Gamma',\psi\myvdash{T}\theta & \mbox{by} &
  \mbox{propositional reasoning}
\end{eqnarray*}
Finally, take $\Gamma\equiv\Gamma',x_i$; then $\Delta\equiv\Delta',x_i$ with
$\Delta'$ obtained from $\Gamma'$ by replacing each formula $\psi$ with
$\ofun\psi$.  Assume $\Delta',x_i\myvdash{T}\theta$; then:
\begin{eqnarray*}
\Delta'\myvdash{T}\forall x_i\sep\theta & \mbox{by} &
  \mbox{(\emph{$\forall$-I})} \\
\Gamma'\myvdash{T}\forall x_i\sep\theta & \mbox{by} &
  \mbox{induction hypothesis} \\
\Gamma'\myvdash{T}\theta & \mbox{by} &
  \mbox{(\emph{$\forall$-E})} \\
\Gamma',x_i\myvdash{T}\theta & \mbox{by} &
  \mbox{weakening}
\end{eqnarray*}
\item If $\Gamma\myvdash{FOL}\varphi$ then $\Delta\myvdash{T}\varphi$ by
Proposition~\ref{FOLtoT}.  If $\Delta\myvdash{T}\varphi$ then
$\ofun\Delta\myvdash{FOL}\ofun\varphi$ by Proposition~\ref{TtoFOL};
by Proposition~\ref{ofunFOLprops}
both $\myvdash{FOL}\varphi\leftrightarrow\ofun\varphi$
and $\myvdash{FOL}\psi\leftrightarrow\ofun\psi$ for all $\psi\in\Delta$,
whence $\Delta\myvdash{FOL}\varphi$.
\end{enumerate}
\end{proof}

\subsection{Semantic equivalence of {\FOL} and {\T}}

The equivalence of {\FOL} and {\T} at the semantic level follows the
same structure as the syntactic proof; some steps are simpler, since
{\T}-models and {\FOL}-models are the same thing.

Some of the lemmas below could be proved by appealing to their syntactic
counterparts and the Completeness Theorem; however, a direct proof is
usually quite straightforward, so we present this instead.

\begin{notation}
Throughout this section we often want to see a {\FOL}-model as a {\T}-model
and vice-versa (which is legal, since they are the same thing).  In such
situations we will use the term `model'.
\end{notation}

\begin{definition}\label{defn:lessdefined} Let $\mathfrak M$ be a model
and $\rho,\rho'$ be respectively a {\T}- and a {\FOL}-substitution
for $\mathfrak M$.  We say that $\rho$ is \emph{less defined} than
$\rho'$, denoted $\rho\subseteq\rho'$, if $\rho(x_i)=\rho'(x_i)$
for all $x_i$ such that $\rho(x_i)\defined$.
\end{definition}

To begin with, we prove a semantic analogue of
Proposition~\ref{ofunFOLprops}.
\begin{proposition}\label{ofunFOLequiv} Let $\varphi\in\lang{FOL}$,
$\mathfrak M$ be a {\FOL}-model and $\rho$ be a {\FOL}-substitution
for $\mathfrak M$.  Then $\mymodelss{FOL}M\rho\varphi$ iff
$\mymodelss{FOL}M\rho\ofun\varphi$.
\end{proposition}
\begin{proof}
By induction on $\varphi$.  If $\varphi$ is $\bot$ the result is
trivial; the cases $\vartheta\to\psi$ and $\forall x_i\sep\psi$
immediately follow from the induction hypothesis.  Finally, in the
cases $t_1=t_2$ and $P_i(t_1,\ldots,t_{r_i})$ we conclude from
Proposition~\ref{ofunFOLterms} that $\ofun\varphi$ is
$\psi\to\varphi$, with $\psi$ a propositional tautology, and the
result is again immediate since every first-order model satisfies all
propositional tautologies.
\end{proof}

\bigskip\noindent
The following is the semantic counterparts to Lemmas~\ref{ofunfalse}
and~\ref{ofuntrue}.
\begin{lemma}\label{ofunFOLinterpretation} Let $\mathfrak M$ be a
{\FOL}-model and $\rho$ be a {\FOL}-substitution for $\mathfrak M$.
For any $t\in\terms{T}$, there is exactly one pair
$\pair\varphi{t'}\in\ofun{t}$ such that $\mymodelss{FOL}M\rho\varphi$.
\end{lemma}
\begin{proof}
By structural induction on $t$.
If $t$ is a variable or a constant symbol, then the thesis is immediate.

Let $t$ be $f_i(t_1,\ldots,t_{a_i})$.  By induction hypothesis, for
each $k=1,\ldots,a_i$ there is exactly one pair
$\pair{\varphi_k}{t'_k}\in\ofun{t_k}$ such that
$\mymodelss{FOL}M\rho\varphi_k$; but then
$\mymodelss{FOL}M\rho\bigwedge_{k=1}^{a_i}\varphi_k$, and this is an
element of a pair in $\ofun{(f(t_1,\ldots,t_{a_i}))}$.  Reciprocally,
if $\mymodelss{FOL}M\rho\psi$ for any pair
$\pair\psi{t'}\in\ofun{(f(t_1,\ldots,t_{a_i}))}$ then by definition
$\psi$ must be of the form $\bigwedge_{k=1}^{a_i}\psi_k$ with $\psi_k$
as above; hence by induction hypothesis $\psi_k$ is $\varphi_k$, and
the thesis holds.

Suppose now that $t$ is {\ifthelse\vartheta{t_1}{t_2}}.  
By induction hypothesis for $k=1,2$ there is exactly one pair
$\pair{\psi_k}{t'_k}\in\ofun{t_k}$ such that
$\mymodelss{FOL}M\rho\varphi_k$, so at most two of the formulas
in {\ofun{t}} can be satisfied by $\mathfrak M$ and $\rho$
(namely, $\ofun\vartheta\wedge\psi_1$ and $\neg\ofun\vartheta\wedge\psi_2$).
By definition of $\mymodels{FOL}$ either $\mymodelss{FOL}M\rho\ofun\vartheta$
or $\mymodelss{FOL}M\rho\neg\ofun\vartheta$; in the first case,
$\mymodelss{FOL}M\rho\ofun\vartheta\wedge\psi_1$, and in the second case
$\mymodelss{FOL}M\rho\ofun\neg\vartheta\wedge\psi_2$.  Both cases are obviously
mutually exclusive.
\end{proof}

\bigskip\noindent
Now we deviate from the line of reasoning in the previous section.
The following result compares {\FOL} and {\T} locally.
\begin{proposition}\label{ofunTinterpretation} Let $\mathfrak M$ be a
model and $\rho,\rho'$ be respectively a {\T}- and a {\FOL}-substitution
for $\mathfrak M$ such that $\rho\subseteq\rho'$.  Then the following hold.
\begin{enumerate}[(i)]
\item For any $t\in\terms{T}$ and $\pair\varphi{t'}\in\ofun{t}$ such
that $\mymodelss{FOL}M{\rho'}\varphi$, if $\mymodelss{T}M\rho{t}\wf$
then $\ints{T}M\rho{t}=\ints{FOL}M{\rho'}{t'}$.
\item For any $\varphi\in\lang{T}$, if $\mymodelss{T}M\rho\varphi\wf$
then $\mymodelss{T}M\rho\varphi$ iff $\mymodelss{FOL}M{\rho'}\ofun\varphi$.
\end{enumerate}
Notice that the first statement completely characterizes
$\ints{T}M\rho{t}$, in view of Proposition~\ref{ofunFOLinterpretation}.
\end{proposition}
\begin{proof}
By structural induction on $t$ and $\varphi$.
If $t$ is a variable or a constant symbol, then the thesis is immediate.

Let $t$ be $f_i(t_1,\ldots,t_{a_i})$.  Suppose that
$\ints{T}M\rho{f(t_1,\ldots,t_{a_i})}$ is defined;
then $\ints{T}M\rho{f(t_1,\ldots,t_{a_i})}=%
\intm{T}Mf(\ints{T}M\rho{t_1},\ldots,\ints{T}M\rho{t_{a_i}})$.  Let
{\pair\varphi{t'}} be the element of {\ofun{t}} for which
$\mymodelss{FOL}M{\rho'}\varphi$; then
$\varphi\equiv\bigwedge_{k=1}^{a_i}\varphi_k$ and $t'\equiv
f_i(t'_1,\ldots,t'_{a_i})$ with $\pair{\varphi_k}{t'_k}\in\ofun{t_k}$
for each $k$.  By definition of satisfaction
$\mymodelss{FOL}M{\rho'}\varphi_k$, and the induction hypothesis can
be applied to conclude that
$\ints{T}M\rho{t_k}=\ints{FOL}M{\rho'}{t'_k}$.  Hence
$\ints{T}M\rho{f(t_1,\ldots,t_{a_i})}=%
\intm{T}Mf(\ints{T}M\rho{t_1},\ldots,\ints{T}M\rho{t_{a_i}})=%
\intm{FOL}Mf(\ints{FOL}M\rho{t'_1},\ldots,\ints{FOL}M\rho{t'_{a_i}})=%
\ints{FOL}M{\rho'}{f(t'_1,\ldots,t'_{a_i})}=%
\ints{FOL}M{\rho'}{t'}$.

Suppose now that $t$ is {\ifthelse\vartheta{t_1}{t_2}} and
$\mymodelss{T}M\rho t\wf$.  Then in particular
$\mymodelss{T}M\rho\vartheta\wf$.  Let {\pair\varphi{t'}} be the element of
{\ofun{t}} for which $\mymodelss{FOL}M{\rho'}\varphi$; then either
$\varphi\equiv\ofun\vartheta\wedge\psi$ and $\pair{\psi}{t'}\in\ofun{t_1}$
or $\varphi\equiv\neg\ofun\vartheta\wedge\psi$ and
$\pair{\psi}{t'}\in\ofun{t_2}$.  We consider the first case, the
second being similar.  By definition of satisfaction
$\mymodelss{FOL}M{\rho'}\ofun\vartheta$, hence by induction hypothesis
$\mymodelss{T}M\rho\vartheta$.  By definition of $\ints{T}M\rho\cdot$,
$\ints{T}M\rho{t}=\ints{T}M\rho{t_1}$, and the latter is again by
induction hypothesis equal to $\ints{FOL}M{\rho'}{t'}$ since
$\mymodelss{FOL}M{\rho'}\psi$.

\medskip
We now look at the possible cases for $\varphi$.  If $\varphi$ is
$\bot$ the result is immediate, while the cases $\psi\to\theta$ and
$\forall x_i\sep\psi$ follow from directly from the induction
hypothesis.

The cases $t_1=t_2$ and $P_i(t_1,\ldots,t_{r_i})$ are similar, so we
consider only the first one.  If $\mymodelss{T}M\rho t_1=t_2\wf$, then
$\ints{T}M\rho{t_1}$ and $\ints{T}M\rho{t_2}$ are both defined;
furthermore, $\mymodelss{T}M\rho t_1=t_2$ iff
$\ints{T}M\rho{t_1}=\ints{T}M\rho{t_2}$.  By
Lemma~\ref{ofunFOLinterpretation} there are exactly one pair
$\pair{\varphi_1}{t'_1}\in\ofun{t_1}$ and one pair
$\pair{\varphi_2}{t'_2}\in\ofun{t_2}$ such that
$\mymodelss{FOL}M\rho\varphi_k$; by induction hypothesis
$\ints{T}M\rho{t_k}=\ints{FOL}M{\rho'}{t'_k}$.  Hence
$\mymodelss{FOL}M{\rho'}\psi_1\wedge\psi_2\to t'_1=t'_2$ holds iff
$\mymodelss{T}M\rho t_1=t_2$.  But $\mymodelss{FOL}M{\rho'}\psi$ for
all other conjuncts $\psi$ in {\ofun{(t_1=t_2)}}, since their
antecedent does not hold.  Therefore
$\mymodelss{FOL}M{\rho'}\ofun{(t_1=t_2)}$ iff
$\mymodelss{T}M\rho t_1=t_2$.
\end{proof}

\bigskip\noindent
This result generalizes to consequence.
\begin{proposition}\label{ofunconsequence}
Let $\Gamma$ be a {\T}-context, $\varphi\in\lang{T}$, $\mathfrak M$ be
a model and $\rho$ be a {\T}-substitution for $\mathfrak M$ such
that $\rho(x_i)\undefined$ for all $x_i\in\Gamma$ and
$\Gamma\mymodelss{T}M\rho\varphi\wf$.  Then
$\Gamma\mymodelss{T}M\rho\varphi$ iff
$\ofun\Gamma\mymodelss{FOL}M{\rho'}\ofun\varphi$
for all {\FOL}-substitutions $\rho'$ such that $\rho\subseteq\rho'$.
\end{proposition}
\begin{proof}
By induction on $\Gamma$.  If $\Gamma=\epsilon$, then the thesis
reduces to~(ii) of Proposition~\ref{ofunTinterpretation}.

Suppose $\Gamma\equiv\psi,\Gamma'$.  By definition of $\mymodels{T}$,
since $\Gamma\mymodelss{T}M\rho\varphi\wf$,
$\Gamma\mymodelss{T}M\rho\varphi$ iff
$\mymodelss{T}M\rho\neg\psi$ or $\Gamma'\mymodelss{T}M\rho\varphi$.
The first case is equivalent to $\mymodelss{FOL}M{\rho'}\neg\ofun\psi$
by Proposition~\ref{ofunTinterpretation}; the second case is equivalent
to $\ofun{\Gamma'}\mymodelss{FOL}M{\rho'}\ofun\varphi$ by induction
hypothesis.  In either case the thesis holds.

Finally take $\Gamma\equiv x_i,\Gamma'$.  Again we may conclude that
$\Gamma\mymodelss{T}M\rho\varphi$ iff
$\Gamma'\mymodelss{T}M{\rho[x_i:=a]}\varphi$ for any $a\in A$.  For each
$a$, the latter is equivalent by induction hypothesis (applicable since
$x_i$ does not occur in $\Gamma'$, see Remark~\ref{remark}) to
$\ofun{\Gamma'}\mymodelss{FOL}M{\rho'}\ofun\varphi$ for any $\rho'$
such that $\rho'(x_i)=a$ and $\rho\subseteq\rho'$.  Quantifying first over
all such $\rho'$ and then over all $a$ is equivalent to quantifying over
all $\rho'$ such that $\rho\subseteq\rho'$, since $\rho(x_i)\undefined$,
so again the thesis holds.
\end{proof}

\bigskip\noindent
As a corollary we get equivalence at the semantic level.
\begin{theorem}\label{TequivFOLsem}
\emph{(Equivalence)}
\begin{enumerate}[(i)]
\item Let $\Gamma$ be a {\T}-context and $\varphi\in\lang{T}$ such that
$\Gamma\yields{T}\varphi\wf$.  Then
$\Gamma\yields{T}\varphi$ iff $\ofun\Gamma\yields{FOL}\ofun\varphi$.
\item Let $\Gamma$ be a {\FOL}-context and $\varphi\in\lang{FOL}$.
Let $\Delta$ be any {\T}-context containing exactly the same formulas
as $\Gamma$ such that $\Delta\yields{T}\varphi\wf$.  Then
$\Gamma\yields{FOL}\varphi$ iff $\Delta\yields{T}\varphi$.
\end{enumerate}
\end{theorem}
\begin{proof}
\begin{enumerate}[(i)]
\item Let $\mathfrak M$ be a model.  Under the hypothesis, 
$\Gamma\mymodelss{T}M\emptyset\varphi$ iff
$\ofun\Gamma\mymodelss{FOL}M\rho\ofun\varphi$ for all {\FOL}-substitutions
$\rho$, according to Proposition~\ref{ofunconsequence}.  But the former is
the definition of $\Gamma\mymodelsm{T}M\varphi$, whereas the latter is
the definition of $\ofun\Gamma\mymodelsm{FOL}M\ofun\varphi$.
Since $\mathfrak M$ is arbitrary it follows that 
$\Gamma\yields{T}\varphi$ iff $\ofun\Gamma\yields{FOL}\ofun\varphi$.
\item By the previous alinea, under the hypothesis
$\Delta\yields{T}\varphi$ iff $\ofun\Delta\yields{FOL}\ofun\varphi$.
For each model $\mathfrak M$ and {\FOL}-substitution $\rho$ for $\mathfrak M$,
Proposition~\ref{ofunFOLequiv} implies that $\mymodelss{FOL}M\rho\ofun\Delta$
iff $\mymodelss{FOL}M\rho\Gamma$, since $\ofun\Delta$ contains the formulas
$\ofun\psi$ for $\psi\in\Gamma$.  The same proposition implies that
$\mymodelss{FOL}M\rho\ofun\varphi$ iff $\mymodelss{FOL}M\rho\varphi$.
It then follows that $\ofun\Delta\yields{FOL}\ofun\varphi$ iff
$\Gamma\yields{FOL}\varphi$, whence the thesis holds.
\end{enumerate}
\end{proof}

\subsection{System {\T}: correctness and completeness}

The following is a simple corollary of Theorems~\ref{TequivFOL}
and~\ref{TequivFOLsem}.
\begin{theorem}\emph{(Completeness)}\label{Tcomplete}
Let $\Gamma$ be a {\T}-context and $\varphi\in\lang{T}$.
Then $\Gamma\myvdash{T}\varphi$ iff $\Gamma\yields{T}\varphi$.
\end{theorem}
\begin{proof}
By Theorem~\ref{TequivFOL}, $\Gamma\myvdash{T}\varphi$ iff (1)
$\ofun\Gamma\myvdash{FOL}\ofun\varphi$ and (2)
$\Gamma\myvdash{T}\varphi\wf$.  By completeness of {\FOL}, (1) is
equivalent to $\ofun\Gamma\yields{FOL}\ofun\varphi$; by
Lemma~\ref{Twfsyntsem} (2) is equivalent to
$\Gamma\yields{T}\varphi\wf$.  By Theorem~\ref{TequivFOLsem}, the
conjunction of these two holds iff $\Gamma\yields T\varphi$.
\end{proof}

%%%%%%%%%%%%%%%%%%%%%%%%%%%%%%%%%%%%%%%%%%%%%%%%%%%%%%%%%%%%%%%

\section{Equivalence of {\T} and {\D}}

We now focus on the relationship between {\T} and {\D}.  In order to
do this, we first introduce the domain conditions -- judgements that
internalize the {\D}-notion of well-formedness into the language of
{\T}.  Then we can show the two systems to be equivalent, in the sense
that $\varphi$ follows from $\Gamma$ in {\D} (w.r.t.\ either syntactic
or semantic consequence) iff $\varphi$ follows from $\Gamma$ in {\T}
and the (syntactic or semantic) domain conditions of $\varphi$ w.r.t.\
$\Gamma$ hold.

In this section we ignore the explicit presence of equality in the
language, seeing $t_1=t_2$ as a particular case of $P_i(t_1,t_2)$ for
some $i$ with $r_i=2$.

\subsection{The domain conditions}

We begin by introducing the domain conditions of an expression.  For each
{\T}-term, -formula or -judgement, its domain conditions are a set of
judgements that state that in the expression no function is applied
outside its domain.  According to whether we want to look at derivations
or at models, we get two different notions of domain conditions.

\subsubsection*{The syntactic domain conditions}

The syntactic domain conditions were already introduced
in~\cite{wie:zwa:03}.

\begin{definition}\label{defn:synDCGamma}
Let $\Gamma$ be a {\T}-context.  The domain conditions of a term or
formula relative to $\Gamma$ are defined inductively as follows.
\begin{eqnarray*}
%\synDC[\Gamma] : \terms{T} & \to & \wp(\judg{T}) \\
\synDC[\Gamma](x_i) & = & \emptyset \\
\synDC[\Gamma](c_i) & = & \emptyset \\
\synDC[\Gamma](f_i(t_1,\ldots,t_{a_i})) & = &
 \synDC[\Gamma](t_1)\cup\ldots\cup\synDC[\Gamma](t_{a_i})
 \cup\left\{\Gamma\myvdash{T} D_{f_i}(t_1,\ldots,t_{a_i})\right\} \\
\synDC[\Gamma](\ifthelse{\vartheta}{t_1}{t_2}) & = &
 \synDC[\Gamma](\vartheta)\cup\synDC[\Gamma,\vartheta](t_1)
 \cup\synDC[\Gamma,\neg\vartheta](t_2)\\
\\
%\synDC[\Gamma] : \lang{T} & \to & \wp(\judg{T}) \\
\synDC[\Gamma](\bot) & = & \emptyset \\
\synDC[\Gamma](P_i(t_1,\ldots,t_{r_i})) & = &
 \synDC[\Gamma](t_1)\cup\ldots\cup\synDC[\Gamma](t_{r_i}) \\
%\synDC[\Gamma](t_1=t_2) & = & \synDC[\Gamma](t_1)\cup\synDC[\Gamma](t_2) \\
\synDC[\Gamma](\varphi\to\psi) & = &
 \synDC[\Gamma](\varphi)\cup\synDC[\Gamma,\varphi](\psi) \\
\synDC[\Gamma](\forall x_i\sep\varphi) & = & \synDC[\Gamma,x_i](\varphi)
\end{eqnarray*}
\end{definition}

\begin{definition}\label{defn:synDC}
Let $\Gamma$ be a {\T}-context.  The domain conditions of $\Gamma$ are defined
inductively as follows, where {\judg{T}} is as in the previous
definition.
\begin{eqnarray*}
\synDC(\epsilon) & = & \emptyset \\
\synDC(\Gamma,\varphi) & = & \synDC(\Gamma)\cup\synDC[\Gamma](\varphi) \\
\synDC(\Gamma,x_i) & = & \synDC(\Gamma)
\end{eqnarray*}
\end{definition}

Notice that satisfying the domain conditions in itself does not imply
well-formedness even in {\T}.  At this stage we are not concerned with
bookkeeping affairs, but simply with making sure that all functions are
correctly applied.

\subsubsection*{The semantic domain conditions}

For the semantic domain conditions we begin with a model and a substitution.

\begin{definition}\label{defn:semDCmodel}
Let $\mathfrak M$ be a {\T}-model and $\rho$ be a {\T}-substitution
for $\mathfrak M$.  The domain conditions relative to $\mathfrak M$
and $\rho$ are a set of judgements inductively defined as follows.
\begin{eqnarray*}
%\semDC{} : \terms{T} & \to & \wp(\judg{T}) \\
\semDC{}(x_i) & = & \top \\
\semDC{}(c_i) & = & \top \\
\semDC{}(f_i(t_1,\ldots,t_{a_i})) & = &
 \semDC{}(t_1)\wedge\ldots\wedge\semDC{}(t_{a_i})
 \wedge\left(\ints{T}M\rho{t_1},\ldots,\ints{T}M\rho{t_{a_i}}\right)\in
  \intm{T}M{D_{f_i}} \\
\semDC{}(\ifthelse{\vartheta}{t_1}{t_2}) & = &
 \left\{\begin{array}{ll}
  \semDC{}(\vartheta)\wedge\semDC{}(t_1) & \mbox{if $\mymodelss{T}M\rho\vartheta$}\\
  \semDC{}(\vartheta)\wedge\semDC{}(t_2) & \mbox{if $\mymodelss{T}M\rho\neg\vartheta$}
 \end{array}\right. \\
\\
%\semDC{} : \lang{T} & \to & \wp(\judg{T}) \\
\semDC{}(\bot) & = & \top \\
\semDC{}(P_i(t_1,\ldots,t_{r_i})) & = &
 \semDC{}(t_1)\wedge\ldots\wedge\semDC{}(t_{r_i})\\
%\semDC{}(t_1=t_2) & = & \semDC{}(t_1)\cup\semDC{}(t_2) \\
\semDC{}(\varphi\to\psi) & = &
 \left\{\begin{array}{ll}
  \semDC{}(\varphi)\wedge\semDC{}(\psi) & \mbox{if $\mymodelss{T}M\rho\varphi$}\\
  \semDC{}(\varphi) & \mbox{if $\mymodelss{T}M\rho\neg\varphi$}
 \end{array}\right. \\
\semDC{}(\forall x_i\sep\varphi) & = &
 \forall_{a\in A}\sep\semDC[{\mathfrak M},{\rho[x_i:=a]}]{}(\varphi)
\end{eqnarray*}
\end{definition}

\begin{definition}\label{defn:semDCGamma}
Let $\Gamma$ be a {\T}-context and $\cal X$ stand for an arbitrary term
or formula.  The domain conditions of $\cal X$ relative to $\Gamma$ in
a {\T}-model $\mathfrak M$ with a substitution $\rho$ are defined
inductively as follows.
\begin{eqnarray*}
\semDC{\epsilon}({\cal X}) & = & \semDC{}({\cal X}) \\
\semDC{\varphi,\Gamma}({\cal X}) & = &
 \left\{\begin{array}{ll}
  \semDC{}(\varphi)\wedge\semDC{\Gamma}({\cal X}) &
   \mbox{if $\mymodelss{T}M\rho\varphi$}\\
  \semDC{}(\varphi) &
   \mbox{if $\mymodelss{T}M\rho\neg\varphi$}
 \end{array}\right. \\
\semDC{x_i,\Gamma}({\cal X}) & = &
 \forall_{a\in A}\sep\semDC[{\mathfrak M},{\rho[x_i:=a]}]{\Gamma}({\cal X})
\end{eqnarray*}
\end{definition}

We now quantify over all models to get the domain conditions of a term
or formula w.r.t.\ a context $\Gamma$.

\begin{definition}\label{defn:semDC}
Let $\Gamma$ be a {\T}-context and $\cal X$ stand for an arbitrary term
or formula.  The domain conditions of $\cal X$ relative to $\Gamma$ are
defined as follows.
\[
\semDC[{}]{\Gamma}({\cal X}) =
 \forall_{\mathfrak M}\sep\semDC[{\mathfrak M},\emptyset]{\Gamma}({\cal X})
\]
\end{definition}

\subsubsection*{Equivalence}

The main result in this section is showing that, for well-formed terms,
the syntactic and semantic domain conditions are equivalent.  For this
we first need a couple of simple lemmas, due to the fact that contexts
`grow' in opposite directions in the two definitions.

\begin{lemma}\label{synDCpsi}
Let $\varphi$ be a {\T}-formula, $x_i$ be a variable, $\Gamma$ be a
{\T}-context and $\cal X$ stand for either a {\T}-term or a
{\T}-formula.  Then the following hold.
\begin{enumerate}[(i)]
\item $\synDC[\varphi,\Gamma]({\cal X})=\{\varphi,\Delta\myvdash{T}\psi\alt%
\Delta\myvdash{T}\psi\in\synDC[\Gamma]({\cal X})\}$
\item $\synDC[x_i,\Gamma]({\cal X})=\{x_i,\Delta\myvdash{T}\psi\alt%
\Delta\myvdash{T}\psi\in\synDC[\Gamma]({\cal X})\}$
\item $\synDC(\varphi,\Gamma)=\synDC[\epsilon](\varphi)\cup%
\{\varphi,\Delta\myvdash{T}\psi\alt\Delta\myvdash{T}\psi\in\synDC(\Gamma)\}$
\item $\synDC(x_i,\Gamma)=%
\{x_i,\Delta\myvdash{T}\psi\alt\Delta\myvdash{T}\psi\in\synDC(\Gamma)\}$
\end{enumerate}
\end{lemma}
\begin{proof}
~\begin{enumerate}[(i)]
\item By structural induction on ${\cal X}$ using
definition~\ref{defn:synDCGamma} (straightforward).
\item Analogous.
\item By induction on $\Gamma$ using Definition~\ref{defn:synDC}.  If
$\Gamma$ is $\epsilon$, then the result is immediate.  Take $\Gamma$ to
be $\Gamma',\varphi'$; then
\begin{eqnarray*}
\lefteqn{\synDC(\varphi,\Gamma',\varphi') = }\\
 & = & \synDC(\varphi,\Gamma')\cup\synDC[\varphi,\Gamma'](\varphi') \\
 & = & \synDC[\epsilon](\varphi)\cup\{\varphi,\Delta\myvdash{T}\psi\alt
  \Delta\myvdash{T}\psi\in\synDC(\Gamma')\}\cup \\
  & & \ \ \cup\{\varphi,\Delta\myvdash{T}\psi\alt
  \Delta\myvdash{T}\psi\in\synDC[\Gamma'](\varphi') \\
 & = & \synDC[\epsilon](\varphi)\cup
  \{\varphi,\Delta\myvdash{T}\psi\alt\Delta\myvdash{T}\psi\in
  (\synDC(\Gamma')\cup\synDC[\Gamma'](\varphi'))\} \\
 & = & \synDC[\epsilon](\varphi)\cup
  \{\varphi,\Delta\myvdash{T}\psi\alt\Delta\myvdash{T}\psi\in
  \synDC(\Gamma',\varphi'))\}
\end{eqnarray*}
where in the second step we used the induction hypothesis and part~(i).
Finally, if $\Gamma$ is $\Gamma',x_i$ then
$\synDC(\varphi,\Gamma',x_i)=%
\synDC(\varphi,\Gamma')=%
\synDC[\epsilon](\varphi)\cup
\{\varphi,\Delta\myvdash{T}\psi\alt\Delta\myvdash{T}\psi\in\synDC(\Gamma')\}=%
\synDC[\epsilon](\varphi)\cup
\{\varphi,\Delta\myvdash{T}\psi\alt\Delta\myvdash{T}\psi\in\synDC(\Gamma',x_i)\}$
again using the induction hypothesis on the second step.
\item Analogous.
\end{enumerate}
\end{proof}

\bigskip\noindent
We now assume a fixed {\T}-model $\mathfrak M$ and a {\T}-substition $\rho$
for $\mathfrak M$.

\begin{lemma}\label{synDCwf}
Let $\varphi,\psi$ be {\T}-formulas, $\Gamma,\Delta$ be {\T}-contexts
and $\cal X$ denote some {\T}-term or -formula.  Then the following hold.
\begin{enumerate}[(i)]
\item If $\mymodelss{T}M\rho{\cal X}\wf$, then
$\Delta\mymodelss{T}M\rho\psi\wf$ whenever
$\Gamma,\Delta\myvdash{T}\psi\in\synDC[\Gamma](\varphi)$.
\item If $\Gamma\mymodelss{T}M\rho{\cal X}\wf$, then
$\Delta\mymodelss{T}M\rho\psi\wf$ whenever
$\Delta\myvdash{T}\psi\in\synDC[\Gamma](\varphi)$.
\end{enumerate}
\end{lemma}
\begin{proof}
~\begin{enumerate}[(i)]
\item By structural induction on ${\cal X}$, noticing that only subterms of
${\cal X}$ can occur in $\Delta$ or $\psi$.
\item By induction on $\Gamma$ (straightforward).
\end{enumerate}
\end{proof}

\bigskip\noindent
We can now prove our main lemma.
\begin{lemma}\label{synDCsemDClemma}
Let $\varphi$ be a {\T}-formula, $\Gamma$ be a {\T}-context
and $\cal X$ denote some {\T}-term or -formula.  Then the following hold.
\begin{enumerate}[(i)]
\item If $\mymodelss{T}M\rho{\cal X}\wf$, then
$\semDC{}(\varphi)$ hold iff $\Delta\mymodelss{T}M\rho\psi$
whenever $\Gamma,\Delta\myvdash{T}\psi\in\synDC[\Gamma]({\cal X})$.
\item If $\Gamma\mymodelss{T}M\rho{\cal X}\wf$, then
$\semDC{\Gamma}(\varphi)$ hold iff $\Delta\mymodelss{T}M\rho\psi$ whenever
$\Delta\myvdash{T}\psi\in(\synDC[\Gamma]({\cal X})\cup\synDC(\Gamma))$.
\end{enumerate}
\end{lemma}
\begin{proof}
~\begin{enumerate}[(i)]
\item By structural induction on ${\cal X}$.

If ${\cal X}$ is a constant or a variable, or $\bot$, then both
$\semDC{}({\cal X})$ and $\synDC{\Gamma}({\cal X})$ are $\top$ and
the result is trivial.

If ${\cal X}$ is $f_i(t_1,\ldots,t_{a_i})$, then $\semDC{}({\cal X})$
is $\semDC{}(t_1)\wedge\ldots\wedge\semDC{}(t_{a_i})\wedge%
\{(\ints{T}M\rho{t_1},\ldots,\ints{T}M\rho{t_{a_i}})\in\intm{T}M{D_{f_i}}\}$.
By induction hypothesis, the first $a_i$ conjuncts hold iff
$\Delta\mymodelss{T}M\rho\psi$ whenever
$\Gamma,\Delta\myvdash{T}\psi\in%
\synDC[\Gamma](t_1)\cup\ldots\synDC[\Gamma](t_{a_i})$;
the last conjunct is equivalent to
$\mymodelss{T}M\rho D_{f_i}(t_1,\ldots,t_n)$.  But
$\synDC[\Gamma]({\cal X})=\synDC[\Gamma](t_1)\cup\ldots\cup%
\synDC[\Gamma](t_{a_i})\cup\{\Gamma\myvdash{T}D_{f_i}(t_1,\ldots,t_{a_i})\}$,
which establishes the thesis.

If ${\cal X}$ is $\ifthelse\vartheta{t_1}{t_2}$, then (since
$\mymodelss{T}M\rho\vartheta\wf$) either $\mymodelss{T}M\rho\vartheta$
or $\mymodelss{T}M\rho\neg\vartheta$.  We consider the first case;
here $\semDC{}({\cal X})=\semDC{}(\vartheta)\wedge\semDC{}(t_1)$
and $\synDC[\Gamma]({\cal X})=\synDC[\Gamma](\vartheta)\cup%
\synDC[\Gamma,\vartheta](t_1)\cup\synDC[\Gamma,\neg\vartheta](t_2)$.
By induction hypothesis $\semDC{}(\vartheta)$ holds iff
$\Delta\mymodelss{T}M\rho\psi$ whenever
$\Gamma,\Delta\myvdash{T}\psi\in\synDC[\Gamma](\vartheta)$.
Similarly, $\semDC{}(t_1)$ holds iff $\Delta\mymodelss{T}M\rho\psi$ whenever
$\Gamma,\Delta\myvdash{T}\psi\in\synDC[\Gamma](t_1)$; but
$\mymodelss{T}M\rho\vartheta$, hence $\Delta\mymodelss{T}M\rho\psi$ iff
$\vartheta,\Delta\mymodelss{T}M\rho\psi$.
Therefore, $\semDC{}(t_1)$ hold iff $\varphi,\Delta\mymodelss{T}M\rho\psi$
whenever $\Gamma,\Delta\myvdash{T}\psi\in\synDC[\Gamma](t_1)$.
But it can easily be seen (or proved by a trivial induction) that
$\Gamma,\Delta\myvdash{T}\psi\in\synDC[\Gamma](t_1)$ iff
$\Gamma,\varphi,\Delta\myvdash{T}\psi\in\synDC[\Gamma,\varphi](t_1)$, hence
$\semDC{}(t_1)$ holds iff $\Delta\mymodelss{T}M\rho\psi$
whenever $\Gamma,\Delta\myvdash{T}\psi\in\synDC[\Gamma,\varphi](t_1)$.
Finally, in this situation $\mymodelss{T}M\rho\neg\neg\vartheta$,
hence (using Lemma~\ref{synDCwf}) $\neg\vartheta,\Delta\mymodelss{T}M\rho\psi$
always holds whenever $\Gamma,\Delta\myvdash{T}\psi\in\synDC[\Gamma](t_2)$;
reasoning as above, we conclude that $\Delta\mymodelss{T}M\rho\psi$
whenever $\Gamma,\Delta\myvdash{T}\psi\in\synDC[\Gamma,\neg\vartheta](t_2)$.

If ${\cal X}$ is $P_i(t_1,\ldots,t_{r_i})$ then the result follows
directly from the induction hypothesis.

If ${\cal X}$ is $\varphi\to\varphi'$, we again must consider two cases, one
of which always holds because $\mymodelss{T}M\rho\varphi\wf$.

If $\mymodelss{T}M\rho\varphi$, then
$\semDC{}(\varphi\to\varphi')=\semDC{}(\varphi)\wedge\semDC{}(\varphi')$.
By induction hypothesis these hold iff
$\Delta\mymodelss{T}M\rho\psi$ whenever $\Gamma,\Delta\myvdash{T}\psi$ is
in either $\synDC[\Gamma](\varphi)$ or $\synDC[\Gamma](\varphi')$.  It
remains to check that the latter case is equivalent to
$\Delta'\mymodelss{T}M\rho\psi$ whenever $\Gamma,\Delta'\myvdash{T}\psi$
is an element of $\synDC[\Gamma,\varphi](\varphi')$; but as remarked
above the latter contains all elements of the form
$\Gamma,\varphi,\Delta\myvdash{T}\psi$ with
$\Gamma,\Delta\myvdash{T}\psi\in\synDC[\Gamma](\varphi')$, and in this
situation $\varphi,\Delta\mymodelss{T}M\rho\psi$ iff
$\Delta\mymodelss{T}M\rho\psi$, hence we are done.

If $\mymodelss{T}M\rho\neg\varphi$, then
$\semDC{}(\varphi\to\varphi')=\semDC{}(\varphi)$.
By induction hypothesis the latter holds iff
$\Delta\mymodelss{T}M\rho\psi$ whenever
$\Gamma,\Delta\myvdash{T}\psi\in\synDC[\Gamma](\varphi)$.  It
remains to show that necessarily
$\Delta\mymodelss{T}M\rho\psi$ whenever
$\Gamma,\Delta\myvdash{T}\psi\in\synDC[\Gamma,\varphi](\varphi')$;
but again these are just the judgements of the form
$\varphi,\Delta\mymodelss{T}M\rho\psi$ with
$\Gamma,\Delta\myvdash{T}\psi\in\synDC[\Gamma](\varphi')$, and from
Lemma~\ref{synDCwf} we can conclude that indeed
$\varphi,\Delta\mymodelss{T}M\rho\psi$.

Finally, if ${\cal X}$ is $\forall x_i\sep\varphi$, then the
result is also straightforward:
$\semDC{}(\forall x_i\sep\varphi)$ holds iff
$\semDC[\mathfrak M,{\rho[x_i:=a]}]{}(\varphi)$ holds for all $a\in A$;
by induction hypothesis these hold iff, whenever
$\Gamma,\Delta\myvdash{T}\psi\in\synDC[\Gamma](\varphi)$,
$\Delta\mymodelss{T}M{\rho[x_i:=a]}\psi$; this in turn can be
written as $x_i,\Delta\mymodelss{T}M{\rho}\psi$ (since $a$ is arbitrary),
and again a look at
$\synDC[\Gamma](\forall x_i\sep\varphi)=\synDC[\Gamma,x_i](\varphi)$ shows
that this is exactly what we needed to prove.
\item By induction on $\Gamma$.

Take $\Gamma=\epsilon$.  Then $\semDC{\epsilon}({\cal X})=\semDC{}({\cal X})$,
while $\synDC(\epsilon)=\emptyset$; by~(i), $\semDC{}({\cal X})$ holds iff
$\Delta\mymodelss{T}M\rho\psi$ whenever
$\Delta\myvdash{T}\psi\in\synDC[\epsilon]({\cal X})$.  Thus we have
established the thesis.

Take $\Gamma=\varphi,\Gamma'$.  By Lemma~\ref{synDCpsi},
(1) $\synDC[\varphi,\Gamma']({\cal X})=\{\varphi,\Delta\myvdash{T}\psi\alt%
\Delta\myvdash{T}\psi\in\synDC[\Gamma']({\cal X})\}$ and (2)
$\synDC(\varphi,\Gamma')=\synDC[\epsilon](\varphi)\cup%
\{\varphi,\Delta\myvdash{T}\psi\alt\Delta\myvdash{T}\psi\in\synDC(\Gamma')\}$.
Since $\varphi,\Gamma'\mymodelss{T}M\rho{\cal X}\wf$, both
$\mymodelss{T}M\rho\varphi\wf$ and $\Gamma'\mymodelss{T}M\rho{\cal X}\wf$.
Based on the first of these, there are two cases to consider.

If $\mymodelss{T}M\rho\varphi$, then $\semDC{\varphi,\Gamma'}({\cal X})=%
\semDC{}(\varphi)\wedge\semDC{\Gamma'}({\cal X})$.
By~(i) we know that $\semDC{}(\varphi)$ holds iff
$\Delta\mymodelss{T}M\rho\psi$ for all
$\Delta\myvdash{T}\psi\in\synDC[\epsilon](\varphi)$.
By induction hypothesis, $\semDC{\Gamma'}({\cal X})$ holds iff
$\Delta\mymodelss{T}M\rho\psi$ for all $\Delta\myvdash{T}\psi$ in
either $\synDC[\Gamma']({\cal X})$ or $\synDC(\Gamma')$.  Since
$\mymodelss{T}M\rho\varphi$, this is equivalent to
$\varphi,\Delta\mymodelss{T}M\rho\psi$ for all such $\Delta$ and $\psi$,
and together with the above this establishes the thesis thanks to (1) and
(2).

On the other hand, if $\mymodelss{T}M\rho\neg\varphi$ then
$\semDC{\varphi,\Gamma'}({\cal X})=\semDC{}(\varphi)$.  As above,
these hold iff $\Delta\mymodelss{T}M\rho\psi$ for all
$\Delta\myvdash{T}\psi\in\synDC[\epsilon](\varphi)$.  Finally, if
$\Delta\myvdash{T}\in\synDC(\Gamma')\cup\synDC[\Gamma']({\cal X})$,
we can apply Lemma~\ref{synDCwf} to conclude that
$\Delta\mymodelss{T}M\rho\psi\wf$, whence
$\varphi,\Delta\mymodelss{T}M\rho\psi$ and again we have established the
thesis using (1) and (2).

Finally, take $\Gamma$ to be $x_i,\Gamma'$.  Then
$\semDC{x_i,\Gamma'}({\cal X})$ hold iff, for every $a\in A$,
$\semDC[\mathfrak M,{\rho[x_i:=a]}]{\Gamma'}({\cal X})$ holds.
By induction hypothesis this happens iff
$\Delta\mymodelss{T}M{\rho[x_i:=a]}\psi$ whenever
$\Delta\myvdash{T}\psi\in\synDC(\Gamma')\cup\synDC[\Gamma']({\cal X})$.
Since $a$ is arbitrary, this amounts to saying that
$x_i,\Delta\mymodelss{T}M\rho\psi$ for all
$\Delta\myvdash{T}\psi\in\synDC(\Gamma')\cup\synDC[\Gamma']({\cal X})$,
and by Lemma~\ref{synDCpsi} this is the same as saying that
$\Delta'\mymodelss{T}M\rho\psi'$ for all
$\Delta'\myvdash{T}\psi'\in\synDC(x_i,\Gamma')\cup\synDC[x_i,\Gamma']({\cal X})$.
\end{enumerate}
\end{proof}

\begin{corollary}\label{synDCsemDC}
Let $\varphi$ be a {\T}-formula and $\Gamma$ be a {\T}-context such
that $\Gamma\mymodels{T}\varphi\wf$.  Then
$\semDC[{}]{\Gamma}(\varphi)$ hold iff $\synDC[\Gamma](\varphi)$ hold
and $\synDC(\Gamma)$ hold.
\end{corollary}
\begin{proof}
By definition, $\semDC[{}]{\Gamma}(\varphi)$ holds iff for every {\T}-model
$\semDC[\mathfrak M,\emptyset]{\Gamma}(\varphi)$ holds.  Since by hypothesis
for every such $\mathfrak M$ also
$\Gamma\mymodelss{T}M\emptyset\varphi\wf$, Lemma~\ref{synDCsemDClemma}
is applicable and we conclude that this is equivalent to saying that
$\Delta\mymodelss{T}M\emptyset\psi$ for every
$\Delta\myvdash{T}\psi\in\synDC[\Gamma](\varphi)\cup\synDC(\Gamma)$.
By arbitrariness of $\mathfrak M$, the latter is again equivalent to
$\Delta\mymodels{T}\psi$ for every such $\Delta\myvdash{T}\psi$, and
from Theorem~\ref{Tcomplete} this is again equivalent to
$\Delta\myvdash{T}\psi$ for every
$\Delta\myvdash{T}\psi\in\synDC[\Gamma](\varphi)\cup\synDC(\Gamma)$.
But $\synDC[\Gamma](\varphi)\cup\synDC(\Gamma)$ contains only judgements
of the form $\Delta\myvdash{T}\psi$, so the last statement simply
expresses that $\synDC[\Gamma](\varphi)$ and $\synDC(\Gamma)$ hold.
\end{proof}

\subsection{From {\T} to {\D}: the $\ast$-functions}

For both the syntactic and the semantic equivalences between {\T} and {\D}
we will need to define auxiliary maps between terms and models of both
systems.  These maps are interesting in themselves, so we describe them
in this section together with their main properties.

The {\starmap} operation maps system {\T} to system {\D}.
It makes the partial functions total by setting them to the constant
$c_1$ outside their domain.
Then system {\T} proofs are interpreted in system {\D} as talking about
these `extended' functions.
\begin{definition}\label{defn:starmap} The functions
$\starmap : \terms{T} \to \terms{D}$ and
$\starmap : \lang{T} \to \lang{D}$ are simultaneously defined as follows.
\begin{eqnarray*}
x_i & \mapsto & x_i \\
c_i & \mapsto & c_i \\
f_i(t_1,\ldots,t_{a_i}) & \mapsto &
 \ifthelse{D_{f_i}(\starfun{t_1},\ldots,\starfun{t_{a_i}})}%
{f_i(\starfun{t_1},\ldots,\starfun{t_{a_i}})}{c_1} \\
\ifthelse\vartheta{t_1}{t_2} & \mapsto &
  \ifthelse{\starfun\vartheta}{\starfun{t_1}}{\starfun{t_2}} \\
\\
\bot & \mapsto & \bot \\
P_i(t_1,\ldots,t_{r_i}) & \mapsto &
 P_i(\starfun{t_1},\ldots,\starfun{t_{r_i}}) \\
%t_1=t_2 & \mapsto & \starfun{t_1}=\starfun{t_2} \\
\varphi\to\psi & \mapsto & \starfun\varphi\to\starfun\psi \\
\forall x_i\sep\varphi & \mapsto & \forall x_i\sep\starfun\varphi
\end{eqnarray*}
This function is extended trivially to contexts: $\starfun\epsilon=\epsilon$,
$\starfun{(\Gamma,x_i)}=\starfun\Gamma,x_i$ and
$\starfun{(\Gamma,\varphi)}=\starfun\Gamma,\starfun\varphi$.
\end{definition}

Dually, we can lift every {\D}-model to a {\T}-model using the same
construction.  Given any function symbol $f_i$, we make it total by
assigning the value $c_1$ to $f_i(t_1,\ldots,t_{a_i})$ whenever
$(t_1,\ldots,t_{a_i})$ lies outside of the domain of $f_i$.
\begin{definition}\label{defn:modelmap} Let
$\mathfrak M=\langle A,F,P,C\rangle$ be a {\D}-model.  Then $\modelfunm M$
is the {\T}-model defined by $\modelfunm M=\langle A,\modelfun F,P,C\rangle$,
where $\modelfun F=%
\{\intm{T}{\modelfunm M}{f_1},\ldots,\intm{D}{\modelfunm M}{f_n}\}$ with
\[\intm{T}{\modelfunm M}{f_i}(e_1,\ldots,e_{a_i})=%
\left\{\begin{array}{ll} \intm{D}M{f_i}(e_1,\ldots,e_{a_i}) &%
 \mbox{ if $\intm{D}M{f_i}(e_1,\ldots,e_{a_i})$ is defined} \\
\intm{D}M{c_1} & \mbox{ otherwise}\end{array}\right.\]
Notice that by definition a {\D}-substitution for $\mathfrak M$ is a
{\T}-substitution for $\modelfunm M$ and vice-versa.
\end{definition}

Although they will not be needed in the sequence, the following properties
show the close relationship between these two operations and justify the
notation used.
\begin{proposition}\label{starfunprops}
Let $\mathfrak M$ be a {\D}-model, $\rho$ be a {\D}-substitution for
$\mathfrak M$, $t\in\terms{T}$ and $\varphi\in\lang{T}$.
Then\footnote{In this lemma, $t_1\simeq t_2$ should be read as `$t_1$ and
$t_2$ are both undefined or they are both defined and $t_1=t_2$'.} the
following hold.
\begin{enumerate}[(i)]
\item $\ints{D}M\rho{\starfun t}\simeq\ints{T}{\modelfunm M}\rho{t}$
\item $\mymodelss{D}M\rho\starfun\varphi\wf$ iff
$\mymodelss{T}{\modelfunm M}\rho\varphi\wf$
\item $\mymodelss{D}M\rho\starfun\varphi$ iff
$\mymodelss{T}{\modelfunm M}\rho\varphi$
\end{enumerate}
\end{proposition}
\begin{proof}
By induction on the structure of $t$ and $\varphi$.

%If $t$ is a constant or a variable, the result is trivial, since then
%$\starfun{t}=t$ and its interpretation only depends on $C$ and $\rho$,
%which are the same in $\mathfrak M$ and {\modelfunm M}.

%If $t$ is $f_i(t_1,\ldots,t_{a_i})$ we consider two cases.
%First, $\ints{T}{\modelfunm M}\rho{f_i(t_1,\ldots,t_{a_i})}$ is
%undefined iff $\ints{T}{\modelfunm M}\rho{t_k}$ for some $k$, since
%functions in {\T}-models are total; by induction hypothesis this is
%equivalent to $\ints{D}M\rho{\starfun{t_k}}$, which in turn happens
%iff $\ints{D}M\rho{\starfun{f_i(t_1,\ldots,t_{a_i})}}$ is undefined:
%the definition of $\starmap$ ensures that $\intm{D}M{f_i}$ will only
%be applied if its arguments are inside its domain, and will otherwise
%return a term that is always defined.

%We now assume that $\ints{T}{\modelfunm M}\rho{f_i(t_1,\ldots,t_{a_i})}$ is
%defined.  By induction hypothesis, for all $k$ we know that
%$\ints{D}M\rho{\starfun{t_k}}=\ints{T}{\modelfunm M}\rho{t_k}$;
%hence
%$\ints{T}{\modelfunm M}\rho{f_i(t_1,\ldots,t_{a_i})}=%
%\intm{T}{\modelfunm M}{f_i}(\ints{T}{\modelfunm M}\rho{t_1},\ldots,\ints{T}{\modelfunm M}\rho{t_{a_i}})=%
%\intm{T}{\modelfunm M}{f_i}(\ints{D}M\rho{\starfun{t_1}},\ldots,\ints{D}M\rho{\starfun{t_{a_i}}})$.
%We now have two cases to consider.  If
%$(\ints{D}M\rho{\starfun{t_1}},\ldots,\ints{D}M\rho{\starfun{t_{a_i}}})\in%
%\intm{D}M{D_{f_i}}$, then from the definition of {\starmap} follows that
%$\intm{T}{\modelfunm M}{f_i}(\ints{D}M\rho{\starfun{t_1}},\ldots,\ints{D}M\rho{\starfun{t_{a_i}}})=%
%\intm{D}M{f_i}(\ints{D}M\rho{\starfun{t_1}},\ldots,\ints{D}M\rho{\starfun{t_{a_i}}})=%
%\ints{D}M\rho{f_i(\starfun{t_1},\ldots,\starfun{t_{a_i}})}=%
%\ints{D}M\rho{\starfun{(f_i(t_1,\ldots,t_{a_i}))}}$ since the condition
%in the outermost {\ifte} of the last term is satisfied.
%Otherwise
%$\intm{T}{\modelfunm M}{f_i}(\ints{D}M\rho{\starfun{t_1}},\ldots,\ints{D}M\rho{\starfun{t_{a_i}}})=%
%\intm{D}M{c_1}=%
%\ints{D}M\rho{\starfun{(f_i(t_1,\ldots,t_{a_i}))}}$ since the condition
%in the outermost {\ifte} of the last term is not satisfied.

%The case $t=\ifthelse{\vartheta}{t_1}{t_2}$ follows from the induction
%hypothesis:
%\begin{eqnarray*}
%\ints{D}M\rho{\starfun{(\ifthelse{\vartheta}{t_1}{t_2})}}
% & = & \left\{\begin{array}{ll}
%  \ints{D}M\rho{\starfun{t_1}} & \mbox{ if } \mymodelss{D}M\rho\starfun\vartheta \\
%  \ints{D}M\rho{\starfun{t_2}} & \mbox{ if } \mymodelss{D}M\rho\neg\starfun\vartheta \end{array}\right. \\
% & = & \left\{\begin{array}{ll}
%  \ints{T}{\modelfunm M}\rho{t_1} & \mbox{ if } \mymodelss{T}{\modelfunm M}\rho\vartheta \\
%  \ints{T}{\modelfunm M}\rho{t_2} & \mbox{ if } \mymodelss{T}{\modelfunm M}\rho\neg\vartheta \end{array}\right. \\
% & = & \ints{T}{\modelfunm M}\rho{\ifthelse{\vartheta}{t_1}{t_2}}
%\end{eqnarray*}

%As for formulas, the case when $\varphi$ is $\bot$ is trivial and the cases
%$\psi\to\vartheta$ and $\forall x_i\sep\psi$ follow directly from the
%induction hypothesis.  The cases $P_i(t_1,\ldots,t_{r_i})$ and
%$t_1=t_2$ are similar, so we only consider the first.  Using the induction
%hypothesis we see that
%\begin{eqnarray*}
%\mymodelss{D}M\rho\starfun{(P_i(t_1,\ldots,t_{r_i}))}
% & \mbox{iff} & (\ints{D}M\rho{\starfun{t_1}},\ldots,\ints{D}M\rho{\starfun{t_{r_i}}})\in\intm{D}M{P_i} \\
% & \mbox{iff} & (\ints{T}{\modelfunm M}\rho{t_1},\ldots,\ints{T}{\modelfunm M}\rho{t_{r_i}})\in\intm{T}M{P_i} \\
% & \mbox{iff} & \mymodelss{T}{\modelfunm M}\rho P_i(t_1,\ldots,t_{r_i})
%\end{eqnarray*}
%which proves the last point for this case; finally, observe that the
%expressions in the last chain of inequalities are defined under the same
%circumstances, so that also
%$\mymodelss{D}M\rho\starfun{(P_i(t_1,\ldots,t_{r_i}))}\wf$ iff
%$\mymodelss{T}{\modelfunm M}\rho P_i(t_1,\ldots,t_{r_i})$\wf.
\end{proof}

\subsection{Syntactic equivalence}

We will not prove the syntactic equivalence between systems {\T}
and {\D}, since this is the subject of~\cite{wie:zwa:03}.  The
final result is the following.
\begin{theorem}\label{TequivD}
Let $\Gamma$ be a context and $\varphi$ be a formula.  Then the
following statements are equivalent.
\begin{enumerate}[(i)]
\item $\Gamma\myvdash{D}\varphi$
\item $\Gamma\myvdash{T}\varphi$ and $\synDC{}(\Gamma)$ hold and
$\synDC[{\Gamma}](\varphi)$ hold.
\end{enumerate}
\end{theorem}

%In this subsection we summarize the proof originally published
%in~\cite{wie:zwa:03} of the syntactic equivalence between system {\T}
%with domain conditions and system {\D}.  The proofs can be found in
%that same reference.

%\begin{proposition}\label{freek8}
%~\begin{enumerate}[(i)]
%\item If $\Gamma\myvdash{D}\ok$, then $\Gamma\myvdash{T}\ok$ and
%$\synDC(\Gamma)$ hold.
%\item If $\Gamma\myvdash{D} t\wf$, then $\Gamma\myvdash{T} t\wf$,
%$\synDC(\Gamma)$ hold and $\synDC[\Gamma](t)$ hold.
%\item If $\Gamma\myvdash{D}\varphi\wf$, then $\Gamma\myvdash{T}\varphi\wf$,
%$\synDC(\Gamma)$ hold and $\synDC[\Gamma](\varphi)$ hold.
%\item If $\Gamma\myvdash{D}\varphi$, then $\Gamma\myvdash{T}\varphi$,
%$\synDC(\Gamma)$ hold and $\synDC[\Gamma](\varphi)$ hold.
%\end{enumerate}
%\end{proposition}

%\begin{proposition}\label{freek10}
%~\begin{enumerate}[(i)]
%\item If $\Gamma\myvdash{D} t\wf$ then $\Gamma\myvdash{D} t=\starfun{t}$.
%\item If $\Gamma\myvdash{D}\varphi\wf$ then
%$\Gamma\myvdash{D}\varphi\leftrightarrow\starfun{\varphi}$.
%\end{enumerate}
%\end{proposition}

%\begin{proposition}\label{freek11} If $\Gamma\myvdash{D}\ok$ then
%$\Gamma\myvdash{D}\varphi$ iff $\starfun\Gamma\myvdash{D}\varphi$.
%\end{proposition}

%\begin{proposition}\label{freek12}
%~\begin{enumerate}[(i)]
%\item If $\Gamma\myvdash{T}\ok$, then $\starfun\Gamma\myvdash{D}\ok$.
%\item If $\Gamma\myvdash{T} t\wf$, then $\starfun\Gamma\myvdash{D}\starfun{t}\wf$.
%\item If $\Gamma\myvdash{T}\varphi\wf$, then $\starfun\Gamma\myvdash{D}\starfun{\varphi}\wf$.
%\item If $\Gamma\myvdash{T}\varphi$, then $\starfun\Gamma\myvdash{D}\starfun\varphi$.
%\end{enumerate}
%\end{proposition}

%\begin{proposition}\label{freek13}\marginpar{wrong}
%~\begin{enumerate}[(i)]
%\item If $\Gamma\myvdash{D}\ok$ and $\synDC[\Gamma](t)$ hold, then
%$\Gamma\myvdash{D} t=\starfun t$.
%\item If $\Gamma\myvdash{D}\ok$ and $\synDC[\Gamma](\varphi)$ hold, then
%$\Gamma\myvdash{D}\varphi\leftrightarrow\varphi$.
%\end{enumerate}
%\end{proposition}

%\begin{proposition}\label{freek14} If $\synDC(\Gamma)$ hold, then
%$\Gamma\myvdash{D}\varphi$ iff $\starfun\Gamma\myvdash{D}\varphi$.
%\end{proposition}

%\begin{proposition}\label{freek15} If $\synDC(\Gamma)$ and
%$\synDC[\Gamma](\varphi)$ hold and $\Gamma\myvdash{T}\varphi$, then
%$\Gamma\myvdash{D}\varphi$.
%\end{proposition}

\subsection{Semantic equivalence}

The proof of semantic equivalence between system {\T} with domain
conditions and system {\D} proceeds in two steps.  First, we define
the obvious map {\restrmap} from {\T}-models to {\D}-models and prove
that {\D}-consequence implies {\T}-consequence and the domain
conditions; then we use {\modelmap} and prove the converse implication.

The proofs of the two implications are in themselves very similar, and
quite straightforward.

We begin by mapping {\T}-models to {\D}-models.  The trick is simply
to restrict each function to its domain.

\begin{definition}\label{defn:restrmap} Let
$\mathfrak M=\langle A,F,P,C\rangle$ be a {\T}-model.  Then $\restrfunm M$
is the {\D}-model defined by $\restrfunm M=\langle A,\restrfun F,P,C\rangle$,
where $\restrfun F=%
\{\intm{D}{\restrfunm M}{f_1},\ldots,\intm{D}{\restrfunm M}{f_n}\}$ with
\[\intm{D}{\restrfunm M}{f_i}(e_1,\ldots,e_{a_i})=%
\intm{T}M{f_i}(e_1,\ldots,e_{a_i})
 \mbox{ if $\intm{T}M{f_i}(e_1,\ldots,e_{a_i})\in\intm{T}M{D_{f_i}}$}\]
Notice that, again by definition, a {\T}-substitution for $\mathfrak M$ is a
{\D}-substitution for $\restrfunm M$ and vice-versa.
\end{definition}

In the following, we assume a fixed {\T}-model $\mathfrak M$ and a
{\T}-substitution $\rho$ for $\mathfrak M$.  Also, since the languages
of {\T} and {\D} coincide and we want to see the same terms and formulas
as elements of both, we will simply speak of `terms' and `formulas'.

We begin by stating a trivial result.
\begin{lemma}\label{DwwfTwf} Let $\Gamma$ be a context and ${\cal X}$
stand for either a term $t$ or a formula $\varphi$.
If $\Gamma\mymodelss{D}{\restrfunm M}\rho{\cal X}\wwf$,
then $\Gamma\mymodelss{T}M\rho{\cal X}\wf$.
\end{lemma}
\begin{proof}
Straightforward induction.
\end{proof}

\bigskip\noindent
The proof itself proceeds in three steps: first we relate local
interpretations and local validity in $\mathfrak M$ and $\restrfunm M$;
then we add a context and look at consequence; and finally we abstract
from $\mathfrak M$ to get the general result.
\begin{proposition}\label{DintTint}
Let $t$ be a term and $\varphi$ be a formula.  Then the following hold.
\begin{enumerate}[(i)]
\item If $\mymodelss{D}{\restrfun M}\rho t\wf$, then $\semDC{}(t)$
hold and $\ints{D}{\restrfun M}\rho t=\ints{T}{M}\rho t$.
\item If $\mymodelss{D}{\restrfun M}\rho\varphi\wf$, then
$\semDC{}(\varphi)$ hold and $\mymodelss{D}{\restrfun M}\rho\varphi$
iff $\mymodelss{T}{M}\rho\varphi$.
\end{enumerate}
\end{proposition}
\begin{proof}
By simultaneous induction on $t$ and $\varphi$.

If $t$ is a constant $c_i$, then $\mymodelss{D}{\restrfun M}\rho{c_i}\wf$
and the thesis is trivial.
If $t$ is a variable $x_i$, then $\mymodelss{D}{\restrfun M}\rho{x_i}\wf$
iff $\rho(x_i)$ is defined, and in that case the result also follows.

Suppose that $t$ is $f_i(t_1,\ldots,t_{a_i})$ and that
$\mymodelss{D}{\restrfun M}\rho{f_i(t_1,\ldots,t_{a_i})}\wf$.  Then
$\mymodelss{D}{\restrfun M}\rho{t_j}\wf$ for $j=1,\ldots,a_i$ and
$(\ints{D}{\restrfun M}\rho{t_1},\ldots,\ints{D}{\restrfun M}\rho{t_j})%
\in\intm{D}{\restrfun M}{D_{f_i}}$.
By induction hypothesis then
$\ints{D}{\restrfun M}\rho{t_j}=\ints{T}M\rho{t_j}$ for each $j$
and $\semDC{}{t_j}$ hold; furthermore, since
$\intm{D}{\restrfun M}{D_{f_i}}=\intm{T}M{D_{f_i}}$, also
$(\ints{T}M\rho{t_1},\ldots,\ints{T}M\rho{t_j})\in\intm{T}M{D_{f_i}}$,
hence $\semDC{}{t}$ also hold.  Finally,
$\ints{D}{\restrfun M}\rho{f_i(t_1,\ldots,t_{a_i})}=%
\intm{D}{\restrfun M}{f_i}(\ints{D}{\restrfun M}\rho{t_1},\ldots,\ints{D}{\restrfun M}\rho{t_j})=%
\intm{T}M{f_i}(\ints{T}M\rho{t_1},\ldots,\ints{T}M\rho{t_j})=%
\ints{T}M\rho{f_i(t_1,\ldots,t_{a_i})}$.

Now take $t$ to be $\ifthelse\vartheta{t_1}{t_2}$ and suppose
$\mymodelss{D}{\restrfun M}\rho{\ifthelse\vartheta{t_1}{t_2}}\wf$.
Then there are two cases, which can be treated similarly; we consider
the case when $\mymodelss{D}{\restrfun M}\rho\neg\vartheta$, which has
a slight nuance.  In this case,
$\ints{D}{\restrfun M}\rho{\ifthelse\vartheta{t_1}{t_2}}=\ints{D}{\restrfun M}\rho{t_2}$
and $\mymodelss{D}{\restrfun M}\rho t_1\wwf$.
By induction hypothesis $\ints{D}{\restrfun M}\rho{t_2}=\ints{T}M\rho{t_2}$
and $\semDC{}(t_2)$ hold; by Lemma~\ref{DwwfTwf} also
$\mymodelss{T}M\rho t_1\wf$.
Finally, we would like to apply the induction hypothesis and conclude
that $\mymodelss{T}M\rho\neg\vartheta$, but this cannot be done
because $\neg\vartheta$ is not structurally smaller than $t$.  What we
can conclude is that $\not\mymodelss{T}M\rho\vartheta$ and
$\semDC{}(\vartheta)$ hold; but we also know that
$\mymodelss{D}{\restrfun M}\rho\neg\vartheta\wwf$, hence by
Lemma~\ref{DwwfTwf}, Proposition~\ref{modelTwf} we arrive at the
desired conclusion.
Therefore $\semDC{}(\ifthelse\vartheta{t_1}{t_2})=%
\semDC{}\vartheta\cup\semDC{}(t_2)$ hold and
$\ints{T}M\rho{\ifthelse\vartheta{t_1}{t_2}}=\ints{T}M\rho{t_2}=%
\ints{D}{\restrfun M}\rho{\ifthelse\vartheta{t_1}{t_2}}$.

As for the formulas, the case of $\bot$ is trivial and the case when
$\varphi$ is $P_i(t_1,\ldots,t_{r_i})$ follows directly from the
induction hypothesis and the fact that
$\intm{D}{\restrfun M}{P_i}=\intm{T}M{P_i}$.

Consider now the case $\varphi\to\psi$ and assume that
$\mymodelss{D}{\restrfun M}\rho\varphi\to\psi\wf$.  Then
$\mymodelss{D}{\restrfun M}\rho\varphi\wf$, whence by induction
hypothesis $\semDC{}(\varphi)$ hold and
$\mymodelss{D}{\restrfun M}\rho\varphi$ iff $\mymodelss{T}M\rho\varphi$.

Suppose first that $\mymodelss{D}{\restrfun M}\rho\varphi$.  Then
$\mymodelss{D}{\restrfun M}\rho\psi\wf$, and we can again apply the
induction hypothesis to conclude that $\semDC{}(\psi)$ hold and
$\mymodelss{D}{\restrfun M}\rho\psi$ iff $\mymodelss{T}M\rho\psi$.
It is then easy to see that the thesis holds in this case.

Otherwise, we can reason as above to conclude that
$\mymodelss{T}M\rho\neg\varphi$, whence $\mymodelss{T}M\rho\varphi\to\psi$
and the thesis is immediately proved.

Finally we consider the case when $\varphi$ is $\forall x_i\sep\psi$.
If $\mymodelss{D}{\restrfun M}\rho\forall x_i\sep\psi\wf$, then
$\mymodelss{D}{\restrfun M}{\rho[x_i:=a]}\psi\wf$ for any $a\in A$.
By induction hypothesis, $\semDC[{\mathfrak M},\rho[x_i:=a]]{}(\psi)$
holds and $\mymodelss{D}{\restrfun M}{\rho[x_i:=a]}\psi$ iff
$\mymodelss{T}M{\rho[x_i:=a]}\psi$; hence $\semDC{}(\forall x_i\sep\psi)$
and $\mymodelss{D}{\restrfun M}\rho\forall x_i\sep\psi$ iff
$\mymodelss{T}M\rho\forall x_i\sep\psi$.
\end{proof}

\begin{proposition}\label{DconsTcons}
Let $\varphi$ be a formula and $\Gamma$ be a context.  If
$\Gamma\mymodelss{D}{\restrfun M}\rho\varphi$, then
$\semDC{\Gamma}(\varphi)$ hold and $\Gamma\mymodelss{T}{M}\rho\varphi$.
\end{proposition}
\begin{proof}
By induction on $\Gamma$.

Let $\Gamma$ be $\epsilon$.  By definition,
$\epsilon\mymodelss{D}{\restrfun M}\rho\varphi$ iff
$\mymodelss{D}{\restrfun M}\rho\varphi$; if the latter holds then also
$\mymodelss{D}{\restrfun M}\rho\varphi\wf$, whence by
Proposition~\ref{DintTint} we conclude that
$\mymodelss{T}M\rho\varphi$, which is equivalent to
$\epsilon\mymodelss{T}M\rho\varphi$, and $\semDC{}(\varphi)$, which is
by definition $\semDC{\epsilon}(\varphi)$.

Now take $\Gamma$ to be $\psi,\Gamma'$.  If
$\psi,\Gamma'\mymodelss{D}{\restrfun M}\rho\varphi$, then there are
two possibilities to consider.

Suppose that $\mymodelss{D}{\restrfun M}\rho\psi$ and
$\Gamma'\mymodelss{D}{\restrfun M}\rho\varphi$.  As in the previous case
we can then conclude from Proposition~\ref{DintTint} that
$\mymodelss{T}M\rho\psi$ and $\semDC{}(\psi)$ hold; furthermore, by
induction hypothesis $\Gamma'\mymodelss{T}M\rho\varphi$ and
$\semDC{\Gamma'}(\varphi)$ hold.  But then $\semDC{\psi,\Gamma'}(\varphi)$
hold and $\psi,\Gamma'\mymodelss{T}M\rho\varphi$.

The other case is easier.  If $\mymodelss{D}{\restrfun M}\rho\neg\psi$
and $\Gamma'\mymodelss{D}{\restrfun M}\rho\varphi\wwf$, we can again
conclude from Proposition~\ref{DintTint} that $\mymodelss{T}M\rho\neg\psi$
and $\semDC{}(\neg\psi)$ hold; the latter can immediately be seen to
coincide with $\semDC{}(\psi)$.  Furthermore, by Lemma~\ref{DwwfTwf}
also $\Gamma'\mymodelss{T}M\rho\varphi\wf$.  Hence again
$\semDC{\psi,\Gamma'}(\varphi)$ hold and
$\psi,\Gamma'\mymodelss{T}M\rho\varphi$.

Finally take $\Gamma$ to be $x_i,\Gamma'$.  Then
$x_i,\Gamma'\mymodelss{D}{\restrfun M}\rho\varphi$ iff
$\Gamma'\mymodelss{D}{\restrfun M}{\rho[x_i:=a]}\varphi$ for all $a\in
A$.  From the latter we conclude by induction hypothesis that for any
$a\in A$ also $\Gamma'\mymodelss{T}M{\rho[x_i:=a]}\varphi$ and
$\semDC{\Gamma'}(\varphi)$ hold, whence by definition
$x_i,\Gamma'\mymodelss{T}M\rho\varphi$ and
$\semDC{x_i,\Gamma'}(\varphi)$ hold.
\end{proof}

\begin{proposition}\label{DvalTval}
Let $\varphi$ be a formula and $\Gamma$ be a context.  If
$\Gamma\mymodels{D}\varphi$, then $\semDC[{}]{\Gamma}(\varphi)$ hold
and $\Gamma\mymodels{T}\varphi$.
\end{proposition}
\begin{proof}
Suppose that $\Gamma\mymodels{D}\varphi$ and let $\mathfrak M$ be
a {\T}-model; then in particular
$\Gamma\mymodelss{D}{\restrfun M}\emptyset\varphi$, whence we may apply
Proposition~\ref{DconsTcons} to conclude that
$\Gamma\mymodelss{T}M\emptyset\varphi$ and
$\semDC[{\mathfrak M},\emptyset]{\Gamma}(\varphi)$ hold.  But since
$\mathfrak M$ is arbitrary we conclude that
$\Gamma\mymodels{T}\varphi$ and $\semDC[{}]{\Gamma}(\varphi)$.
\end{proof}

\bigskip\noindent
We now prove the converse implication.  The proof follows exactly the same
three steps, with an auxiliary lemma as before.

We now assume a fixed {\D}-model $\mathfrak M$ and a {\D}-substitution $\rho$
for $\mathfrak M$.
\begin{lemma}\label{TwfDwwf} Let $\Gamma$ be a context and ${\cal X}$
stand for either a term $t$ or a formula $\varphi$.
If $\Gamma\mymodelss{T}{\modelfun M}\rho{\cal X}\wf$,
then $\Gamma\mymodelss{D}M\rho{\cal X}\wwf$.
\end{lemma}
\begin{proof}
Straightforward induction.
\end{proof}

\begin{proposition}\label{TintDint}
Let $t$ be a term and $\varphi$ be a formula.  Then the following hold.
\begin{enumerate}[(i)]
\item If $\mymodelss{T}{\modelfun M}\rho t\wf$ and $\semDCa{}(t)$
hold, then $\mymodelss{D}M\rho t\wf$ and
$\ints{T}{\modelfun M}\rho t=\ints{D}M\rho t$.
\item If $\mymodelss{T}{\modelfun M}\rho\varphi\wf$ and $\semDCa{}(\varphi)$
hold, then $\mymodelss{T}{\modelfun M}\rho\varphi$
iff $\mymodelss{D}{M}\rho\varphi$.
\end{enumerate}
\end{proposition}
\begin{proof}
By simultaneous induction on $t$ and $\varphi$.  If $t$ is a constant or a
variable, the statement is trivial; likewise if $\varphi$ is $\bot$.

Suppose $t$ is $f_i(t_1,\ldots,t_{a_i})$.  If
$\mymodelss{T}{\modelfun M}\rho f_i(t_1,\ldots,t_{a_i})\wf$, then
$\mymodelss{T}{\modelfun M}\rho t_j\wf$ for $j=1,\ldots,a_i$.
On the other hand, from $\semDCa{}(f_i(t_1,\ldots,t_{a_i}))$ we conclude
that $\semDCa{}(t_j)$ also hold for each $j$ and that
$(\ints{T}{\modelfun M}\rho{t_1},\ldots,\ints{T}{\modelfun M}\rho{t_{a_i}})\in\intm{T}{\modelfun M}\rho{D_{f_i}}$.
The induction hypothesis is therefore applicable, and we conclude that
for each $j$ both $\mymodelss{D}M\rho t_j\wf$ and
$\ints{D}M\rho{t_j}=\ints{T}{\modelfun M}\rho{t_j}$.  Since
$\intm{D}M{D_{f_i}}=\intm{T}M{D_{f_i}}$, we conclude that
$\mymodelss{D}M\rho f_i(t_1,\ldots,t_{a_i})\wf$ and
$\ints{D}M\rho{f_i(t_1,\ldots,t_{a_i})}=\ints{T}{\modelfun M}\rho{f_i(t_1,\ldots,t_{a_i})}$.

Suppose now that $t$ is $\ifthelse\vartheta{t_1}{t_2}$.  From
$\mymodelss{T}{\modelfun M}\rho\ifthelse\vartheta{t_1}{t_2}\wf$ we
conclude that $\mymodelss{T}{\modelfun M}\rho\vartheta\wf$ and
$\mymodelss{T}{\modelfun M}\rho t_i\wf$ for $i=1,2$.

We consider the case when $\mymodelss{T}{\modelfun M}\rho\vartheta$,
since the other one is similar.  In this case,
$\semDCa{}(\ifthelse\vartheta{t_1}{t_2})$ become simply
$\semDCa{}(\vartheta)\cup\semDCa{}(t_1)$.  Assuming these to hold, we
can apply the induction hypothesis twice to conclude on the one hand
that $\mymodelss{D}M\rho\vartheta\wf$ and $\mymodelss{D}M\rho\vartheta$,
and on the other hand that $\mymodelss{D}M\rho t_1\wf$ and
$\ints{D}M\rho{t_1}=\ints{T}{\modelfun M}\rho{t_1}$.  Finally, since
$\mymodelss{T}{\modelfun M}\rho{t_2}\wf$ we may apply Lemma~\ref{TwfDwwf}
to conclude that $\mymodelss{D}M\rho t_2\wwf$.  Hence
$\ints{D}M\rho{\ifthelse\vartheta{t_1}{t_2}}=\ints{D}M\rho{t_1}=%
\ints{T}{\modelfun M}\rho{t_1}=%
\ints{T}{\modelfun M}\rho{\ifthelse\vartheta{t_1}{t_2}}$.

The case when $\varphi$ is $P_i(t_1,\ldots,t_{r_i})$ follows directly
from the induction hypothesis.

In the case $\varphi\to\psi$, supposing that
$\mymodelss{T}{\modelfun M}\rho\varphi\to\psi\wf$ amounts to assuming
that $\mymodelss{T}{\modelfun M}\rho\varphi\wf$ and
$\mymodelss{T}{\modelfun M}\rho\psi\wf$.  There are again two cases to
consider.

If $\mymodelss{T}{\modelfun M}\rho\varphi$, then $\semDCa{}(\varphi\to\psi)$
are equivalent to $\semDCa{}(\varphi)\cup\semDCa{}(\psi)$.  The induction
hypothesis then yields that $\mymodelss{D}M\rho\varphi$ and that
$\mymodelss{D}M\rho\psi$ iff $\mymodelss{T}{\modelfun M}\rho\psi$;
in either case it then follows that $\mymodelss{D}M\rho\varphi\to\psi$
iff $\mymodelss{T}{\modelfun M}\rho\varphi\to\psi$.

On the other hand, if $\mymodelss{T}{\modelfun M}\rho\neg\varphi$ then
$\semDCa{}(\varphi\to\psi)$ become simply $\semDCa{}(\varphi)$.
Applying the induction hypothesis and Proposition~\ref{modelDprops} we
conclude that $\mymodelss{D}M\rho\neg\varphi$; also applying
Lemma~\ref{TwfDwwf} to the assumption
$\mymodelss{T}{\modelfun M}\rho\psi\wf$ allows us to conclude that
$\mymodelss{D}M\rho\psi\wwf$, from which follows that
$\mymodelss{D}M\rho\varphi\to\psi$.  Since in this case necessarily
$\mymodelss{T}{\modelfun M}\rho\varphi\to\psi$, this establishes the
thesis.

Finally assume that $\varphi$ is $\forall x_i\sep\psi$.  If
$\mymodelss{T}{\modelfun M}\rho\forall x_i\sep\psi\wf$, then
$\mymodelss{T}{\modelfun M}{\rho[x_i:=a]}\forall x_i\sep\psi\wf$ for all
$a\in A$; similarly, from $\semDCa{}(\forall x_i\sep\psi)$ we conclude
that $\semDC[{\mathfrak M},\rho[x_i:=a]]{}(\psi)$ hold for all $a\in A$.
Given any such $a$, then, the induction hypothesis allows us to conclude
that $\mymodelss{D}M{\rho[x_i:=a]}$, from which follows that
$\mymodelss{D}M\rho\forall x_i\sep\psi$.
\end{proof}

\begin{proposition}\label{TconsDcons}
Let $\varphi$ be a formula and $\Gamma$ be a context.  If
$\Gamma\mymodelss{T}{\modelfun M}\rho\varphi$ and $\semDCa{\Gamma}(\varphi)$
hold, then $\Gamma\mymodelss{D}{M}\rho\varphi$.
\end{proposition}
\begin{proof}
By induction on $\Gamma$.

Let $\Gamma$ be $\epsilon$.  The hypotheses then translate to
$\mymodelss{T}{\modelfun M}\rho\varphi$ and $\semDCa{}(\varphi)$.
By Proposition~\ref{modelTwf} we conclude that
$\mymodelss{T}{\modelfun M}\rho\varphi\wf$, and
applying Proposition~\ref{TintDint} yields
$\mymodelss{D}M\rho\varphi$, which is by definition the same as
$\epsilon\mymodelss{D}M\rho\varphi$.

Take now $\Gamma$ to be $\psi,\Gamma'$.  There are two cases to
consider.

Suppose that $\mymodelss{T}{\modelfun M}\rho\psi$ and
$\Gamma'\mymodelss{T}{\modelfun M}\rho\varphi$.  By definition of
$\semDC[{}]{}$ in this case also $\semDCa{}(\psi)$ and
$\semDCa{\Gamma'}(\varphi)$ must hold.  Then we may apply
Proposition~\ref{TintDint} as in the previous case to conclude that
$\mymodelss{D}M\rho\psi$; by induction hypothesis we conclude that
$\Gamma'\mymodelss{D}M\rho\varphi$, and the conjunction of these two
statements means that $\psi,\Gamma'\mymodelss{D}M\rho\varphi$.

On the other hand, if $\mymodelss{T}{\modelfun M}\rho\neg\psi$
then $\Gamma'\mymodelss{T}{\modelfun M}\rho\varphi\wf$ and
$\semDCa{}(\psi)$ hold.  Again we can conclude, using
Proposition~\ref{TintDint} and Lemma~\ref{modelDprops}, that
$\mymodelss{D}M\rho\neg\psi$; by Lemma~\ref{TwfDwwf} we conclude that
$\Gamma'\mymodelss{D}M\rho\varphi\wwf$, whence again
$\psi,\Gamma'\mymodelss{D}M\rho\varphi$.

Finally, let $\Gamma$ be $x_i,\Gamma'$.  If
$x_i,\Gamma'\mymodelss{T}{\modelfun M}\rho\varphi$ then
$\Gamma'\mymodelss{T}{\modelfun M}{\rho[x_i:=a]}\varphi$ for all $a\in A$.
Similarly, from $\semDCa{x_i,\Gamma'}(\varphi)$ we conclude that
$\semDC[{\modelfunm M},{\rho[x_i:=a]}]{\Gamma'}(\varphi)$ hold for all
$a\in A$.  Given any such $a$, the induction hypothesis allows us then
to conclude that $\Gamma'\mymodelss{D}M{\rho[x_i:=a]}\varphi$,
whence $x_i,\Gamma'\mymodelss{D}M\rho\varphi$.
\end{proof}

\begin{proposition}\label{TvalDval}
Let $\varphi$ be a formula and $\Gamma$ be a context.  If
$\Gamma\mymodels{T}\varphi$ and $\semDC[{}]{\Gamma}(\varphi)$ hold,
then $\Gamma\mymodels{D}\varphi$.
\end{proposition}
\begin{proof}
Suppose that $\Gamma\mymodels{T}\varphi$ and let $\mathfrak M$ be
a {\D}-model; then in particular
$\Gamma\mymodelss{T}{\modelfun M}\emptyset\varphi$.
Assume furthermore that $\semDC[{}]{\Gamma}(\varphi)$ hold; then also
$\semDC[{\modelfunm M},\emptyset]{\Gamma}(\varphi)$ hold.
We may then apply Proposition~\ref{TconsDcons} to conclude that
$\Gamma\mymodelss{D}M\emptyset\varphi$; since
$\mathfrak M$ is arbitrary we conclude that
$\Gamma\mymodels{D}\varphi$.
\end{proof}

\bigskip\noindent
We summarize these results in a theorem.
\begin{theorem}\label{TequivDsem}
Let $\Gamma$ be a context and $\varphi$ be a formula.  Then the
following statements are equivalent.
\begin{enumerate}[(i)]
\item $\Gamma\mymodels{D}\varphi$
\item $\Gamma\mymodels{T}\varphi$ and $\semDC[{}]{\Gamma}(\varphi)$ hold.
\end{enumerate}
\end{theorem}

%\begin{proposition}\label{freek10sem} Let $t\in\terms{D}$,
%$\varphi\in\lang{D}$ and $\rho$ be a {\D}-substitution for $\mathfrak M$.
%\begin{enumerate}[(i)]
%\item If $\mymodelss{D}M\rho t\wf$ then
%$\Gamma\mymodelss{D}M\rho t=\starfun{t}$.
%\item If $\mymodelss{D}M\rho\varphi\wf$ then
%$\mymodelss{D}M\rho\starfun\varphi\wf$ and
%$\Gamma\mymodelss{D}M\rho\varphi\leftrightarrow\starfun{\varphi}$.
%\end{enumerate}
%\end{proposition}
%\begin{proof}
%We prove this for all $\rho$, by structural induction on $t$ and
%$\varphi$.
%The only case which neither is trivial nor follows directly from the
%induction hypothesis is the case $t=f_i(t_1,\ldots,t_{a_i})$.

%Suppose that $\mymodelss{D}M\rho f_i(t_1,\ldots,t_{a_i})\wf$; this
%means that $\ints{D}M\rho{f_i(t_1,\ldots,t_{a_i})}$ is defined, which
%in turn is equivalent to saying that $\ints{D}M\rho{t_k}$ is defined for
%each $k$ and that
%$\intm{D}M{f_i}(\ints{D}M\rho{t_1},\ldots,\ints{D}M\rho{t_{a_i}})$ is
%defined, which in turn implies that
%$(\ints{D}M\rho{t_1},\ldots,\ints{D}M\rho{t_{a_i}})\in\intm{D}M{D_{f_i}}$.
%From these and the induction hypothesis we can conclude that
%$(\ints{D}M\rho{\starfun{t_1}},\ldots,\ints{D}M\rho{\starfun{t_{a_i}}})\in%
%\intm{D}M{D_{f_i}}$,
%from which follows that
%$\mymodelss{D}M\rho D_{f_i}(\ints{D}M\rho{\starfun{t_1}},\ldots,\ints{D}M\rho{\starfun{t_{a_i}}})$
%and hence
%$\ints{D}M\rho{\starfun{f_i(t_1,\ldots,t_{a_i})}}=%
%\ints{D}M\rho{\ifthelse{D_{f_i}(\starfun{t_1},\ldots,\starfun{t_{a_i}})}{f_i(\starfun{t_1},\ldots,\starfun{t_{a_i}})}{c_1}}=%
%\ints{D}M\rho{f_i(\starfun{t_1},\ldots,\starfun{t_{a_i}})}=%
%\ints{D}M\rho{f_i(t_1,\ldots,t_{a_i})}$.

%The strengthened conclusion in the second statement is present for
%the case $\varphi\to\psi$, where the side condition
%$\mymodelss{D}M\rho\starfun{(\varphi\to\psi)}\wf$ has to be ensured.
%\end{proof}

%\begin{proposition}\label{freek12sem}Let $\Gamma$ be a {\D}-context,
%$t\in\terms{D}$, $\varphi\in\lang{D}$ and $\rho$ be a {\D}-substitution
%for $\mathfrak M$.
%\begin{enumerate}[(i)]
%\item $\Gamma\mymodelss{T}{\modelfunm M}\rho\ok$ iff
%$\starfun\Gamma\mymodelss{D}M\rho\ok$.
%\item $\Gamma\mymodelss{T}{\modelfunm M}\rho t\wf$ iff
%$\starfun\Gamma\mymodelss{D}M\rho\starfun{t}\wf$.
%\item $\Gamma\mymodelss{T}{\modelfunm M}\rho\varphi\wf$ iff
%$\starfun\Gamma\mymodelss{D}M\rho\starfun{\varphi}\wf$.
%\item $\Gamma\mymodelss{T}{\modelfunm M}\rho\varphi$ iff
%$\starfun\Gamma\mymodelss{D}M\rho\starfun{\varphi}$.
%\end{enumerate}
%\end{proposition}
%\begin{proof}
%All four points are proved simultaneously by induction on $\Gamma$.

%If $\Gamma$ is $\epsilon$ then the first case is trivial while the
%other three follow directly from Proposition~\ref{starfunprops}.

%If $\Gamma$ is $\psi,\Gamma'$, the three first cases are analogous; we
%consider the third.  By definition,
%$\psi,\Gamma'\mymodelss{T}{\modelfunm M}\rho\varphi\wf$ iff
%$\mymodelss{T}{\modelfunm M}\rho\psi\wf$ and
%$\Gamma'\mymodelss{T}{\modelfunm M}\rho\varphi\wf$.
%By Proposition~\ref{starfunprops} and the induction hypothesis these are
%equivalent to
%$\mymodelss{D}M\rho\starfun\psi\wf$ and
%$\starfun{(\Gamma')}\mymodelss{D}M\rho\starfun\varphi\wf$,
%the conjunction of which is the definition of
%$\starfun{(\psi,\Gamma')}\mymodelss{D}M\rho\starfun\varphi\wf$.
%The fourth case is similar, using the result of the third case.

%Finally, if $\Gamma$ is $x_i,\Gamma'$ all cases are similar, so we
%again consider the third.  By definition,
%$x_i,\Gamma'\mymodelss{T}{\modelfunm M}\rho\varphi\wf$ iff
%$\Gamma'\mymodelss{T}{\modelfunm M}{\rho[x_i:=a]}\varphi\wf$ for all
%$a\in A$.  By induction hypothesis this is equivalent to
%$\starfun{(\Gamma')}\mymodelss{D}M{\rho[x_i:=a]}\starfun\varphi\wf$ for
%all $a\in A$, which is the definition of
%$\starfun{(x_i,\Gamma')}\mymodelss{D}M\rho\starfun\varphi\wf$.
%\end{proof}

%\begin{proposition}\label{freek14sem} Let $\Gamma$ be a {\D}-context,
%$\varphi\in\lang{D}$ and $\rho$ be a {\D}-substitution for
%$\mathfrak M$ such that $\Gamma\mymodelss{D}M\rho\varphi\wf$.  If
%$\starfun\Gamma\mymodelss{D}M\rho\starfun\varphi$, then
%$\Gamma\mymodelss{D}M\rho\varphi$.
%\end{proposition}
%\begin{proof}
%By induction on $\Gamma$.  If $\Gamma$ is $\epsilon$, then the thesis
%follows from Proposition~\ref{freek10sem}.

%If $\starfun{(\psi,\Gamma)}\mymodelss{D}M\rho\starfun\varphi$,
%then there are two cases to consider.  If
%$\mymodelss{D}M\rho\neg\starfun\psi$ then Proposition~\ref{freek10sem} is
%applicable, yielding $\mymodelss{D}M\rho\neg\psi$; from the hypothesis
%$\psi,\Gamma\mymodelss{D}M\rho\varphi\wf$ we conclude that
%$\Gamma\mymodelss{D}M\rho\varphi\wf$, hence
%$\psi,\Gamma\mymodelss{D}M\rho\varphi$.

%In the second case both $\mymodelss{D}M\rho\starfun\psi$ and
%$\starfun\Gamma\mymodelss{D}M\rho\starfun\varphi$; applying
%Proposition~\ref{freek10sem} to the first and the induction hypothesis
%to the second yields respectively $\mymodelss{D}M\rho\psi$ and
%$\Gamma\mymodelss{D}M\rho\varphi$, from which we can again conclude
%that $\psi,\Gamma\mymodelss{D}M\rho\varphi$.

%Finally, $\starfun{(x_i,\Gamma)}\mymodelss{D}M\rho\starfun\varphi$ iff
%$\starfun\Gamma\mymodelss{D}M{\rho[x_i:=a]}\starfun\varphi$ for all
%$a\in A$; from the induction hypothesis we get
%$\Gamma\mymodelss{D}M{\rho[x_i:=a]}\varphi$ for all $a\in A$, whence
%$x_i,\Gamma\mymodelss{D}M\rho\varphi$.
%\end{proof}

%\begin{proposition}\label{freek15sem} Let $\Gamma$ be a {\D}-context,
%$\varphi\in\lang{D}$ and $\rho$ be a {\D}-substitution for
%$\mathfrak M$ such that $\Gamma\mymodelss{D}M\rho\varphi\wf$.  If
%$\Gamma\mymodelss{T}{\modelfunm M}\rho\varphi$, then
%$\Gamma\mymodelss{D}M\rho\varphi$.
%\end{proposition}
%\begin{proof}
%By Proposition~\ref{freek12sem}, if
%$\Gamma\mymodelss{T}{\modelfunm M}\rho\varphi$ then
%$\starfun\Gamma\mymodelss{D}M\rho\starfun\varphi$.  Since
%$\Gamma\mymodelss{D}M\rho\varphi\wf$, by Proposition~\ref{freek14sem} we
%conclude that $\Gamma\mymodelss{D}M\rho\varphi$.
%\end{proof}

%\begin{corollary}\label{freek15sema} Let $\Gamma$ be a {\D}-context and
%$\varphi\in\lang{D}$ such that $\Gamma\mymodelsm{D}M\varphi\wf$.  If
%$\Gamma\mymodelsm{T}{\modelfunm M}\varphi$, then
%$\Gamma\mymodelsm{D}M\varphi$.
%\end{corollary}
%\begin{proof} Just apply the previous proposition with $\rho=\emptyset$.
%\end{proof}

%\begin{corollary}\label{freek15semb} Let $\Gamma$ be a {\D}-context and
%$\varphi\in\lang{D}$ such that $\Gamma\mymodels{D}\varphi\wf$.  If
%$\Gamma\mymodels{T}\varphi$, then
%$\Gamma\mymodels{D}\varphi$.
%\end{corollary}
%\begin{proof}
%Let $\mathfrak M$ be an arbitrary {\D}-model.  Then
%$\Gamma\mymodelss{T}{\modelfunm M}\varphi$ and
%$\Gamma\mymodelss{D}M\varphi\wf$, hence by the previous result
%$\Gamma\mymodelss{D}M\varphi$.  Since $\mathfrak M$ is arbitrary, we conclude
%that $\Gamma\mymodels{D}\varphi$.
%\end{proof}

\subsection{System {\D}: correctness and completeness}

Again we can use the equivalence we showed above to establish completeness
of system {\D}.
\begin{theorem}\emph{(Completeness)}\label{Dcomplete}
Let $\Gamma$ be a context and $\varphi$ be a formula.
Then $\Gamma\myvdash{D}\varphi$ iff $\Gamma\yields{D}\varphi$.
\end{theorem}
\begin{proof}
First notice that if $\Gamma\myvdash{T}\varphi$ then
$\Gamma\yields{T}\varphi\wf$ because of Proposition~3
of~\cite{wie:zwa:03} and of Lemma~\ref{Twfsyntsem}; also if
$\Gamma\yields{T}\varphi$ then $\Gamma\yields{T}\varphi\wf$
(this is a trivial consequence of Proposition~\ref{modelTwf}).
This justifies the application of Theorem~\ref{TequivDsem} in the
following reasoning.

By Theorem~\ref{TequivD}, $\Gamma\myvdash{D}\varphi$ iff (1)
$\Gamma\myvdash{T}\varphi$, (2) $\synDC(\Gamma)$ hold and
(3) $\synDC[\Gamma](\varphi)$ hold.  By Theorem~\ref{Tcomplete},
(1) is equivalent to $\Gamma\yields{T}\varphi$; by
Corollary~\ref{synDCsemDC}, (2) and (3) are equivalent to
$\semDC[{}]{\Gamma}(\varphi)$ holding.  By Theorem~\ref{TequivDsem},
the conjunction of these two holds iff $\Gamma\yields{D}\varphi$.
\end{proof}

\section{Conclusions}

%\section*{Acknowledgments}

\bibliographystyle{plain}
\bibliography{folp-i}

\end{document}
